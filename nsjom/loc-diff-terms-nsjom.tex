%% FILE: loc-diff-terms-ijac.tex
%% AUTHOR: William DeMeo, Ralph Freese
%% DATE: 1 Dec 2016
%% COPYRIGHT: (C) 2016 William DeMeo 

%%%%%%%%%%%%%%%%%%%%%%%%%%%%%%%%%%%%%%%%%%%%%%%%%%%%%%%%%%
%%                         BIBLIOGRAPHY FILE            %%
%%%%%%%%%%%%%%%%%%%%%%%%%%%%%%%%%%%%%%%%%%%%%%%%%%%%%%%%%%
%% The `filecontents` command will crete a file in the inputs directory called 
%% refs.bib containing the references in the document, in case this file does 
%% not exist already.
%% If you want to add a BibTeX entry, please don't add it directly to the
%% refs.bib file.  Instead, add it in this file between the
%% \begin{filecontents*}{refs.bib} and \end{filecontents*} lines
%% then delete the existing refs.bib file so it will be automatically generated 
%% again with your new entry the next time you run pdfaltex.
\begin{filecontents*}{refs.bib}
@article {MR1637665,
    AUTHOR = {Lipparini, Paolo},
     TITLE = {A characterization of varieties with a difference term. {II}.
              {N}eutral {$=$} meet semi-distributive},
   JOURNAL = {Canad. Math. Bull.},
  FJOURNAL = {Canadian Mathematical Bulletin. Bulletin Canadien de
              Math\'ematiques},
    VOLUME = {41},
      YEAR = {1998},
    NUMBER = {3},
     PAGES = {318--327},
      ISSN = {0008-4395},
   MRCLASS = {08B05},
  MRNUMBER = {1637665},
MRREVIEWER = {E. W. Kiss},
       DOI = {10.4153/CMB-1998-044-9},
       URL = {http://dx.doi.org/10.4153/CMB-1998-044-9},
}
@article{Bergman-DeMeo-2016,
  author    = {Clifford Bergman and William DeMeo},
  title     = {Universal Algebraic Methods for Constraint Satisfaction Problems},
  journal   = {CoRR},
  volume    = {abs/1611.02867},
  year      = {2016},
  url       = {http://arxiv.org/abs/1611.02867},
  timestamp = {Thu, 01 Dec 2016 19:32:08 +0100},
  biburl    = {http://dblp.uni-trier.de/rec/bib/journals/corr/BergmanD16},
  bibsource = {dblp computer science bibliography, http://dblp.org}
}
@article {MR1257643,
    AUTHOR = {Lipparini, Paolo},
     TITLE = {Commutator theory without join-distributivity},
   JOURNAL = {Trans. Amer. Math. Soc.},
  FJOURNAL = {Transactions of the American Mathematical Society},
    VOLUME = {346},
      YEAR = {1994},
    NUMBER = {1},
     PAGES = {177--202},
      ISSN = {0002-9947},
   MRCLASS = {08B10 (08A30 08B05)},
  MRNUMBER = {1257643},
MRREVIEWER = {H. Peter Gumm},
       DOI = {10.2307/2154948},
       URL = {http://dx.doi.org/10.2307/2154948},
}
@article {MR1145556,
    AUTHOR = {Day, A. and Gumm, H. P.},
     TITLE = {Some characterizations of the commutator},
   JOURNAL = {Algebra Universalis},
  FJOURNAL = {Algebra Universalis},
    VOLUME = {29},
      YEAR = {1992},
    NUMBER = {1},
     PAGES = {61--78},
      ISSN = {0002-5240},
   MRCLASS = {08B10 (08B99)},
  MRNUMBER = {1145556},
MRREVIEWER = {David Hobby},
       DOI = {10.1007/BF01190756},
       URL = {http://dx.doi.org/10.1007/BF01190756},
}
@article {MR541622,
    AUTHOR = {Hagemann, Joachim and Herrmann, Christian},
     TITLE = {A concrete ideal multiplication for algebraic systems and its
              relation to congruence distributivity},
   JOURNAL = {Arch. Math. (Basel)},
  FJOURNAL = {Archiv der Mathematik. Archives of Mathematics. Archives
              Math\'ematiques},
    VOLUME = {32},
      YEAR = {1979},
    NUMBER = {3},
     PAGES = {234--245},
      ISSN = {0003-889X},
   MRCLASS = {08B10},
  MRNUMBER = {541622},
MRREVIEWER = {Ralph Freese},
       DOI = {10.1007/BF01238496},
       URL = {http://dx.doi.org/10.1007/BF01238496},
}
@book {MR0432511,
    AUTHOR = {Smith, Jonathan D. H.},
     TITLE = {Mal$\prime$cev varieties},
    SERIES = {Lecture Notes in Mathematics, Vol. 554},
 PUBLISHER = {Springer-Verlag, Berlin-New York},
      YEAR = {1976},
     PAGES = {viii+158},
   MRCLASS = {08A15},
  MRNUMBER = {0432511},
MRREVIEWER = {V. A. Artamonov},
}
@article {MR590312,
    AUTHOR = {Gumm, H.-Peter},
     TITLE = {An easy way to the commutator in modular varieties},
   JOURNAL = {Arch. Math. (Basel)},
  FJOURNAL = {Archiv der Mathematik. Archives of Mathematics. Archives
              Math\'ematiques},
    VOLUME = {34},
      YEAR = {1980},
    NUMBER = {3},
     PAGES = {220--228},
      ISSN = {0003-889X},
   MRCLASS = {08B10},
  MRNUMBER = {590312},
MRREVIEWER = {Alden Pixley},
       DOI = {10.1007/BF01224955},
       URL = {http://dx.doi.org/10.1007/BF01224955},
}
@article {MR1871085,
    AUTHOR = {Bergman, Clifford and Slutzki, Giora},
     TITLE = {Computational complexity of some problems involving
              congruences on algebras},
   JOURNAL = {Theoret. Comput. Sci.},
  FJOURNAL = {Theoretical Computer Science},
    VOLUME = {270},
      YEAR = {2002},
    NUMBER = {1-2},
     PAGES = {591--608},
      ISSN = {0304-3975},
     CODEN = {TCSDI},
   MRCLASS = {08A30 (05C85 08A35 68Q17)},
  MRNUMBER = {1871085 (2002i:08002)},
MRREVIEWER = {Radim B{\v{e}}lohl{\'a}vek},
       DOI = {10.1016/S0304-3975(01)00009-3},
       URL = {http://dx.doi.org/10.1016/S0304-3975(01)00009-3},
}
@article {MR1695293,
    AUTHOR = {Bergman, Clifford and Juedes, David and Slutzki, Giora},
     TITLE = {Computational complexity of term-equivalence},
   JOURNAL = {Internat. J. Algebra Comput.},
  FJOURNAL = {International Journal of Algebra and Computation},
    VOLUME = {9},
      YEAR = {1999},
    NUMBER = {1},
     PAGES = {113--128},
      ISSN = {0218-1967},
   MRCLASS = {68Q17 (08A70 68Q15)},
  MRNUMBER = {1695293 (2000b:68088)},
       DOI = {10.1142/S0218196799000084},
       URL = {http://dx.doi.org/10.1142/S0218196799000084},
}
@article {MR3449235,
    AUTHOR = {Kearnes, Keith and Szendrei, {\'A}gnes and Willard, Ross},
     TITLE = {A finite basis theorem for difference-term varieties with a
              finite residual bound},
   JOURNAL = {Trans. Amer. Math. Soc.},
  FJOURNAL = {Transactions of the American Mathematical Society},
    VOLUME = {368},
      YEAR = {2016},
    NUMBER = {3},
     PAGES = {2115--2143},
      ISSN = {0002-9947},
   MRCLASS = {03C05 (08B05 08B10)},
  MRNUMBER = {3449235},
       DOI = {10.1090/tran/6509},
       URL = {http://dx.doi.org/10.1090/tran/6509},
}
@article {MR1663558,
    AUTHOR = {Kearnes, Keith A. and Szendrei, {\'A}gnes},
     TITLE = {The relationship between two commutators},
   JOURNAL = {Internat. J. Algebra Comput.},
  FJOURNAL = {International Journal of Algebra and Computation},
    VOLUME = {8},
      YEAR = {1998},
    NUMBER = {4},
     PAGES = {497--531},
      ISSN = {0218-1967},
   MRCLASS = {08A05 (08A30)},
  MRNUMBER = {1663558},
MRREVIEWER = {M. G. Stone},
       DOI = {10.1142/S0218196798000247},
       URL = {http://dx.doi.org/10.1142/S0218196798000247},
}
@article{KSW,
title = {Simpler maltsev conditions for (weak) difference terms in locally finite varieties},
author = {Kearnes, Keith and Szendrei, \'{A}gnes and Willard, Ross},
note = {to appear}
}

@article {MR3239624,
    AUTHOR = {Valeriote, M. and Willard, R.},
     TITLE = {Idempotent {$n$}-permutable varieties},
   JOURNAL = {Bull. Lond. Math. Soc.},
  FJOURNAL = {Bulletin of the London Mathematical Society},
    VOLUME = {46},
      YEAR = {2014},
    NUMBER = {4},
     PAGES = {870--880},
      ISSN = {0024-6093},
   MRCLASS = {08A05 (06F99 68Q25)},
  MRNUMBER = {3239624},
       DOI = {10.1112/blms/bdu044},
       URL = {http://dx.doi.org/10.1112/blms/bdu044},
}
@article {MR3350327,
    AUTHOR = {Kozik, Marcin and Krokhin, Andrei and Valeriote, Matt and
              Willard, Ross},
     TITLE = {Characterizations of several {M}altsev conditions},
   JOURNAL = {Algebra Universalis},
  FJOURNAL = {Algebra Universalis},
    VOLUME = {73},
      YEAR = {2015},
    NUMBER = {3-4},
     PAGES = {205--224},
      ISSN = {0002-5240},
   MRCLASS = {08B05 (08A70 08B10)},
  MRNUMBER = {3350327},
MRREVIEWER = {David Hobby},
       DOI = {10.1007/s00012-015-0327-2},
       URL = {http://dx.doi.org/10.1007/s00012-015-0327-2},
}
@article {MR1358491,
    AUTHOR = {Kearnes, Keith A.},
     TITLE = {Varieties with a difference term},
   JOURNAL = {J. Algebra},
  FJOURNAL = {Journal of Algebra},
    VOLUME = {177},
      YEAR = {1995},
    NUMBER = {3},
     PAGES = {926--960},
      ISSN = {0021-8693},
     CODEN = {JALGA4},
   MRCLASS = {08B10 (08B05)},
  MRNUMBER = {1358491},
MRREVIEWER = {H. Peter Gumm},
       DOI = {10.1006/jabr.1995.1334},
       URL = {http://dx.doi.org/10.1006/jabr.1995.1334},
}
@book {MR2839398,
    AUTHOR = {Bergman, Clifford},
     TITLE = {Universal algebra},
    SERIES = {Pure and Applied Mathematics (Boca Raton)},
    VOLUME = {301},
      NOTE = {Fundamentals and selected topics},
 PUBLISHER = {CRC Press, Boca Raton, FL},
      YEAR = {2012},
     PAGES = {xii+308},
      ISBN = {978-1-4398-5129-6},
   MRCLASS = {08-02 (06-02 08A40 08B05 08B10 08B26)},
  MRNUMBER = {2839398 (2012k:08001)},
MRREVIEWER = {Konrad P. Pi{\'o}ro},
}
@article {MR0434928,
    AUTHOR = {Taylor, Walter},
     TITLE = {Varieties obeying homotopy laws},
   JOURNAL = {Canad. J. Math.},
  FJOURNAL = {Canadian Journal of Mathematics. Journal Canadien de
              Math\'ematiques},
    VOLUME = {29},
      YEAR = {1977},
    NUMBER = {3},
     PAGES = {498--527},
      ISSN = {0008-414X},
   MRCLASS = {08A25},
  MRNUMBER = {0434928 (55 \#7891)},
MRREVIEWER = {James B. Nation},
}
  @BOOK{HM:1988,
    AUTHOR = {Hobby, David and McKenzie, Ralph},
    TITLE = {The structure of finite algebras},
    SERIES = {Contemporary Mathematics},
    VOLUME = {76},
    PUBLISHER = {American Mathematical Society},
    ADDRESS = {Providence, RI},
    YEAR = {1988},
    PAGES = {xii+203},
    ISBN = {0-8218-5073-3},
    MRCLASS = {08A05 (03C05 08-02 08B05)},
    MRNUMBER = {958685 (89m:08001)},
    MRREVIEWER = {Joel Berman},
    note = {Available from:
      \href{http://math.hawaii.edu/~ralph/Classes/619/HobbyMcKenzie-FiniteAlgebras.pdf}{math.hawaii.edu}}
  }
@article {MR0455543,
    AUTHOR = {Jones, Neil D. and Laaser, William T.},
     TITLE = {Complete problems for deterministic polynomial time},
   JOURNAL = {Theoret. Comput. Sci.},
  FJOURNAL = {Theoretical Computer Science},
    VOLUME = {3},
      YEAR = {1976},
    NUMBER = {1},
     PAGES = {105--117 (1977)},
      ISSN = {0304-3975},
   MRCLASS = {68A20},
  MRNUMBER = {0455543},
MRREVIEWER = {Forbes D. Lewis},
       DOI = {10.1016/0304-3975(76)90068-2},
       URL = {http://dx.doi.org/10.1016/0304-3975(76)90068-2},
}
@article{DeMeo:2017-ext,
  author    = {William DeMeo},
  title     = {On the Complexity of Difference Term Existence},
  year      = {2017},
  note      = {This is a place-holder to be used after posting long version on arXiv.},
  url       = {https://github.com/UniversalAlgebra/diff-term-existence}
}
@article{Bergman-DeMeo,
  author    = {Clifford Bergman and William DeMeo},
  title     = {Universal Algebraic Methods for Constraint Satisfaction Problems},
  journal   = {CoRR},
  year      = {2016},
  note      = {Available at: \url{http://arxiv.org/abs/1611.02867}},
  url       = {http://arxiv.org/abs/1611.02867},
  timestamp = {Thu, 01 Dec 2016 19:32:08 +0100},
  biburl    = {http://dblp.uni-trier.de/rec/bib/journals/corr/BergmanD16},
  bibsource = {dblp computer science bibliography, http://dblp.org}
}
@unpublished{Bergman-DeMeo-alt,
    AUTHOR = {Bergman, Clifford and DeMeo, William},
    TITLE = {Universal Algebraic Methods for Constraint Satisfaction Problems},
    YEAR = {2016},
    NOTE = {unpublished notes; available online},
    URL = {https://github.com/UniversalAlgebra/algebraic-csp}
}
  @article {Freese:2009,
    AUTHOR = {Freese, Ralph and Valeriote, Matthew A.},
    TITLE = {On the complexity of some {M}altsev conditions},
    JOURNAL = {Internat. J. Algebra Comput.},
    FJOURNAL = {International Journal of Algebra and Computation},
    VOLUME = {19},
    YEAR = {2009},
    NUMBER = {1},
    PAGES = {41--77},
    ISSN = {0218-1967},
    MRCLASS = {08B05 (03C05 08B10 68Q25)},
    MRNUMBER = {2494469 (2010a:08008)},
    MRREVIEWER = {Clifford H. Bergman},
    DOI = {10.1142/S0218196709004956},
    URL = {http://dx.doi.org/10.1142/S0218196709004956}
  }
@article {MR3076179,
    AUTHOR = {Kearnes, Keith A. and Kiss, Emil W.},
     TITLE = {The shape of congruence lattices},
   JOURNAL = {Mem. Amer. Math. Soc.},
  FJOURNAL = {Memoirs of the American Mathematical Society},
    VOLUME = {222},
      YEAR = {2013},
    NUMBER = {1046},
     PAGES = {viii+169},
      ISSN = {0065-9266},
      ISBN = {978-0-8218-8323-5},
   MRCLASS = {08B05 (08B10)},
  MRNUMBER = {3076179},
MRREVIEWER = {James B. Nation},
       DOI = {10.1090/S0065-9266-2012-00667-8},
       URL = {http://dx.doi.org/10.1090/S0065-9266-2012-00667-8},
}
@incollection {MR1404955,
    AUTHOR = {Kearnes, Keith A.},
     TITLE = {Idempotent simple algebras},
 BOOKTITLE = {Logic and algebra ({P}ontignano, 1994)},
    SERIES = {Lecture Notes in Pure and Appl. Math.},
    VOLUME = {180},
     PAGES = {529--572},
 PUBLISHER = {Dekker, New York},
      YEAR = {1996},
   MRCLASS = {08B05 (06F25 08A05 08A30)},
  MRNUMBER = {1404955 (97k:08004)},
MRREVIEWER = {E. W. Kiss},
}
@misc{william_demeo_2016_53936,
  author       = {DeMeo, William and Freese, Ralph},
  title        = {AlgebraFiles v1.0.1},
  month        = May,
  year         = 2016,
  doi          = {10.5281/zenodo.53936},
  url          = {http://dx.doi.org/10.5281/zenodo.53936}
}
@article{FreeseMcKenzie2016,
	Author = {Freese, Ralph and McKenzie, Ralph},
	Date-Added = {2016-08-22 19:43:56 +0000},
	Date-Modified = {2016-08-22 19:45:50 +0000},
	Journal = {Algebra Universalis},
	Title = {Mal'tsev families of varieties closed under join or Mal'tsev product},
	Year = {to appear}
}
@article {MR2333368,
    AUTHOR = {Kearnes, Keith A. and Tschantz, Steven T.},
     TITLE = {Automorphism groups of squares and of free algebras},
   JOURNAL = {Internat. J. Algebra Comput.},
  FJOURNAL = {International Journal of Algebra and Computation},
    VOLUME = {17},
      YEAR = {2007},
    NUMBER = {3},
     PAGES = {461--505},
      ISSN = {0218-1967},
   MRCLASS = {08A35 (08B20 20B25)},
  MRNUMBER = {2333368},
MRREVIEWER = {Giovanni Ferrero},
       DOI = {10.1142/S0218196707003615},
       URL = {http://dx.doi.org/10.1142/S0218196707003615},
}
@article {MR2504025,
    AUTHOR = {Valeriote, Matthew A.},
     TITLE = {A subalgebra intersection property for congruence distributive
              varieties},
   JOURNAL = {Canad. J. Math.},
  FJOURNAL = {Canadian Journal of Mathematics. Journal Canadien de
              Math\'ematiques},
    VOLUME = {61},
      YEAR = {2009},
    NUMBER = {2},
     PAGES = {451--464},
      ISSN = {0008-414X},
     CODEN = {CJMAAB},
   MRCLASS = {08B10 (08A30 08B05)},
  MRNUMBER = {2504025},
MRREVIEWER = {Jarom{\'{\i}}r Duda},
       DOI = {10.4153/CJM-2009-023-2},
       URL = {http://dx.doi.org/10.4153/CJM-2009-023-2},
}
@misc{UACalc,
	Author = {Ralph Freese and Emil Kiss and Matthew Valeriote},
	Date-Added = {2014-11-20 01:52:20 +0000},
	Date-Modified = {2014-11-20 01:52:20 +0000},
	Note = {Available at: {\verb+www.uacalc.org+}},
	Title = {Universal {A}lgebra {C}alculator},
	Year = {2011}
}
@article {MR2470585,
    AUTHOR = {Freese, Ralph},
     TITLE = {Computing congruences efficiently},
   JOURNAL = {Algebra Universalis},
  FJOURNAL = {Algebra Universalis},
    VOLUME = {59},
      YEAR = {2008},
    NUMBER = {3-4},
     PAGES = {337--343},
      ISSN = {0002-5240},
   MRCLASS = {08A30 (08A40 68W30 68W40)},
  MRNUMBER = {2470585 (2009j:08003)},
MRREVIEWER = {Clifford H. Bergman},
       DOI = {10.1007/s00012-008-2073-1},
       URL = {http://dx.doi.org/10.1007/s00012-008-2073-1},
}
@incollection {MR1191235,
    AUTHOR = {Szendrei, {\'A}gnes.},
     TITLE = {A survey on strictly simple algebras and minimal varieties},
 BOOKTITLE = {Universal algebra and quasigroup theory ({J}adwisin, 1989)},
    SERIES = {Res. Exp. Math.},
    VOLUME = {19},
     PAGES = {209--239},
 PUBLISHER = {Heldermann, Berlin},
      YEAR = {1992},
   MRCLASS = {08-02 (08A40 08B05)},
  MRNUMBER = {1191235 (93h:08001)},
MRREVIEWER = {Ivan Chajda},
}
@unpublished{Bergman-DeMeo,
    AUTHOR = {Bergman, Clifford and DeMeo, William},
    TITLE = {Universal Algebraic Methods for Constraint Satisfaction Problems:
      with applications to commutative idempotent binars},
    YEAR = {2016},
    NOTE = {unpublished notes; soon to be available online},
    URL = {https://github.com/UniversalAlgebra/algebraic-csp}
}
\end{filecontents*}
%:biblio
%%%%%%%%%%%%%%%%%%%%%%%%%%%%%%%%%%%%%%%%%%%%%%%%%%%%%%%%%%%%%%%%%%%%%%%%%%%%%%%%%%%%
%%                                     PREAMBLE                                   %%
%%%%%%%%%%%%%%%%%%%%%%%%%%%%%%%%%%%%%%%%%%%%%%%%%%%%%%%%%%%%%%%%%%%%%%%%%%%%%%%%%%%%
\documentclass[leqno,twoside]{article}
\headsep 0.5cm \pagestyle{myheadings}
\usepackage{amssymb,amsmath,latexsym, amsthm,enumerate, nsjom, amsfonts}
\title{}\author{}\date{}
\markboth{W.~DeMeo and R.~Freese}{Complexity of Difference Term Existence}\setcounter{page}{1}
%% \usepackage[dvipdfm, pdfstartview=FitH]{hyperref} % Use LaTeX and then DVI2PDF,
% or, if you plan to use packages only compatible with PDFLaTeX
\usepackage[pdftex, pdfstartview=FitH]{hyperref}

%% \usepackage{graphicx}
% Use this package if you plan to put some pictures

%% \usepackage{pinlabel}  %%% was the recommended graphics+labelling package for JLA

%If you want equations number as (2.1), where, 2 is the section number then use
\numberwithin{equation}{section} %You may omit this line if you want numbering as (1)

 \newtheorem{thm}{Theorem}[section]
 \newtheorem*{theorem}{Theorem}        % theorem environment without numbering
 \newtheorem{cor}[thm]{Corollary}
 \newtheorem*{corollary}{Corollary}     % Corollary environment without numbering 
 \newtheorem{lem}[thm]{Lemma}
 \newtheorem*{lemma}{Lemma}            % Lemma environment without numbering 
 \newtheorem*{zlem}{Zorn's Lemma}      % A special unnumbered lemma.
 \newtheorem*{monotonicity}{Monotonicity of the Commutator} % A special unnumbered lemma.
 \newtheorem{prop}[thm]{Proposition}

 \theoremstyle{definition}
 \newtheorem{defn}[thm]{Definition}
 \newtheorem{prob}{Problem}    
 \newtheorem{ex}[thm]{Example}

 \theoremstyle{remark}
 \newtheorem{rem}[thm]{Remark}
 \newtheorem{Fact}{Fact}[section]
 \newtheorem{remarks}{Remarks}



 \usepackage[paperwidth=165mm, paperheight=235mm, twoside, hmargin={25mm,20mm}, vmargin={20mm,20mm} ]{geometry}

% You may use some variations of this environments, for instance, with separate global counters
% \newtheorem{defn}{Definition}
% Also, you may use your own names for environments like
% \newtheorem{df}{Definition}
% or whatever you prefer, but, please, use them

 %%%%%%%%%%%%%%%%%%%%%%%%%%%%%%%%%%%%%%%%%%%%%%%%%%%%%%%%%%%%%%%%%
 %%% MY STUFF
 \usepackage{macros}
 \usepackage{comment}
 %%%%%%%%%%%%%%%%%%%%%%%%%%%%%%%%%%%%%%%%%%%%%%%
 %% showkeys: just comment out in the final version
 \usepackage[notref,notcite]{showkeys}
 %%%%%%%%%%%%%%%%%%%%%%%%%%%%%%%%%%%%%%%%%%%%%%%
 %% removed these for jla
 % \usepackage{amsmath,amssymb,amsthm} %, amsmath are included by default}
 %% \usepackage{amscd}
 \usepackage{mathtools}
 % \usepackage{scrextend}
 \usepackage{lmodern}% http://ctan.org/pkg/lm
 %\usepackage{bm}
 \usepackage{latexsym,enumerate,scalefnt,ifthen} %,mathrsfs,
 \usepackage{stmaryrd}
 \SetSymbolFont{stmry}{bold}{U}{stmry}{m}{n}
 \usepackage[mathscr]{euscript}
 %% \usepackage[colorlinks=true,urlcolor=black,linkcolor=black,citecolor=black]{hyperref}
 \usepackage{scalefnt}
 \usepackage{tikz}
 \usepackage{color}
 %% \usepackage[margin=1.5in]{geometry}

 \newboolean{todos}
 \setboolean{todos}{true}  % set to true to include TODO statements
 %   \setboolean{todos}{false}  % set to false to exclude TODO statements

   %%% INTENDED USE OF THE arxiv AND extralong BOOLEAN VARIABLES
   %%% -- The brief journal version should have `arxiv` and `extralong` variables set to false.
   %%% -- The arxiv version should have `arxiv` set to true and `extralong` set to false.
   %%% -- The extralong version may contain notes intended for our own personal reference.
   %%%    For the extralong version, set both `arxiv` and `extralong` to true.   
   \newboolean{arxiv}
   \setboolean{arxiv}{true}  % set to true to include almost everything
   \setboolean{arxiv}{false}  % set to false for the brief version

   \newboolean{extralong}
   \setboolean{extralong}{true}  % set to true to include everything
   %% \setboolean{extralong}{false}  % set to false for the long (but not too long) version

   \newboolean{footnotes}
   \setboolean{footnotes}{true}  % set to true to include footnotes
   \setboolean{footnotes}{false}  % set to false for no footnotes


   \newboolean{draftsecskip}
   \setboolean{draftsecskip}{true}
   % \setboolean{draftsecskip}{false}

   \newboolean{thetanotation}
   \setboolean{thetanotation}{true}
   \setboolean{thetanotation}{false}


   % \newcommand\draftsecskip{\ifthenelse{\boolean{draftsecskip}}{\medskip}{}}
   \newcommand\draftsecskip{\ifthenelse{\boolean{draftsecskip}}{\newpage}{}}

   %%%% wjd: adding pagebreaks for ``draft mode'' to reduce printing costs
   %%%%      To turn off these unnecessary page breaks, set `draft` to false:
   \newboolean{draft}
   \setboolean{draft}{true}  % set to true to include footnotes

   \newcommand\extendedref{\cite{DeMeo:2017-ext}}
   %% \newcommand{\glocal}{global-local} %% This is a weird name... changing it:
   \newcommand{\glocal}{local\xspace}
   

%%%%%%%%%%%%%%%%%%%%%%%%%%%%%%%%%% End of user-defined macros %%%





\begin{document}
\thispagestyle{empty}

% \begin{flushleft}
% \vspace*{-1.1cm} {\sc  Novi Sad J.\ Math.}\\ {\sc Vol.\ XX, No.\ Y, 20ZZ, ??-??}
% \end{flushleft}
\begin{flushright}
\vspace*{-1.1cm} {{\small \sffamily Last updated: 8 March 2017}}
\end{flushright}
\vspace{0.8cm}
% PRAVLJENJE NASLOVA
\begin{center}
{\large \bf ON THE COMPLEXITY OF DIFFERENCE\\ TERM EXISTENCE
%\footnote{This is one place where you can put acknowledgement}
} \vspace*{3mm}

% Title should be in upper case

{\bf William DeMeo\footnote{Department of Mathematics, University of Hawaii,\\
    e-mail: \href{mailto:williamdemeo@gmail.com}{williamdemeo@gmail.com}}
and Ralph Freese\footnote{Department of Mathematics, University of Hawaii,\\
    e-mail: \href{mailto:ralph@math.hawaii.edu}{ralph@math.hawaii.edu}}}
\end{center}
% Authors should be ordered alphabetically
% Please, use your full name and put an email address in affiliation. The example with underline in e-mail address:
%\href{mailto:my_address@wikibooks.org}{my\_address@wikibooks.org}

\begin{abstract}
We consider the following practical question: given a finite algebra $\bA$ in a
finite language, can we efficiently decide whether the variety generated by
$\bA$ has a difference term?  We define %In~\cite{DeMeo:2017}
``local difference terms'' and used to solve 
a related but easier problem---namely, we show that
there is a polynomial-time algorithm for deciding whether any finite idempotent
algebra has a difference term operation. 
Thereafter, we define ``global-local difference terms'' which we use to 
make some progress toward an efficient algorithm for deciding whether the
variety generated by a finite idempotent algebra has a difference term.
\\[2mm] {\it AMS Mathematics  Subject Classification $(2000)$}: 08B05, 08B10, 68Q25
% Please do not use not complete classification like 54Axx
\\[1mm] {\it Key words:} difference term, idempotent algebra, commutator

\end{abstract}

%%%%%%%%%%%%%%%%%%%%   Start of main body of article


\section{Introduction}
\label{sec:introduction}
Let $\sV$ be a variety (equational class) of algebras.
A ternary term $d$ in the language of $\sV$ is called 
a \defin{difference term for $\sV$} if it satisfies the following:
for all $\bA = \<A, \dots \> \in \sV$, for all $a, b \in A$, for every 
congruence $\theta\in \Con\bA$ containing $(a,b)$, we have
\begin{equation}
\label{eq:3}  
d^{\bA}(a,a,b) = b \quad \text{ and } \quad
d^{\bA}(a,b,b) \comr{\theta} a,
\end{equation}
where $[\cdot, \cdot]$ denotes the (term condition) commutator
defined in Section~\ref{sec:definitions} below
(see also~\cite{HM:1988} or~\cite{MR3076179}).
(By the monotonicity of the commutator, we could replace $\theta$ in the
definition by $\Cg^{\bA}(a,b)$.)
If for all $a, b \in A$ the relations in (\ref{eq:3}) hold 
with $\theta = \Cg^{\bA}(a,b)$, then we call
$d^{\bA}$ a \defin{difference term operation} for $\bA$.

Difference terms are studied extensively in the universal algebra literature.
(See, for example, \cite{HM:1988,KSW,MR3449235,MR1358491,MR3076179,MR1663558}.)
There are many reasons to study difference terms, but
perhaps the most obvious is that knowing a variety 
has a difference term allows us to deduce many useful
properties of the algebras in that variety.
(Very roughly speaking, having a difference term is slightly stronger than having
a Taylor term and slightly weaker than having a Mal'tsev term.
Note that if
$\bA$ is an \defin{abelian} algebra---that is, $[1_A, 1_A] = 0_A$---then by
the monotonicity of the commutator we have
$[\theta, \theta] = 0_A$ for all $\theta \in \Con \bA$, in which case
(\ref{eq:3}) says that $d^{\bA}$ is a Mal'tsev term operation.)

Digital computers have turned out to be invaluable tools for exploring and
understanding algebras and the varieties they inhabit, and this is largely due
to the fact that researchers have found ingenious ways
to get computers to solve abstract decision problems---such as
whether a variety is 
congruence-modular (\cite{Freese:2009}) or
congruence-$n$-permutable (\cite{MR3239624})---and to do so efficiently.
%% , in a way that scales well with problem size.

Consider the following problems:
\begin{prob}
  \label{prob:1}
  Is there a polynomial-time algorithm that takes a finite
  idempotent algebra $\bA$ as input and decides whether the variety generated by
  $\bA$ has a difference term?
\end{prob}
\begin{prob}
  \label{prob:2}
  Is there a polynomial-time algorithm that takes a finite
  idempotent algebra $\bA$ as input and decides whether 
  $\bA$ has a difference term operation?
\end{prob}
In the present paper we solve Problem~\ref{prob:2} and then discuss the
progress we have made toward a solution to Problem~\ref{prob:1}. 

The rest of the paper is organized as follows:
Section~\ref{sec:definitions} introduces notation and definitions and some of
the background that we expect the reader to have.
In~\cite{MR1358491} 
it was shown that a locally finite idempotent variety $\sV$ has a difference
term if and only if $\sansH \sansS \sansP(\bF_{\sV}(2))$ 
has a difference term (where $\bF_{\sV}(2)$ denotes the 2-generated free algebra in $\sV$).
In Section~\ref{sec:equiv-cond-exist}
we observe that this is also equivalent to
the free algebra $\bF_{\sV}(2)$ itself having a difference term operation.
In~\cite{MR3239624},
Valeriote and Willard define 
a ``local Hagemann-Mitschke sequence'' which they use as the basis of
an efficient algorithm for deciding for a given $n$ whether an idempotent
variety is $n$-permutable. 
In Section~\ref{sec:local-diff-terms}
we devise a similar construct, called
a ``local difference term,'' that we use, in Section~\ref{sec:algor-1},
to give a polynomial-time algorithm for deciding the existence of a difference term
operation for $\bA$.  In Section~\ref{sec:glob-local-diff} we extend local
difference terms from points to universes and then, in 
Section~\ref{sec:algor-2}, we describe some recent progress toward a polynomial-time
algorithm for deciding whether the variety generated by a finite idempotent algebra has a
difference term. We conclude with a brief description of our software
implementation of the main algorithm (TODO!!!), and mention some related 
problems that remain open (TODO!!!).


\section{Background, Notation, Definitions}
\label{sec:definitions}
%% The centralizer, term condition, and abelian congruences}
%% We review some useful properties of centralizers and abelian
Our arguments depend on some basic results of universal algebra that we now review.
For the most part we use standard notation such as those found in~\cite{MR2839398}.
\ifthenelse{\boolean{thetanotation}}{
  One exception is the old-fashioned notation we use for congruence generation:
  if $\bA =\<A, \dots\>$ is an algebra with elements $a, b \in A$,
  then we use $\Theta(a,b)$ to denote the congruence of $\bA$ generated by $a$
  and $b$. If $X\subseteq A$, then $\Theta(X)$ is the congruence generated by $X$.  
  \renewcommand{\Cg}{\ensuremath{\Theta}}
  Most other notation we use is standard.
}{}

Let $A$ and $B$ be sets and let $\alpha \subseteq A\times A$ and $\beta \subseteq B\times B$
be binary relations on $A$ and $B$, respectively.
We define
%% let $\alpha \mytimes \beta$ denote the
the \emph{pairwise product} of $\alpha$ and $\beta$ by
\begin{equation}
\label{eq:pair-product}
\alpha \mytimes \beta = \{((a, b), (a', b')) 
\in (A\times B)^2 \mid a\mathrel{\alpha} a'\, \text{ and } \,  b\mathrel{\beta} b'\},
\end{equation}
and we let $\alpha \times \beta$ denote the usual Cartesian product of sets; that is,
%% of the sets $\alpha$ and $\beta$, that is,
\begin{equation}
\label{eq:set-product}
\alpha \times \beta = \{((a, a'), (b, b')) 
\in A^2\times B^2 \mid a\mathrel{\alpha} a' \, \text{ and } \, b\mathrel{\beta} b'\}.
\end{equation}
Notice that $\alpha \mytimes \beta$ defines an equivalence relation on
$A\times B$, whereas in general $\alpha\times \beta$ is 
not even a binary relation on a single set.
The equivalence class of $\alpha \mytimes \beta$ containing the pair
$(a, b)$ is denoted and defined by % \in A\times B$ is
\[(a,b)/(\alpha \mytimes \beta) = a/\alpha \times b/\beta= 
    \{(a', b') \in A\times B \mid a\mathrel{\alpha} a' \, \text{ and } \,  b\mathrel{\beta} b'\},
    \]
the Cartesian product of the sets $a/\alpha$ and $b/\beta$.
The collection of all such equivalence classes is also a Cartesian product, namely,
\[
(A\times B)/(\alpha \mytimes \beta) =
A/\alpha \times B/\beta  = \{(a, b)/(\alpha \mytimes \beta) \mid a\in A \, \text{ and } \, b \in B\}.\]
%% \begin{remarks}\
%%   \begin{enumerate}
%%   \item The equivalence relations $\alpha$ and $\beta$ are subsets
%%     of $A^2$ and $B^2$, respectively, so it might seem more natural to interpret 
%%     $\alpha \times \beta$ as a subset of $A^2 \times B^2$. Nonetheless, in this 
%%     context it is usually more convenient to arrange the coordinates 
%%     so that $\alpha \times \beta \subseteq (A \times B)^2$, as in
%%     item~(\ref{item:9}).
%%   \end{enumerate}
%% \end{remarks}




Let $\bA = \<A, F^{\bA}\>$ be an algebra.
A reflexive, symmetric, compatible binary relation $T\subseteq A^2$ is called a
\defin{tolerance of $\bA$}.  
Given a pair $(\bu, \bv) \in A^m\times A^m$ of $m$-tuples of $A$, we sometimes
use the short-hand $\bu \mathrel{\bT} \bv$ to mean
``$\bu(i) \mathrel{T} \bv(i)$ for all $0\leq i<m$.'' 
%% A \defin{tolerance} of an algebra $\bA$ is a reflexive symmetric compatible relation.

The set of all tolerance relations of $\bA$ is denoted $\Tol(\bA)$, 
and the set of all congruence relations is denoted $\Con(\bA)$.
The subalgebra of $\bA$ generated by a set $X \subseteq A$ is denoted by
$\Sg^{\bA}(X)$, and
the congruence relation of $\bA$ generated by a set $X \subseteq A\times A$ is
denoted by $\Cg^{\bA}(X)$.
(We sometimes leave off the parentheses and write $\Sg^{\bA}X$ or $\Cg^{\bA}X$.
If $X$ is finite, say, $X = \{(a,b)\}$, we typically write 
$\Cg^{\bA}(a,b)$ instead of $\Cg^{\bA}\{(a,b)\}$.)
%% If $X$ is finite, say, $X = \{a,b\}$, then we
%% often write $\Sg^{\bA}\{a,b\}$, instead of $\Sg^{\bA}(\{a,b\})$.


We state a number of definitions in this section using tolerance relations, but 
the definitions don't change when the tolerance in question happens to be
a congruence relation (i.e., a transitive tolerance).

Suppose $S$ and $T$ are tolerances on $\bA$.  An \defin{$S,T$-matrix} 
is a $2\times 2$ array of the form
\[
\begin{bmatrix*}[r] t(\ba,\bu) & t(\ba,\bv)\\ t(\bb,\bu)&t(\bb,\bv)\end{bmatrix*},
\]
where $t$, $\ba$, $\bb$, $\bu$, $\bv$ have the following properties:
\begin{enumerate}[(i)] %[label=(\roman*)]
\item $t\in \sansClo_{\ell + m}(\bA)$,
\item $(\ba, \bb)\in A^\ell\times A^\ell$ and $\ba \mathrel{\bS} \bb$,
\item $(\bu, \bv)\in A^m\times A^m$ and $\bu \mathrel{\bT} \bv$.
\end{enumerate}
Let $\delta$ be a congruence relation of $\bA$.
If the entries of every $S,T$-matrix satisfy
\begin{equation}
  \label{eq:22}
t(\ba,\bu) \mathrel{\delta} t(\ba,\bv)\quad \iff \quad t(\bb,\bu) \mathrel{\delta} t(\bb,\bv),
\end{equation}
then we say that $S$ \defin{centralizes $T$ modulo} $\delta$ and we write 
$\CC{S}{T}{\delta}$.
That is, $\CC{S}{T}{\delta}$  means that 
(\ref{eq:22}) holds \emph{for all}
$\ell$, $m$, $t$, $\ba$, $\bb$, $\bu$, $\bv$ satisfying properties (i)--(iii).

The \defin{commutator} of $S$ and $T$, denoted by $[S, T]$,
is the least congruence $\delta$ such that $\CC{S}{T}{\delta}$ 
holds.  
Note that $\CC{S}{T}{0_A}$ is equivalent to $[S,T] = 0_A$, and this
is sometimes called the \defin{$S, T$-term condition};
when it holds we say  that
$S$ \defin{centralizes} $T$. %% , and write $\C{S}{T}$.
A tolerance $T$ is called \defin{abelian} if
%% $\C{T}{T}$ (i.e., $[T, T] = 0_A$).  
$[T, T] = 0_A$.  
An algebra $\bA$ is called \defin{abelian} if $1_A$ is abelian
(i.e., $[1_A,1_A] = 0_A$).

%% \begin{rem}
%%   An algebra $\bA$ is abelian iff %$\C{1_A}{1_A}$ iff
%%   \[
%%   \forall \ell, m \in \N,
%%   \quad \forall t\in \sansClo_{\ell + m}(\bA),
%%   \quad \forall (\ba, \bb)\in A^\ell\times A^\ell,
%%   \]
%%   \[
%%   \ker t(\ba, \cdot)=\ker t(\bb, \cdot).
%%   \]
%% \end{rem}

%% wjd: deleting the iterated commutator stuff (I don't think we need it)
%% It is sometimes useful to iterate the commutator, for example,
%% $[[\alpha, \alpha], [\alpha, \alpha]]$, and for this purpose
%% we define $[\alpha]^n$ recursively as
%% follows:
%% $[\alpha]^0 = \alpha$ and
%% $[\alpha]^{n+1} = [[\alpha]^n, [\alpha]^n]$.  A congruence $\alpha$ of $\bA$
%% is called \defin{solvable} if $[\alpha]^n = 0_A$ for some $n$.


Here are some properties of the centralizer relation
that are well-known and not too hard to prove
(see \cite[Prop~3.4]{HM:1988} or~\cite[Thm~2.19]{MR3076179}).
\begin{lem}
\label{lem:centralizers}
Let $\bA$ be an algebra and suppose
$\bB$ is a subalgebra of $\bA$. 
Let $\alpha$, $\beta$, $\gamma$, $\delta$, $\alpha_i$
$\beta_j$, $\gamma_k$
be congruences of $\bA$, for all 
$i \in I$, $j\in J$, $k \in K$. Then the following hold:
\begin{enumerate}
\item \label{centralizing_over_meet}
  $\CC{\alpha}{\beta}{\alpha \meet \beta}$;
\item \label{centralizing_over_meet2}
  if $\CC{\alpha}{ \beta}{ \gamma_k}$ for all $k \in K$, then
  $\CC{\alpha}{ \beta}{ \Meet_{K}\gamma_k}$;
\item \label{centralizing_over_join1}
  if $\CC{\alpha_i}{ \beta}{ \gamma}$ for all $i\in I$, then
  $\CC{\Join_{I}\alpha_i}{ \beta}{\gamma}$;
\item \label{monotone_centralizers1}
  if $\CC{\alpha}{ \beta}{ \gamma}$ and $\alpha' \leq \alpha$, then 
  $\CC{\alpha'}{ \beta}{ \gamma}$;
\item \label{monotone_centralizers2}
  if $\CC{\alpha}{ \beta}{ \gamma}$ and $\beta' \leq \beta$, then
  $\CC{\alpha}{ \beta'}{ \gamma}$;
\item \label{centralizing_over_subalg}
  if $\CC{\alpha}{ \beta}{ \gamma}$ in $\bA$, 
  then $\CC{\alpha\cap B^2}{ \beta\cap B^2}{\gamma\cap B^2}$ in $\bB$;
\item \label{centralizing_factors}
  if $\gamma \leq \delta$, then $\CC{\alpha}{ \beta}{ \delta}$
  in $\bA$ if and only if $\CC{\alpha/\gamma}{ \beta/\gamma}{ \delta/\gamma}$
  in $\bA/\gamma$.
\end{enumerate}
\end{lem}


\begin{rem}
By (\ref{centralizing_over_meet}), 
if $\alpha \meet \beta = 0_{A}$,  
then %$\C{\beta}{\alpha}$ and $\C{\alpha}{\beta}$.
$[\beta, \alpha] = 0_A = [\alpha, \beta]$.
\end{rem}

% \draftsecskip
%%%%%%%%%%%%%%%%%%%%%%%%%%%%%%%%%%%%%%%%%%%%%%%%%%%%%%%%%%%%%%%%%%%%%%%

%% \subsection{The commutator}
%% \label{sec:facts-about-comm}
Before proceeding, we collect some facts about the commutator that are
sometimes useful, especially when reasoning about difference terms.
The next two lemmas are easy consequences
of Lemma~\ref{lem:centralizers}, so 
we omit the proofs.
(See Section~\ref{sec-lem:monotone-comm}
of the extended version of this paper~\extendedref.)

\begin{lem}[Monotonicity of the Commutator]
  \label{lem:monotone-comm}
  Let $\bA$ be an algebra
  with congruences
  $\alpha$, $\alpha'$, $\beta$, $\beta'$ satisfying
  $\alpha\leq \alpha'$ and $\beta \leq \beta'$.
  Then $\comm \alpha \beta \leq \comm{\alpha'}{\beta'}$.
\end{lem}
%% \begin{proof}
%%   For every $\delta \in \Con\bA$, $\CC{\alpha'}{\beta'}{\delta}$ implies
%%   $\CC{\alpha}{\beta}{\delta}$, since $\alpha\leq \alpha'$ and $\beta \leq \beta'$.
%%   In particular, $\CC{\alpha'}{\beta'}{[\alpha', \beta']}$ implies
%%   $\CC{\alpha}{\beta}{[\alpha', \beta']}$, so
%%   $\comm{\alpha}{\beta} \leq \comm{\alpha'}{\beta'}$.
%% \end{proof}

\begin{lem}
  \label{lem:complete-meet-join-monotone}
Let $\bA$ be an algebra with congruences
$\alpha_i$ and 
$\beta_i$ %% $\gamma_k$
%% are congruences of $\bA$, 
for all $i \in I$.
Then
\[
\bigl[ \Meet \alpha_i, \Meet \beta_i \bigr] \leq
\Meet \comm {\alpha_i} {\beta_i}
\quad \text{ and } \quad
\Join \comm {\alpha_i} {\beta_i} \leq
\bigl[ \Join \alpha_i, \Join \beta_i\bigr].
\]
\end{lem}

%% \begin{proof}
%%   By Lemma~\ref{lem:monotone-comm}, $\bigl[\Meet \alpha_i, \Meet \beta_i\bigr] \leq
%%   \comm {\alpha_i} {\beta_i} \leq \bigl[\Join \alpha_i, \Join \beta_i\bigr]$,
%%   for all $i \in I$.
%% \end{proof}

The next result is useful when considering a commutator computed
with respect to a subalgebra $\bB \leq \bA$, rather than with respect to the
whole algebra $\bA$.
In the statement of the lemma, we use shorthand notation that will come in handy below.
The commutator %% , $[\theta, \theta]$,
of a congruence $\theta$ with itself, which appears so often in the sequel,
will be abbreviated as follows:\footnote{This is similar to the standard notational convention
  for the iterated commutator:
  \[
    [\theta]^0 =  \theta, \quad
    [\theta]^1 =  [\theta, \theta],  \quad
    [\theta]^2 =  \bigl[[\theta, \theta],[\theta, \theta]\bigr],  \; \dots, \;
    [\theta]^n =  \bigl[[\theta]^{n-1}, [\theta]^{n-1}\bigr], \; \dots.
    \]
}
\[
\Com{\theta}:= [\theta, \theta].
\]



\begin{lem}
\label{lem:subalgebra-comm}
If $\bB \leq \bA$ and $a, b \in B$, then 
$\Com{\Cg^{\bB} (a, b)} \subseteq \Com{\Cg^{\bA} (a, b)}$. %%  \label{eq:6-app}
\end{lem}
\begin{proof}
  Let $\alpha = \Cg^{\bA} (a, b)$, $\beta = \Cg^{\bB} (a, b)$, and
  $\delta=\Com{\Cg^{\bA} (a, b)} \cap B^2$. 
  To prove the lemma it suffices to show that
  $\CC{\beta}{\beta}{\delta}$ holds, since this will give us
  the required $\leq$ relation in the following:
  \[\Com{\Cg^{\bB}(a,b)} = \com{\beta} \leq \delta \subseteq \Com{\Cg^{\bA}(a,b)}.\]
  Let $\br$, $\bs \in B^{k}$, 
  $\bu$, $\bv \in B^{\ell}$, and $t\in \Clo_{k+\ell}(\bB)$.
  Assume $r_i \rbeta s_i$ and $u_i \rbeta v_i$ and $t(\br, \bu) \rdelta t(\br, \bv)$.
  We must prove $t(\bs, \bu) \rdelta t(\bs, \bv)$.
  Clearly, $\beta \subseteq \alpha$, so $r_i \rbeta s_i$ and $u_i \rbeta v_i$ imply
  $r_i \ralpha s_i$ and $u_i \ralpha v_i$.  Therefore, 
  $(t(\br, \bu), t(\br, \bv)) \in \delta  %% =\com{\alpha} \cap B^2 
  \subseteq \com{\alpha}$ implies
  $(t(\bs, \bu), t(\bs, \bv)) \in \com{\alpha}$.
  Of course, $(t(\bs, \bu), t(\bs, \bv)) \in B^2$, since $\bB$ is a subalgebra. 
  Therefore, $(t(\bs, \bu), t(\bs, \bv)) \in \com{\alpha}\cap B^2 = \delta$, as desired.
\end{proof}



%% We will apply the preceding result in a simple special case involving
%% just four congruences; we record this version of the result for convenience.
%% %% (delete the corollary later)
%% \begin{cor}
%%   \label{cor:facts-about-comm-1}
%% Let $\bA$ be an algebra with congruences
%% $\alpha$, $\beta$, $\gamma$, $\delta$.  Then,
%% \[
%% \comm {\alpha \meet \gamma} {\beta \meet \delta} {\leq}
%% \comm \alpha \beta \meet \comm \gamma \delta
%% \quad \text{ and } \quad
%% \comm \alpha \beta \join \comm \gamma \delta {\leq}
%% \comm {\alpha \join \gamma} {\beta \join \delta}.
%% \]
%% \end{cor}
Before stating the next result, we remind the reader of a 
standard notational convention. 
If $\phi \in \Hom(\bA, \bB)$ and $\theta \in \Con (\bA)$, then by
$\phi(\theta)$ we mean the set $\{(\phi(x), \phi(y)) \mid x \thetar y\}$.
\begin{lem}[\protect{\cite[Theorem 2.10]{MR1358491}}]
\label{lem:hom-image-diff-term}
  Let $\bA$, $\bB$ be algebras of the same similarity type and suppose
  $\phi: \bA \to \bB$ is a surjective homomorphism.  If
  $\alpha, \beta \in \Con \bA$, then
  $\phi([\alpha, \beta]) \subseteq [\phi(\alpha), \phi(\beta)]$.
  Moreover, if there exists a homomorphism $\psi: \bB \to \bA$ such that
  $\phi \circ \psi = \id_B$ and $\rho, \sigma \in \Con \bB$, then
 \[\psi^{-1} \{[\psi(\rho), \psi(\sigma)]\} = \phi\bigl( [\psi(\rho), \psi(\sigma)]\bigr)
  = [\rho, \sigma].\]
\end{lem}
In fact, this result holds even if $\phi$ is not surjective. (See the remark
after the proof of~\cite[Theorem 2.10]{MR1358491}.)


\begin{lem}
  \label{lem:hom-cong}
  Let $\bA$, $\bB$ be algebras of the same similarity type and suppose
  $\phi: \bA \to \bB$ is a surjective homomorphism.  If $a, a' \in A$, then 
  $\phi \bigl(\Cg^{\bA}(a,a')\bigr) =\Cg^{\bB}\bigl(\phi (a),\phi(a')\bigr)$.
\end{lem}
\begin{proof}

For the opposite inclusion, note that $\phi$ 
takes congruences to congruences, so $\phi \bigl(\Cg^{\bA}(a,a')\bigr)$ is a congruence
of $\bB$ that clearly contains $(\phi (a),\phi(a'))$. Therefore,
$\Cg^{\bB}\bigl(\phi (a),\phi(a')\bigr) \subseteq \phi \bigl(\Cg^{\bA}(a,a')\bigr)$. 
\end{proof}


%%%%%%%%%%%%%%%%%%%%%%%%%%%%%%%%%%%%%%%%%%%%%%%%%%%%%%%%%%%%%%%%%%%%%%%


\section{Conditions for Existence of a Difference Term}
\label{sec:equiv-cond-exist}
The main result of this section is Theorem~\ref{thm:F}, which 
is essentially due to Keith Kearnes and is based on an observation
in~\cite{MR1358491}
asserting that a variety $\sV$ has a difference term if and only if
$\sansH \sansS \sansP(\bF_{\sV}(2))$ 
has a difference term.
%% (where $\bF_{\sV}(2)$ denotes the 2-generated free algebra in $\sV$).
The forward implication of this claim is trivial;
the argument for the converse goes as follows:
assume that $d(x, y, z)$ is a difference term for $\sansH \sansS \sansP(\bF)$.
Choose $\bA \in \sV$ and $a, b \in A$. Let $\bB = \Sg^{\bA} \{a, b\}$.
Since $\bB$ is 2-generated, $B \in \sansH \sansS \sansP (\bF)$.
Hence $d(x, y, z)$ interprets as a difference term in $\bB$. This means that
$d^{\bA} (a, a, b) = d^{\bB} (a, a, b) = b$.
Furthermore,
$d^{\bA} (a, b, b) = d^{\bB} (a, b, b)
%% \mathrel{[\Cg^{\bB} (a, b), \Cg^{\bB} (a, b)]}
\Comr{\Cg^{\bB} (a, b)} a$.
However, 
%% $[\Cg^{\bB} (a, b), \Cg^{\bB} (a,b)]\subseteq [\theta, \theta]$
$\Com{\Cg^{\bB} (a, b)} \subseteq \Com{\theta}$
for every congruence
$\theta \in \Con \bA$ containing $(a, b)$. Consequently
$d^{\bA} (a, b, b)
%% \mathrel{[\theta, \theta]}
\Comr{\theta}
a$ as desired.

Considering the goal of our project, it is natural
to ask whether the existence of a difference term for
 $\sV$ is equivalent to the  existence  of a difference term
operation for a specific algebra in $\sV$.  This is achieved in
Theorem~\ref{thm:F}, which will play a key role
in our main complexity argument in Section~\ref{sec:glob-local-diff}.
First, the lemma that does the heavy lifting in the proof of 
Theorem~\ref{thm:F} is the following:
\begin{lem}
  \label{lem:equiv-cond-exist-1}
  Let $\bA$ be an algebra, let $t(x,y,z)$ be a ternary term in the language
  of $\bA$, and let $\bF = \bF_{\bbV(\bA)}(x,y)$. Consider the following statements:
  \begin{enumerate}[(A)]
  \item \label{item:6} $t^{\bA}$ is not a difference term operation for $\bA$.
  \item \label{item:7} There exists a 2-generated subalgebra $\bB \leq \bA$
    such that $t^{\bB}$ is not a difference term operation for $\bB$.
  \item \label{item:8} $t^{\bF}$ is not a difference term operation for $\bF$.
  \end{enumerate}
  Then (\ref{item:6}) implies (\ref{item:7}) and (\ref{item:7}) implies (\ref{item:8}).
\end{lem}
\begin{proof}
  (\ref{item:6}) $\Rightarrow $ (\ref{item:7}):
  Suppose  $t^{\bA}$ fails to be a difference term operation for $\bA$ and let $a, b \in
  A$ witness this failure. That is, either
  \begin{enumerate}
  \item\label{item:9} $d^{\bA}(a,a,b) \neq b$, or
  \item\label{item:10} $(d^{\bA}(a,b,b), a) \notin \Com{\Cg^{\bA} (a, b)}$.
  \end{enumerate}
  Let $\bB = \Sg^{\bA} \{a, b\}$.  In case
  (\ref{item:9}), 
  $d^{\bB}(a,a,b) = d^{\bA}(a,a,b) \neq b$, so $d^{\bB}(x,y,z)$ is not a difference
  term operation for $\bB$.
  In case (\ref{item:10}), observe that
  $(d^{\bB}(a,b,b), a) = (d^{\bA}(a,b,b), a)\notin \Com{\Cg^{\bA} (a, b)}$.
  By Lemma~\ref{lem:subalgebra-comm}, $\Com{\Cg^{\bB} (a, b)} \subseteq \Com{\Cg^{\bA} (a, b)}$,
  from which it follow that $(d^{\bB}(a,b,b), a) \notin \Com{\Cg^{\bB} (a, b)}$.
  Therefore, $d^{\bB}(x,y,z)$ is not a difference term operation for $\bB$.
  \\[4pt]
  (\ref{item:7}) $\Rightarrow$ (\ref{item:8}):
  Since there is a surjective homomorphism from $\bF$ to $\bB$,
  Lemma~\ref{lem:hom-image-diff-term} implies that $d^{\bF}(x,y,z)$ 
  is not a difference term operation for $\bF$.
\end{proof}

Obviously, we could have stated~\ref{lem:equiv-cond-exist-1} in the following positive form:
\begin{cor}
  Let $\bA$ be an algebra, let $t(x,y,z)$ be a ternary term in the language
  of $\bA$, and let $\bF = \bF_{\bbV(\bA)}(x,y)$. Consider the following statements:
  \begin{enumerate}[(A')]
  \item \label{item:6'} $t^{\bA}$ is a difference term operation for $\bA$.
  \item \label{item:7'} for all 2-generated $\bB \leq \bA$,
    $t^{\bB}$ is a difference term operation for $\bB$;
  \item \label{item:8'} $t^{\bF}$ is a difference term operation for $\bF$;
  \end{enumerate}
  Then (\ref{item:8'}') $\Rightarrow$ (\ref{item:7'}') $\Rightarrow$ (\ref{item:6'}').
\end{cor}


%% Cliff's comment:
%% F is a subalgebra of A^n for some n. We can assume that V is neither congruence
%% modular nor meet-semidistributive. Can we use that fact to put a  bound on n?
%% (On the face of it, all we know is that n < a^2 where a is the cardinality of A.) 

\begin{thm}
  \label{thm:F}
Let $\sV$ be a variety and $\bF = \bF_{\sV}(2)$, the 2-generated
free algebra in $\sV$. The following are equivalent:
\begin{enumerate}[(i)]
\item \label{item:1010}
  $\sV$ has a difference term;
\item \label{item:2}
  $\sansH \sansS \sansP (\bF)$ has a difference term;
\item \label{item:3}
  $\bF$ has a difference term operation.
\end{enumerate}
\end{thm}
\begin{proof}
  The implications
  (\ref{item:1010}) $\Rightarrow$  (\ref{item:2}) $\Rightarrow$  (\ref{item:3}) are
  obvious. We prove
  (\ref{item:3}) $\Rightarrow$  (\ref{item:1010}) by contraposition.
  Suppose $\sV$ has no difference term and let
  %% (We show $\bF$ has no difference term operation.)
  $d(x,y,z)$ be an arbitrary ternary term in the language of $\sV$.
  Let $\bA\in \sV$ be such that $d^{\bA}(x,y,z)$ is not a difference term
  operation for $\bA$. Then by Lemma~\ref{lem:equiv-cond-exist-1}, $d^{\bF}(x,y,z)$ 
  is not a difference term operation for $\bF$.
  %% Since $d(x,y,z)$ is arbitrary, it follows that
  %% $\bF$ has no difference term operation whatsoever.
  %% , as we set out to prove.
\end{proof}

\section{Local Difference Terms}
\label{sec:local-diff-terms}
In~\cite{MR3239624},
Valeriote and Willard define %% an \defin{$\bA$-triple for $\bp$}
%% to be a triple $(a,b,i)$ such that $a, b \in A$ and
%% $p_i(a,b,b) = p_{i+1}(a,a,b)$. They use this to define 
a ``local Hagemann-Mitschke sequence'' which they use as the basis of
an efficient algorithm for deciding for a given $n$ whether an idempotent
variety is $n$-permutable. 
Inspired by that work, we devise a similar construct, called
a ``local difference term,'' that we use to develop a polynomial-time
algorithm for deciding the existence of a difference term operation.

Let $\bA=\< A, \dots\>$ be an algebra, fix $a, b \in A$ and
$i \in \{0,1\}$.
%% An \defin{$\bA$-local difference term for
A \defin{local difference term for
  $(a,b,i)$} is a ternary term $d$ satisfying the following:
\begin{align}
%% \text{ if $i=0$, then } & a \comm{\Cg^{\bA}(a,b)}{\Cg^{\bA}(a,b)} d(a,b,b); \label{eq:diff-triple}\\
\text{ if $i=0$, then } & a \Comr{\Cg(a,b)} d(a,b,b); \label{eq:diff-triple}\\
\text{ if $i=1$, then } &d(a,a,b) = b. \nonumber
\end{align}
If $d$ satisfies~(\ref{eq:diff-triple}) for all triples
in some subset $S\subseteq A \times A \times \{0,1\}$, then we call $d$
a \defin{local difference term for $S$}.

Let 
$\sS = A \times A \times \{0,1\}$ and
suppose that every pair
$((a_0, b_0, \chi_0), (a_1, b_1, \chi_1))$
in $\sS^2$ has a local difference term.
That is, for each pair $((a_0, b_0, \chi_0), (a_1, b_1, \chi_1))$, there exists
$d$ such that for each $i \in \{0,1\}$ we have
\begin{align}
  a_i \Comr{\Cg(a_i,b_i)} d(a_i,b_i,b_i), & \;
  \text{ if $\chi_i=0$, and }  \label{eq:d-trip-i1}\\
  d(a_i,a_i,b_i) =b_i, & \;
  \text{ if $\chi_i=1$.}\label{eq:d-trip-i2} %\\\nonumber
\end{align}
Under these hypothesis we will prove that every subset $S\subseteq \sS$
has a local difference term.
That is, there is a single term $d$ that works (i.e., satisfies
(\ref{eq:d-trip-i1}) and (\ref{eq:d-trip-i2})) for all $(a_i, b_i, \chi_i) \in S$.
The statement and proof of this new result follows.

%% \subsection{Main Results}
\begin{thm} %[\protect{cf.~\cite[Theorem 2.2]{MR3239624}}]
  \label{thm:local-diff-terms}
  Let $\sV$ be an idempotent variety and
  $\bA \in \sV$. Define
  $\sS= A \times A \times \{0,1\}$
  and suppose that every pair
  $((a_0, b_0, \chi_0), (a_1, b_1, \chi_1)) \in \sS^2$
  has a local difference term.
  Then every subset $S \subseteq \sS$
  has a local difference term.
\end{thm}
\begin{proof}
The proof is by induction on the size of $S$.  In the base case, $|S| = 2$,
the claim holds by assumption.
Fix $n\geq 2$ and assume that every subset of $\sS$ of size $2\leq k \leq n$ has a local
difference term. Let
\[
S = \{(a_0, b_0, \chi_0), (a_1, b_1, \chi_1), \dots, (a_{n}, b_{n},\chi_{n})\} \subseteq \sS,\]
so that $|S| = n+1$.  We prove $S$ has a local difference term.

Since $|S| \geq 3$ and $\chi_i \in \{0,1\}$ for all $i$, there must exist
indices $i\neq j$ such that $\chi_i = \chi_j$. Assume without loss of generality
that one of these indices is $j=0$.  Define
the set
$S' = S \setminus \{(a_0, b_0, \chi_0)\}$.
Since $|S'| < |S|$, the set $S'$ has a local difference term $p$.
We split the remainder of the proof into two cases.
%% In the first case $\chi_0 = 0$ and in the second $\chi_0 = 1$.

\medskip

%--------------------------------------
\noindent \underline{Case $\chi_0 = 0$}:
Without loss of generality, suppose that $\chi_1 = %% \chi_2 =
\cdots =\chi_k = 1$,
and $\chi_{k+1} %% = \chi_{k+2} 
= \cdots = \chi_{n} = 0$. Define %% $T$ to be the set
\[T = \{(a_0, p(a_0, b_0, b_0), 0),
(a_1, b_1, 1), (a_2, b_2, 1), 
\dots, (a_k, b_k, 1)\},\] and 
note that $|T| < |S|$.
Let $t$ be a local difference term for $T$.
Define
\[
d(x,y,z) = t(x, p(x,y,y), p(x,y,z)).
\]
We show that $d$ is a local difference term for $S$.
Since $\chi_0 =0$, we first verify that
$(a_0, d(a_0,b_0,b_0))$ belongs to $\Com{\Cg(a_0,b_0)}$.
Indeed,
\begin{equation}
    \label{eq:100000}
  d(a_0,b_0,b_0) =
  %% t(a_0, p(a_0,b_0,b_0), p(a_0,b_0,b_0))\comr{\tau} a_0,
  t(a_0, p(a_0,b_0,b_0), p(a_0,b_0,b_0))\Comr{\Cg(a_0, p(a_0,b_0,b_0))} a_0.
\end{equation}
%% where we have used $\tau$ to denote $\Cg(a_0, p(a_0,b_0,b_0))$.
Note that the pair $(a_0, p(a_0,b_0,b_0))$ is equal to
$(p(a_0,a_0,a_0), p(a_0,b_0,b_0))$ (by idempotence) and 
belongs to $\Cg(a_0, b_0)$, so $\Cg(a_0, p(a_0,b_0,b_0))\leq \Cg(a_0,b_0)$.
Therefore,
%% so $\tau\leq \Cg(a_0,b_0)$. Therefore,
by monotonicity of the commutator we have
$\Com{\Cg(a_0, p(a_0,b_0,b_0))} \leq \Com{\Cg(a_0,b_0)}$.
It follows from this and (\ref{eq:100000}) that
%% $d(a_0,b_0,b_0)\comm{\Cg(a_0,b_0)}{\Cg(a_0,b_0)} a_0$,
$d(a_0,b_0,b_0)\Comr{\Cg(a_0,b_0)} a_0$, as desired.

For the indices $1\leq i \leq k$ we have $\chi_i =1$, so we prove
$d(a_i,a_i,b_i) = b_i$ for such indices. Observe,
\[
  d(a_i,a_i,b_i) =
  t(a_i, p(a_i,a_i,a_i), p(a_i,a_i,b_i)) % \label{eq:200000}\\
  =t(a_i, a_i, b_i) % \label{eq:200001}\\
  =b_i. % \label{eq:200002}
\]
The first equation holds by definition of $d$, the second
because $p$ is an idempotent local difference term for
$S'$, and the third because $t$ is a local difference term for $T$.

The remaining triples in our original set $S$
have indices satisfying $k<j\leq n$ and $\chi_j = 0$.
Thus, for these triples we want
$d(a_j,b_j,b_j)\Comr{\Cg(a_j,b_j)} a_j$.
By definition,
\begin{equation}
  \label{eq:450000}
d(a_j,b_j,b_j) =t(a_j, p(a_j,b_j,b_j), p(a_j,b_j,b_j)).  
\end{equation}
Since $p$ is a local difference term for $S'$, %we have
the pair $(p(a_j,b_j,b_j), a_j)$ belongs to $[\Cg(a_j,b_j), \Cg(a_j,b_j)]$.
%% $(p(a_j,b_j,b_j), a_j)\in [\Cg(a_j,b_j), \Cg(a_j,b_j)]$.
This and 
(\ref{eq:450000}) imply
that 
$(d(a_j, b_j,b_j), t(a_j,a_j,a_j))$
belongs to
$\Com{\Cg(a_j,b_j)}$.
Finally, by idempotence of $t$ we have
$d(a_j,b_j,b_j)\Comr{\Cg(a_j,b_j)} a_j$,
as desired.
\\[4pt]
%--------------------------------------
\underline{Case $\chi_0 = 1$}:
%% \\[4pt]
%% Assume $\chi_0 = 1$ and, 
Without loss of generality, suppose $\chi_1 = \chi_2 =\cdots =\chi_k = 0$,
and $\chi_{k+1} = \chi_{k+2} = \cdots = \chi_{n} = 1$. Define 
\[
T = \{(p(a_0, a_0, b_0), b_0, 1),
(a_1, b_1, 0), (a_2, b_2 0), \dots, (a_k, b_k, 0)\},
\]
and note that $|T| < |S|$.
Let $t$ be a local difference term for $T$ and
define
%% \[d(x,y,z) = t(p(x,y,z), p(y,y,z), z).\] 
$d(x,y,z) = t(p(x,y,z), p(y,y,z), z)$. 
Since $\chi_0 =1$, we want $d(a_0,a_0,b_0) = b_0$. By the definition of
$d$,
% \begin{equation*}
%   d(a_0,a_0,b_0) =
%   t(p(a_0,a_0,b_0), p(a_0,a_0,b_0), b_0) =b_0.
% \end{equation*}
$d(a_0,a_0,b_0) =
t(p(a_0,a_0,b_0), p(a_0,a_0,b_0), b_0) =b_0$.
The last equality holds since $t$ is a local difference term for $T$, thus,
for $(p(a_0, a_0, b_0), b_0, 1)$.

If $1\leq i \leq k$, then $\chi_i =0$, so for these indices we prove
that $(a_i, d(a_i,b_i,b_i))$ belongs to $\Com{\Cg(a_i,b_i)}$.
Again, starting from the definition of $d$ and using idempotence of $p$, we have
\begin{equation}
  \label{eq:40000}
  d(a_i,b_i,b_i) =
  t(p(a_i,b_i,b_i), p(b_i,b_i,b_i), b_i)=
  t(p(a_i,b_i,b_i), b_i, b_i).
\end{equation}
% \begin{align}
%   d(a_i,b_i,b_i) &=
%   t(p(a_i,b_i,b_i), p(b_i,b_i,b_i), b_i)   \label{eq:40000}\\
%   &=t(p(a_i,b_i,b_i), b_i, b_i). \nonumber
% \end{align}
Next, since $p$ is a local difference term for $S'$, we have
\begin{equation}
  \label{eq:50000}
  t(p(a_i,b_i,b_i), b_i, b_i)
 \Comr{\Cg(a_i,b_i)}
 t(a_i, b_i, b_i).
\end{equation}
Since $t$ is a local difference term for $T$, hence for
$(a_i, b_i, b_i)$,  %% $(1\leq i \leq k)$,
we see that 
$t(a_i, b_i, b_i) \Comr{\Cg(a_i,b_i)} a_i$.
This plus (\ref{eq:40000}) and (\ref{eq:50000}) yields
$d(a_i,b_i,b_i) \Comr{\Cg(a_i,b_i)} a_i$,
as desired.

The remaining elements of our original set $S$
have indices $j$ satisfying $k<j\leq n$ and $\chi_j = 1$.
For these we want $d(a_j,a_j,b_j) = b_j$.
Since $p$ is a local difference term for $S'$, we have
$p(a_j,a_j,b_j) = b_j$, and this along with idempotence of $t$ yields
\[ d(a_j,a_j,b_j) =  t(p(a_j,a_j,b_j), p(a_j,a_j,b_j), b_j)=  t(b_j, b_j, b_j) =b_j,\]
% \begin{align*}
% d(a_j,a_j,b_j) &=
% t(p(a_j,a_j,b_j), p(a_j,a_j,b_j), b_j)\\
% &=t(b_j, b_j, b_j) =b_j,
% \end{align*}
as desired.
\end{proof}

\begin{cor}
  \label{cor:loc-diff-term}
  A finite idempotent algebra $\bA$ has a difference term operation if and
  only if each pair $((a,b,i), (a',b',i')) \in (A\times A \times \{0,1\})^2$ has a local
  difference term.
\end{cor}
\begin{proof}
  One direction is clear, since a difference term operation for $\bA$ is
  obviously a local difference term for the whole set 
  $A\times A \times \{0,1\}$.
  For the converse, suppose
  each pair in $(A\times A \times \{0,1\})^2$ has a local
  difference term. Then, by Theorem~\ref{thm:local-diff-terms},
  there is a single local difference term for the whole set $A\times A \times \{0,1\}$,
  and this is a difference term operation for $\bA$.  Indeed, if $d$ is a
  local difference term for $A\times A \times \{0,1\}$, then 
  for all $a, b \in A$, we have
  $a \Comr{\Cg(a,b)} d(a,b,b)$,
  since $d$ is a local difference term for $(a,b,0)$, and we have
  $d(a,a,b) = b$, since $d$ is also a local difference term for
  $(a,b,1)$.
\end{proof}

% \draftsecskip

\subsection{Algorithm 1: existence of a difference term operation}
\label{sec:algor-1}
%% In this subsection we prove the following:
%% \subsection*{Algorithm 1: existence of a difference term operation}
\begin{cor}
  \label{cor:algor-1}
  There is a polynomial-time algorithm that takes as input
  any finite idempotent algebra $\bA$ and decides whether
  %% the variety $\bbV(\bA)$ that it generates
  $\bA$ has a difference term operation.
\end{cor}
\begin{proof}
  %% and let  $\sV = \bbV(\bA)$.
  We describe an efficient algorithm for deciding,
  given a finite idempotent algebra $\bA$,
  whether every pair $((a,b,i), (a',b',i')) \in (A\times A \times \{0,1\})^2$ has a local
  difference term.  By Corollary~\ref{cor:loc-diff-term}, this will prove we
  can decide in polynomial-time whether $\bA$ has a difference term operation.
  %% We will then complete the
  %% proof by explaining why $\bA$ has a difference term operation iff the variety
  %% it generates has a difference term. 

  Fix a pair
  $((a,b,i), (a',b',i'))$ in $(A\times A \times \{0,1\})^2$. If $i = i' = 0$,
  then the first projection is a local difference term. If $i = i' = 1$,  
    then the third projection is a local difference term. The two remaining cases to
    consider are (1) $i = 0$ and $i'=1$, and (2)
    $i = 1$ and $i'=0$. Since these are completely symmetric, we only handle the
    first case. Assume  the given pair of triples is
    $((a,b,0), (a',b',1))$.  By definition, a term $t$ is local difference term
    for this pair iff
    \[
    a\Comr{\Cg(a,b)} t^{\bA}(a,b,b) \; \text{ and } \;
    t^{\bA}(a',a',b') = b'.
    \]
    We can rewrite this condition more compactly by
    considering 
    \[t^{\bA\times \bA}((a,a'), (b,a'), (b,b')) =
    (t^{\bA}(a,b,b),t^{\bA}(a',a',b')).\]
    Clearly $t$ is a local difference term for
    $((a,b,0), (a',b',1))$ iff
    \[
    t^{\bA\times \bA}((a,a'), (b,a'), (b,b'))\in a/\delta \times \{b'\},
    \]
    where $\delta = \Com{\Cg(a,b)}$ and $a/\delta$ denotes the
    $\delta$-class containing $a$.
    (Observe that $a/\delta \times \{b'\}$ is a subalgebra of $\bA \times \bA$
    by idempotence.)
    It follows that the pair
    $((a,b,0), (a',b',1))$ has a local difference term iff
    the subuniverse of $\bA\times \bA$ generated by
    $\{(a,a'), (b,a'), (b,b')\}$ intersects nontrivially with the subuniverse
    $a/\delta \times \{b'\}$.

    Thus, the algorithm takes as input $\bA$ and, for each 
    $((a,a'), (b,a'), (b,b'))$ in $(A\times A)^3$, computes
    $\delta = \Com{\Cg(a,b)}$, computes the subalgebra
    $\bS$ of $\bA\times \bA$ generated by
    %% \Sg^{\bA\times \bA}\{(a,a'), (b,a'), (b,b')\}$, and then
    $\{(a,a'), (b,a'), (b,b')\}$, and then
    tests whether $S \cap (a/\delta \times \{b'\})$ is empty.
    If we find an empty intersection at any point, then
    $\bA$ has no difference term operation.
    Otherwise the algorithm halts without witnessing an empty
    intersection, in which case $\bA$ has a difference term operation.

    Most of the operations carried out by this algorithm are well known to be
    polynomial-time.  For example, that the running time of subalgebra generation is
    polynomial has been known for a long time (see~\cite{MR0455543}).
    The time complexity of congruence generation is also known to be polynomial
    (see~\cite{MR2470585}).  The only operation whose tractability might be 
    called into question is the commutator, but we have a straight-forward 
    algorithm for computing it that we describe in detail in 
    Appendix Section~\ref{sec:interlude:-an-easy}.
    % which, after the congruences have been computed, simply
    % involves generating more subalgebras.
    %% Finally, we observe that if $\bA$ has a difference term operation, then the
    %% variety it generates has a difference term.
\end{proof}


More details on the complexity of operations carried out by the algorithm, as well as many other algebraic operations, can be found in the references mentioned, as well as~\cite{MR1871085,MR1695293,Freese:2009}. 


\section{Generalizations and Extensions}

%% %%% NEW SECTION
\subsection{Mixed local difference terms}
\label{sec:mixed-local-diff}
Let $\sV$ be a variety and let $\bA_0=\< A_0, \dots\>$ and  $\bA_1=\< A_1, \dots\>$ be
algebras in $\sV$.  Fix $a_0, b_0 \in A_0$,  $a_1, b_1 \in A_i$,  and
$\chi_i \in \{0,1\}$.
%% An \defn{$\bA$-local difference term for
Define a \defn{mixed local difference term for
$((a_0, b_0,\chi_0), (a_1, b_1,\chi_1))$}
is a ternary term $d$ satisfying, for each $i \in \{0,1\}$,
\begin{align}
%% \text{ if $i=0$, then } & a \comm{\Cg^{\bA}(a,b)}{\Cg^{\bA}(a,b)} d(a,b,b); \label{eq:diff-triple}\\
\text{ if $\chi_i=0$, then } & a_i \comm{\Cg^{\bA_i}(a_i,b_i)}{\Cg^{\bA_i}(a_i,b_i)} d^{\bA_i}(a_i,b_i); \label{eq:mixed-diff-triple}\\
\text{ if $\chi_i=1$, then } &d^{\bA_i}(a_i,a_i,b_i) = b_i. \nonumber
\end{align}
If $d$ satisfies~(\ref{eq:mixed-diff-triple}) for all triples
in some subset $S$ of the disjoint union
\[
\sU(A_0, A_1)  := (A_0 \times A_0 \times \{0,1\}) + (A_1\times A_1 \times \{0,1\}),
\]
%% $(A_0^2 \times \{0,1\}) + (A_1^2 \times \{0,1\})$,
then we call $d$ a \defn{mixed local difference term for $S$}.
Elements of the disjoint union $\sU(A_0, A_1)$ will be denoted by
$\<(a, b, \chi), 0\>$ if $a, b \in A_0$ and
$\<(a, b, \chi), 1\>$ if $a, b \in A_1$.
We will use $\sU$ in place of $\sU(A_0, A_1)$ when the context renders the
universes involved either obvious or irrelevant.

Suppose that all pairs of triples 
in $\sU$ have mixed local difference terms.
%% That is, for each pair
%% $(\<(a, b, \chi), i\>, \<(a', b', \chi'), i'\>)$
%% there exists a term $d$ such that % for each $i \in \{0,1\}$ we have
%% \begin{align}
%%   a \comm{\Cg^{\bA_i}(a,b)}{\Cg^{\bA_i}(a,b)} d^{\bA_i}(a,b,b), & \;
%%   \text{ if $\chi_i=0$, and }  \label{eq:d-trip-i1}\\
%%   d^{\bA_i}(a,a,b) =b, & \;
%%   \text{ if $\chi_i=1$,}\label{eq:d-trip-i2} %\\\nonumber
%% \end{align}
%% and such that similar relations hold for $\<(a', b', \chi'), i'\>$.
Under this hypothesis we prove that every subset $S\subseteq \sU$
has a mixed local difference term.
That is, there is a single term $d$ that works (i.e., satisfies
%% (\ref{eq:mixed-trip-i1}) and (\ref{eq:mixed-trip-i2})) for all $\<(a, b, \chi), i\> \in S$.
the relations (\ref{eq:mixed-diff-triple})) for all $\<(a, b, \chi), i\> \in S$.
The statement and proof of this result follows.

\begin{thm} %[\protect{cf.~\cite[Theorem 2.2]{MR3239624}}]
  \label{thm:mixed-local-diff-terms}
  Let $\sV$ be an idempotent variety and let
  $\bA_0 = \<A_0, \dots\>$ and   $\bA_1 = \<A_1, \dots\>$ be algebras in $\sV$. Define
  $\sU  = (A_0^2 \times \{0,1\}) + (A_1^2\times \{0,1\})$
  and suppose that every pair
  $(\<(a, b, \chi), i\>, \<(a', b', \chi'), i'\>) \in \sU^2$
  has a mixed local difference term. Then every subset $S \subseteq \sU$,
  has a mixed local difference term.
\end{thm}

\begin{proof}
The proof is by induction on the size of $S$.  In the base case, $|S| = 2$,
the claim holds by assumption.
Fix $n\geq 2$ and assume that every subset of $\sU$ of size $2\leq k \leq n$ has
a mixed local
difference term. Let
\[
S =
\{\<(a_0, b_0, \chi_0), \iota_0\>, \<(a_1, b_1, \chi_1), \iota_1\>,
\dots, \<(a_{n}, b_{n},\chi_{n}), \iota_n\>\} \subseteq \sU,\]
so that $|S| = n+1$.  We prove $S$ has a mixed local difference term.

Since $|S| \geq 3$ and $\chi_i \in \{0,1\}$ for all $i$, there must exist
indices $i\neq j$ such that $\chi_i = \chi_j$. Assume without loss of generality
that one of these indices is $j=0$.  Define
the set
$S' = S \setminus \{\<(a_0, b_0, \chi_0), \iota_0\>\}$.
Since $|S'| < |S|$, the set $S'$ has a mixed local difference term $p$.
We split the remainder of the proof into two cases.
%% In the first case $\chi_0 = 0$ and in the second $\chi_0 = 1$.

\vskip3mm

%--------------------------------------
\noindent \underline{Case $\chi_0 = 0$}:
Without loss of generality, suppose that $\chi_1 = %% \chi_2 =
\cdots =\chi_k = 1$,
and $\chi_{k+1} %% = \chi_{k+2} 
= \cdots = \chi_{n} = 0$. Define %% $T$ to be the set
\[T = \{\<(a_0, p(a_0, b_0, b_0), 0), \iota_0\>,
\<(a_1, b_1, 1), \iota_1\>, %% (a_2, b_2, 1), \iota_2\>, 
\dots, \<(a_k, b_k, 1), \iota_k\>\},\] and 
note that $|T| < |S|$.
Let $t$ be a mixed local difference term for $T$.
Define
\[
d(x,y,z) = t(x, p(x,y,y), p(x,y,z)).
\]
We show that $d$ is a mixed local difference term for $S$.
Since $\chi_0 =0$, we need to show
that $(a_0, d^{\bA_{\iota_0}}(a_0,b_0,b_0))$ belongs to $\comm{\Cg^{\bA_{\iota_0}}(a_0,b_0)}{\Cg^{\bA_{\iota_0}}(a_0,b_0)}$.
We have
\begin{equation}
    \label{eq:100000}
  d(a_0,b_0,b_0) =
  t(a_0, p(a_0,b_0,b_0), p(a_0,b_0,b_0))\comm{\tau}{\tau} a_0,
\end{equation}
where we have used $\tau$ to denote $\Cg(a_0, p(a_0,b_0,b_0))$.
Note that
\[(a_0, p(a_0,b_0,b_0)) = (p(a_0,a_0,a_0), p(a_0,b_0,b_0)) \in \Cg(a_0, b_0),\]
%% $(a_0, p(a_0,b_0,b_0))$ is equal to $(p(a_0,a_0,a_0), p(a_0,b_0,b_0))$ which 
%% belongs to $\Cg(a_0, b_0)$,
so $\tau\leq \Cg(a_0,b_0)$. Therefore,
by monotonicity of the commutator we have
$\comm{\tau}{\tau} {\leq} \comm{\Cg(a_0,b_0)}{\Cg(a_0,b_0)}$.
It follows from this and (\ref{eq:100000}) that
%% $d(a_0,b_0,b_0)\comm{\Cg(a_0,b_0)}{\Cg(a_0,b_0)} a_0$,
\[d(a_0,b_0,b_0)\comm{\Cg(a_0,b_0)}{\Cg(a_0,b_0)} a_0,\]
as desired.

For the indices $1\leq i \leq k$ we have $\chi_i =1$, so we prove
$d(a_i,a_i,b_i) = b_i$ for such $i$. Observe,
\begin{align}
  d(a_i,a_i,b_i) &=
  t(a_i, p(a_i,a_i,a_i), p(a_i,a_i,b_i)) \label{eq:200000}\\
  &=t(a_i, a_i, b_i) \label{eq:200001}\\
  &=b_i. \label{eq:200002}
\end{align}
Equation~(\ref{eq:200000}) holds by definition of $d$,~(\ref{eq:200001})
because $p$ is an idempotent mixed local difference term for
$S'$, and~(\ref{eq:200002}) because $t$ is a mixed local difference term for $T$.

The remaining triples in our original set $S$
have indices satisfying $k<j\leq n$ and $\chi_j = 0$.
Thus, for these triples we want
$d(a_j,b_j,b_j)\comm{\Cg(a_j,b_j)}{\Cg(a_j,b_j)} a_j$.
By definition,
\begin{equation}
  \label{eq:450000}
d(a_j,b_j,b_j) =t(a_j, p(a_j,b_j,b_j), p(a_j,b_j,b_j)).  
\end{equation}
Since $p$ is a mixed local difference term for $S'$, %we have
the pair $(p(a_j,b_j,b_j), a_j)$ belongs to $[\Cg(a_j,b_j), \Cg(a_j,b_j)]$.
%% $(p(a_j,b_j,b_j), a_j)\in [\Cg(a_j,b_j), \Cg(a_j,b_j)]$.
This and 
(\ref{eq:450000}) imply
that 
$(d(a_j, b_j,b_j), t(a_j,a_j,a_j))$
belongs to
$\comm{\Cg(a_j,b_j)}{\Cg(a_j,b_j)}$.
Finally, by idempotence of $t$ we have
\[
d(a_j,b_j,b_j)\comm{\Cg(a_j,b_j)}{\Cg(a_j,b_j)} a_j,\]
as desired.
\\[6pt]
%--------------------------------------
\underline{Case $\chi_0 = 1$}:
%% \\[4pt]
%% Assume $\chi_0 = 1$ and, 
Without loss of generality, suppose $\chi_1 = \chi_2 =\cdots =\chi_k = 0$,
and $\chi_{k+1} = \chi_{k+2} = \cdots = \chi_{n} = 1$. Define 
\[
T = \{(p(a_0, a_0, b_0), b_0, 1),
(a_1, b_1, 0), (a_2, b_2 0), \dots, (a_k, b_k, 0)\},
\]
and note that $|T| < |S|$.
Let $t$ be a mixed local difference term for $T$ and
define
%% \[d(x,y,z) = t(p(x,y,z), p(y,y,z), z).\] 
$d(x,y,z) = t(p(x,y,z), p(y,y,z), z)$. 
Since $\chi_0 =1$, we want $d(a_0,a_0,b_0) = b_0$. By the definition of
$d$,
\begin{equation*}
  d(a_0,a_0,b_0) =
  t(p(a_0,a_0,b_0), p(a_0,a_0,b_0), b_0) =b_0.
\end{equation*}
The last equality holds since $t$ is a mixed local difference term for $T$, thus,
for $(p(a_0, a_0, b_0), b_0, 1)$.

If $1\leq i \leq k$, then $\chi_i =0$, so for these indices we prove
that $(a_i, d(a_i,b_i,b_i))$ belongs to $\comm{\Cg(a_i,b_i)}{\Cg(a_i,b_i)}$.
Again, starting from the definition of $d$ and using idempotence of $p$, we have
%% \begin{equation}
%%   \label{eq:40000}
%%   d(a_i,b_i,b_i) =
%%   t(p(a_i,b_i,b_i), p(b_i,b_i,b_i), b_i)=
%%   t(p(a_i,b_i,b_i), b_i, b_i).
%% \end{equation}
\begin{align}
  d(a_i,b_i,b_i) &=
  t(p(a_i,b_i,b_i), p(b_i,b_i,b_i), b_i)   \label{eq:40000}\\
  &=t(p(a_i,b_i,b_i), b_i, b_i). \nonumber
\end{align}
Next, since $p$ is a mixed local difference term for $S'$, we have
\begin{equation}
  \label{eq:50000}
  t(p(a_i,b_i,b_i), b_i, b_i)
 \comm{\Cg(a_i,b_i)}{\Cg(a_i,b_i)}
 t(a_i, b_i, b_i).
\end{equation}
Finally, since $t$ is a mixed local difference term for $T$, hence for
$(a_i, b_i, b_i)$,  %% $(1\leq i \leq k)$,
we have 
$t(a_i, b_i, b_i) \comm{\Cg(a_i,b_i)}{\Cg(a_i,b_i)} a_i$.
Combining this with (\ref{eq:40000}) and (\ref{eq:50000}) yields
$d(a_i,b_i,b_i) \comm{\Cg(a_i,b_i)}{\Cg(a_i,b_i)} a_i$,
as desired.

The remaining elements of our original set $S$
have indices $j$ satisfying $k<j\leq n$ and $\chi_j = 1$.
For these we want $d(a_j,a_j,b_j) = b_j$.
Since $p$ is a mixed local difference term for $S'$, we have
$p(a_j,a_j,b_j) = b_j$, and this along with idempotence of $t$ yields
%%\[ d(a_j,a_j,b_j) =  t(p(a_j,a_j,b_j), p(a_j,a_j,b_j), b_j)=  t(b_j, b_j, b_j) =b_j,\]
\begin{align*}
d(a_j,a_j,b_j) &=
t(p(a_j,a_j,b_j), p(a_j,a_j,b_j), b_j)\\
&=t(b_j, b_j, b_j) =b_j,
\end{align*}
as desired.
\end{proof}

\begin{cor}
  \label{cor:loc-diff-term}
  A finite idempotent algebra $\bA$ has a difference term operation if and
  only if each pair $((a,b,i), (a',b',i')) \in (A\times A \times \{0,1\})^2$ has a mixed local
  difference term.
\end{cor}
\begin{proof}
  One direction is clear, since a difference term operation for $\bA$ is
  obviously a mixed local difference term for the whole set 
  $A\times A \times \{0,1\}$.
  For the converse, suppose
  each pair in $(A\times A \times \{0,1\})^2$ has a mixed local
  difference term. Then, by Theorem~\ref{thm:mixed local-diff-terms},
  there is a single mixed local difference term for the whole set $A\times A \times \{0,1\}$,
  and this is a difference term operation for $\bA$.  Indeed, if $d$ is a
  mixed local difference term for $A\times A \times \{0,1\}$, then 
  for all $a, b \in A$, we have
  $a \comm{\Cg(a,b)}{\Cg(a,b)} d(a,b,b)$,
  since $d$ is a mixed local difference term for $(a,b,0)$, and we have
  $d(a,a,b) = b$, since $d$ is also a mixed local difference term for
  $(a,b,1)$.
\end{proof}


%%% NEW SECTION
\subsection{Mixed local difference terms}
\label{sec:mixed-local-diff}
In this section, we observe that the proofs in the previous section
did not hinge on the fact that we only considered a single algebra.  
Let $\sV$ be a variety and let $\bA_0=\< A_0, \dots\>$ and  $\bA_1=\< A_1, \dots\>$ be
algebras in $\sV$.
The direct sum (or coproduct) of $\bA_0$ and
$\bA_1$ is denoted by $\bA_0 + \bA_1$
(or by $\coprod_{i=0}^1 \bA_i$, especially when there are more than two
factors).
An element of (the universe of)
$\bA_0 + \bA_1$ is often denoted by $\<a, i\>$,
where $i\in \{0,1\}$ and $a \in A_i$.
The (universe of the) coproduct %% $\coprod_{i=0}^1 (\bA_i\times \bA_i)$ has elements
$\bA_0^2 + \bA_1^2$ has elements
$\<(a,b), i\>$ where $i\in \{0,1\}$ and $(a,b) \in A_i^2$.
An element of  the set $(A_0^2 + A_1^2) \times \{0,1\}$---and
now the notation has already become a bit unwieldy---has
the form $(\<(a,b), i\>, \chi)$, where $i\in \{0,1\}$, $(a,b) \in A_i^2$,
and $\chi\in \{0,1\}$. 

Fix two elements $(\<(a, b),i\>, \chi)$ and $(\<(a', b'),i'\>, \chi')$ of the
set $(A_0^2 + A_1^2) \times \{0,1\}$.
By a \defin{mixed local difference term} for this pair
we mean a ternary term $d$ satisfying both
\begin{align}
\text{ if $\chi=0$, then } & a \Comr{\Cg^{\bA_i}(a,b)} d^{\bA_i}(a,b,b); \label{eq:mixed-diff-triple}\\
\text{ if $\chi=1$, then } &d^{\bA_i}(a,a,b) = b; \nonumber
\end{align}
and the same set of relations with $a$, $b$, $i$, $\chi$ replaced 
by $a'$, $b'$, $i'$, $\chi'$, respectively.

Let $S$ be a sequence of triples drawn from the set
\[
\sU(A_0, A_1)  := (A_0^2 + A_1^2) \times \{0,1\}.
\]
If $d$ satisfies~(\ref{eq:mixed-diff-triple}) for all triples in $S$.
then we call $d$ is a \defin{mixed local difference term for $S$}.
We may use $\sU$ to denote the set $\sU(A_0, A_1)$ when the context renders the
universes involved obvious or immaterial.

Now, suppose that all pairs of triples 
in $\sU$ have mixed local difference terms.
Under this hypothesis the same argument that we used to prove
Theorem~\ref{thm:local-diff-terms} above can be used to prove that, for every $n$,
%% and every sequence $S\in \sU^n$, there is a term $d$ that is a mixed local
%% difference term for $S$.
every sequence $S\in \sU^n$ has a mixed local difference term.
That is, there is a single term $d$ that works (i.e., satisfies
the relations (\ref{eq:mixed-diff-triple})) for all $(\<(a, b), i\>,\chi)$ in $S$.
Here is the full statement of this slightly more general version of
Theorem~\ref{thm:local-diff-terms}.
From now on we drop the ``mixed'' qualifier since
it is inconsequential.

\begin{thm} %[\protect{cf.~\cite[Theorem 2.2]{MR3239624}}]
  \label{thm:mixed-local-diff-terms}
  Let $\sV$ be an idempotent variety and let
  $\bA_0 = \<A_0, \dots\>$ and   $\bA_1 = \<A_1, \dots\>$ be algebras in $\sV$. Define
  $\sU  = (A_0^2 + A_1^2)\times \{0,1\}$
  and suppose that every pair
  $((\<(a, b), i\>, \chi), (\<(a', b'), i'\>\chi')) \in \sU^2$
  has a local difference term. Then, for every $n$, every sequence $S \in \sU^n$
  has a local difference term.
\end{thm}

Corollary~\ref{cor:loc-diff-term} also generalizes, as follows:
\begin{cor}
  \label{cor:mix-loc-diff-term}
  Let $\sV$ be an idempotent variety and let
  $\bA_0 = \<A_0, \dots\>$ and   $\bA_1 = \<A_1, \dots\>$ be algebras in $\sV$. Define
  $\sU  = (A_0^2 + A_1^2)\times \{0,1\}$
  and suppose that every pair
  $((\<(a, b), i\>, \chi), (\<(a', b'), i'\>\chi')) \in \sU^2$
  has a local difference term. Then, there is a term $d$ that interprets as a
  difference term operation for both $\bA_0$ and $\bA_1$.
\end{cor}
%% Corollary~\ref{cor:loc-diff-term} also generalizes, as follows:
%% \begin{cor}
%%   \label{cor:mix-loc-diff-term}
%%   Let $\scA = \{\<A_i, \dots\> \mid i \in I\}$ be a collection of finite
%%   idempotent algebras. Then there exists a term $d$ that interprets as a
%%   difference term operation in every $\<A_i, \dots\> \in \scA$ if and only if each pair
%%   $((\<(a_i,b_i),\iota_i\>,\chi_i),  (\<(a_j,b_j),\iota_j\>,\chi_j))$ in
%%   $\bigl(\coprod_{i \in I} A_i^2\times \{0,1\}\bigr)^2$
%%   % \[\bigl(\coprod_{i \in I} (A_i^2\times \{0,1\})\bigr)
%%   %   \times \bigl(\coprod_{i \in I} (A_i^2\times \{0,1\})\bigr),\]
%%   has a local difference term.
%% \end{cor}





\subsection{Local difference terms on universes}
\label{sec:glob-local-diff}
The methods from earlier sections can be lifted up to work globally---that is,
on universes rather than elements---as we now explain. 
Let $\sV$ be a variety, let $\bA = \<A, \dots\> \in \sV$ 
and $i\in \{0,1\}$.
We call a term $d$ a \defin{local difference term
for $(A, i)$}
provided $d$ is a local difference term for every triple
$(a,b,i) \in A \times A \times \{i\}$. That is, for all $a, b \in A$,
\begin{align}
\text{ if $i=0$, then } & \, a \Comr{\Cg^{\bA}(a,b)} d^{\bA}(a,b,b);
\label{eq:global-diff-triple}\\
\text{ if $i=1$, then } & \,
d^{\bA}(a,a,b) = b. 
\end{align}

Let $\sV$ be a variety and let $\scA$ be a collection of algebras that belong to $\sV$.
Let $\scS(\scA)$ be the collection of all pairs $(A, i)$ where $A$ is the universe
of some algebra in $\scA$ and $i\in \{0,1\}$.  That is,
\[
\scS(\scA) = \{(A, i) \mid \<A, \dots\> \in \scA \text{ and } i \in \{0,1\}\}.
\]
Given a sequence
$S = ((A_0, \chi_0), (A_1, \chi_1), \dots,
(A_{n-1},\chi_{n-1})) \in \scS(\scA)^n$,
(or a subset $S \subseteq \scS(\scA)$),
a term $d$ is called a \defin{\glocal difference term for $S$}
if it is a \glocal difference term for every pair $(A_i, \chi_i)$ in $S$.
In addition to these definitions, in the proof of the next theorem we use
$|S|$ to denote the \emph{length of the sequence $S$}
(or, in case $S$ is a set, then $|S|$ denotes the cardinality of $S$, as usual).
\begin{thm}
  \label{thm:glob-loc-diff-terms}
  Let $\sV$ be a variety.  Let $\scA$ be a collection of finite idempotent
  algebras in $\sV$. Fix $n\geq 2$ and 
  let $S= ((A_0, \chi_0), (A_1, \chi_1), \dots, (A_{n-1}, \chi_{n-1}))\in \scS(\scA)^n$.
  Then there exists a term that is a \glocal difference term for $S$
  if and only if each 2-element subsequence $((A_i,\chi_i), (A_j,\chi_j))$ of $S$
  has a \glocal difference term.
\end{thm}
We relegate the proof of Theorem~\ref{thm:glob-loc-diff-terms} 
to the appendix (see Section~\ref{sec:proof-thm:glob}), since the argument 
is nearly identical to the one used to prove Theorem~\ref{thm:local-diff-terms}. 

\begin{cor}
  \label{cor:glob-loc-diff-term}
  Let $\sV$ be a variety.  Let $\scA$ be a collection of finite idempotent
  algebras in $\sV$. %% that is closed under the taking of subalgebras.
  Then there exists a term $d$ that interprets as a difference term operation
  for every algebra in $\scA$
  if and only if each pair $((A,i), (B,j)) \in \scS(\scA)^2$ has a \glocal
  difference term.
\end{cor}
Since the proof of Corollary~\ref{cor:glob-loc-diff-term}
is easy and similar to the proof
of Corollary~\ref{cor:loc-diff-term}, we consign it to 
appendix Section~\ref{sec:proof-cor:glob}.

We now pause to fix some more notation.
If $\alpha \in \Con(\bA)$ and $\beta \in \Con(\bB)$, then we let
$\alpha \mytimes \beta$ denote the set of pairs $((a,b),(a',b'))\in (A\times B)^2$ satisfying
$a \mathrel{\alpha} a'$ and $b \mathrel{\beta} b'$.  The relation 
$\alpha \mytimes \beta$ is clearly a congruence of $\bA \times \bB$.

% \pagebreak[1]
\begin{lem}
  \label{lem:products}
  Let $\sV$ be a variety and let $\bA$ and $\bB$ be finite idempotent
  algebras in $\sV$. Suppose the term $d$ 
  interprets as a difference term operation for $\bA\times \bB$.
  Then $d^{\bA}$ (resp., $d^{\bB}$) is a difference term operation for 
  $\bA$  (resp., $\bB$).
\end{lem}
\begin{proof}
  Assume that for all $(a, b)$ and $(a', b')$ in $A \times B$, the term $d$ satisfies
  \begin{align}
    d^{\bA \times \bB}((a, b), (a, b), (a', b')) &= (a', b'),\; \text{ and } \label{eq:60002}\\
    d^{\bA \times \bB}((a, b), (a', b'), (a', b'))
    &\Comr{\Cg^{\bA \times \bB}((a, b), (a', b'))} (a, b). \label{eq:60003}
  \end{align}
  We prove that $d^{\bA}$ is a difference term operation
  for $\bA$. (Obviously, the proof for $\bB$ is identical.)
  Thus, fixing $a, a' \in A$, we will show 
  \begin{align}
    d^{\bA}(a, a, a') &= a',\; \text{ and } \label{eq:60004}\\
    d^{\bA}(a,a',a')
    &\Comr{\Cg^{\bA}(a, a')} a. \label{eq:60005}
  \end{align}
  Equation (\ref{eq:60004}) is obvious by (\ref{eq:60002}),
  so we proceed to~(\ref{eq:60005}).
  Observe that 
  \[
  (d^{\bA}(a, a',a'), d^{\bB}(b, b',b'))
  \Comr{\Cg^{\bA \times \bB}((a, b), (a', b'))} (a, b),
  \]
  by~(\ref{eq:60003}). Therefore, Lemma~\ref{lem:hom-image-diff-term}
  implies\footnote{The first projection 
    $\pi_A : \bA \times \bB \to \bA$ is a surjective 
    homomorphism, so 
  $\pi_A \Com{\theta} \subseteq \Com{\pi_A(\theta)}$
    for all $\theta \in \Con (\bA \times \bB)$,
    by Lemma~\ref{lem:hom-image-diff-term}. Recall, that
    $\pi_A$ is defined on a congruence $\theta \in \Con(\bA \times \bB)$ as follows:
    $\pi_A(\theta) = \{(a,a') \in A^2 \mid ((a,b),(a',b')) \in \theta \text{ for
      some $(b,b')\in B^2$}\}$.}
  \[
    (d^{\bA}(a, a',a'), a)\in \pi_A \Com{\Cg^{\bA \times \bB}((a, b), (a', b'))}
    \subseteq \Com{\pi_A\bigl(\Cg^{\bA \times \bB}((a, b), (a', b'))\bigr)}.
  \]
  Next, observe that 
  $\Cg^{\bA}(a,a') \mytimes \Cg^{\bB}(b,b')$  %%  \in \Con(\bA \times \bB)$
  is a product of two congruences, one in $\Con(\bA)$ and the
  other in $\Con(\bB)$, so it is a congruence of $\bA\times \bB$.
  Moreover, it contains the pair $((a,b), (a',b'))$, so
  %% belongs to $\Cg^{\bA}(a,a') \times \Cg^{\bB}(b,b')$, 
  \[
  \Cg^{\bA \times \bB}((a, b), (a', b'))\leq \Cg^{\bA}(a,a') \mytimes \Cg^{\bB}(b,b').
  \]
  Therefore, 
  \[
  \pi_A\bigl(\Cg^{\bA \times \bB}((a, b), (a', b'))\bigr)
  \leq
  \pi_A\bigl(\Cg^{\bA}(a,a') \mytimes \Cg^{\bB}(b,b')\bigr) = 
  \Cg^{\bA}(a,a').
  \]
  Pulling all of these observations together and applying monotonicity of the
  commutator, we arrive at $d^{\bA}(a, a',a')  \Comr{\Cg^{\bA}(a,a')} a$, as desired.
\end{proof}

The converse of Lemma~\ref{lem:products} is harder to prove.
\begin{lem}
  \label{lem:products-conv}
 Let $\sV$ be a variety and let $\bA$ and $\bB$ be finite idempotent
 algebras in $\sV$. Suppose there is a single term $d$ that
 interprets as a difference term operation for $\bA$ and for $\bB$.
 Then $d^{\bA \times \bB}$ is a difference term operation for the product
 $\bA \times \bB$.
\end{lem}
\begin{proof}
  Fix $(a, b)$ and $(a', b')$ in $A \times B$.
  We must prove
  \begin{align}
    d^{\bA \times \bB}((a, b), (a, b), (a', b')) &= (a', b'),\; \text{ and } \label{eq:60000}\\
    d^{\bA \times \bB}((a, b), (a', b'), (a', b'))
    &\Comr{\Cg^{\bA \times \bB}((a, b), (a', b'))} (a, b). \label{eq:60001}
  \end{align}
  %% where $\theta:= \Cg^{\bA \times \bB}((a, b), (a', b'))$.
  Since $d^{\bA}$ and $d^{\bB}$ are difference term operations for $\bA$ and
  $\bB$, respectively, it's easy to see that~(\ref{eq:60000}) is satisfied:
  \[
  d^{\bA \times \bB}((a, b), (a, b), (a', b')) =
  (d^{\bA}(a, a, a'),  d^{\bB}(b, b, b')) = (a', b').
  \]
  It remains to check (\ref{eq:60001}).
  Again, since $d^{\bA}$ and
  $d^{\bB}$ are difference term operations,
  % \begin{align*}
  % d^{\bA}(a,a',a')
  % &\Comr{\Cg^{\bA}(a,a')} a\\
  % d^{\bB}(b, b', b')
  % &\Comr{\Cg^{\bB}(b,b')} b.
  % \end{align*}
  \[
  d^{\bA}(a,a',a')
  \Comr{\Cg^{\bA}(a,a')} a \quad \text{ and } \quad
  d^{\bB}(b, b', b')
  \Comr{\Cg^{\bB}(b,b')} b.
  \]
  Therefore, 
  % the pair $\bigl((d^{\bA}(a,a',a'),d^{\bB}(b, b', b')), (a,b)\bigr)$ belongs
  % to the relation \[ \Com{\Cg^{\bA}(a,a')} \mytimes \Com{\Cg^{\bB}(b,b')}.\]
  $(d^{\bA}(a,a',a'),d^{\bB}(b, b', b'))
    \mathrel{\Com{\Cg^{\bA}(a,a')}\mytimes\Com{\Cg^{\bB}(b,b')}}
    (a,b)$.
  We claim that the latter is equal to
  $\Com{\Cg^{\bA}(a,a') \mytimes \Cg^{\bB}(b,b')}$.
  Recall from above that
  \[
  \Cg^{\bA \times \bB}((a, b), (a', b'))\leq \Cg^{\bA}(a,a') \times \Cg^{\bB}(b,b').
  \]
  Therefore, if we prove that
  \[
  \Com{\Cg^{\bA}(a,a')}  \mytimes  \Com{\Cg^{\bB}(b,b')} = 
  \Com{\Cg^{\bA}(a,a') \mytimes \Cg^{\bB}(b,b')},
  \]
  then we could complete the proof by showing that
  \begin{equation}
    \label{eq:655}    
  \Cg^{\bA \times \bB}((a, b), (a', b'))\geq \Cg^{\bA}(a,a') \mytimes \Cg^{\bB}(b,b').
  \end{equation}

  \smallskip

  TODO: (\ref{eq:655}) is false in general; maybe false here
  too; then we need a new idea.

  \medskip
  
  TODO: Prove Lemma~\ref{lem:products-conv} somehow!!!
  %% Therefore, by monotonicity of the commutator,
  %% (\ref{eq:60001}) holds.
  %% It remains to check the claim... 
\end{proof}


\subsection{Algorithm 2: existence of a difference term}
\label{sec:algor-2}
%% \subsection{Algorithm 2: existence of difference terms}
%% \label{sec:algor-2:-exist}
%% In this subsection we prove the following
\begin{cor}
\label{cor:algor-2}
  There is a polynomial-time algorithm that takes as input
  any finite idempotent algebra $\bA$ and decides whether
  the variety $\bbV(\bA)$ that it generates
  has a difference term operation.
\end{cor}
\begin{proof}
  Let $\sV = \bbV(\bA)$ and let $\bF = \bF_{\sV}(x,y)$ be the free algebra in
  $\sV$ generated by $x$ and $y$.
  By Theorem~\ref{thm:F}, deciding whether $\sV$ has a difference term is equivalent to 
  deciding whether $\bF$ has a difference term operation.
  We can assume $\bF$ is a subdirect product of
  $\bA_0 \times \bA_1 \times \cdots \times \bA_{n-1}$, where $n\leq |A|^2$ and
  where each $\bA_i$ is a 2-generated subalgebra of $\bA$.
  Let $\scA = \{A_0, A_1, \dots, A_{n-1}\}$ and (as above) let $\scS(\scA)$ denote 
  all pairs $(A, i)$ such that $\bA = \<A, \dots\> \in \scA$ and $i\in \{0,1\}$.

  We begin by proving that we can check in polynomial time (in $|A|$)
  whether or not the product
  $\bA_0 \times \bA_1 \times \cdots \times \bA_{n-1}$ has a difference term operation.
  By Corollary~\ref{cor:glob-loc-diff-term} and Lemma~\ref{lem:products-conv},
  it suffices to check that each of the 
  $n^2$ pairs  in 
  $\scS(\scA)^2$ has a \glocal
  difference term.  Fix a pair
  $((A_i, \chi_i), (A_j, \chi_j)) \in  \scS(\scA)^2$,
  and let $\sU  = (A_i^2 + A_j^2)\times \{0,1\}$.
  By Theorem~\ref{thm:mixed-local-diff-terms},
  to prove that every sequence $S \in \sU^n$
  has a local difference term, it suffices to check that every pair
  $\bigl( (\<(a, b), i\>, \chi), (\<(a', b'), i'\>, \chi')\bigr) \in \sU^2$
  has a local difference term. It follows from the argument given 
  in the proof of Corollary~\ref{cor:algor-1} that the number of 
  operations required to check whether
  $\bigl( (\<(a, b), i\>, \chi), (\<(a', b'), i'\>, \chi')\bigr)$
  has a local difference term is bounded by a
  polynomial in $|A_i||A_j|\leq |A|^2$.  Since there are 
  $4|A_i|^2|A_j|^2 \leq 4|A|^4$ pairs in $\sU^2$, 
  it still takes only a polynomial in $|A|$ number of steps to test whether
  the pair $((A_i, \chi_i), (A_j, \chi_j))$ has a \glocal difference term.
  There are $n^2 \leq |A|^4$ such pairs to test, so the number of steps required to 
  test whether
  $\bA_0 \times \bA_1 \times \cdots \times \bA_{n-1}$ has a difference term
  operation is bounded by a constant times a power of $|A|$.

  \medskip

  \noindent TODO: complete the proof by showing that if the product 
  $\bA_0 \times \bA_1 \times \cdots \times \bA_{n-1}$ has a difference term operation,
  then so does the subdirect product $\bF$.
\end{proof}

% \draftsecskip

\section*{Acknowledgments}
The author would like to thank
Ralph Freese for proposing this project and calling attention to
the work of Valeriote and Willard.
Other helpful suggestions came from Cliff Bergman and the
following members of the {\it Hawaii Universal Algebra 
Seminar:} Alex Guillen, Tristan Holmes, Bill Lampe, and J.~B.~Nation.
These contributions are gratefully acknowledged.

This work was partially supported by the National Science 
Foundation Grant No.~1500235.

%% \newpage


\appendix



\section{An Easy Route to the Commutator}
\label{sec:interlude:-an-easy}
For an algebra $\bA$ with congruence relations $\alpha$, $\beta\in \Con\bA$,
let $\bbeta$ denote the subalgebra of $\bA\times \bA$ with universe 
$\beta$, and let $0_A$ denote the least equivalence relation on $A$.
%% (note that $0_A$ is a subuniverse of $\bbeta$).
Thus, $0_A = \{(a,a) \mid a\in A\} \leq \beta$.
Denote by $D_\alpha$ the following subset of $\beta \times \beta$:
\begin{equation}
  \label{eq:9009}
D_\alpha =(\alpha \mytimes \alpha) \cap (0_A \times 0_A)
= \{((a,a), (b,b)) \in (0_A \times 0_A) \mid a\alphar b\}.
\end{equation}
Let $\Delta_{\beta, \alpha} = \Cg^{\bbeta}(D_\alpha)$ denote the congruence relation
of $\bbeta$ generated by $D_\alpha$.
The condition $\CC{\alpha}{\beta}{\gamma}$
holds iff for all $a \alphar b$, for all $u_i \betar v_i$ ($1\leq i\leq n$), and for all 
$t\in \Pol_{n+1}(\bA)$, we have
$t(a,\bu) \mathrel{\gamma} t(a, \bv)$
iff $t(b,\bu) \mathrel{\gamma} t(b, \bv)$.
There are a number of different ways to define a commutator.
See, for example,~\cite{MR0432511,MR541622,MR590312,MR1145556,MR1663558,MR1257643}.
In this note, the commutator $[\alpha, \beta]$ is defined to be the least
congruence $\gamma$ such that $\CC{\alpha}{\beta}{\gamma}$ holds.
\begin{comment}
Occasionally it is more convenient to write such an equivalence as a (two-way) derivation tree,
as follows:
\[
\infer=[\CC{\alpha}{\beta}{\gamma}]{\Gamma \vdash t(b,\bu) \mathrel{\gamma} t(b, \bv)}{\Gamma \vdash t(a,\bu) \mathrel{\gamma} t(a, \bv)}\]
where $\Gamma$ is a context containing
$a \alphar b$, $u_i \betar v_i$ ($1\leq i\leq n$), and 
$t\in \Pol_{n+1}(\bA)$.
\end{comment}


%% \section{Alternate Description of the Commutator}
%% \label{sec:altern-descr}

We now %% In~\cite{com-fix-poi} we gave 
describe an alternate way to express the commutator---specifically,
it is the least fixed point of a certain closure operator.
This description was inspired by the one that is mentioned in passing by
Keith Kearnes in~\cite[p.~930]{MR1358491}.  Our objective here is
to prove that the description we present is correct (i.e., describes the commutator)
and to show that it leads to a simple, efficient procedure for computing the commutator.

Let $\Tol(A)$ denote the collection of all tolerances (reflexive symmetric relations)
on the set $A$,\footnote{Actually, a
  \emph{tolerance} of an algebra $\bA = \<A, \dots\>$
  is a reflexive symmetric subalgebra of $\bA \times \bA$.
  Therefore, the set of all tolerances of $\bA$ forms an
  algebraic (hence complete) lattice.
  If we drop the operations and consider only the set $A$, then a tolerance relation on $A$
  is simply a reflexive symmetric binary relation.
}
and let 
$\Psi_{\beta, \alpha} \colon \Tol(A) \to \Tol(A)$ be the function defined
for each $T \in  \Tol(A)$ follows:
\begin{equation}
  \label{eq:7}
  \Psi_{\beta, \alpha}(T)
  = \{ (x,y) \in A\times A \mid
  (\exists\, (a,b) \in T)\,
 (a,b) \mathrel{\Delta_{\beta, \alpha}} (x,y)\},
\end{equation}
where
$\Delta_{\beta, \alpha} = \Cg^{\bbeta}(D_\alpha)$ and
$D_\alpha =(\alpha \mytimes \alpha) \cap (0_A \times 0_A)$
(as in~(\ref{eq:9009})).

\begin{remarks}\
  \begin{enumerate}
\item
  It's easy to see that $\Psiba (T)$ is reflexive and symmetric
  whenever $T$ has these properties; similarly, $\Psiba (T)$ is compatible with the
  operations of $\bA$ whenever $T$ is. In other words $\Psiba$ maps tolerances of
  $A$ ($\bA$, resp.) to tolerances of $A$ ($\bA$, resp.).
  \item 
  Since $\Psiba$ is clearly a monotone increasing function on the complete
  lattice $\Tol(A)$, it is guaranteed to have a least fixed
  point---that is, there is a point $\tau\in \Tol(A)$ such that $\Psiba(\tau) = \tau$
  and $\tau \leq T$, for every $T \in \Tol(A)$
  satisfying $\Psiba(T) = T$.
\item
  Here are two ways the least fixed point of $\Psiba$ could be computed:
  \begin{equation}
    \label{eq:4}
  \tau = \Meet \{ T \in \Tol(A) \mid \Psiba(T) \leq T\}
  \quad \text{ and } \quad
     \tau = \Join_{k\geq 0} \Psiba^{k}(0_A).
  \end{equation}
  In Lemma~\ref{lem:fixed-point-comm} we will show that the least
  fixed point of $\Psiba$ is, in fact, the commutator,
  $\tau = [\alpha, \beta]$, so either
  expression in~(\ref{eq:4}) could potentially be used to compute it.
  However, Lemma~\ref{lem:fixed-point-comm} also shows that 
  $\Psiba$ is a closure operator; in particular, it is idempotent. Therefore,
  $\Psiba^{k}(0_A) = \Psiba(0_A)$ for all $k$, so we have 
  the following simple description of the commutator:
  \begin{align*}
    %% \label{eq:55}
          [\alpha, \beta] =
          \Psiba(0_A)
          &= \{ (x,y) \in A\times A \mid
          (\exists\, (a,b) \in 0_A)\, (a,b) \mathrel{\Delta_{\beta, \alpha}} (x,y)\}\\
          &= \{ (x,y) \in A\times A \mid
          (\exists a \in A)\, (a,a) \mathrel{\Delta_{\beta, \alpha}} (x,y)\}.
  \end{align*}
  \end{enumerate}
\end{remarks}


\subsection{Fixed Point Lemma}
\begin{lemma}
  \label{lem:fixed-point-comm}
  If $\alpha$, $\beta\in \Con(\bA)$ and 
  if $\Psi_{\beta, \alpha}$ is defined by~(\ref{eq:7}), then 
  \begin{enumerate}[(i)]
  \item $\Psiba$ is a closure operator on $\Tol(A)$;
  \item $[\alpha, \beta]$ is the least fixed point of $\Psiba$.
  \end{enumerate}
\end{lemma}
\begin{proof}\
  \begin{enumerate}[(i)]
  \item 
  %% In fact, $\Psi_{\beta, \alpha}$ is a closure operator on all
  %% of $\Tol(A)$ as we 
    To prove (i) we verify that
    $\Psi_{\beta, \alpha}$ has the three properties that define a closure
    operator---namely for all $T$, $T' \in \Tol(A)$,
  \begin{enumerate}[(c.1)]
  \item \label{item:c1} $T  \leq \Psiba(T )$;     
  \item \label{item:c2} $T  \leq T'  \Rightarrow \Psiba(T) \leq \Psiba(T')$;    
  \item \label{item:c3} $\Psiba(\Psiba(T))  = \Psiba(T)$. 
  \end{enumerate}

  \vskip2mm

  \noindent {\it Proof of (c.1):} $(a,b) \in T $
  implies $(a,b) \in \Psiba(T )$ because $(a,b)\mathrel{\Delta_{\beta, \alpha}} (a,b)$.\\[4pt]
  %% this proves~(c.\ref{item:c1}).
  \noindent {\it Proof of (c.2):} $(x,y) \in \Psiba(T )$ iff there exists
  $(a,b) \in T  \leq T'$ such that
  $(a,b) \mathrel{\Delta_{\beta, \alpha}} (x,y)$; this and $(a,b) \in T'$ implies
  $(x,y) \in \Psiba(T')$.\\[4pt]
  \noindent {\it Proof of (c.3):} $(x,y) \in \Psiba(\Psiba(T))$ if and only if
  there exists $(a,b) \in \Psiba(T)$ such that
  $(a,b) \Deltabar (x,y)$, and $(a,b) \in \Psiba(T)$ is in turn equivalent to 
  the existence of $(c,d) \in T $ such that
  $(c,d) \Deltabar (a,b)$. By transitivity of $\Deltaba$, we have that
  $(c,d) \Deltabar (a,b) \Deltabar (x,y)$ implies
  $(c,d) \Deltabar (x,y)$, proving that there exists $(c,d) \in T $ such that
  $(c,d) \Deltabar (x,y)$; equivalently, $(x,y) \in T $.

  \medskip

\item
  %% \noindent (ii) 
  As remarked above, from part (i) follows 
  $\Psiba^{k}(0_A) = \Psiba(0_A)$ for all $k$, so the least fixed point of
  $\Psiba$ that appears in the formula on the right in~(\ref{eq:4}) reduces
  to $\tau = \Psiba(0_A)$.  Therefore, to complete the proof it suffices to show
  $[\alpha, \beta] = \Psiba(0_A)$.

  We first prove $[\alpha, \beta]\leq \Psiba(0_A)$.
  Since $[\alpha, \beta]$ is the least congruence $\gamma$
  satisfying $\CC{\alpha}{\beta}{\gamma}$, it suffices to prove
    $\CC{\alpha}{\beta}{\Psiba(0_A)}$ holds.
    Suppose $a \alphar a'$ and $b_i \betar b_i'$ %% ($1\leq i \leq k$)
    and $t^{\bA} \in \Pol_{k+1}(\bA)$ satisfy
    $t^{\bA}(a, \bb) \mathrel{\Psiba(0_A)} t^{\bA}(a, \bb')$,
    where $\bb = (b_1, \dots, b_k)$ and $\bb' = (b_1', \dots, b_k')$.
    We must show $t(a', \bb) \mathrel{\Psiba(0_A)} t(a', \bb')$.  
    By definition of $\Psiba$,
    the antecedent $t^{\bA}(a, \bb) \mathrel{\Psiba(0_A)} t^{\bA}(a, \bb')$ is equivalent to    
    the existence of $c \in A$ such that $(c,c) \Deltabar (t^{\bA}(a, \bb), t^{\bA}(a, \bb'))$.
    Now
    \[
    (t^{\bA}(a, \bb), t^{\bA}(a, \bb')) = t^{\bbeta}((a,a),(b_1, b_1'), \dots,(b_k, b_k')),
    \]
    and since $a \alphar a'$, we have
    \[
    t^{\bbeta}((a,a),(b_1, b_1'), \dots,(b_k, b_k'))
    \Deltabar
    t^{\bbeta}((a',a'),(b_1, b_1'), \dots,(b_k, b_k')).
    \]
    The latter is equal to $(t^{\bA}(a', \bb), t^{\bA}(a', \bb'))$, and  it follows
    by transitivity of $\Deltaba$ that
    $(c,c) \Deltabar (t^{\bA}(a', \bb), t^{\bA}(a', \bb'))$.
    Therefore, $t(a', \bb) \mathrel{\Psiba(0_A)} t(a', \bb')$, as desired.  



  We now prove $\Psiba(0_A)\leq   [\alpha, \beta]$.
  If $(x,y)\in \Psiba(0_A)$ then there exists $a \in A$ such that 
  \begin{equation}
    \label{eq:1100}
    (a,a) \mathrel{\Delta_{\beta, \alpha}} (x,y).
  \end{equation}
  From the definition of $\Delta_{\beta, \alpha}$ and 
  \malcev's congruence generation theorem,~(\ref{eq:1100})
  holds if and only if for there exist
  $(z_i, z_i') \in \beta$ ($0\leq i \leq n+1$), and $(u_i, v_i) \in \alpha$,
  $f_i \in \Pol_1(\bbeta)$ ($0\leq i \leq n$), such that
  $(a, a) = (z_0,z_0')$ and $(x, y)=(z_{n+1},z'_{n+1})$ hold, and so do the
  following equations of sets: 
  \begin{align}
    \label{eq:001}
    \{(a, a),(z_1,z_1')\} &= \{f_0(u_0,u_0), f_0(v_0,v_0)\},\\
    \label{eq:011}
    \{(z_1,z_1'),(z_2,z_2')\} &= \{f_1(u_1,u_1), f_1(v_1,v_1)\},\\
    \nonumber
    &\; \; \vdots\\
    %% \label{eq:n-1}
    \nonumber
    \{(z_{n},z_{n}'),(x, y)\} &= \{f_{n}(u_{n},u_{n}), f_{n}(v_{n},v_{n})\}.
  \end{align}
  Now $f_i \in \Pol_1(\bbeta)$ for all $i$, so
  \newcommand\gA{\ensuremath{g^{\bA}}}%
  \[
  f_i(c, c') = g_i^{\bbeta}((c, c'), (b_1, b_1'), \dots, (b_k, b_k') )
  = (\gA_i(c, \bb), \gA_i(c', \bb')),%
  \]
  \renewcommand\gA{\ensuremath{g}}%
  for some $k$, some $(k+1)$-ary term $\gA_i$, and some constants
  $\bb = (b_1, \dots, b_k)$ and $\bb' = (b_1', \dots, b_k')$ satisfying
  $b_i \betar b_i'$ ($1\leq i\leq k$). 
  By~(\ref{eq:001}), either
  \[
  (a, a) = \bigl(\gA_0(u_0, \bb), \gA_0(u_0, \bb')\bigr)
  \quad \text{ and } \quad 
  (z_1,z_1')= \bigl(\gA_0(v_0, \bb), \gA_0(v_0, \bb')\bigr),
  \]
  or vice-versa. 
  We assumed $u_0 \alphar v_0$ and $b_i \betar b_i'$ ($1\leq i\leq k$),
  so the $\alpha,\beta$-term condition entails
  $\gA_0(u_0, \ba) \commr{\alpha}{\beta} \gA_0(u_0, \ba')$
  iff 
  $\gA_0(v_0, \ba) \commr{\alpha}{\beta} \gA_0(v_0, \ba')$.
  %% \[
  %%   \gA_0(u_0, \bb) \commr{\alpha}{\beta} \gA_0(u_0, \bb')
  %%   \quad \Longleftrightarrow \quad 
  %%   \gA_0(v_0, \bb) \commr{\alpha}{\beta} \gA_0(v_0, \bb').
  %%   \]
  From this and~(\ref{eq:001}) we deduce that 
  $(a,a)\in [\alpha, \beta]$ iff $(z_1,z_1')\in [\alpha, \beta]$.
  Similarly~(\ref{eq:011}) and $u_1 \alphar v_1$ imply
  $(z_1,z_1')\in [\alpha, \beta]$ iff
  $(z_2,z_2')\in [\alpha, \beta]$.  Inductively, and by transitivity of
  $[\alpha, \beta]$, we conclude $(a,a)\in [\alpha, \beta]$ iff
  $(x,y)\in [\alpha, \beta]$.
  Since $(a,a)\in [\alpha, \beta]$, we have $(x,y)\in [\alpha, \beta]$, as desired.

  \end{enumerate}
\end{proof}

\subsection{Computing the Commutator}
As a consequence of the description of the commutator given in the last section,
we now have the following simple method for computing it.

\smallskip

\noindent {\bf Input} \hskip2mm A finite algebra, $\bA = \<A, \dots\>$, and two congruence relations $\alpha$, $\beta \in \Con \bA$.

\smallskip
\noindent {\bf Procedure}
\begin{itemize}
\item {\bf Step 1} \hskip2mm Compute the congruence relation
  $\Deltaba = \Cg^{\bbeta}\bigl\{((a,a), (b,b)) \mid a \alphar b \bigr\}$.
\item {\bf Step 2} \hskip2mm Compute the commutator
  \[[\alpha, \beta] 
  = \bigl\{(x,y) \in A\times A \mid (\exists a \in A) \, (a,a) \Deltabar (x,y)\bigr\} 
  =\bigcup_{a\in A} (a,a)/\Deltaba
    \]
\end{itemize}
Note that $\Deltaba$ is a subalgebra of $\bA^2 \times \bA^2$ and such a congruence
can be computed in polynomial-time in the size of $\bA$. (See~\cite{MR2470585}.)




%%%%%%%%%%%%%%%%%%%%%%%%%%%%%%%%%%%%%%%%%%%%%%%%%%% OMITTED PROOFS

\section{Proofs}
\label{sec:proofs}

\ifthenelse{\boolean{extralong}}{

  \subsection{Proof of Lemmas~\ref{lem:monotone-comm} and~\ref{lem:complete-meet-join-monotone}}
  \label{sec-lem:monotone-comm}

  \begin{lemma}[Monotonicity of the Commutator]
    %% \label{lem:monotone-comm}
    Let $\bA$ be an algebra with congruences $\alpha$, $\alpha'$, $\beta$, $\beta'$ satisfying
    $\alpha\leq \alpha'$ and $\beta \leq \beta'$.
    Then $\comm \alpha \beta \leq \comm{\alpha'}{\beta'}$.
  \end{lemma}
  \begin{proof}
    For every $\delta \in \Con\bA$, $\CC{\alpha'}{\beta'}{\delta}$ implies
    $\CC{\alpha}{\beta}{\delta}$, since $\alpha\leq \alpha'$ and $\beta \leq \beta'$.
    In particular, $\CC{\alpha'}{\beta'}{[\alpha', \beta']}$ implies
    $\CC{\alpha}{\beta}{[\alpha', \beta']}$, so
    $\comm{\alpha}{\beta} \leq \comm{\alpha'}{\beta'}$.
  \end{proof}

  \begin{lemma}[\ref{lem:complete-meet-join-monotone}]
    Let $\bA$ be an algebra with congruences
    $\alpha_i$ and 
    $\beta_i$ %% $\gamma_k$
    %% are congruences of $\bA$, 
    for all $i \in I$.
    Then
    \[
    \bigl[ \Meet \alpha_i, \Meet \beta_i \bigr] 
    \leq \Meet \comm {\alpha_i} {\beta_i}  \quad \text{ and } \quad
    \Join \comm {\alpha_i} {\beta_i} \leq \bigl[ \Join \alpha_i, \Join \beta_i\bigr].
    \]
  \end{lemma}

  \begin{proof}
    By monotonicity, $\bigl[\Meet \alpha_i, \Meet \beta_i\bigr] \leq
    \comm {\alpha_i} {\beta_i} \leq \bigl[\Join \alpha_i, \Join \beta_i\bigr]$,
    for all $i \in I$.
  \end{proof}
}{}


\subsection{Proof of Theorem~\ref{thm:glob-loc-diff-terms}}
\label{sec:proof-thm:glob}
\begin{theorem} %[\ref{thm:glob-loc-diff-terms}]
  Let $\sV$ be a variety.  Let $\scA$ be a collection of finite idempotent
  algebras in $\sV$. Fix $n\geq 2$ and 
  let $S= ((A_0, \chi_0), (A_1, \chi_1), \dots, (A_{n-1}, \chi_{n-1}))\in \scS(\scA)^n$.
  Then there exists a term that is a \glocal difference term for $S$
  if and only if each 2-element subsequence $((A_i,\chi_i), (A_j,\chi_j))$ of $S$
  has a \glocal difference term.
\end{theorem}
\begin{proof}
  One direction is clear; if $d$ is a \glocal difference term for every
  element $(A_i, \chi_i)$ of $S$, then  every pair $((A_i,\chi_i),
  (A_j,\chi_j))$ of elements of $S$ also has a \glocal difference
  term---namely, $d$.

  For the converse, suppose that
  for each pair $((A_i,\chi_i), (A_j,\chi_j))$ of elements of $S$ there exists a
  term $p_{ij}$ that is a \glocal difference term for both
  $(A_i,\chi_i)$ and $(A_j,\chi_j)$.
  We will prove by induction on the length of $S$ that
  %% , for every $n\geq 2$ every subsequence $S \in \scS(\scA)^n$
  there exists a term $d$ that is a \glocal difference term for every
  $(A_i, \chi_i)$ in $S$.

  In the base case, $n = |S| = 2$, the claim holds by assumption.
  Fix $n\geq 2$ and assume for every
  $2\leq k \leq n$ that every sequence in $\scS(\scA)^k$
  has a \glocal difference term. Let
  $S = ((A_0, \chi_0), (A_1, \chi_1), \dots, (A_{n},\chi_{n})) \in \scS(\scA)^{n+1}$.
  %% We prove there exists a term $d$ that is a \glocal difference term for every
  %% $(A_i, \chi_i)$ in $S$.
  We prove  $S$ has a \glocal difference term.

  Since $|S| \geq 3$ and $\chi_i \in \{0,1\}$ for all $i$, there must exist
  indices $i\neq j$ such that $\chi_i = \chi_j$. Assume without loss of generality
  that one of these indices is $j=0$.  Define the subsequence
  $S' = ((A_1, \chi_1), \dots,(A_{n},\chi_{n}))$ of $S$. %% \in \scS(\scA)^{n}$
  Since $|S'| = n$, the sequence $S'$ has a \glocal difference term $p$.
  Thus, for all $1\leq i \leq n$,
  for all $a, b\in A_i$  we have
  \begin{align*}
    \text{ if $\chi_i=0$, then } &
    a \Comr{\Cg(a,b)} d(a,b,b);\\
    \text{ if $\chi_i=1$, then } &
    %% d^{\bB_i}(b,b,b') = b'.
    d(a,a,b) = b.
  \end{align*}
  We split the remainder of the proof into two cases.

  \medskip

%--------------------------------------
\noindent \underline{Case $\chi_0 = 0$:}
Without loss of generality, suppose that
$\chi_1 = \chi_2 = \cdots =\chi_k = 1$,
and
$\chi_{k+1} = \chi_{k+2} = \cdots = \chi_{n} = 0$.
Define
%% the set
%% \[P_0 = \{p(b, b', b')  \in B_0 \mid b, b' \in B_0\},\]
%% and let
\[
T = ((A_0, 0), (A_1, 1), (A_2, 1), \dots, (A_k, 1)).
\]
Note that $|T| < |S|$.
Let $t$ be a \glocal difference term for $T$.
We will prove that the term $d(x,y,z) = t(x, p(x,y,y), p(x,y,z))$
is a \glocal difference term for $S$.

The first element of $S$ is $(A_0, 0)$, so we need to show for all $a$, $b \in A_0$
that
\[
d(a,b,b) \Comr{\Cg(a,b)} a.
\]
Fix $a, b \in A_0$.
By definition of $d$, and since
$t$ is a \glocal difference term for $(A_0, 0)$,
\begin{equation}
  \label{eq:100100}
  d(a,b,b) 
  =t(a, c, c)\Comr{\Cg(a, c)} a,
\end{equation}
where $c = p(a,b,b)$.
Now, $(a, c) = (p(a,a,a), p(a,b,b)) \in \Cg(a, b)$, therefore,
$\Cg(a, c) \leq \Cg(a,b')$.
It follows from this and monotonicity of the commutator that
$\Com{\Cg(a, c)} \leq \Com{\Cg(a,b)}$,
This and~(\ref{eq:100100}) imply
$d(a,b,b)\Comr{\Cg(a,b)} a$,
as desired.

Next, consider the (possibly empty) set of indices $\{i \mid 1\leq i \leq k\}$.
For such indices $\chi_i =1$, so we will prove
for all $a$, $b \in A_i$ that $d(a,a,b) = b$.
Fix $a, b \in A_i$ and observe that
% \begin{align}
%   %% d^{\bB_i}(b,b,b') &=
%   d(a,a,b) &=
%   t(a, p(a,a,a), p(a,a,b)) \label{eq:210200}\\
%   &=t(a,a,b) \label{eq:220201}\\
%   &=b. \label{eq:230202}
% \end{align}
% Equation~(\ref{eq:210200}) holds by definition of $d$,~(\ref{eq:220201})
% because $p$ is an idempotent \glocal difference term for
% $S'$, and~(\ref{eq:230202}) because $t$ is a \glocal difference term for $T$.
\[
  d(a,a,b) =
  t(a, p(a,a,a), p(a,a,b)) % \label{eq:210200}\\
  =t(a,a,b) % \label{eq:220201}\\
  =b. % \label{eq:230202}
\]
The first equation holds by definition of $d$, the second
because $p$ is an idempotent \glocal difference term for
$S'$, and the third because $t$ is a \glocal difference term for $T$.

The indices of the remaining elements of $S$
belong to the set $\{j \mid k<j\leq n\}$ (which is nonempty since we
assumed $\chi_0 = \chi_i = 0$ for some $i>0$).
For such indices we have $\chi_j = 0$.
Thus, fixing $a, b \in A_j$, we check that
$d(a,b,b)\Comr{\Cg(a,b)} a$.
%% Fix $a, b \in A_j$. 
By definition,
\begin{equation}
  \label{eq:451}
%% d^{\bA_j}(a,b,b) =t^{\bA_j}(a, p^{\bA_j}(a,b,b), p^{\bA_j}(a,b,b)).  
d(a,b,b) =t(a, p(a,b,b), p(a,b,b)).  
\end{equation}
Also, $p(a,b,b) \Comr{\Cg(a,b)} a$,
since $p$ is a \glocal difference term for $S'$.
%% $(p(a,b,b), a)\in [\Cg(a,b), \Cg(a,b)]$.
This and (\ref{eq:451}) imply
that
%% $(d(a, b,b), t(a,a,a))$ belongs to $\Com{\Cg(a,b)}$.
$d(a, b,b) \Comr{\Cg(a,b)} t(a,a,a))$.
Finally, by idempotence of $t$ we have
$d(a,b,b)\Comr{\Cg(a,b)} a$,
as desired.
\\[4pt]
%--------------------------------------
\underline{Case $\chi_0 = 1$:}
Without loss of generality, suppose $\chi_1 = \chi_2 =\cdots =\chi_k = 0$,
and $\chi_{k+1} = \chi_{k+2} = \cdots = \chi_{n} = 1$.
%% Define the set
%% \[P_1 = \{p(b, b', b')  \in B_0 \mid b, b' \in B_0\},\]
Define
\[
T = ((A_0, 1), (A_1, 0), (A_2, 0), \dots, (A_k, 0)).
\]
and note that $|T| < |S|$, so $T$ has a \glocal difference term $t$.
We will prove that the term $d(x,y,z) = t(p(x,y,z), p(y,y,z), z)$
is a \glocal difference term for $S$.

The first pair in $S$ is $(A_0, 1)$, so we want to show for all $a$, $b \in A_0$ that
$d(a,a,b) = b$.
Fix $a$, $b \in A_0$. By definition of $d$,
we have $d(a,a,b) = t(p(a,a,b), p(a,a,b), b) =b$.
The last equality holds since $t$ is a \glocal difference term for $T$, in particular,
for $(A_0, 1)$.

Next, consider the (possibly empty) set of indices $\{i \mid 1\leq i \leq k\}$.
For such indices $\chi_i =0$, so we will prove
for all $a$, $b \in A_i$ that
$d(a,b,b) \Comr{\Cg(a,b)} a$.
Fix $a, b\in A_i$.
By definition of $d$ and idempotence of $p$, 
% \begin{align}
%   d(a,b,b) &=
%   t(p(a,b,b), p(b,b,b), b)   \label{eq:444}\\
%   &=t(p(a,b,b), b, b). \nonumber
% \end{align}
\begin{equation}
\label{eq:444}  
  d(a,b,b) = t(p(a,b,b), p(b,b,b), b) =t(p(a,b,b), b, b).
\end{equation}
Next, since $p$ is a \glocal difference term for $S'$,
hence for $(A_i, 0)$,
\begin{equation}
  \label{eq:555}
  t(p(a,b,b), b, b)
 \Comr{\Cg(a,b)}
 t(a, b, b).
\end{equation}
Finally, since $t$ is a \glocal difference term for $T$, hence for
$(A_i, 0)$  $(1\leq i \leq k)$, we have  %% 
$t(a, b, b) \Comr{\Cg(a,b)} a$.
This, (\ref{eq:444}), and (\ref{eq:555}) yield
$d(a,b,b) \Comr{\Cg(a,b)} a$,
as desired.

The indices of the remaining elements of $S$
belong to the set $\{j \mid k<j\leq n\}$ (which is nonempty since we
assumed $\chi_0 = \chi_i = 1$ for some $i>0$).
For such indices we have $\chi_j = 1$.
Thus, fixing $a, b \in A_j$, we check that $d(a,a,b) = b$.
Indeed, $p(a,a,b) = b$, since $p$ is a \glocal difference term for $S'$; 
this, along with idempotence of $t$, yields
$d(a,a,b) =t(p(a,a,b), p(a,a,b), b)=t(b, b, b) =b$.
\end{proof}

\subsection{Proof of Corollary~\ref{cor:glob-loc-diff-term}}
\label{sec:proof-cor:glob}
\begin{corollary} %[\ref{cor:glob-loc-diff-term}]
  Let $\sV$ be a variety.  Let $\scA$ be a collection of finite idempotent
  algebras in $\sV$. %% that is closed under the taking of subalgebras.
  Then there exists a term $d$ that interprets as a difference term operation
  for every algebra in $\scA$
  if and only if each pair $((A,i), (B,j)) \in \scS(\scA)^2$ has a \glocal
  difference term.
\end{corollary}
%% The proof of this result is also easy and very similar to the proof
%% of~\ref{cor:loc-diff-term}; nevertheless, it appears in the appendix
%% (see Section~\ref{sec:proof-thm:glob}).
\begin{proof}
  One direction is clear, since a term that is a difference term operation for
  every $\bA \in \scA$ is obviously a \glocal difference term for
  every $(A, i) \in \scS(\scA)$.
  For the converse, suppose
  each pair in $\scS(\scA)^2$ has a \glocal
  difference term. Then, by Theorem~\ref{thm:glob-loc-diff-terms},
  there is a single term $d$ that is a \glocal difference term for every 
  $(A, i) \in \scS(\scA)$, 
  and therefore $d$ interprets as a difference term operation for every $\bA \in \scA$.
  To see this, choose an arbitrary $\bA = \<A, \dots\> \in \scA$ and fix $a, b \in A$. 
  Then $a \Comr{\Cg(a,b)} d^{\bA}(a,b,b)$,
  since $d$ is a \glocal difference term for $(A,0)$,
  and $d^{\bA}(a,a,b) = b$, since $d$ is a \glocal
  difference term for $(A,1)$. 
  %% Since $a, b$ were arbitrary elements of $A$, we conclude that
  %% $d$ is a difference term operation for $\bA$.
\end{proof}



%%%%%%%%%%%%%%%%%%%%   End of main body of article
%
%                             References
%
%   BiBTeX users uncomment the following line:
%
\bibliographystyle{plainurl} %jloganal}
%
%% \bibliographystyle{ws-ijac}
%% \bibliographystyle{alphaurl}
\bibliography{refs}
%% \begin{thebibliography}
%% \end{thebibliography}

\end{document}



% References must be ordered alphabetically and all have to be cited in the text. Citation should be done using, for instance,  \cite{1}
% Please use the following model. If your references are not in such model, we have to change it, which rises a possibility for a mistake.

\begin{thebibliography}{9}
\bibitem{1}  Bohl, E., Discrete versus continuous models for dissipative
systems. In: Numerical models for bifurcation problems. (T. Kr\"{u}pper,
H.D. Mittelmann, H. Weber, eds.), pp. 68-78. Basel, Boston, Stuttgart:
Birkh\"{a}user Verlag 1984.


\bibitem{dm2}
  Ili\'c, A., Ma\v sulovi\'c, D., Rajkovi\'c, U.,
  Finite homomorphism-homogeneous tournaments with loops.
  Journal of Graph Theory, Vol.\ 59 No.\ 1 (2008), 45--58.

\bibitem{2}  O Reilly, M.J., A uniform scheme for the singularly perturbed
Riccati equation. Numer. Math. 50 (1987), 483-501.

\bibitem{3}  Ortega, J.M., Rheinboldt, W.C., Iterative solution of nonlinear
equations in several variables. New York and London: Academic Press 1970.


\bibitem{4} J\"{o}rg, B., Ferle\v{z}, J., Grabczewski, E., Public IST
World Deliverable 1.2 -- Data Model for Knowledge Organisation.  2005. Available
at: \url
{http://ist-world.dfki.de/downloads/deliverables/ISTWorld_D1.2_DataModelForKnowledgeOrganisation.pdf}
(accessed 4 September 2010)

%Please use \url{} for hyperlinks

\end{thebibliography}



\end{document} 
