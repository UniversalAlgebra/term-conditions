\documentclass[leqno,twoside]{article}
\headsep 0.5cm \pagestyle{myheadings}
\usepackage{amssymb,amsmath,latexsym, amsthm,enumerate, nsjom, amsfonts}
\title{} \author{} \date{}
\markboth{Dragoslav Herceg, Aleksandar Pavlovi\'c}{Sample file}\setcounter{page}{1}
\usepackage[dvipdfm, pdfstartview=FitH]{hyperref} % Use LaTeX and then DVI2PDF,
% or, if you plan to use packages only compatible with PDFLaTeX
% \usepackage[pdftex, pdfstartview=FitH]{hyperref}

\usepackage{graphicx}
% Use this package if you plan to put some pictures

%If you want equations number as (2.1), where, 2 is the section number then use
\numberwithin{equation}{section} %You may omit this line if you want numbering as (1)

 \newtheorem{thm}{Theorem}[section]
 \newtheorem{cor}[thm]{Corollary}
 \newtheorem{lem}[thm]{Lemma}
 \newtheorem{prop}[thm]{Proposition}
 \theoremstyle{definition}
 \newtheorem{defn}[thm]{Definition}
  \newtheorem{ex}[thm]{Example}
 \theoremstyle{remark}
 \newtheorem{rem}[thm]{Remark}

 \usepackage[paperwidth=165mm, paperheight=235mm, twoside, hmargin={25mm,20mm}, vmargin={20mm,20mm} ]{geometry}

% You may use some variations of this environments, for instance, with separate global counters
% \newtheorem{defn}{Definition}
% Also, you may use your own names for environments like
% \newtheorem{df}{Definition}
% or whatever you prefer, but, please, use them

\begin{document}
\thispagestyle{empty}

\begin{flushleft}
\vspace*{-1.1cm} {\sc  Novi Sad J.\ Math.}\\ {\sc Vol.\ XX, No.\ Y,
20ZZ, ??-??}
\end{flushleft}
\vspace{0.8cm}
% PRAVLJENJE NASLOVA
\begin{center}
{\large \bf SAMPLE FILE\footnote{This is one place where you can put acknowledgement}

} \vspace*{3mm}

% Title should be in upper case

{\bf Dragoslav Herceg\footnote{Department of Mathematics and Informatics, Faculty of Science, University of Novi Sad,\\ e-mail: \href{mailto:hercegd@dmi.uns.ac.rs}{hercegd@dmi.uns.ac.rs}} and Aleksandar Pavlovi\'c\footnote{Department of Mathematics and Informatics, Faculty of Science, University of Novi Sad,\\ e-mail: \href{mailto:apavlovic@dmi.uns.ac.rs}{apavlovic@dmi.uns.ac.rs}}}
\end{center}
% Authors should be ordered alphabetically
% Please, use your full name and put an email address in affiliation. The example with underline in e-mail address:
%\href{mailto:my_address@wikibooks.org}{my\_address@wikibooks.org}

\begin{abstract}
Abstract has to be selfcontained and without any references.
\\[2mm] {\it AMS Mathematics  Subject Classification $(2010)$}: 00A11, 55B55
% Please do not use not complete classification like 54Axx
\\[1mm] {\it Key words and phrases:} sample file, NSJOM
% put as many key words as you like

\end{abstract}










\section{Introduction}

This is just a small sample file in which are given few rules and examples for preparing file for publishing in NSJOM.  The idea is to have fully functional PDF file with active hyperlinks, quotations and citations in a document. Therefore, using theorem environments is important. Of course, you may use your own definitions.

We intend in future to put more examples which can make preparation of LaTeX file for NSJOM more easier.


If you have any question or doubts, please do not hesitate to ask:

   \href{mailto:nsjom@dmi.uns.ac.rs}{nsjom@dmi.uns.ac.rs}.


\section{Definitions}

If you are putting references, please do in the following way: \cite{dm2}, or \cite{1,2,3}. Also you may use something like \cite[ch. 3]{3}.

Definitions, theorems, lemmas have to be in the appropriate environments.
\begin{defn}
This is just a sample of a definition.
\end{defn}

You may do something like:
\begin{defn}[\cite{dm2}]
This is just a sample of a definition with citation.
\end{defn}
or
\begin{rem}[Gauss]
Here goes a remark.
\end{rem}


\section{Main results}

\begin{thm}\label{TH001}
After theorem usually goes its proof.
\end{thm}
\begin{proof}
Proof  has its own environment which ends with QED symbol.
\end{proof}


\begin{lem}
There holds
\begin{equation}\label{EQ001}
1+1=2.
\end{equation}
\end{lem}
\begin{proof}
The proof of Equation \eqref{EQ001} is easy.
\end{proof}

Please use labels for easier referring of previous theorems, lemmas and similar. Like

Theorem \ref{TH001} is just an example of \emph{proof} environment.

If you want to use pictures, use it in greyscale:
\begin{figure}[htbp]
\begin{center}
\includegraphics[bb=0 0 645 642, width=5truecm,height=5truecm]{PMF.JPG}
\end{center}
\caption{\small Faculty of Sciences, Novi Sad} \label{PIC01}
\end{figure}


For a JPG file you have to define Bounding Box. For EPS it is within a file.

\section*{Acknowledgement}
This is also a place where you can put acknowledgement.


% References must be ordered alphabetically and all have to be cited in the text. Citation should be done using, for instance,  \cite{1}
% Please use the following model. If your references are not in such model, we have to change it, which rises a possibility for a mistake.

\begin{thebibliography}{9}
\bibitem{1}  Bohl, E., Discrete versus continuous models for dissipative
systems. In: Numerical models for bifurcation problems. (T. Kr\"{u}pper,
H.D. Mittelmann, H. Weber, eds.), pp. 68-78. Basel, Boston, Stuttgart:
Birkh\"{a}user Verlag 1984.


\bibitem{dm2}
  Ili\'c, A., Ma\v sulovi\'c, D., Rajkovi\'c, U.,
  Finite homomorphism-homogeneous tournaments with loops.
  Journal of Graph Theory, Vol.\ 59 No.\ 1 (2008), 45--58.

\bibitem{2}  O Reilly, M.J., A uniform scheme for the singularly perturbed
Riccati equation. Numer. Math. 50 (1987), 483-501.

\bibitem{3}  Ortega, J.M., Rheinboldt, W.C., Iterative solution of nonlinear
equations in several variables. New York and London: Academic Press 1970.


\bibitem{4} J\"{o}rg, B., Ferle\v{z}, J., Grabczewski, E., Public IST
World Deliverable 1.2 -- Data Model for Knowledge Organisation.  2005. Available
at: \url
{http://ist-world.dfki.de/downloads/deliverables/ISTWorld_D1.2_DataModelForKnowledgeOrganisation.pdf}
(accessed 4 September 2010)

%Please use \url{} for hyperlinks

\end{thebibliography}



\end{document} 