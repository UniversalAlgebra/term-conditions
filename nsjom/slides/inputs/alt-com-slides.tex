
\begin{frame}
  \frametitle{What is the commutator?}
  Let $\bA$ be an algebra.

  $\CC{1_A}{1_A}{0_A}$ denotes the following statement:

  $\forall \, t\in \sansClo_{\ell + m}(\bA)$, 
  $\forall \, \ba, \bb\in A^\ell$,
  $\forall \, \bu, \bv\in A^m$,
  \begin{equation}
    \label{eq:210}
    t(\ba,\bu) = t(\ba,\bv)\quad \iff \quad t(\bb,\bu) = t(\bb,\bv),
  \end{equation}
  \pause
  
  We say ``$\CC{1_A}{1_A}{0_A}$ holds'' and write $\com{1_A} = 0_A$.
\end{frame}

\begin{frame}
  \frametitle{What is the commutator?}
  \framesubtitle{the rule $\CC{1_A}{1_A}{0_A}$}
  Let $\Gamma$ be a ``context'' containing
  \[
  \bA \oftype \sV, \qquad 
  t \oftype \sansClo_{\ell + m}(\bA), \qquad
  \ba, \bb \oftype A^\ell, \qquad \bu, \bv\oftype A^m
  \]
  Then the statement ``$\CC{1_A}{1_A}{0_A}$ holds'' can be viewed as a
  \emph{derivation rule}
  \[
  %% \infer[\CC{1_A}{1_A}{0_A}]%
  \infer{\Gamma \vdash  t(\bb,\bu) = t(\bb, \bv)}%
        {\Gamma \vdash  t(\ba,\bu) = t(\ba, \bv)}
  \]
  %% \[
  %% \infer[\CC{1_A}{1_A}{0_A}]%
  %%        {\Gamma \vdash \bigl(\forall \,\bb \oftype A^\ell\bigr) \; t(\bb,\bu) = t(\bb, \bv)}%
  %%        {\Gamma \vdash \bigl(\exists \,\ba \oftype A^\ell\bigr) \; t(\ba,\bu) = t(\ba, \bv)}
  %% \]
\end{frame}


\begin{frame}
  \frametitle{What is the commutator?}
  \framesubtitle{the rule $\CC{S}{T}{0_A}$}
  In a context $\Gamma$ containing
  \begin{gather*}
    \bA  \oftype \sV \qquad      \bu, \bv \oftype A^m   \qquad \ba, \bb \oftype A^\ell \\
    t \oftype \sansClo_{\ell + m}(\bA) \qquad S, T \oftype \Tol(\bA)
  \end{gather*}
  %% \begin{gather*}
  %%   \bA \oftype \sV \qquad 
  %%   t \oftype \sansClo_{\ell + m}(\bA) \qquad
  %%   S, T \oftype \Tol(\bA) \\
  %%   \ba, \bb \oftype A^\ell \qquad
  %%   \bu, \bv \oftype A^m
  %% \end{gather*}
  $\CC{S}{T}{0_A}$ corresponds to the derivation rule
  \[
  %% \infer[\CC{S}{T}{0_A}]%
  \infer{\Gamma \vdash t(\bb,\bu) = t(\bb, \bv)}%
        {\Gamma \vdash \ba \mathrel{\bS} \bb & \Gamma \vdash \bu \mathrel{\bT} \bv & \Gamma \vdash t(\ba,\bu) = t(\ba, \bv)}
  \]
\end{frame}

\begin{frame}
  \frametitle{What is the commutator?}
  \framesubtitle{the rule $\CC{S}{T}{\delta}$}
  In a context $\Gamma$ containing
  %% \begin{alignat*}{3}
  %%   \bA  &\oftype \sV \qquad    &        t &\oftype \sansClo_{\ell + m}(\bA)\\
  %%   \bu, \bv &\oftype A^m \qquad &S, T &\oftype \Tol(\bA) \\
  %%   \ba, \bb &\oftype A^\ell \qquad & \delta &\oftype \Con(\bA)
  %%   \end{alignat*}
  %% \begin{alignat*}{4}
  %%   \bA  &\oftype \sV \qquad    &  \bu, \bv &\oftype A^m   \qquad &\ba, \bb &\oftype A^\ell \\
  %%   t &\oftype \sansClo_{\ell + m}(\bA) \qquad &S, T &\oftype \Tol(\bA)\qquad & \delta &\oftype \Con(\bA)
  %%   \end{alignat*}
  \begin{gather*}
    \bA  \oftype \sV \qquad      \bu, \bv \oftype A^m   \qquad \ba, \bb \oftype A^\ell \\
    t \oftype \sansClo_{\ell + m}(\bA) \qquad S, T \oftype \Tol(\bA)\qquad  \delta \oftype \Con(\bA)
  \end{gather*}
  $\CC{S}{T}{\delta}$ is the rule
  \[
  %% \infer[\CC{S}{T}{\delta}]%
  \infer{\Gamma \vdash t(\bb,\bu) \deltar t(\bb, \bv)}%
        {\Gamma \vdash \ba \mathrel{\bS} \bb & \Gamma \vdash \bu \mathrel{\bT} \bv & \Gamma \vdash t(\ba,\bu) \deltar t(\ba, \bv)} 
        \]
        \pause
        %% The \defn{commutator} $\comm{S}{T}$ is the least $\delta \in \Con \bA$ such that
        %% $\CC{S}{T}{\delta}$ is a rule.
        The \defn{commutator} is the least $\delta$ such that
        $\CC{S}{T}{\delta}$ is a valid rule.
        \pause
        \[\comm{S}{T} = \Meet \{\delta \in \Con \bA \mid \CC{S}{T}{\delta} \text{ holds}\}\]

\end{frame}



\begin{frame}{Definitions: Term Condition, Commutator}
  Suppose $S$ and $T$ are \emph{tolerances} on $\bA$\\[4pt]
  {\small (i.e., $S$ and $T$ are compatible, reflexive, and symmetric)}

  \pause
  An \defin{$S,T$-matrix} is a $2\times 2$ array of the form
  \[
  \begin{bmatrix*}[r] t(\ba,\bu) & t(\ba,\bv)\\ t(\bb,\bu)&t(\bb,\bv)\end{bmatrix*},
  \]
  where 
  \begin{enumerate}[(i)] %[label=(\roman*)]
  \item $t\in \sansClo_{\ell + m}(\bA)$
  \item $(\ba, \bb)\in A^\ell\times A^\ell$ and $\ba \mathrel{\bS} \bb$
    (i.e. $\forall i$ $a_i \mathrel{S} b_i$)
  \item $(\bu, \bv)\in A^m\times A^m$ and $\bu \mathrel{\bT} \bv$.
  \end{enumerate}

  \pause
  For $\delta \in \Con\bA$ if every $S,T$-matrix satisfies
  \begin{equation}
    \label{eq:22}
    t(\ba,\bu) \mathrel{\delta} t(\ba,\bv)\quad \iff \quad t(\bb,\bu) \mathrel{\delta} t(\bb,\bv),
  \end{equation}
  we say \defin{$S$ centralizes $T$ modulo $\delta$} and write 
  $\CC{S}{T}{\delta}$.

  That is, $\CC{S}{T}{\delta}$  means
  (\ref{eq:22}) holds \emph{for all}
  $\ell$, $m$, $t$, $\ba$, $\bb$, $\bu$, $\bv$ satisfying (i)--(iii).
\end{frame}









\begin{comment}










%--------------------------------------------------------------------
%% \subsection{Definitions}

\begin{frame}{Definitions: Term Condition, Commutator}
  Suppose $S$ and $T$ are \emph{tolerances} on $\bA$\\[4pt]
  {\small (i.e., $S$ and $T$ are compatible, reflexive, and symmetric)}

  \pause
  An \defin{$S,T$-matrix} is a $2\times 2$ array of the form
  \[
  \begin{bmatrix*}[r] t(\ba,\bu) & t(\ba,\bv)\\ t(\bb,\bu)&t(\bb,\bv)\end{bmatrix*},
  \]
  where 
  \begin{enumerate}[(i)] %[label=(\roman*)]
  \item $t\in \sansClo_{\ell + m}(\bA)$
  \item $(\ba, \bb)\in A^\ell\times A^\ell$ and $\ba \mathrel{\bS} \bb$
    (i.e. $\forall i$ $a_i \mathrel{S} b_i$)
  \item $(\bu, \bv)\in A^m\times A^m$ and $\bu \mathrel{\bT} \bv$.
  \end{enumerate}

  \pause
  For $\delta \in \Con\bA$ if every $S,T$-matrix satisfies
  \begin{equation}
    \label{eq:22}
    t(\ba,\bu) \mathrel{\delta} t(\ba,\bv)\quad \iff \quad t(\bb,\bu) \mathrel{\delta} t(\bb,\bv),
  \end{equation}
  we say \defin{$S$ centralizes $T$ modulo $\delta$} and write 
  $\CC{S}{T}{\delta}$.

  That is, $\CC{S}{T}{\delta}$  means
  (\ref{eq:22}) holds \emph{for all}
  $\ell$, $m$, $t$, $\ba$, $\bb$, $\bu$, $\bv$ satisfying (i)--(iii).
\end{frame}


\begin{frame}{Definitions: Term Condition, Commutator}
  The \defin{commutator} $[S, T]$ is the least
  $\delta$ such that $\CC{S}{T}{\delta}$.

  The \defin{$S, T$-term condition} is $\CC{S}{T}{0_A}$ (i.e., $[S,T] = 0_A$)

  A tolerance $T$ is called \defin{abelian} if $[T, T] = 0_A$.  

  An algebra $\bA$ is called \defin{abelian} if $1_A$ is abelian
  (i.e., $[1_A,1_A] = 0_A$).

\end{frame}
\end{comment}
