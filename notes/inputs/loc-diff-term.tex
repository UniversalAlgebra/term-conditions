
For the most part we use standard notation, definitions, and
results of universal algebra, such as those found in~\cite{MR2839398}.
However, we make a few exceptions for notational simplicity.
For example, if $\bA =\<A, \dots\>$ is an algebra with elements 
$a, b \in A$, then we use $\theta(a,b)$ to denote
the congruence of $\bA$ generated by $a$ and $b$.

Let $\bA=\< A, \dots\>$ be an algebra, fix $a, b \in A$ and
$i \in \{0,1\}$.
%% An \defn{$\bA$-local difference term for
A \defn{local difference term for
  $(a,b,i)$} is a ternary term $p$ satisfying the following:
\begin{align}
%% \text{ if $i=0$, then } & a \comm{\Cg^{\bA}(a,b)}{\Cg^{\bA}(a,b)} p(a,b,b); \label{eq:diff-triple}\\
\text{ if $i=0$, then } & a \comm{\Cg(a,b)}{\Cg(a,b)} p(a,b,b); \label{eq:diff-triple}\\
\text{ if $i=1$, then } &p(a,a,b) = b. \nonumber
\end{align}
%% We often drop the
%% $\bA$ when the algebra is clear from context.
%% For example, we write $\Cg$ in place of $\Cg^{\bA}$, and 
%% call the term $p$ above a local difference term for 
%% $(a,b,i)$.
If $p$ satisfies~(\ref{eq:diff-triple}) for all triples
in some subset $S\subseteq A \times A \times \{0,1\}$, then we call $p$
a \defn{local difference term for $S$}.

Let 
$\sS = A \times A \times \{0,1\}$ and
suppose that every pair
$((a_0, b_0, \chi_0), (a_1, b_1, \chi_1))$
in $\sS^2$ has a local difference term.
That is, for each pair $((a_0, b_0, \chi_0), (a_1, b_1, \chi_1))$, there exists
$p$ such that for each $i \in \{0,1\}$ we have
\begin{align}
  a_i \comm{\Cg(a_i,b_i)}{\Cg(a_i,b_i)} p(a_i,b_i,b_i), & \;
  \text{ if $\chi_i=0$, and }  \label{eq:d-trip-i1}\\
  p(a_i,a_i,b_i) =b_i, & \;
  \text{ if $\chi_i=1$.}\label{eq:d-trip-i2} %\\\nonumber
\end{align}
Under these hypothesis we will prove that every subset $S\subseteq \sS$
has a local difference term.
That is, there is a single term $p$ that works (i.e., satisfies
(\ref{eq:d-trip-i1}) and (\ref{eq:d-trip-i2})) for all $(a_i, b_i, \chi_i) \in S$.
The statement and proof of this new result follows.

\begin{thm}[\protect{cf.~\cite[Theorem 2.2]{MR3239624}}]
  \label{thm:local-diff-terms}
  Let $\sV$ be an idempotent variety and
  $\bA \in \sV$. Define
  $\sS= A \times A \times \{0,1\}$
  and suppose that every pair
  $((a_0, b_0, \chi_0), (a_1, b_1, \chi_1)) \in \sS^2$
  has a local difference term.
  Then every subset $S \subseteq \sS$,
  has a local difference term.
\end{thm}
\begin{proof}
The proof is by induction on the size of $S$.  In the base case, $|S| = 2$,
the claim holds by assumption.
Fix $n>2$ and assume that every subset of $\sS$ of size $2\leq k \leq n$ has a local
difference term. Let
$S = \{(a_0, b_0, \chi_0), (a_1, b_1, \chi_1), \dots, (a_{n}, b_{n},\chi_{n})\} \subseteq \sS$,
so that $|S| = n+1$.  We prove $S$ has a local difference term.

Since $|S| \geq 3$ and $\chi_i \in \{0,1\}$ for all $i$, there must exist
indices $i\neq j$ such that $\chi_i = \chi_j$. Assume without loss of generality
that one of these indices is $j=0$.  Define
the set
$S' = S \setminus \{(a_0, b_0, \chi_0)\}$.
Since $|S'| < |S|$, the set $S'$ has a local difference term $p$.
We split the remainder of the proof into two cases. In the first case
$\chi_0 = 0$ and in the second
$\chi_0 = 1$.

\vskip3mm

%--------------------------------------
\noindent \underline{Case 1:} $\chi_0 = 0$.
%% \\[4pt]
%% Assume $\chi_0 = 0$ and, 
%% w
Without loss of generality, suppose that $\chi_1 = %% \chi_2 =
\cdots =\chi_k = 1$,
and $\chi_{k+1} %% = \chi_{k+2} 
= \cdots = \chi_{n} = 0$. Define %% $T$ to be the set
$T = \{(a_0, p(a_0, b_0, b_0), 0),
(a_1, b_1, 1), (a_2, b_2, 1), 
\dots, (a_k, b_k, 1)\}$, and 
note that $|T| < |S|$.
Let $t$ be a local difference term for $T$.
Define
\[
d(x,y,z) = t(x, p(x,y,y), p(x,y,z)).
\]
Since $\chi_0 =0$, we need to show
$(a_0, d(a_0,b_0,b_0))$ belongs to $\comm{\Cg(a_0,b_0)}{\Cg(a_0,b_0)}$.
We have
\begin{equation}
    \label{eq:100000}
  d(a_0,b_0,b_0) =
  t(a_0, p(a_0,b_0,b_0), p(a_0,b_0,b_0))\comm{\tau}{\tau} a_0,
\end{equation}
where we have used $\tau$ to denote $\Cg(a_0, p(a_0,b_0,b_0))$.
Note that
%% \[(a_0, p(a_0,b_0,b_0)) = (p(a_0,a_0,a_0), p(a_0,b_0,b_0)) \in \Cg(a_0, b_0),\]
$(a_0, p(a_0,b_0,b_0)) = (p(a_0,a_0,a_0), p(a_0,b_0,b_0))$
belongs to $\Cg(a_0, b_0)$,
so $\tau\leq \Cg(a_0,b_0)$. Therefore,
by monotonicity of the commutator,
$\comm{\tau}{\tau} {\leq} \comm{\Cg(a_0,b_0)}{\Cg(a_0,b_0)}$.
It follows from this and (\ref{eq:100000}) that
%% $d(a_0,b_0,b_0)\comm{\Cg(a_0,b_0)}{\Cg(a_0,b_0)} a_0$,
\[d(a_0,b_0,b_0)\comm{\Cg(a_0,b_0)}{\Cg(a_0,b_0)} a_0,\]
as desired.

For the indices $1\leq i \leq k$ we have $\chi_i =1$, so we wish to prove
$d(a_i,a_i,b_i) = b_i$ for such $i$. Observe,
\begin{align}
  d(a_i,a_i,b_i) &=
  t(a_i, p(a_i,a_i,a_i), p(a_i,a_i,b_i)) \label{eq:200000}\\
  &=t(a_i, a_i, b_i) \label{eq:200001}\\
  &=b_i. \label{eq:200002}
\end{align}
Equation~(\ref{eq:200000}) holds by definition of $d$,~(\ref{eq:200001})
because $p$ is an idempotent local difference term for
$S'$, and~(\ref{eq:200002}) because $t$ is a local difference term for $T$.

The remaining triples in our original set $S$
have indices satisfying $k<j\leq n$ and $\chi_j = 0$.
Thus, for these triples we want
$d(a_j,b_j,b_j)\comm{\Cg(a_j,b_j)}{\Cg(a_j,b_j)} a_j$.
By definition,
\begin{equation}
  \label{eq:450000}
d(a_j,b_j,b_j) =t(a_j, p(a_j,b_j,b_j), p(a_j,b_j,b_j)).  
\end{equation}
Since $p$ is a local difference term for $S'$, we have
%% the pair $(p(a_j,b_j,b_j), a_j)$ belongs to $[\Cg(a_j,b_j), \Cg(a_j,b_j)]$.
\[
(p(a_j,b_j,b_j), a_j)\in [\Cg(a_j,b_j), \Cg(a_j,b_j)].
\]
This and 
(\ref{eq:450000}) imply
%% \begin{equation*}
%%     %% \label{eq:300000}
%%   d(a_j,b_j,b_j) =
%%   t(a_j, p(a_j,b_j,b_j), p(a_j,b_j,b_j))
%%   \comm{\Cg(a_j,b_j)}{\Cg(a_j,b_j)} t(a_j,a_j,a_j) = a_j,
%% \end{equation*}
that 
%% $(t(a_j, p(a_j,b_j,b_j), p(a_j,b_j,b_j)), t(a_j,a_j,a_j))$
$(d(a_j, b_j,b_j), t(a_j,a_j,a_j))$
belongs to
$\comm{\Cg(a_j,b_j)}{\Cg(a_j,b_j)}$.
Finally, by idempotence of $t$ we have
$d(a_j,b_j,b_j)\comm{\Cg(a_j,b_j)}{\Cg(a_j,b_j)} a_j$,
%% \begin{align*}
%%     %% \label{eq:300000}
%%   d(a_j,b_j,b_j) &=
%%   t(a_j, p(a_j,b_j,b_j), p(a_j,b_j,b_j))\\
%%   &\comm{\Cg(a_j,b_j)}{\Cg(a_j,b_j)} t(a_j,a_j,a_j)\\
%%   &= a_j,
%% \end{align*}
as desired.
\\[6pt]
%--------------------------------------
\underline{Case 2:}
$\chi_0 = 1$.
%% \\[4pt]
%% Assume $\chi_0 = 1$ and, 
Without loss of generality, suppose $\chi_1 = \chi_2 =\cdots =\chi_k = 0$,
and $\chi_{k+1} = \chi_{k+2} = \cdots = \chi_{n} = 1$. Define $T$ to be the set
\[
\{(p(a_0, a_0, b_0), b_0, 1),
(a_1, b_1, 0), (a_2, b_2 0), \dots, (a_k, b_k, 0)\},
\]
and note that $|T| < |S|$.
Let $t$ be a local difference term for $T$ and
define
%% \[d(x,y,z) = t(p(x,y,z), p(y,y,z), z).\] 
$d(x,y,z) = t(p(x,y,z), p(y,y,z), z)$. 
Since $\chi_0 =1$, we want $d(a_0,a_0,b_0) = b_0$. By the definition of
$d$,
\begin{equation*}
  d(a_0,a_0,b_0) =
  t(p(a_0,a_0,b_0), p(a_0,a_0,b_0), b_0) =b_0.
\end{equation*}
The last equality holds since $t$ is a local difference term for $T$, thus,
for $(p(a_0, a_0, b_0), b_0, 1)$.

If $1\leq i \leq k$, then $\chi_i =0$, so for these indices we want
$d(a_i,b_i,b_i) \comm{\Cg(a_i,b_i)}{\Cg(a_i,b_i)} a_i$.
Again, starting from the definition of $d$ and using idempotence of $p$, we have
%% \begin{equation}
%%   \label{eq:40000}
%%   d(a_i,b_i,b_i) =
%%   t(p(a_i,b_i,b_i), p(b_i,b_i,b_i), b_i)=
%%   t(p(a_i,b_i,b_i), b_i, b_i).
%% \end{equation}
\begin{align}
  d(a_i,b_i,b_i) &=
  t(p(a_i,b_i,b_i), p(b_i,b_i,b_i), b_i)   \label{eq:40000}\\
  &=t(p(a_i,b_i,b_i), b_i, b_i). \nonumber
\end{align}
Next, since $p$ is a local difference term for $S'$, we have
\begin{equation}
  \label{eq:50000}
  t(p(a_i,b_i,b_i), b_i, b_i)
 \comm{\theta(a_i,b_i)}{\theta(a_i,b_i)}
 t(a_i, b_i, b_i).
\end{equation}
Finally, since $t$ is a local difference term for $T$, hence for
$(a_i, b_i, b_i)$,  %% $(1\leq i \leq k)$,
we have 
$t(a_i, b_i, b_i) \comm{\Cg(a_i,b_i)}{\Cg(a_i,b_i)} a_i$.
Combining this with (\ref{eq:40000}) and (\ref{eq:50000}) yields
\[
d(a_i,b_i,b_i) \comm{\Cg(a_i,b_i)}{\Cg(a_i,b_i)} a_i,
\]
as desired.

The remaining elements of our original set $S$
have indices $j$ satisfying $k<j\leq n$ and $\chi_j = 1$.
For these we want $d(a_j,a_j,b_j) = b_j$.
Since $p$ is a local difference term for $S'$, we have
$p(a_j,a_j,b_j) = b_j$, and this along with idempotence of $t$ yields
%%\[ d(a_j,a_j,b_j) =  t(p(a_j,a_j,b_j), p(a_j,a_j,b_j), b_j)=  t(b_j, b_j, b_j) =b_j,\]
\begin{align*}
d(a_j,a_j,b_j) &=
t(p(a_j,a_j,b_j), p(a_j,a_j,b_j), b_j)\\
&=t(b_j, b_j, b_j) =b_j,
\end{align*}
as desired.
\end{proof}

\begin{cor}
  \label{cor:loc-diff-term}
  A finite idempotent algebra $\bA$ has a difference term operation if and
  only if every pair $((a,b,i), (a',b',i')) \in (A\times A \times \{0,1\})^2$ has a local
  difference term.
\end{cor}
\begin{proof}
  One direction is clear, since a difference term operation for $\bA$ is
  obviously a local difference term for the whole set 
  $A\times A \times \{0,1\}$.
  For the converse, suppose
  each pair in $(A\times A \times \{0,1\})^2$ has a local
  difference term. Then, by Theorem~\ref{thm:local-diff-terms},
  there is a single local difference term for the whole set $A\times A \times \{0,1\}$,
  and this is a difference term operation for $\bA$.  Indeed, if $d$ is a
  local difference term for $A\times A \times \{0,1\}$, then 
  for all $a, b \in A$, we have
  $a \comm{\Cg(a,b)}{\Cg(a,b)} d(a,b,b)$,
  since $d$ is a local difference term for $(a,b,0)$, and we have
  $d(a,a,b) = b$, since $d$ is also a local difference term for
  $(a,b,1)$.
\end{proof}

\begin{cor}
  If $\sV$ is a variety generated by a finite idempotent algebra, then
  there is a polynomial-time algorithm for deciding
  whether or not $\sV$ has a difference term.
\end{cor}
\begin{proof}
  Let $\bA$ be a finite idempotent algebra and let
  $\sV = \bbV(\bA)$.
  We describe a polynomial-time algorithm for deciding whether
  the hypothesis of Corollary~\ref{cor:loc-diff-term} holds for $\bA$, thereby
  proving that we can decide in polynomial-time whether
  there is a difference term  operation for $\bA$. We will then complete the
  proof by explaining why $\bA$ has a difference term operation iff the variety
  it generates has a difference term. 

  Fix a pair
  $((a,b,i), (a',b',i'))$ in $(A\times A \times \{0,1\})^2$. If $i = i' = 0$,
  then the first projection is a local difference term. If $i = i' = 1$,  
    then the third projection is a local difference term. The two remaining cases to
    consider are (1) $i = 0$ and $i'=1$, and (2)
    $i = 1$ and $i'=0$. Since these are completely symmetric, we only handle the
    first case. Assume  the given pair of triples is
    $((a,b,0), (a',b',1))$.  By definition, a term $t$ is local difference term
    for this pair iff
    \[
    a\comm{\Cg(a,b)}{\Cg(a,b)} t^{\bA}(a,b,b) \; \text{ and } \;
    t^{\bA}(a',a',b') = b'.
    \]
    We can rewrite this condition more compactly by %% using the triple
    %% $((a,a'), (b,a'), (b,b'))$ in $(A\times A)^3$, and 
    considering 
    $t^{\bA\times \bA}((a,a'), (b,a'), (b,b')) =
    (t^{\bA}(a,b,b),t^{\bA}(a',a',b'))$.
    Clearly $t$ is a local difference term for
    $((a,b,0), (a',b',1))$ iff
    \[
    t^{\bA\times \bA}((a,a'), (b,a'), (b,b'))\in a/\delta \times \{b'\},
    \]
    where $\delta = [\Cg(a,b), \Cg(a,b)]$ and $a/\delta$ denotes the
    $\delta$-class containing $a$.
    (Observe that $a/\delta \times \{b'\}$ is a subalgebra of $\bA \times \bA$
    by idempotence.)
    It follows that the pair
    $((a,b,0), (a',b',1))$ has a local difference term iff
    the subuniverse of $\bA\times \bA$ generated by
    $\{(a,a'), (b,a'), (b,b')\}$ intersects nontrivially with the subuniverse
    $a/\delta \times \{b'\}$.

    Thus, the algorithm takes as input $\bA$ and, for each triple
    $((a,a'), (b,a'), (b,b'))$ in $(A\times A)^3$, computes
    $\delta = [\Cg(a,b), \Cg(a,b)]$, computes the subalgebra
    $\bS$ of $\bA\times \bA$ generated by
    %% \Sg^{\bA\times \bA}\{(a,a'), (b,a'), (b,b')\}$, and then
    $\{(a,a'), (b,a'), (b,b')\}$, and then
    tests whether $S \cap (a/\delta \times \{b'\})$ is empty.
    If we find an empty intersection at any point, then the algorithm returns
    the answer ``no difference term operation.'' Otherwise,
    $\bA$ has a difference term operation.

    Finally, we observe that if $\bA$ has a difference term operation, then the
    variety it generates has a difference term.
\end{proof}
    TODO: justify the last sentence of the last proof.

