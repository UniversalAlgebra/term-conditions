\documentclass[11pt]{amsart}
% The following \documentclass options may be useful:
% preprint      Remove this option only once the paper is in final form.
% 10pt          To set in 10-point type instead of 9-point.
% 11pt          To set in 11-point type instead of 9-point.
% numbers       To obtain numeric citation style instead of author/year.

%% \usepackage{setspace}\onehalfspacing

\usepackage{amsmath}
\usepackage{amscd,amssymb,amsthm} %, amsmath are included by default
\usepackage{latexsym,stmaryrd,mathrsfs,enumerate,scalefnt,ifthen}
\usepackage{mathtools}
\usepackage[mathcal]{euscript}
\usepackage[colorlinks=true,urlcolor=black,linkcolor=black,citecolor=black]{hyperref}
\usepackage{url}
\usepackage{scalefnt}
\usepackage{tikz}
\usepackage{color}
\usepackage[margin=1in]{geometry}
\usepackage{scrextend}

%%////////////////////////////////////////////////////////////////////////////////
%% Theorem styles
\numberwithin{equation}{section}
\theoremstyle{plain}
\newtheorem{theorem}{Theorem}[section]
\newtheorem{lemma}[theorem]{Lemma}
\newtheorem{proposition}[theorem]{Proposition}
\newtheorem{prop}[theorem]{Proposition}
\theoremstyle{definition}
\newtheorem{claim}[theorem]{Claim}
\newtheorem{corollary}[theorem]{Corollary}
\newtheorem{definition}[theorem]{Definition}
\newtheorem{notation}[theorem]{Notation}
\newtheorem{Fact}[theorem]{Fact}
\newtheorem*{fact}{Fact}
\newtheorem{example}[theorem]{Example}
\newtheorem{examples}[theorem]{Examples}
\newtheorem{exercise}{Exercise}
\newtheorem*{lem}{Lemma}
\newtheorem*{cor}{Corollary}
\newtheorem*{remark}{Remark}
\newtheorem*{remarks}{Remarks}
\newtheorem*{obs}{Observation}


%%%%%%%%%%%%%%%%%%%%%%%%%%%%%%%%%%%%%%%%
% Acronyms
%%%%%%%%%%%%%%%%%%%%%%%%%%%%%%%%%%%%%%%%
%% \usepackage[acronym, shortcuts]{glossaries}
%\usepackage[smaller]{acro}
\usepackage[smaller]{acronym}
\usepackage{xspace}

%% \acs{CSP} -- short version of the acronym\\
%% \acl{CSP} -- expanded acronym without mentioning the acronym.\\
%% \acp{CSP} -- plurals.\\
%% \acfp{CSP} -- long forms into plurals.\\
%% \acsp{CSP} -- short form into a plural.\\
%% \aclp{CSP} -- long form into a plural.\\
%% \acfi{CSP} -- Full Name acronym in italics and abbreviated form in upshape.\\
%% \acsu{CSP} -- short form of the acronym and marks it as used.\\
%% \aclu{CSP} -- Prints the long form of the acronym and marks it as used.\\

\acrodef{lics}[LICS]{Logic in Computer Science}
\acrodef{sat}[SAT]{satisfiability}
\acrodef{nae}[NAE]{not-all-equal}
\acrodef{ctb}[CTB]{cube term blocker}
\acrodef{tct}[TCT]{tame congruence theory}
\acrodef{wnu}[WNU]{weak near-unanimity}
\acrodef{CSP}[CSP]{constraint satisfaction problem}
\acrodef{MAS}[MAS]{minimal absorbing subuniverse}
\acrodef{MA}[MA]{minimal absorbing}
\acrodef{cib}[CIB]{commutative idempotent binar}
\acrodef{sd}[SD]{semidistributive}
\acrodef{NP}[NP]{nondeterministic polynomial time}
\acrodef{P}[P]{polynomial time}
\acrodef{PeqNP}[P $ = $ NP]{P is NP}
\acrodef{PneqNP}[P $ \neq $ NP]{P is not NP}

%%%%%%%%%%%%%%%%%%%%%%%%%%%%%%%%%%%%%%%%%%%%%%%%%%%%%%%%%%%%%%%%%

\usepackage{inputs/proof-dashed}


%%%%%%%%%%%%%%%%%%%%%%%%%%%%%%%%%%%%%%%%%%%%%%%%%%%%%%%%%%%%%%%%%

%% Put new macros in the macros.sty file
\usepackage{inputs/macros}

\usepackage[backend=bibtex]{biblatex}
\bibliography{inputs/refs.bib}

\begin{document}

\title{Local Difference Terms}
\date{\today}
%% \author[W.~DeMeo]{William DeMeo}
\address{University of Hawaii}
\email{ralph@math.hawaii.edu}
\email{williamdemeo@gmail.com}

%% \thanks{The authors would like to extend special thanks to...}

\maketitle

%% \begin{abstract}\end{abstract}


%% \section{Local difference terms}
Let $\bp = (p_0, p_1, \dots, p_n)$ be an $(n+1)$-tuple of ternary terms, where
$p_0(x,y,z) \approx x$ and $p_n(x,y,z) \approx z$, the first and third
ternary projections, respectively. 
Let $\bA=\< A, \dots\>$ be an algebra.
Following Willard and Valeriote, we define an \defn{$\bA$-triple for $\bp$}
to be a triple $(a,b,i)$ such that $a, b \in A$ and
$p_i(a,b,b) = p_{i+1}(a,a,b)$.

Below we will use  $\pi^3_0$ and $\pi^3_2$
to denote the first and third ternary projections; that is,
$\pi^3_0(x,y,z) \approx x$ and $\pi^3_2(x,y,z) \approx z$.

Consider the following modification of this idea.  Let $\bp = (p_0, p_1, p_2)$
be triple of ternary terms. Define an
\defn{$\bA$-difference-triple for
$(p_0, p_1, p_2)$} %(or more simply for $p$)
  %% $(\pi^3_0, p, \pi^3_2)$} (or more simply for $p$)
to be a triple $(a,b,i)$ where $i$ is either 0 or 1 and 
\begin{align}
\text{ if $i=0$, then } & p_0(a,b,b) \comm{\Cg(a,b)}{\Cg(a,b)} p_1(a,b,b); \label{eq:diff-trip-1}\\
\text{ if $i=1$, then } &p_1(a,a,b) = p_2(a,a,b). \label{eq:diff-trip-2}
\end{align}
We intend to apply this definition in the special case where
$p_0 = \pi^3_0$ and
$p_2 = \pi^3_2$ are the first and third projections, respectively, in which case
the conditions (\ref{eq:diff-trip-1}) and (\ref{eq:diff-trip-2}) become
\begin{align*}
\text{ if $i=0$, then } &a \comm{\Cg(a,b)}{\Cg(a,b)} p(a,b,b); \\
\text{ if $i=1$, then } &p(a,a,b) =b.
\end{align*}

Let $\sS$ denote the set of all 2-element sets of triples of the form
$\{(a_0, b_0, \chi_0), (a_1, b_1, \chi_1)\}$, such that
$a_i, b_i \in A$ and $\chi_i \in \{0,1\}$.
That is,
\[
\sS= \{\{(a_0, b_0, \chi_0), (a_1, b_1, \chi_1)\} \mid
a_i, b_i \in A \; \text{ and } \; \chi_i \in \{0,1\}\}.
\]
Suppose that for each
$S = \{(a_0, b_0, \chi_0), (a_1, b_1, \chi_1)\} \in \sS$ there exists a term $p$
for which $S$ is a pair of $\bA$-difference-triples. That is, 
\begin{align*}
\text{ if $\chi_i=0$, then } &a_i \comm{\Cg(a_i,b_i)}{\Cg(a_i,b_i)} p(a_i,b_i,b_i); \\
\text{ if $\chi_i=1$, then } &p(a_i,a_i,b_i) =b_i.
\end{align*}
We prove that under this hypothesis, for any set
$S = \{(a_0, b_0, \chi_0), (a_1, b_1, \chi_1), \dots, (a_{n-1}, b_{n-1},\chi_{n-1})\}$
such that $a_i, b_i \in A$ and $\chi_i \in \{0,1\}$, there exists a term $p$
such that every member of $S$ is an $\bA$-difference-triple for $p$.
%% \appendix
%% \section{Appendix Title}
%% This is the text of the appendix, if you need one.

%\bibliographystyle{amsplain} %% or amsalpha
%% \bibliographystyle{plain-url}
\printbibliography


\end{document}
