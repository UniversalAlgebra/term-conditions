%% FILE: alg-csp-cib.tex
%% AUTHORS: Clifford Bergman, William DeMeo
%% DATE: 25 July 2016
%% COPYRIGHT: (C) 2016 Clifford Bergman, William DeMeo,

%%%%%%%%%%%%%%%%%%%%%%%%%%%%%%%%%%%%%%%%%%%%%%%%%%%%%%%%%%%%%%%%%%%%%%%%%%%%%%%
%%                         BIBLIOGRAPHY FILE                                 %%
%%%%%%%%%%%%%%%%%%%%%%%%%%%%%%%%%%%%%%%%%%%%%%%%%%%%%%%%%%%%%%%%%%%%%%%%%%%%%%%
%% The `filecontents` command will crete a file in the inputs directory called 
%% refs.bib containing the references in the document, in case this file does 
%% not exist already.
%% If you want to add a BibTeX entry, please don't add it directly to the
%% refs.bib file.  Instead, add it in this file between the
%% \begin{filecontents*}{refs.bib} and \end{filecontents*} lines
%% then delete the existing refs.bib file so it will be automatically generated 
%% again with your new entry the next time you run pdfaltex.
\begin{filecontents*}{inputs/refs2.bib}
  @article {MR2893395,
    AUTHOR = {Barto, Libor and Kozik, Marcin},
    TITLE = {Absorbing subalgebras, cyclic terms, and the constraint
      satisfaction problem},
    JOURNAL = {Log. Methods Comput. Sci.},
    FJOURNAL = {Logical Methods in Computer Science},
    VOLUME = {8},
    YEAR = {2012},
    NUMBER = {1},
    PAGES = {1:07, 27},
    ISSN = {1860-5974},
    MRCLASS = {68Q17 (08A70)},
    MRNUMBER = {2893395},
    DOI = {10.2168/LMCS-8(1:7)2012},
    URL = {http://dx.doi.org/10.2168/LMCS-8(1:7)2012}
  }
  @article {MR3374664,
    AUTHOR = {Barto, Libor and Kozik, Marcin and Stanovsk{\'y}, David},
    TITLE = {Mal'tsev conditions, lack of absorption, and solvability},
    JOURNAL = {Algebra Universalis},
    FJOURNAL = {Algebra Universalis},
    VOLUME = {74},
    YEAR = {2015},
    NUMBER = {1-2},
    PAGES = {185--206},
    ISSN = {0002-5240},
    MRCLASS = {08B05 (08A05)},
    MRNUMBER = {3374664},
    DOI = {10.1007/s00012-015-0338-z},
    URL = {http://dx.doi.org/10.1007/s00012-015-0338-z},
  }  
}
\end{filecontents*}
%:biblio
\documentclass[11pt]{amsart}
% The following \documentclass options may be useful:
% preprint      Remove this option only once the paper is in final form.
% 10pt          To set in 10-point type instead of 9-point.
% 11pt          To set in 11-point type instead of 9-point.
% numbers       To obtain numeric citation style instead of author/year.

%% \usepackage{setspace}\onehalfspacing

\usepackage{amsmath}
\usepackage{amscd,amssymb,amsthm} %, amsmath are included by default
\usepackage{latexsym,stmaryrd,mathrsfs,enumerate,scalefnt,ifthen}
\usepackage{mathtools}
\usepackage[mathcal]{euscript}
\usepackage[colorlinks=true,urlcolor=black,linkcolor=black,citecolor=black]{hyperref}
\usepackage{url}
\usepackage{scalefnt}
\usepackage{tikz}
\usepackage{color}
\usepackage[margin=1in]{geometry}
\usepackage{scrextend}

%%////////////////////////////////////////////////////////////////////////////////
%% Theorem styles
\numberwithin{equation}{section}
\theoremstyle{plain}
\newtheorem{theorem}{Theorem}[section]
\newtheorem{lemma}[theorem]{Lemma}
\newtheorem{proposition}[theorem]{Proposition}
\newtheorem{prop}[theorem]{Proposition}
\theoremstyle{definition}
\newtheorem{claim}[theorem]{Claim}
\newtheorem{corollary}[theorem]{Corollary}
\newtheorem{definition}[theorem]{Definition}
\newtheorem{notation}[theorem]{Notation}
\newtheorem{Fact}[theorem]{Fact}
\newtheorem*{fact}{Fact}
\newtheorem{example}[theorem]{Example}
\newtheorem{examples}[theorem]{Examples}
\newtheorem{exercise}{Exercise}
\newtheorem*{lem}{Lemma}
\newtheorem*{cor}{Corollary}
\newtheorem*{remark}{Remark}
\newtheorem*{remarks}{Remarks}
\newtheorem*{obs}{Observation}


%%%%%%%%%%%%%%%%%%%%%%%%%%%%%%%%%%%%%%%%
% Acronyms
%%%%%%%%%%%%%%%%%%%%%%%%%%%%%%%%%%%%%%%%
%% \usepackage[acronym, shortcuts]{glossaries}
%\usepackage[smaller]{acro}
\usepackage[smaller]{acronym}
\usepackage{xspace}

%% \acs{CSP} -- short version of the acronym\\
%% \acl{CSP} -- expanded acronym without mentioning the acronym.\\
%% \acp{CSP} -- plurals.\\
%% \acfp{CSP} -- long forms into plurals.\\
%% \acsp{CSP} -- short form into a plural.\\
%% \aclp{CSP} -- long form into a plural.\\
%% \acfi{CSP} -- Full Name acronym in italics and abbreviated form in upshape.\\
%% \acsu{CSP} -- short form of the acronym and marks it as used.\\
%% \aclu{CSP} -- Prints the long form of the acronym and marks it as used.\\

\acrodef{lics}[LICS]{Logic in Computer Science}
\acrodef{sat}[SAT]{satisfiability}
\acrodef{nae}[NAE]{not-all-equal}
\acrodef{ctb}[CTB]{cube term blocker}
\acrodef{tct}[TCT]{tame congruence theory}
\acrodef{wnu}[WNU]{weak near-unanimity}
\acrodef{CSP}[CSP]{constraint satisfaction problem}
\acrodef{MAS}[MAS]{minimal absorbing subuniverse}
\acrodef{MA}[MA]{minimal absorbing}
\acrodef{cib}[CIB]{commutative idempotent binar}
\acrodef{sd}[SD]{semidistributive}
\acrodef{NP}[NP]{nondeterministic polynomial time}
\acrodef{P}[P]{polynomial time}
\acrodef{PeqNP}[P $ = $ NP]{P is NP}
\acrodef{PneqNP}[P $ \neq $ NP]{P is not NP}

%%%%%%%%%%%%%%%%%%%%%%%%%%%%%%%%%%%%%%%%%%%%%%%%%%%%%%%%%%%%%%%%%

%% \usepackage{inputs/proof-dashed}


%%%%%%%%%%%%%%%%%%%%%%%%%%%%%%%%%%%%%%%%%%%%%%%%%%%%%%%%%%%%%%%%%

%% Put new macros in the macros.sty file
\usepackage{inputs/macros}

\begin{document}

\title[Product Congruences]{Product Congruences in Idempotent Varieties}
\date{\today}
%% \author[W.~DeMeo]{William DeMeo}
\address{University of Hawaii}
\email{williamdemeo@gmail.com}

%% \thanks{The authors would like to extend special thanks to...}

\maketitle

%% \begin{abstract}\end{abstract}

%% \section{Introduction}
%% \label{sec:introduction}
\section{Skew congruences of idempotent algebras}
Let $\sV$ be a variety and let $\bA$ and $\bB$ be idempotent
algebras in $\sV$.
Recall the following standard notation:
if $\alpha \in \Con(\bA)$ and $\beta \in \Con(\bB)$, then
$\alpha \times \beta$ denotes the set of pairs $((a,b),(a',b'))$ satisfying
$a \mathrel{\alpha} a'$ and $b \mathrel{\beta} b'$.  The relation 
$\alpha \times \beta$ is clearly a congruence of $\bA \times \bB$.

Fix $(a, b)$ and $(a', b')$ in $A \times B$.
We claim that
\[\Cg^{\bA}(a,a') \times \Cg^{\bB}(b,b')
=\Cg^{\bA \times \bB}((a, b), (a', b')).\]

Let $\tau := \Cg^{\bA}(a,a') \times \Cg^{\bB}(b,b')$  %%  \in \Con(\bA \times \bB)$
and $\theta:= \Cg^{\bA \times \bB}((a, b), (a', b'))$.
First note that $\tau$ is a product of a congruence
of $\bA$ with a congruence of $\bB$, so $\tau$
is a congruence of $\bA\times \bB$.
Moreoever, the pair $((a,b), (a',b'))$ clearly belongs to $\tau$, so
$\theta\leq \tau$.  We must prove that $\tau \leq \theta$.

Fix $((x,y),(x'y')) \in \tau$. This means that
$(x,x') \in \Cg^{\bA}(a,a')$ and $(y,y')\in \Cg^{\bB}(b,b')$.
Therefore, there exist $n>0$, $m>0$, 
%% $(c_0, c_0'), (c_1, c_1'), \dots, (c_n, c_n')$ in $,  
$c_0$, $c_1$, $\dots$, $c_n$ $\in A$, $f_0$, $f_1$, $\dots$,
$f_{n-1} \in \Pol_1(\bA)$, $d_0$, $d_1$, $\dots$, $d_n\in B$, and $g_0$, $g_1$, $\dots$,
$g_{n-1} \in \Pol_1(\bB)$ such that 
\begin{align*}
  x = c_0, \; c_n=x', \text{ and } \; \{c_i, c_{i+1}\} &= \{f_i(a), f_i(a')\},
  \text{ for all $0\leq i < n$, and}\\
  y = d_0, \; d_n=y',  \text{ and }  
  \{d_i, d_{i+1}\} &= \{g_i(b), g_i(b')\}, \text{ for all $0\leq i < m$.}
\end{align*}
We can assume without loss of generality that $n=m$, since we can insert dummy
terms to extend the shorter of the two sequences.

We wish to prove $((x,y),(x'y')) \in \theta:= \Cg^{\bA \times \bB}((a, b), (a', b'))$,
which is equivalent to the following: there exist $n>0$, 
$(e_0, e_0')$, $(e_1,e_1')$, $\dots$, $(e_n, e_n') \in A\times B$, 
$h_0$, $h_1$, $\dots$, $h_{n-1} \in \Pol_1(\bA\times \bB)$ such that
\begin{align*}
  (x,y) = (e_0,e_0'), \; (e_n, e_n') &=(x',y'), \text{ and for all $0\leq i < n$, } \\
  \{(e_i, e_i'), (e_{i+1}, e_{i+1}')\} &= \{h_i(a,b), h_i(a',b')\}.
\end{align*}


Let $f = f_0$.  Since $f\in \Pol_1(\bA)$, for some $\ell>0$ there
exist a term $s^{\bA} \in \Clo_{\ell+1}(\bA)$ and $a_0, a_1, \dots, a_\ell$ in $A$ such that
\[
f^{\bA}(x) = s^{\bA}(a_0, \dots, a_{i-1}, x, a_{i+1}, \dots, a_\ell), \text{ for some $i$.}
\]
Similarly, letting $g = g_0$, for some $k>0$ there
exist a term $t^{\bB} \in \Clo_{k+1}(\bB)$ and $b_0, b_1, \dots, b_k$ in $B$ such that
\[
g^{\bB}(y) = t^{\bB}(b_0, \dots, b_{j-1}, y, b_{j+1}, \dots, b_k), \text{ for some $j$.}
\]
Without loss of generality, we can assume $j=i$. (check this)

%% Now, for
%% $\ba  = (a_0, a_1, \dots, a_{\ell})\in A^{\ell+1}$ and
%% $\bb  = (b_0, b_1, \dots, b_{\ell})\in B^{\ell+1}$, we have
%% \[
%% s^{\bA \times \bB}((a_0, b_0), \dots, (a_{\ell}, b_{\ell}))
%% = (s^{\bA}(\ba), s^{\bB}(\bb)).
%% \]
Consider the polynomial $p \in \Pol_1(\bA\times \bB)$ defined via
$s$ as follows:
\begin{align*}
p(x,y) &= s^{\bA \times \bB}((a_0, b), (a_1, b), \dots, (a_{i-1}, b),(x,y),
(a_{i+1}, b), \dots, (a_{\ell}, b))\\
&= (s^{\bA} (a_0, a_1,\dots, a_{i-1}, x, a_{i+1}, \dots, a_{\ell}), 
s^{\bB}(b, \dots, b, y, b,\dots, b))\\
&= (f^{\bA} (x), 
s^{\bB}(b, \dots, b, y, b,\dots, b)).
\end{align*}
Consider the polynomial $q \in \Pol_1(\bA\times \bB)$ defined via
$t$ as follows:
\begin{align*}
q(x,y) &= t^{\bA \times \bB}((f^{\bA}(a), b_0), (f^{\bA}(a), b_1), \dots, (f^{\bA}(a), b_{i-1}),(x,y),
(f^{\bA}(a), b_{i+1}), \dots, (f^{\bA}(a), b_{\ell}))\\
&= (t^{\bA} (f^{\bA}(a), \dots, f^{\bA}(a), x, f^{\bA}(a), \dots, f^{\bA}(a)), 
t^{\bB}(b_0, \dots, b_{i-1}, y, b_{i+1},\dots, b_{\ell}))\\
&= (t^{\bA} (f^{\bA}(a), \dots, f^{\bA}(a), x, f^{\bA}(a), \dots, f^{\bA}(a)), 
g^{\bB}(y)).
\end{align*}
Then $p(a,b) = (f^{\bA}(a), b)$ and $q(f^{\bA}(a), b) = (f^{\bA}(a),g^{\bB}(b))$.
Therefore,
\[
(q\circ p)^{\bA\times \bB} (a,b) =
(f^{\bA} (a), g^{\bB} (b)).
\]
We can carry out this construction for each pair $(f_i, g_i)$, $0\leq i < n$,
arriving at a sequence $q_i \circ p_i$ of polynomials in
$\Pol_1(\bA \times \bB)$ such that
\[
(q_i\circ p_i)^{\bA\times \bB} (a,b) =
(f_i^{\bA} (a), g_i^{\bB} (b)).
\]
Unfortunately, evaluating this polynomial at $(a',b')$ yields the less
desirable result:
\[
(q_i\circ p_i)(a',b')=
(t_i^{\bA}(f_i^{\bA} (a), \dots, f_i^{\bA} (a), f_i^{\bA} (a'),f_i^{\bA} (a), \dots, f_i^{\bA} (a)),
g_i(s_i^{\bB}(b, \dots, b, b', b,\dots, b))).
\]
So we try again...

\bigskip
%% \[
%% d_i(x,y) = d^{\bA\times \bB}((a_i,b),(a_i,b),(a_i,y)) = (d^{\bA}(a_i,a_i,a_i),
%% d^{\bB}(b,b,y)) = (a_i, d^{\bB}(b,b,y))
%% \]
Consider the polynomial $p \in \Pol_1(\bA\times \bB)$ defined via
$s$ as follows:
\begin{align*}
p(x,y) &= s^{\bA \times \bB}((a_0, y), \dots, (a_{i-1}, y), (x, y), (a_{i+1}, y), \dots, (a_{\ell}, y))\\
&= (s^{\bA} (a_0, \dots, a_{i-1}, x, a_{i+1}, \dots, a_{\ell}), 
s^{\bB}(y, \dots, y))\\
&= (f^{\bA} (x), y).
\end{align*}
Consider the polynomial $q \in \Pol_1(\bA\times \bB)$ defined via
$t$ as follows:
\begin{align*}
  q(x,y) &= t^{\bA \times \bB}(p(x,b_0),
  \dots, p(x, b_{i-1}),p(x,y),
  p(x, b_{i+1}), \dots, p(x, b_{\ell}))\\
  &= t^{\bA \times \bB}((f^{\bA}(x), b_0), \dots, (f^{\bA}(x), b_{i-1}),(f^{\bA}(x),y),
(f^{\bA}(x), b_{i+1}), \dots, (f^{\bA}(x), b_{\ell}))\\
%% &= (t^{\bA} (f^{\bA}(x), \dots, f^{\bA}(x), x, f^{\bA}(x), \dots, f^{\bA}(x)), 
%% g^{\bB}(y)).
&= (f^{\bA}(x), g^{\bB}(y)).
\end{align*}
Then $p(a,b) = (f^{\bA}(a), b)$ and $q(f^{\bA}(a), b) = (f^{\bA}(a),g^{\bB}(b))$.
Therefore,
\[
(q\circ p)^{\bA\times \bB} (a,b) =
(f^{\bA} (a), g^{\bB} (b)).
\]
We can carry out this construction for each pair $(f_i, g_i)$, $0\leq i < n$,
arriving at a sequence $q_i \circ p_i$ of polynomials in
$\Pol_1(\bA \times \bB)$ such that
\[
(q_i\circ p_i)^{\bA\times \bB} (a,b) =
(f_i^{\bA} (a), g_i^{\bB} (b)).
\]
Unfortunately, evaluating this polynomial at $(a',b')$ yields the less
desirable result:
\[
(q_i\circ p_i)(a',b')=
(t_i^{\bA}(f_i^{\bA} (a), \dots, f_i^{\bA} (a), f_i^{\bA} (a'),f_i^{\bA} (a), \dots, f_i^{\bA} (a)),
g_i(s_i^{\bB}(b, \dots, b, b', b,\dots, b))).
\]




\newpage

  Let $\sV$ be an idempotent variety and suppose $\bF = \<F, \dots\> = \bF_{\sV}(x,y)$ is finite.
  Below it will turn out to be convenient to have indexed copies of $\bF$, viz.,
  we let $\bF_i \cong \bF$ denote the ``$i$-th copy'' of $\bF$. 


\begin{lemma}
  \label{lem:absorpt-theor-bk}
  Let $\sV$ be an idempotent variety and suppose
  $\bF = \bF_{\sV}(x,y)$ is finite.
  %% Assume $\bF \sdp \bA \times \bB$.
  Let $\bF_0 \times \bF_1$ be the product of two copies of $\bF$.
  Let $\bR = \<R, \dots\>  \sdp \bF_0 \times \bF_1$ be the subalgebra generated
  by the set $\{(x, y), (x, x), (y, x)\}$, and let $\eta_i$ be the kernel of the
  $i$-th projection $\pi_i: \bR \onto \bF_i$.  Then 
  $\eta_0 \join \eta_1 = 1_R = \eta_0 \circ \eta_1 \circ \eta_0 = 
  \eta_1 \circ \eta_0 \circ \eta_1$.
\end{lemma}
\begin{proof}
  %% Let $\pi_0$ and $\pi_1$ be the first and second projections of $\bR$ onto $\bF$.
  Since $\bF$ is idempotent, $\{x\}$ is a subuniverse of $\bF$, hence the inverse
  image of $\{x\}$ under the first projection $\pi_0$ is a subuniverse of $\bR$,
  call it $S_0 = \{(a,b) \in R \mid a = x\}$.
  Since $(x,y)$ and $(x,x)$ lie in $S_0$, and $\{x,y\}$
  generates $\bF$, we have that $\{x\}\times F \subseteq S_0$.
  In particular, for every $b \in F$ the pair $(x,b)$ lies in $R$.
  Similarly, 
  $S_1= \{(a,b) \in R \mid b = x\}$ is
  the inverse image of $\{x\}$ under the second
  projection $\pi_1$, which is a subuniverse of $\bR$
  that contains $(x,x)$ and $(y,x)$. Therefore, $F\times \{x\} \subseteq S_1$, 
  so for every $a\in F$ the pair $(a,x)$ lies in $R$.
  %% Note that the kernels of $\pi_A$ and $\pi_B$ are $\eta_A$ and $\eta_B$. 

  It follows from these observations that $\eta_0$ and $\eta_1$ 3-permute:
  \[
  \eta_0 \circ \eta_1 \circ \eta_0 = 
  \eta_1 \circ \eta_0 \circ \eta_1 =
  \eta_0 \join \eta_1 = 1_R.
  \]
  Indeed, if $(a,b)$ and $(a',b')$ are arbitrary pairs in $R$, then
  %% \[
  %% (a_1,a_2) \mathrel{\eta_B} (x,a_2) \mathrel{\eta_A} (x,x) \mathrel{\eta_B} (b_1,x) \mathrel{\eta_A} (b_1,b_2).
  %% \]
  \[
  (a,b) \mathrel{\eta_0} (a,x) \mathrel{\eta_1} (a',x) \mathrel{\eta_0} (a',b').
  \]
  and
  \[
  (a,b) \mathrel{\eta_1} (x,b) \mathrel{\eta_0} (x,b') \mathrel{\eta_1} (a',b').
  \]
\end{proof}

Our goal is to prove the following

\begin{theorem}
Let $\bA = \<A, \dots\>$ be a finite idempotent algebra and $\sV = \bbV(\bA)$ the
variety it generates.  If $\bA$ has a difference term operation, then
$\bF = \bF_{\sV}(x,y)$ has a difference term operation.
%% Assume $\bF \sdp \bA \times \bB$.
\end{theorem}
\begin{proof}
  
We prove this by 
letting $\bF_0 \times \bF_1$ be the product of two copies of $\bF$,
letting $\bR = \<R, \dots\>  \sdp \bF_0 \times \bF_1$ be the subalgebra generated
by the set $\{(x, y), (x, x), (y, x)\}$ (as above), and then showing that
$R \cap \bigl(\{y\} \times y/\com{\Cg^{\bF}(x,y)}\bigr)$ is non-empty.  For if the latter is
non-empty, then there is a term $t$ satisfying
\[
t^{\bF^2}((x,y),(x,x),(y,x)) \in  \{y\} \times y/\com{\Cg^{\bF}(x,y)}.
\]
In other words,
\begin{align}
t^{\bF}(x,x,y) &= y, \; \text{ and }\label{eq:dt1}\\
t^{\bF}(y,x,x) &\comr{\Cg^{\bF}(x,y)} y, \label{eq:dt2}
\end{align}
which means that $t^{\bF}$ is a difference term operation for $\bF$.

Recall $x/\Cg^{F}(x,y)$ denotes the congruence class of $\Cg^{F}(x,y)$
containing $x$. This class obviously contains both $x$ and $y$.

\bigskip

Now, consider $\com{\Cg^{\bF}(x,y)}$, the commutator of $\Cg^{F}(x,y)$ with itself.
Obviously, $\com{\Cg^{\bF}(x,y)} \leq \Cg^{F}(x,y)$.  
Equally obviously, every idempotent term $t$ satisfies
\[
t^{\bF}(y,x,x) \mathrel{\Cg^{\bF}(x,y)} t^{\bF}(y,y,y) = y.
\]
Therefore,
if $\com{\Cg^{\bF}(x,y)} = \Cg^{F}(x,y)$  
every term  $t^{\bF}$ satisfies relation~(\ref{eq:dt2}) above.
Moreover, if the term $d$ interprets as a difference term of $\bA$, then we
have...  TODO: fill in details

\bigskip

%% $t^{\bF}(y,x,x) \comr{\Cg^{\bF}(x,y)} y$, which means that $t^{\bF}$ is a difference term...

%% \bigskip

\[
\bigl( \{x\} \times F\bigr) \cup \bigl(F \times \{x\}\bigr) \subseteq R
\]

\end{proof}

\newpage



\bigskip

\begin{align*}
  f^{\bB}(y) &= s^{\bB}(d(b,b,y), \dots, d(b,b,y), y, d(b,b,y), \dots, d(b,b,y)), \; \text{ and }\\
  g^{\bA}(x) &= t^{\bA}(f^{\bA}(a), \dots, f^{\bA}(a), x, f^{\bA}(a), \dots, f^{\bA}(a)).
\end{align*}
Then
\[
f^{\bA\times \bB} (a,b) =
(f^{\bA} (a), f^{\bB}(b)) = (f^{\bA} (a), s^{\bB}(b, \dots, b, )) = (f^{\bA} (a), b),
\]
and 
\[
g^{\bA\times \bB} (f^{\bA} (a), b) =
(g^{\bA} (f^{\bA} (a)), g^{\bB}(b))=
(t^{\bA}(f^{\bA}(a), \dots, f^{\bA}(a)),g^{\bB}(b))=
(f^{\bA} (a), g^{\bB} (b)).
\]
Therefore,
\[
(g^{\bA\times \bB}\circ f^{\bA\times \bB}) (a,b) =
(f^{\bA} (a), g^{\bB} (b)).
\]

But now what happens when we evaluate 
$(g^{\bA\times \bB}\circ f^{\bA\times \bB}) (a',b')$?
We have
\begin{align*}
  f^{\bA\times \bB} (a',b') &=
  (f^{\bA} (a'), f^{\bB}(b')) \\&= \bigl(f^{\bA} (a'), s^{\bB}(d(b,b,b'), \dots,
  d(b,b,b'), b', d(b,b,b'),\dots, d(b,b,b'))\bigr)\\
  &= \bigl(f^{\bA} (a'), s^{\bB}(b', \dots, b')\bigr)= \bigl(f^{\bA} (a'), b'\bigr).
\end{align*}


\section{Absorption Theorem of BK and Theorem 3.2 of BKS}
Recall,
\begin{theorem}[Absorption Theorem \protect{\cite[Thm~2.3]{MR2893395}}]
\label{thm:absorption}
%% If $\var{V}$ is an idempotent locally finite variety, then the following are equivalent:
%% \begin{itemize}
%% \item $\var{V}$ is a Taylor variety;
%% \item if $\bA_0, \bA_1 \in \var{V}$ are finite absorption-free algebras, 
%%   and if $\bR \sdp \bA_0 \times \bA_1$ is linked, then $\bR = \bA_0 \times \bA_1$.
%% \end{itemize}
Let $\bA$ and $\bB$ be finite
idempotent algebras with Taylor terms; let $R$ be a subdirect subuniverse of
$\bA \times \bB$, and let $\eta_A$ ($\eta_B$, resp.) be the kernel of the
projection $R \to A$ ($R \to B$, resp.).
If $\eta_A \join \eta_B = 1_R$, then either $R = A \times B$, or
$\bA$ has a proper absorbing subuniverse, or $\bB$ has a proper absorbing subuniverse.
\end{theorem}


\begin{theorem}[\protect{\cite[Theorem 3.2]{MR3374664}}]
  Let $\sV$ be a locally finite variety generated by a set $\sA$ of
  idempotent ``hereditarily absorption free'' algebras.
  If $\sV$ has a Taylor term, then it has a Mal'tsev term.
\end{theorem}
\begin{proof}
  Let $\bF = \bF_{\sV}(x,y)$ be the 2-generated free algebra
  %% , with generators $x, y$, 
  in the variety $\sV$. Since it is
  finite, $\bF$ lies in the pseudovariety generated by $\sA$, and thus it is
  absorption-free (see \cite[Proposition 2.1.(1)]{MR3374664}).
  Let $\bR$ be the subalgebra of $\bF^2$ generated by
  $(x, y), (x, x), (y, x)$. It is subdirect in $\bF^2$.
  Since $\bF$ is idempotent, $\eta_A \join \eta_B = 1_R$
  (as explained in the remark %% Remark~\ref{rem:absorpt-theor-bk}
  below).
  Consequently, by Theorem~\ref{thm:absorption},
  $\bR = \bF^2$, and there is a term $m$ witnessing
  that $(y, y) \in R$, i.e., a term $m$ satisfying
  $m((x, y), (x, x), (y, x)) = (y, y)$. This
  term is a Mal'tsev term for $\sV$.
\end{proof}

\begin{remark}
  \label{rem:absorpt-theor-bk}
  Let $\pi_0$ and $\pi_1$ be the first and second projections of $\bR$ onto $\bF$.
  Since $\bF$ is idempotent, $\{x\}$ is a subalgebra of $\bF$, hence the inverse
  image of $\{x\}$ under $\pi_0$ is a subuniverse of $\bR$, call it $S_0$.
  Since $(x,y)$ and $(x,x)$ lie in $S_0$, and $\{x,y\}$
  generates $\bF$, we see that $\{x\}\times F$ is contained in $S_0$.
  In particular, for every $a_1$, the pair $(x,a_1)$ must lie in $R$.
  Similarly, there is a subalgebra $S_1$ containing
  $F\times \{x\}$. %% Note that the kernels of $\pi_A$ and $\pi_B$ are $\eta_A$ and $\eta_B$. 
  Finally, for any $(a_0,a_1)$ and $(a_0',a_1')$ in $R$ we have
  %% \[
  %% (a_1,a_2) \mathrel{\eta_B} (x,a_2) \mathrel{\eta_A} (x,x) \mathrel{\eta_B} (b_1,x) \mathrel{\eta_A} (b_1,b_2).
  %% \]
  \[
  (a_0,a_1) \mathrel{\eta_1} (x,a_1) \mathrel{\eta_0} (x,a_1') \mathrel{\eta_1} (a_0',a_1').
  \]
  %% Cliff: You and I talked about whether Theorem 3.2 required idempotence.
  %%        Apparently it does. Is the idempotent reduct of an abelian algebra still abelian?
\end{remark}

%% \section{Definitions and Notations}
%% \label{sec:defin-notat}

%% Cliff's book~\cite{MR2839398}.
%% My GitHub repo~\cite{overalgebras-github}.
%% \appendix
%% \section{Appendix Title}
%% This is the text of the appendix, if you need one.

%\bibliographystyle{amsplain} %% or amsalpha
%% \bibliographystyle{plain-url}

\bibliographystyle{alphaurl}
\bibliography{inputs/refs2.bib}

\end{document}
