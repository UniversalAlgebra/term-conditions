%% FILE: com-fix-poi.tex
%% AUTHORS: William DeMeo
%% DATE: 18 Feb 2017
%% COPYRIGHT: (C) 2017 William DeMeo
%%%%%%%%%%%%%%%%%%%%%%%%%%%%%%%%%%%%%%%%%%%%%%%%%%%%%%%%%%%%%%%%%%%%%%%%%%%%%%%
%%                         BIBLIOGRAPHY FILE                                 %%
%%%%%%%%%%%%%%%%%%%%%%%%%%%%%%%%%%%%%%%%%%%%%%%%%%%%%%%%%%%%%%%%%%%%%%%%%%%%%%%
%% The `filecontents` command will crete a file in the inputs directory called 
%% refs.bib containing the references in the document, in case this file does 
%% not exist already.
%% If you want to add a BibTeX entry, please don't add it directly to the
%% refs.bib file.  Instead, add it in this file between the
%% \begin{filecontents*}{refs.bib} and \end{filecontents*} lines
%% then delete the existing refs.bib file so it will be automatically generated 
%% again with your new entry the next time you run pdfaltex.
\begin{filecontents*}{com-refs.bib}
@article{Bergman-DeMeo-2016,
  author    = {Clifford Bergman and William DeMeo},
  title     = {Universal Algebraic Methods for Constraint Satisfaction Problems},
  journal   = {CoRR},
  volume    = {abs/1611.02867},
  year      = {2016},
  url       = {http://arxiv.org/abs/1611.02867},
  timestamp = {Thu, 01 Dec 2016 19:32:08 +0100},
  biburl    = {http://dblp.uni-trier.de/rec/bib/journals/corr/BergmanD16},
  bibsource = {dblp computer science bibliography, http://dblp.org}
}
@article {MR1358491,
    AUTHOR = {Kearnes, Keith A.},
     TITLE = {Varieties with a difference term},
   JOURNAL = {J. Algebra},
  FJOURNAL = {Journal of Algebra},
    VOLUME = {177},
      YEAR = {1995},
    NUMBER = {3},
     PAGES = {926--960},
      ISSN = {0021-8693},
     CODEN = {JALGA4},
   MRCLASS = {08B10 (08B05)},
  MRNUMBER = {1358491},
MRREVIEWER = {H. Peter Gumm},
       DOI = {10.1006/jabr.1995.1334},
       URL = {http://dx.doi.org/10.1006/jabr.1995.1334},
}
\end{filecontents*}
\documentclass[12pt]{amsart}
% The following \documentclass options may be useful:
% preprint      Remove this option only once the paper is in final form.
% 10pt          To set in 10-point type instead of 9-point.
% 11pt          To set in 11-point type instead of 9-point.
% numbers       To obtain numeric citation style instead of author/year.

%% \usepackage{setspace}\onehalfspacing

\usepackage{amsmath}
\usepackage{amscd,amssymb,amsthm} %, amsmath are included by default
\usepackage{latexsym,stmaryrd,mathrsfs,enumerate,scalefnt,ifthen}
\usepackage{mathtools}
\usepackage[mathcal]{euscript}
\usepackage[colorlinks=true,urlcolor=black,linkcolor=black,citecolor=black]{hyperref}
\usepackage{url}
\usepackage{scalefnt}
\usepackage{tikz}
\usepackage{color}
\usepackage[margin=1in]{geometry}
\usepackage{scrextend}

%%////////////////////////////////////////////////////////////////////////////////
%% Theorem styles
\numberwithin{equation}{section}
\theoremstyle{plain}
\newtheorem{theorem}{Theorem}[section]
\newtheorem{lemma}[theorem]{Lemma}
\newtheorem{proposition}[theorem]{Proposition}
\newtheorem{prop}[theorem]{Proposition}
\theoremstyle{definition}
\newtheorem{claim}[theorem]{Claim}
\newtheorem{corollary}[theorem]{Corollary}
\newtheorem{definition}[theorem]{Definition}
\newtheorem{notation}[theorem]{Notation}
\newtheorem{Fact}[theorem]{Fact}
\newtheorem*{fact}{Fact}
\newtheorem{example}[theorem]{Example}
\newtheorem{examples}[theorem]{Examples}
\newtheorem{exercise}{Exercise}
\newtheorem*{lem}{Lemma}
\newtheorem*{cor}{Corollary}
\newtheorem*{remark}{Remark}
\newtheorem*{remarks}{Remarks}
\newtheorem*{obs}{Observation}


%%%%%%%%%%%%%%%%%%%%%%%%%%%%%%%%%%%%%%%%
% Acronyms
%%%%%%%%%%%%%%%%%%%%%%%%%%%%%%%%%%%%%%%%
%% \usepackage[acronym, shortcuts]{glossaries}
%\usepackage[smaller]{acro}
\usepackage[smaller]{acronym}
\usepackage{xspace}

%% \acs{CSP} -- short version of the acronym\\
%% \acl{CSP} -- expanded acronym without mentioning the acronym.\\
%% \acp{CSP} -- plurals.\\
%% \acfp{CSP} -- long forms into plurals.\\
%% \acsp{CSP} -- short form into a plural.\\
%% \aclp{CSP} -- long form into a plural.\\
%% \acfi{CSP} -- Full Name acronym in italics and abbreviated form in upshape.\\
%% \acsu{CSP} -- short form of the acronym and marks it as used.\\
%% \aclu{CSP} -- Prints the long form of the acronym and marks it as used.\\

\acrodef{lics}[LICS]{Logic in Computer Science}
\acrodef{sat}[SAT]{satisfiability}
\acrodef{nae}[NAE]{not-all-equal}
\acrodef{ctb}[CTB]{cube term blocker}
\acrodef{tct}[TCT]{tame congruence theory}
\acrodef{wnu}[WNU]{weak near-unanimity}
\acrodef{CSP}[CSP]{constraint satisfaction problem}
\acrodef{MAS}[MAS]{minimal absorbing subuniverse}
\acrodef{MA}[MA]{minimal absorbing}
\acrodef{cib}[CIB]{commutative idempotent binar}
\acrodef{sd}[SD]{semidistributive}
\acrodef{NP}[NP]{nondeterministic polynomial time}
\acrodef{P}[P]{polynomial time}
\acrodef{PeqNP}[P $ = $ NP]{P is NP}
\acrodef{PneqNP}[P $ \neq $ NP]{P is not NP}

%%%%%%%%%%%%%%%%%%%%%%%%%%%%%%%%%%%%%%%%%%%%%%%%%%%%%%%%%%%%%%%%%

%% \usepackage{inputs/proof-dashed}


%%%%%%%%%%%%%%%%%%%%%%%%%%%%%%%%%%%%%%%%%%%%%%%%%%%%%%%%%%%%%%%%%

%% Put new macros in the macros.sty file
\usepackage{inputs/macros}

\begin{document}

\title[Commutator as Fixed Point]{The Commutator as Fixed Point of a Closure Operator}
\date{\today}
\author[W.~DeMeo]{William DeMeo}
\address{University of Hawaii}
\email{williamdemeo@gmail.com}

%% \thanks{The authors would like to extend special thanks to...}

\maketitle

\renewcommand{\etaR}{\ensuremath{\eta}}

%% \begin{abstract}
%% This note provides some tools that should enable us to prove
%% the following: if $\bA$ is a finite idempotent algebra with a
%% difference term operation, then the 2-generated free algebra in $\bbV(\bA)$ has
%% a difference term operation.  
%% \end{abstract}

%% \renewcommand{\bbeta}{\ensuremath{\ubar{\boldsymbol{\beta}}}}

\begin{abstract}
  In this note we elaborate on the following remark of Keith
  Kearnes~\cite[p.~930]{MR1358491}
  giving an alternate description of the commutator % $[\alpha, \beta]$
  (here $\beta$ is a congruence of $\bA$ and $\bA \times_\beta \bA$ denotes the subalgebra of
  $\bA^2$ with universe $\beta$):
  \begin{quote}
  %% ``When $\alpha$ and $\beta$ are reflexive, compatible relations
  %% First, when $\beta$ is a reflexive, compatible binary relation on $A$
  %% we...
  ``Let $\Delta_{\beta,\alpha}$ be the congruence on $\bA \times_\beta \bA$ generated by
  \[\{\<(x, x), (y, y)\> \mid (x, y) \in \alpha\}.\]
  Call a subset $G \subseteq A^2$ {\bf $\Delta$-closed} if
  \[
  \Delta_{\beta,\alpha}\circ G \circ \Delta_{\beta,\alpha} \subseteq G.
  \]
  When $\alpha$ and $\beta$ are reflexive, compatible relations, then $[\alpha, \beta]$
  is the smallest subset $\gamma \subseteq A^2$ such that (i) $\gamma$
  is a congruence of $\bA$ and (ii) $\gamma$ is $\Delta$-closed.''
  \end{quote}
\end{abstract}


\section{Introduction}
To be honest, the genesis of these notes was my futile attempt
to interpret the expression 
$\Delta_{\beta,\alpha}\circ G \circ \Delta_{\beta,\alpha}$.
Noting that 
$\Delta_{\beta,\alpha} \subseteq (A\times A)^2$ while
$G\subseteq A^2$, I wasn't aware of a standard means of composing 
relations of such different arities.
In this note I try to reconcile this by giving an alternative ``closure'' operation that 
acheives the same end and verifies Keith's assertion that ``there is a useful
alternate description of $[\alpha, \beta]$.''
Indeed, the commutator is a fixed point of some
closure operator (i.e., closed set), and the closure operator
involves the relation $\Delta_{\beta, \alpha}$ defined above.

\subsection{Definitions}
\label{sec:defs}
For an algebra $\bA$ with congruence relations $\alpha$, $\beta\in \Con\bA$,
let $\bbeta$ denote the subalgebra of $\bA\times \bA$ with universe 
$\beta$.
Let $D = \{(a,a) \mid a\in A\}$ and 
$D^2_\alpha = \{((a,a), (b,b)) \in D^2 \mid a\alphar b\}$.
Finally, let
$\Delta_{\beta, \alpha} = \Cg^{\bbeta}\bigl(D^2_\alpha \bigr)$ be
the congruence on $\bbeta$ generated by the set $D^2_\alpha$.
The congruence class of $\Delta_{\beta, \alpha}$ that contains $(b,b')$ is
denoted and defined as follows:
\[
(b,b')/\Delta_{\beta,\alpha} = \{(a,a') \in \beta \mid (a,a') \mathrel{\Delta_{\beta,\alpha}} (b,b')\}.
\]
Let $\Phi_{\beta, \alpha} \colon \sP(\beta) \to \sP(\beta)$ be the function that
takes each $B \subseteq \beta$ to
\begin{equation}
  \label{eq:6}
  \Phi_{\beta, \alpha}(B) = \bigcup_{(b,b')\in B} (b,b')/\Delta_{\beta, \alpha}.
\end{equation}
%% \newcommand{\Phiba}{\ensuremath{\Phi_{\beta, \alpha}}}
\newcommand{\Phiba}{\ensuremath{\Phi}}
\section{Fixed Point Lemma}
\begin{lemma}
  \label{lem:fixed-point-comm}
  Let $\bA$ be an algebra with $\alpha$, $\beta\in \Con(\bA)$.
  If $\Phi:=\Phi_{\beta, \alpha}$ is defined as in~(\ref{eq:6}), then 
  \begin{enumerate}[(i)]
  \item \label{item:1} $\Phiba$ is a closure operator on $\sP(\beta)$;
  \item \label{item:2} $[\alpha, \beta]$ is the least fixed point of $\Phiba$.
  \end{enumerate}
\end{lemma}
\begin{proof}\
  %% \begin{enumerate}[(i)]
  %% \item 
\noindent (i) Fix $B \subseteq \beta$. We must prove the following:
    (a) $B \subseteq \Phiba(B)$;    
    (b) $B \subseteq C  \Rightarrow \Phiba(B) \subseteq \Phiba(C)$;    
    and (c) $\Phiba(\Phiba(B))  = \Phiba(B)$.    
    If $(b,b') \in B$, then $(b,b') \in \Phiba(B)$ since the
    operation~(\ref{eq:6}) does not discard any of the pairs that were already in $B$.
    %% Thus, $G \subseteq \Phiba(G)$ proving~(\ref{item:1}).
    As for (b), if $(a,a') \in \Phiba(B)$, then there exists
    $(b,b') \in B \subseteq C$ such that $(a,a') \mathrel{\Delta_{\beta, \alpha}} (b,b')$.
    Since $(b,b')$ belongs to $C$ we have $(a,a') \in \Phiba(C)$ as well.
    As for (c), it clearly follows from
    (a) and (b) that $\Phiba(B)\subseteq \Phiba(\Phiba(B))$, so we prove the
    reverse inclusion.    
    Let $(d,d') \in \Phiba(\Phiba(B))$. Then
    $(c,c') \mathrel{\Delta_{\beta, \alpha}} (d,d')$ for some
    $(c,c') \in \Phiba(B)$, which implies
    $(b,b') \mathrel{\Delta_{\beta, \alpha}} (c,c')$ for some
    $(b,b') \in B$.  By transitivity of $\Delta_{\beta, \alpha}$ we conclude that
    $(d,d') \in \Phiba(B)$, as desired.
    \bigskip

\noindent (ii)
Since $[\alpha, \beta] \in \sP(\beta)$
we have $[\alpha, \beta] \subseteq \Phiba([\alpha, \beta])$,
by~(\ref{item:1}).
    We prove the reverse inclusion.
    %% $\Phiba([\alpha, \beta]) \subseteq [\alpha, \beta]$.
    If $(c,c')\in \Phiba([\alpha, \beta])$, then~(\ref{eq:6})
    implies there exists $(b,b')\in [\alpha, \beta]$ such that
    \begin{equation}
      \label{eq:1000}
      (b,b') \mathrel{\Delta_{\beta, \alpha}} (c,c').
    \end{equation}
    From the definition of $\Delta_{\beta, \alpha}$ and 
    \malcev's theorem on congruence generation,~(\ref{eq:1000})
    holds if and only if
    %% \begin{align*}
    %%   \exists \,& z_i \betar z_i', \quad 0\leq i \leq n,\\
    %%   \exists \,& x_i \alphar y_i, \quad  0\leq i < n,\\
    %%   \exists\, &f_i \in F_{\bA^2}^\ast, \quad 0\leq i < n,
    %% \end{align*}
    $\exists \, z_i \betar z_i'$ $(0\leq i \leq n)$,
    $\exists \, x_i \alphar y_i$ $(0\leq i < n)$,
    $\exists\, f_i \in \Pol_1(\bA\times \bA)$ $(0\leq i < n)$
    such that
    $(b, b') = (z_0,z_0')$ and
    $(z_n,z_n') = (c, c')$, and
    \begin{align}
      \label{eq:0}
      \{(b, b'),(z_1,z_1')\} &= \{f_0(x_0,x_0), f_0(y_0,y_0)\}\\
      \label{eq:1}
      \{(z_1,z_1'),(z_2,z_2')\} &= \{f_1(x_1,x_1), f_1(y_1,y_1)\}\\
      \nonumber
      & \vdots\\
      %% \label{eq:n-1}
      \nonumber
      \{(z_{n-1},z_{n-1}'),(c, c')\} &= \{f_{n-1}(x_{n-1},x_{n-1}), f_{n-1}(y_{n-1},y_{n-1})\}
    \end{align}
    For each $(0\leq i < n)$, $f_i \in \Pol_1(\bA\times \bA)$, which means
    \newcommand\gA{\ensuremath{g^{\bA}}}%
    %% \begin{align*}
    \[      f_i(x, x') = g_i^{\bbeta}((x, x'), (a_1, a_1'), \dots, (a_k, a_k') )
      %% &= (g_i^{\bA}(x, a_1, a_2, \dots, a_k), g_i^{\bA}(x, a_1', a_2', \dots, a_k')),%
      = (g_i^{\bA}(x, \ba), g_i^{\bA}(x, \ba')),%
      %% \end{align*}%
      \]%
    \renewcommand\gA{\ensuremath{g}}%
    for some $k$, $\gA_i \in \sansClo_{k+1}(\bA)$, and constants tuples
    $\ba = (a_1, \dots, a_k)$ and $\ba' = (a_1', \dots, a_k')$ such that
    $a_i \betar a_i'$ ($1\leq i\leq k$). 
    By~(\ref{eq:0}), either
    \[
    (b, b') = (\gA_0(x_0, \ba), \gA_0(x_0, \ba')
    \quad \text{ and } \quad 
    (z_1,z_1')= (\gA_0(y_0, \ba), \gA_0(y_0, \ba')),
    \]
    or vice-versa.  Since $x_0 \alphar y_0$ and 
    $a_i \betar a_i'$ ($1\leq i\leq k$), the $\alpha,\beta$-term condition
    entails
    \[
    \gA_0(x_0, \ba) \commr{\alpha}{\beta} \gA_0(x_0, \ba')
    \quad \Longleftrightarrow \quad 
    \gA_0(y_0, \ba) \commr{\alpha}{\beta} \gA_0(y_0, \ba').
    \]
    This and~(\ref{eq:0}) yield
    $(b,b')\in [\alpha, \beta]$ iff
    $(z_1,z_1')\in [\alpha, \beta]$.
    Similarly~(\ref{eq:1}) and $x_1 \alphar y_1$ imply
    $(z_1,z_1')\in [\alpha, \beta]$ iff
    $(z_2,z_2')\in [\alpha, \beta]$.  Inductively, we arrive at 
    $(b,b')\in [\alpha, \beta]$ iff $(c,c')\in [\alpha, \beta]$, as desired.

    We have thus proved $[\alpha, \beta]$ is a fixed point of $\Phiba$.
    %% Recall that the lattice order on $\Con \bA$ is complete, so t
    In other words, 
    $[\alpha, \beta]$ is a ``$\Phiba$-closed'' subset of $\beta$.
    (A set $B\subseteq \beta$ is called \defn{$\Phiba$-closed} provided
    $\Phiba(B) \subseteq B$.)
    Recall, if $f$ is a monotone increasing function defined on a
    complete poset $\<P, \leq\>$, then the least fixed point of $f$
    is $\Meet \{ p\in P \mid f p \leq p\}$. %(See, for example,~\cite{MR3012378}.)
    Thus,
    Lemma~\ref{lem:fixed-point-comm}~(\ref{item:2}) asserts that
    \begin{equation}
      \label{eq:2}
            [\alpha, \beta] =\Meet \{ B \subseteq \beta \mid \Phiba(B) \subseteq B\}.
    \end{equation}

    We already proved $[\alpha, \beta]$ is
    $\Phiba$-closed, so it remains to check for every $\Phiba$-closed subset
    $B\subseteq \beta$ that $[\alpha, \beta] \subseteq B$.
    Fix a $\Phiba$-closed subset $B\subseteq \beta$. % (i.e., $\Phiba(B)\subseteq B$).
    It suffices to prove $\CC{\alpha}{\beta}{\Phiba(B)}$, since this implies 
    $[\alpha, \beta] \subseteq \Phiba(B) \subseteq B$.
    Thus, our goal is to establish the $\alpha, \beta$-term condition.

    Let $p \in \Pol_{k+1}(\bA)$ and $a \alphar a'$ and $c_i \betar c_i'$ ($1\leq i\leq k$);
    suppose $p(a, \bc) \mathrel{\Phiba(B)} p(a, \bc')$.
    We prove that these hypotheses entail the following relation:
    \begin{equation}
      \label{eq:1001}
      p(a', \bc) \mathrel{\Phiba(B)} p(a', \bc').
    \end{equation}
    By definition of $\Phiba$, (\ref{eq:1001}) 
    is equivalent to the existence of
    some pair $(b,b') \in B$ such that
    $(b,b') \mathrel{\Delta_{\beta, \alpha}} (p(a', \bc), p(a', \bc'))$.
    %% We wish to %% establish the same for $(p(a', \bc), p(a', \bc'))$. That
    %% %% is, we must
    %% find such a pair for $(p(a', \bc), p(a', \bc'))$---that is, 
    %% a pair in $B$ that is $\mathrel{\Delta_{\beta, \alpha}}$-related to
    %% $(p(a', \bc), p(a', \bc'))$.
    Notice that the pair $(p(a, \bc), p(a, \bc'))$
    belongs to $B$ since
    $(p(a, \bc), p(a, \bc')) \in \Phiba(B) \subseteq B$.  Also,
    $c_i \betar c_i'$ ($0\leq i<k$) implies
    \begin{align*}
    ((a,a), (c_1, c_1'), (c_1, c_1'), \dots, (c_{k}, c_{k}'))&\in \beta^{k+1} \quad \text{ and }\\
    ((a',a'), (c_1, c_1'), (c_1, c_1'), \dots, (c_{k}, c_{k}')) &\in \beta^{k+1}.
    \end{align*}
    Therefore,
    \begin{align}
      \label{item:4}
    p^{\beta}((a,a), (c_1, c_1'), (c_1, c_1'), \dots, (c_{k}, c_{k}'))
    &= (p^{\bA}(a, \bc), p^{\bA}(a, \bc')) \in \beta  \quad \text{ and }\\
    \label{item:5}
    p^{\beta}((a',a'), (c_1, c_1'), (c_1, c_1'), \dots, (c_{k}, c_{k}'))
    &= (p^{\bA}(a', \bc), p^{\bA}(a', \bc')) \in \beta.
    \end{align}
    Finally, $a\alphar a'$ implies $p(a, \bc) \alphar p(a', \bc)$, and this---together
    with~(\ref{item:4}) and~(\ref{item:5})---proves the pair
    $\bigl((p(a, \bc), p(a, \bc')), (p(a', \bc), p(a', \bc'))\bigr)$
    belongs to  $\mathrel{\Delta_{\beta, \alpha}}$.
    Since  $(p(a, \bc), p(a, \bc')) \in B$, this proves
    $(p(a', \bc), p(a', \bc')) \in \Phiba(B)$, completing the proof.
  \end{proof}

\bigskip

\bibliographystyle{alphaurl}
\bibliography{com-refs.bib}

\end{document}
