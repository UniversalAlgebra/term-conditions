%% FILE: com-fix-poi.tex
%% AUTHORS: William DeMeo
%% DATE: 14 Feb 2017
%% COPYRIGHT: (C) 2017 William DeMeo
\documentclass[11pt]{amsart}
% The following \documentclass options may be useful:
% preprint      Remove this option only once the paper is in final form.
% 10pt          To set in 10-point type instead of 9-point.
% 11pt          To set in 11-point type instead of 9-point.
% numbers       To obtain numeric citation style instead of author/year.

%% \usepackage{setspace}\onehalfspacing

\usepackage{amsmath}
\usepackage{amscd,amssymb,amsthm} %, amsmath are included by default
\usepackage{latexsym,stmaryrd,mathrsfs,enumerate,scalefnt,ifthen}
\usepackage{mathtools}
\usepackage[mathcal]{euscript}
\usepackage[colorlinks=true,urlcolor=black,linkcolor=black,citecolor=black]{hyperref}
\usepackage{url}
\usepackage{scalefnt}
\usepackage{tikz}
\usepackage{color}
\usepackage[margin=1in]{geometry}
\usepackage{scrextend}

%%////////////////////////////////////////////////////////////////////////////////
%% Theorem styles
\numberwithin{equation}{section}
\theoremstyle{plain}
\newtheorem{theorem}{Theorem}[section]
\newtheorem{lemma}[theorem]{Lemma}
\newtheorem{proposition}[theorem]{Proposition}
\newtheorem{prop}[theorem]{Proposition}
\theoremstyle{definition}
\newtheorem{claim}[theorem]{Claim}
\newtheorem{corollary}[theorem]{Corollary}
\newtheorem{definition}[theorem]{Definition}
\newtheorem{notation}[theorem]{Notation}
\newtheorem{Fact}[theorem]{Fact}
\newtheorem*{fact}{Fact}
\newtheorem{example}[theorem]{Example}
\newtheorem{examples}[theorem]{Examples}
\newtheorem{exercise}{Exercise}
\newtheorem*{lem}{Lemma}
\newtheorem*{cor}{Corollary}
\newtheorem*{remark}{Remark}
\newtheorem*{remarks}{Remarks}
\newtheorem*{obs}{Observation}


%%%%%%%%%%%%%%%%%%%%%%%%%%%%%%%%%%%%%%%%
% Acronyms
%%%%%%%%%%%%%%%%%%%%%%%%%%%%%%%%%%%%%%%%
%% \usepackage[acronym, shortcuts]{glossaries}
%\usepackage[smaller]{acro}
\usepackage[smaller]{acronym}
\usepackage{xspace}

%% \acs{CSP} -- short version of the acronym\\
%% \acl{CSP} -- expanded acronym without mentioning the acronym.\\
%% \acp{CSP} -- plurals.\\
%% \acfp{CSP} -- long forms into plurals.\\
%% \acsp{CSP} -- short form into a plural.\\
%% \aclp{CSP} -- long form into a plural.\\
%% \acfi{CSP} -- Full Name acronym in italics and abbreviated form in upshape.\\
%% \acsu{CSP} -- short form of the acronym and marks it as used.\\
%% \aclu{CSP} -- Prints the long form of the acronym and marks it as used.\\

\acrodef{lics}[LICS]{Logic in Computer Science}
\acrodef{sat}[SAT]{satisfiability}
\acrodef{nae}[NAE]{not-all-equal}
\acrodef{ctb}[CTB]{cube term blocker}
\acrodef{tct}[TCT]{tame congruence theory}
\acrodef{wnu}[WNU]{weak near-unanimity}
\acrodef{CSP}[CSP]{constraint satisfaction problem}
\acrodef{MAS}[MAS]{minimal absorbing subuniverse}
\acrodef{MA}[MA]{minimal absorbing}
\acrodef{cib}[CIB]{commutative idempotent binar}
\acrodef{sd}[SD]{semidistributive}
\acrodef{NP}[NP]{nondeterministic polynomial time}
\acrodef{P}[P]{polynomial time}
\acrodef{PeqNP}[P $ = $ NP]{P is NP}
\acrodef{PneqNP}[P $ \neq $ NP]{P is not NP}

%%%%%%%%%%%%%%%%%%%%%%%%%%%%%%%%%%%%%%%%%%%%%%%%%%%%%%%%%%%%%%%%%

%% \usepackage{inputs/proof-dashed}


%%%%%%%%%%%%%%%%%%%%%%%%%%%%%%%%%%%%%%%%%%%%%%%%%%%%%%%%%%%%%%%%%

%% Put new macros in the macros.sty file
\usepackage{inputs/macros}

\begin{document}

\title[Commutator as Fixed Point]{The Commutator as Fixed Point of a Closure Operator}
\date{\today}
% \author[W.~DeMeo]{William DeMeo}
\address{University of Hawaii}
\email{williamdemeo@gmail.com}

%% \thanks{The authors would like to extend special thanks to...}

\maketitle

\renewcommand{\etaR}{\ensuremath{\eta}}

%% \begin{abstract}
%% This note provides some tools that should enable us to prove
%% the following: if $\bA$ is a finite idempotent algebra with a
%% difference term operation, then the 2-generated free algebra in $\bbV(\bA)$ has
%% a difference term operation.  
%% \end{abstract}

\section{Definitions}
\label{sec:defs}
For an algebra $\bA$ with congruence relations $\alpha$, $\beta\in \Con\bA$, we 
let $\bA \times_\beta \bA$ denote the set $\beta$ when we wish to emphasize
the fact that $\beta$ is a subalgebra of $\bA\times \bA$.
Let $D = \{(a,a) \mid a\in A\}$ and let
\[\Delta_{\beta, \alpha} = \Cg^{\bA\times_\beta \bA}\{((a,a),(b,b))\in D^2 \mid a \alphar b\}.\]
Let $\Phi_{\beta, \alpha} \colon \sP(A\times A) \to \sP(A\times A)$ be defined as
follows: for every $G \subseteq A\times A$,
\begin{equation}
  \label{eq:6}
\Phi_{\beta, \alpha}(G) = \{(x,y) \in A^2 \mid  (\exists (c,d) \in G) \;
(x,x) \mathrel{\Delta_{\beta, \alpha}} (c,c) \text{ and }
(d,d) \mathrel{\Delta_{\beta, \alpha}} (y,y)\}.
\end{equation}
%% \newcommand{\Phiba}{\ensuremath{\Phi_{\beta, \alpha}}}
\newcommand{\Phiba}{\ensuremath{\Phi}}
\section{Fixed Point Lemma}
\begin{lemma} Let $\bA$ be an algebra with $\alpha$, $\beta\in \Con(\bA)$.
  If $\Phi:=\Phi_{\beta, \alpha}$ is defined as in~(\ref{eq:6}), then 
  \begin{enumerate}[(i)]
  \item $\Phiba$ is a closure operator on $\sP(A\times A)$;
  \item $[\alpha, \beta]$ is the least fixed point of $\Phiba$.
  \end{enumerate}
\end{lemma}
\begin{proof}\ 
  \begin{enumerate}[(i)]
  \item Fix $G \subseteq A \times A$. We must prove the following:
    (a) $G \subseteq \Phiba(G)$;    
    (b) $G \subseteq H  \Rightarrow \Phiba(G) \subseteq \Phiba(H)$;    
    and (c) $\Phiba(\Phiba(G))  = \Phiba(G)$.    
    If $(a,b) \in G$, then $(a,b) \in \Phiba(G)$ since
    $(a,a) \mathrel{\Delta_{\beta, \alpha}} (a,a)$ and
    $(b,b) \mathrel{\Delta_{\beta, \alpha}} (b,b)$. Thus (a) holds.
    %% Thus, $G \subseteq \Phiba(G)$ proving~(\ref{item:1}).
    As for (b), if $(x,y) \in \Phiba(G)$, then there exists
    $(c,d) \in G \subseteq H$ such that $(x,x) \mathrel{\Delta_{\beta, \alpha}} (c,c)$
    and $(d,d) \mathrel{\Delta_{\beta, \alpha}} (y,y)$. Since
    $(c,d)$ also belongs to $H$ we have $(x,y) \in \Phiba(H)$ as well.
    Finally, (c) is obvious since
    $\Delta_{\beta, \alpha} \circ \Delta_{\beta, \alpha} = \Delta_{\beta, \alpha}$.

    \bigskip

  \item ???
    
  \end{enumerate}

  \vfill
  
\end{proof}
\end{document}

\bibliographystyle{alphaurl}
\bibliography{inputs/refs2.bib}

\end{document}
