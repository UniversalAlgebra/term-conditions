%% FILE: com-fix-poi.tex
%% AUTHORS: William DeMeo
%% DATE: 14 Feb 2017
%% COPYRIGHT: (C) 2017 William DeMeo
\documentclass[11pt]{amsart}
% The following \documentclass options may be useful:
% preprint      Remove this option only once the paper is in final form.
% 10pt          To set in 10-point type instead of 9-point.
% 11pt          To set in 11-point type instead of 9-point.
% numbers       To obtain numeric citation style instead of author/year.

%% \usepackage{setspace}\onehalfspacing

\usepackage{amsmath}
\usepackage{amscd,amssymb,amsthm} %, amsmath are included by default
\usepackage{latexsym,stmaryrd,mathrsfs,enumerate,scalefnt,ifthen}
\usepackage{mathtools}
\usepackage[mathcal]{euscript}
\usepackage[colorlinks=true,urlcolor=black,linkcolor=black,citecolor=black]{hyperref}
\usepackage{url}
\usepackage{scalefnt}
\usepackage{tikz}
\usepackage{color}
\usepackage[margin=1in]{geometry}
\usepackage{scrextend}

%%////////////////////////////////////////////////////////////////////////////////
%% Theorem styles
\numberwithin{equation}{section}
\theoremstyle{plain}
\newtheorem{theorem}{Theorem}[section]
\newtheorem{lemma}[theorem]{Lemma}
\newtheorem{proposition}[theorem]{Proposition}
\newtheorem{prop}[theorem]{Proposition}
\theoremstyle{definition}
\newtheorem{claim}[theorem]{Claim}
\newtheorem{corollary}[theorem]{Corollary}
\newtheorem{definition}[theorem]{Definition}
\newtheorem{notation}[theorem]{Notation}
\newtheorem{Fact}[theorem]{Fact}
\newtheorem*{fact}{Fact}
\newtheorem{example}[theorem]{Example}
\newtheorem{examples}[theorem]{Examples}
\newtheorem{exercise}{Exercise}
\newtheorem*{lem}{Lemma}
\newtheorem*{cor}{Corollary}
\newtheorem*{remark}{Remark}
\newtheorem*{remarks}{Remarks}
\newtheorem*{obs}{Observation}


%%%%%%%%%%%%%%%%%%%%%%%%%%%%%%%%%%%%%%%%
% Acronyms
%%%%%%%%%%%%%%%%%%%%%%%%%%%%%%%%%%%%%%%%
%% \usepackage[acronym, shortcuts]{glossaries}
%\usepackage[smaller]{acro}
\usepackage[smaller]{acronym}
\usepackage{xspace}

%% \acs{CSP} -- short version of the acronym\\
%% \acl{CSP} -- expanded acronym without mentioning the acronym.\\
%% \acp{CSP} -- plurals.\\
%% \acfp{CSP} -- long forms into plurals.\\
%% \acsp{CSP} -- short form into a plural.\\
%% \aclp{CSP} -- long form into a plural.\\
%% \acfi{CSP} -- Full Name acronym in italics and abbreviated form in upshape.\\
%% \acsu{CSP} -- short form of the acronym and marks it as used.\\
%% \aclu{CSP} -- Prints the long form of the acronym and marks it as used.\\

\acrodef{lics}[LICS]{Logic in Computer Science}
\acrodef{sat}[SAT]{satisfiability}
\acrodef{nae}[NAE]{not-all-equal}
\acrodef{ctb}[CTB]{cube term blocker}
\acrodef{tct}[TCT]{tame congruence theory}
\acrodef{wnu}[WNU]{weak near-unanimity}
\acrodef{CSP}[CSP]{constraint satisfaction problem}
\acrodef{MAS}[MAS]{minimal absorbing subuniverse}
\acrodef{MA}[MA]{minimal absorbing}
\acrodef{cib}[CIB]{commutative idempotent binar}
\acrodef{sd}[SD]{semidistributive}
\acrodef{NP}[NP]{nondeterministic polynomial time}
\acrodef{P}[P]{polynomial time}
\acrodef{PeqNP}[P $ = $ NP]{P is NP}
\acrodef{PneqNP}[P $ \neq $ NP]{P is not NP}

%%%%%%%%%%%%%%%%%%%%%%%%%%%%%%%%%%%%%%%%%%%%%%%%%%%%%%%%%%%%%%%%%

%% \usepackage{inputs/proof-dashed}


%%%%%%%%%%%%%%%%%%%%%%%%%%%%%%%%%%%%%%%%%%%%%%%%%%%%%%%%%%%%%%%%%

%% Put new macros in the macros.sty file
\usepackage{inputs/macros}

\begin{document}

\title[Commutator as Fixed Point]{The Commutator as Fixed Point of a Closure Operator}
\date{\today}
% \author[W.~DeMeo]{William DeMeo}
\address{University of Hawaii}
\email{williamdemeo@gmail.com}

%% \thanks{The authors would like to extend special thanks to...}

\maketitle

\renewcommand{\etaR}{\ensuremath{\eta}}

%% \begin{abstract}
%% This note provides some tools that should enable us to prove
%% the following: if $\bA$ is a finite idempotent algebra with a
%% difference term operation, then the 2-generated free algebra in $\bbV(\bA)$ has
%% a difference term operation.  
%% \end{abstract}

%% \renewcommand{\bbeta}{\ensuremath{\ubar{\boldsymbol{\beta}}}}

\section{Definitions}
\label{sec:defs}
For an algebra $\bA$ with congruence relations $\alpha$, $\beta\in \Con\bA$,
let $\bbeta$ denote the subalgebra of $\bA\times \bA$ with universe 
$\beta$.
Let $D = \{(a,a) \mid a\in A\}$ and 
$D^2_\alpha = \{((a,a), (b,b)) \in D^2 \mid a\alphar b\}$.
Finally, let
$\Delta_{\beta, \alpha} = \Cg^{\bbeta}\bigl(D^2_\alpha \bigr)$ be
the congruence on $\bbeta$ generated by the set $D^2_\alpha$.
The congruence class of $\Delta_{\beta, \alpha}$ that contains $(b,b')$ is
denoted and defined as follows:
\[
(b,b')/\Delta_{\beta,\alpha} = \{(a,a') \in \beta \mid (a,a') \mathrel{\Delta_{\beta,\alpha}} (b,b')\}.
\]
Let $\Phi_{\beta, \alpha} \colon \sP(\beta) \to \sP(\beta)$ be the function that
takes each $B \subseteq \beta$ to
\begin{equation}
  \label{eq:6}
  \Phi_{\beta, \alpha}(B) = \bigcup_{(b,b')\in B} (b,b')/\Delta_{\beta, \alpha}.
\end{equation}
%% \newcommand{\Phiba}{\ensuremath{\Phi_{\beta, \alpha}}}
\newcommand{\Phiba}{\ensuremath{\Phi}}
\section{Fixed Point Lemma}
\begin{lemma} Let $\bA$ be an algebra with $\alpha$, $\beta\in \Con(\bA)$.
  If $\Phi:=\Phi_{\beta, \alpha}$ is defined as in~(\ref{eq:6}), then 
  \begin{enumerate}[(i)]
  \item \label{item:1} $\Phiba$ is a closure operator on $\sP(\beta)$;
  \item $[\alpha, \beta]$ is the least fixed point of $\Phiba$.
  \end{enumerate}
\end{lemma}
\begin{proof}\ 
  \begin{enumerate}[(i)]
  \item Fix $B \subseteq \beta$. We must prove the following:
    (a) $B \subseteq \Phiba(B)$;    
    (b) $B \subseteq C  \Rightarrow \Phiba(B) \subseteq \Phiba(C)$;    
    and (c) $\Phiba(\Phiba(B))  = \Phiba(B)$.    
    If $(b,b') \in B$, then $(b,b') \in \Phiba(B)$ since the
    operation~(\ref{eq:6}) does not discard any of the pairs that were already in $B$.
    %% Thus, $G \subseteq \Phiba(G)$ proving~(\ref{item:1}).
    As for (b), if $(a,a') \in \Phiba(B)$, then there exists
    $(b,b') \in B \subseteq C$ such that $(a,a') \mathrel{\Delta_{\beta, \alpha}} (b,b')$.
    Since $(b,b')$ belongs to $C$ we have $(a,a') \in \Phiba(C)$ as well.
    As for (c), it clearly follows from
    (a) and (b) that $\Phiba(B)\subseteq \Phiba(\Phiba(B))$, so we prove the
    reverse inclusion.    
    Let $(d,d') \in \Phiba(\Phiba(B))$. Then
    $(c,c') \mathrel{\Delta_{\beta, \alpha}} (d,d')$ for some
    $(c,c') \in \Phiba(B)$, which implies
    $(b,b') \mathrel{\Delta_{\beta, \alpha}} (c,c')$ for some
    $(b,b') \in B$.  By transitivity of $\Delta_{\beta, \alpha}$ we conclude that
    $(d,d') \in \Phiba(B)$, as desired.
    \bigskip

  \item
    Since, by~(\ref{item:1}), $\Phiba$ is a closure operator on $\sP(\beta)$
    and since $[\alpha, \beta] \in \sP(\beta)$, 
    we have $[\alpha, \beta] \subseteq \Phiba([\alpha, \beta])$.
    We prove the reverse inclusion.
    %% $\Phiba([\alpha, \beta]) \subseteq [\alpha, \beta]$.
    If $(c,c')\in \Phiba([\alpha, \beta])$, then %(We prove $(c,c')\in [\alpha, \beta]$.)
    by~(\ref{eq:6}) there exists $(b,b')\in [\alpha, \beta]$ such that
    \begin{equation}
      \label{eq:1000}
      (b,b') \mathrel{\Delta_{\beta, \alpha}} (c,c').
    \end{equation}
    From the definition of $\Delta_{\beta, \alpha}$ and 
    \malcev's congruence generation theorem, we see that~(\ref{eq:1000})
    holds if and only if
    %% \begin{align*}
    %%   \exists \,& z_i \betar z_i', \quad 0\leq i \leq n,\\
    %%   \exists \,& x_i \alphar y_i, \quad  0\leq i < n,\\
    %%   \exists\, &f_i \in F_{\bA^2}^\ast, \quad 0\leq i < n,
    %% \end{align*}
    $\exists \, z_i \betar z_i'$ $(0\leq i \leq n)$,
    $\exists \, x_i \alphar y_i$ $(0\leq i < n)$,
    $\exists\, f_i \in F_{\bA^2}^\ast$ $(0\leq i < n)$
    such that
    $(b, b') = (z_0,z_0')$ and
    $(z_n,z_n') = (c, c')$, and
    \begin{align}
      \label{eq:0}
      \{(b, b'),(z_1,z_1')\} &= \{f_0(x_0,x_0), f_0(y_0,y_0)\}\\
      \label{eq:1}
      \{(z_1,z_1'),(z_2,z_2')\} &= \{f_1(x_1,x_1), f_1(y_1,y_1)\}\\
      \nonumber
      & \vdots\\
      %% \label{eq:n-1}
      \nonumber
      \{(z_{n-1},z_{n-1}'),(c, c')\} &= \{f_{n-1}(x_{n-1},x_{n-1}), f_{n-1}(y_{n-1},y_{n-1})\}
    \end{align}
    The notation $f_i\in F^*_{\bA^2}$ means 
    \begin{align*}
      f_i(x, x') &= g_i^{\bbeta}((x, x'), (a_1, a_1'), (a_2, a_2'), \dots, (a_k, a_k') )\\
      &= (g_i^{\bA}(x, a_1, a_2, \dots, a_k), g_i^{\bA}(x, a_1', a_2', \dots, a_k')),
    \end{align*}
    for some $g_i^{\bA} \in \sansClo_{k+1}(\bA)$ and some constants 
    $\ba = (a_1, \dots, a_k)$ and $\ba' = (a_1', \dots, a_k')$ such that
    $\ba\mathrel{\beta^k} \ba'$ (that is, $a_i \betar a_i'$ for all $1\leq i\leq k$). 
    By~(\ref{eq:0}), either
    \[
    (b, b') = (g_0^{\bA}(x_0, \ba), g_0^{\bA}(x_0, \ba')
    \quad \text{ and } \quad 
    (z_1,z_1')= (g_0^{\bA}(y_0, \ba), g_0^{\bA}(y_0, \ba')),
    \]
    or vice-versa.  Since $x_0 \alphar y_0$ and 
    $\ba\mathrel{\beta^k} \ba'$, the defining condition of the commutator gives
    \[
    g_0^{\bA}(x_0, \ba) \commr{\alpha}{\beta} g_0^{\bA}(x_0, \ba')
    \quad \Longleftrightarrow \quad 
    g_0^{\bA}(y_0, \ba) \commr{\alpha}{\beta} g_0^{\bA}(y_0, \ba').
    \]
    This and~(\ref{eq:0}) yield
    $(b,b')\in [\alpha, \beta]$ iff
    $(z_1,z_1')\in [\alpha, \beta]$.
    Similarly~(\ref{eq:1}) and $x_1 \alphar y_1$ imply
    $(z_1,z_1')\in [\alpha, \beta]$ iff
    $(z_2,z_2')\in [\alpha, \beta]$.  Continuing in this way we see that
    $(b,b')\in [\alpha, \beta]$ iff $(c,c')\in [\alpha, \beta]$, as desired.

    We have thus proved that $[\alpha, \beta]$ is a fixed point of $\Phiba$.
    Recall that the lattice order on $\Con \bA$ is complete, so the least fixed
    point of a monotone increasing function $f$ defined on $\Con \bA$ is 
    $p = \Meet \{ c \mid f c \leq c\}$. %(See, for example,~\cite{MR3012378}.)
    Thus, we complete the proof by checking that
    \begin{equation}
      \label{eq:2}
    [\alpha, \beta] =\Meet \{ B \subseteq \beta \mid \Phiba(B) \subseteq B\}.
    \end{equation}
  \end{enumerate}

  \vfill
  
\end{proof}
\end{document}

\bibliographystyle{alphaurl}
\bibliography{inputs/refs2.bib}

\end{document}
