%% FILE: math_notes_template.tex
%% AUTHOR: William DeMeo
%% DATE: 25 July 2016
%% COPYRIGHT: (C) 2016 William DeMeo,

%%%%%%%%%%%%%%%%%%%%%%%%%%%%%%%%%%%%%%%%%%%%%%%%%%%%%%%%%%
%%                         BIBLIOGRAPHY FILE            %%
%%%%%%%%%%%%%%%%%%%%%%%%%%%%%%%%%%%%%%%%%%%%%%%%%%%%%%%%%%
%% The `filecontents` command will crete a file in the inputs directory called 
%% refs.bib containing the references in the document, in case this file does 
%% not exist already.
%% If you want to add a BibTeX entry, please don't add it directly to the
%% refs.bib file.  Instead, add it in this file between the
%% \begin{filecontents*}{refs.bib} and \end{filecontents*} lines
%% then delete the existing refs.bib file so it will be automatically generated 
%% again with your new entry the next time you run pdfaltex.
\begin{filecontents*}{inputs/refs.bib}
@book {MR2839398,
    AUTHOR = {Bergman, Clifford},
     TITLE = {Universal algebra},
    SERIES = {Pure and Applied Mathematics (Boca Raton)},
    VOLUME = {301},
      NOTE = {Fundamentals and selected topics},
 PUBLISHER = {CRC Press, Boca Raton, FL},
      YEAR = {2012},
     PAGES = {xii+308},
      ISBN = {978-1-4398-5129-6},
   MRCLASS = {08-02 (06-02 08A40 08B05 08B10 08B26)},
  MRNUMBER = {2839398 (2012k:08001)},
MRREVIEWER = {Konrad P. Pi{\'o}ro},
}
@article {MR0434928,
    AUTHOR = {Taylor, Walter},
     TITLE = {Varieties obeying homotopy laws},
   JOURNAL = {Canad. J. Math.},
  FJOURNAL = {Canadian Journal of Mathematics. Journal Canadien de
              Math\'ematiques},
    VOLUME = {29},
      YEAR = {1977},
    NUMBER = {3},
     PAGES = {498--527},
      ISSN = {0008-414X},
   MRCLASS = {08A25},
  MRNUMBER = {0434928 (55 \#7891)},
MRREVIEWER = {James B. Nation},
}
  @BOOK{HM:1988,
    AUTHOR = {Hobby, David and McKenzie, Ralph},
    TITLE = {The structure of finite algebras},
    SERIES = {Contemporary Mathematics},
    VOLUME = {76},
    PUBLISHER = {American Mathematical Society},
    ADDRESS = {Providence, RI},
    YEAR = {1988},
    PAGES = {xii+203},
    ISBN = {0-8218-5073-3},
    MRCLASS = {08A05 (03C05 08-02 08B05)},
    MRNUMBER = {958685 (89m:08001)},
    MRREVIEWER = {Joel Berman},
    note = {Available from:
      \href{http://math.hawaii.edu/~ralph/Classes/619/HobbyMcKenzie-FiniteAlgebras.pdf}{math.hawaii.edu}}
  }
  @article {Freese:2009,
    AUTHOR = {Freese, Ralph and Valeriote, Matthew A.},
    TITLE = {On the complexity of some {M}altsev conditions},
    JOURNAL = {Internat. J. Algebra Comput.},
    FJOURNAL = {International Journal of Algebra and Computation},
    VOLUME = {19},
    YEAR = {2009},
    NUMBER = {1},
    PAGES = {41--77},
    ISSN = {0218-1967},
    MRCLASS = {08B05 (03C05 08B10 68Q25)},
    MRNUMBER = {2494469 (2010a:08008)},
    MRREVIEWER = {Clifford H. Bergman},
    DOI = {10.1142/S0218196709004956},
    URL = {http://dx.doi.org/10.1142/S0218196709004956}
  }
@article {MR3076179,
    AUTHOR = {Kearnes, Keith A. and Kiss, Emil W.},
     TITLE = {The shape of congruence lattices},
   JOURNAL = {Mem. Amer. Math. Soc.},
  FJOURNAL = {Memoirs of the American Mathematical Society},
    VOLUME = {222},
      YEAR = {2013},
    NUMBER = {1046},
     PAGES = {viii+169},
      ISSN = {0065-9266},
      ISBN = {978-0-8218-8323-5},
   MRCLASS = {08B05 (08B10)},
  MRNUMBER = {3076179},
MRREVIEWER = {James B. Nation},
       DOI = {10.1090/S0065-9266-2012-00667-8},
       URL = {http://dx.doi.org/10.1090/S0065-9266-2012-00667-8},
}
@incollection {MR1404955,
    AUTHOR = {Kearnes, Keith A.},
     TITLE = {Idempotent simple algebras},
 BOOKTITLE = {Logic and algebra ({P}ontignano, 1994)},
    SERIES = {Lecture Notes in Pure and Appl. Math.},
    VOLUME = {180},
     PAGES = {529--572},
 PUBLISHER = {Dekker, New York},
      YEAR = {1996},
   MRCLASS = {08B05 (06F25 08A05 08A30)},
  MRNUMBER = {1404955 (97k:08004)},
MRREVIEWER = {E. W. Kiss},
}
@misc{william_demeo_2016_53936,
  author       = {William DeMeo and
                  Ralph Freese},
  title        = {AlgebraFiles v1.0.1},
  month        = May,
  year         = 2016,
  doi          = {10.5281/zenodo.53936},
  url          = {http://dx.doi.org/10.5281/zenodo.53936}
}
@article{FreeseMcKenzie2016,
	Author = {Freese, R. and McKenzie, R.},
	Date-Added = {2016-08-22 19:43:56 +0000},
	Date-Modified = {2016-08-22 19:45:50 +0000},
	Journal = {Algebra Universalis},
	Title = {Maltsev families of varieties closed under join or Maltsev product},
	Year = {to appear}
}
@article {MR2333368,
    AUTHOR = {Kearnes, Keith A. and Tschantz, Steven T.},
     TITLE = {Automorphism groups of squares and of free algebras},
   JOURNAL = {Internat. J. Algebra Comput.},
  FJOURNAL = {International Journal of Algebra and Computation},
    VOLUME = {17},
      YEAR = {2007},
    NUMBER = {3},
     PAGES = {461--505},
      ISSN = {0218-1967},
   MRCLASS = {08A35 (08B20 20B25)},
  MRNUMBER = {2333368},
MRREVIEWER = {Giovanni Ferrero},
       DOI = {10.1142/S0218196707003615},
       URL = {http://dx.doi.org/10.1142/S0218196707003615},
}
@article {MR2504025,
    AUTHOR = {Valeriote, Matthew A.},
     TITLE = {A subalgebra intersection property for congruence distributive
              varieties},
   JOURNAL = {Canad. J. Math.},
  FJOURNAL = {Canadian Journal of Mathematics. Journal Canadien de
              Math\'ematiques},
    VOLUME = {61},
      YEAR = {2009},
    NUMBER = {2},
     PAGES = {451--464},
      ISSN = {0008-414X},
     CODEN = {CJMAAB},
   MRCLASS = {08B10 (08A30 08B05)},
  MRNUMBER = {2504025},
MRREVIEWER = {Jarom{\'{\i}}r Duda},
       DOI = {10.4153/CJM-2009-023-2},
       URL = {http://dx.doi.org/10.4153/CJM-2009-023-2},
}
@misc{UACalc,
	Author = {Ralph Freese and Emil Kiss and Matthew Valeriote},
	Date-Added = {2014-11-20 01:52:20 +0000},
	Date-Modified = {2014-11-20 01:52:20 +0000},
	Note = {Available at: {\verb+www.uacalc.org+}},
	Title = {Universal {A}lgebra {C}alculator},
	Year = {2011}
}
@article{Freese2008,
	Author = {Freese, R.},
	Date-Added = {2016-08-29 01:31:23 +0000},
	Date-Modified = {2016-08-29 01:32:09 +0000},
	Journal = {Alg. Univ.},
	Pages = {337--343},
	Title = {Computing congruences efficiently},
	Volume = {59},
	Year = {2008}
}	
@incollection {MR1191235,
    AUTHOR = {Szendrei, {\'A}.},
     TITLE = {A survey on strictly simple algebras and minimal varieties},
 BOOKTITLE = {Universal algebra and quasigroup theory ({J}adwisin, 1989)},
    SERIES = {Res. Exp. Math.},
    VOLUME = {19},
     PAGES = {209--239},
 PUBLISHER = {Heldermann, Berlin},
      YEAR = {1992},
   MRCLASS = {08-02 (08A40 08B05)},
  MRNUMBER = {1191235 (93h:08001)},
MRREVIEWER = {Ivan Chajda},
}
\end{filecontents*}
%:biblio
\documentclass[11pt]{amsart}
% The following \documentclass options may be useful:
% preprint      Remove this option only once the paper is in final form.
% 10pt          To set in 10-point type instead of 9-point.
% 11pt          To set in 11-point type instead of 9-point.
% numbers       To obtain numeric citation style instead of author/year.

%% \usepackage{setspace}\onehalfspacing

\usepackage{amsmath}
\usepackage{amscd,amssymb,amsthm} %, amsmath are included by default
\usepackage{latexsym,stmaryrd,mathrsfs,enumerate,scalefnt,ifthen}
\usepackage{mathtools}
\usepackage[mathcal]{euscript}
\usepackage[colorlinks=true,urlcolor=black,linkcolor=black,citecolor=black]{hyperref}
\usepackage{url}
\usepackage{scalefnt}
\usepackage{tikz}
\usepackage{color}
\usepackage[margin=1in]{geometry}
\usepackage{scrextend}

%%////////////////////////////////////////////////////////////////////////////////
%% Theorem styles
\numberwithin{equation}{section}
\theoremstyle{plain}
\newtheorem{theorem}{Theorem}[section]
\newtheorem{lemma}[theorem]{Lemma}
\newtheorem{proposition}[theorem]{Proposition}
\newtheorem{prop}[theorem]{Proposition}
\theoremstyle{definition}
\newtheorem{claim}[theorem]{Claim}
\newtheorem{corollary}[theorem]{Corollary}
\newtheorem{definition}[theorem]{Definition}
\newtheorem{notation}[theorem]{Notation}
\newtheorem{Fact}[theorem]{Fact}
\newtheorem*{fact}{Fact}
\newtheorem{example}[theorem]{Example}
\newtheorem{examples}[theorem]{Examples}
\newtheorem{exercise}{Exercise}
\newtheorem*{lem}{Lemma}
\newtheorem*{cor}{Corollary}
\newtheorem*{remark}{Remark}
\newtheorem*{remarks}{Remarks}
\newtheorem*{obs}{Observation}


%%%%%%%%%%%%%%%%%%%%%%%%%%%%%%%%%%%%%%%%%%%%%%%%%%%%%%%%%%%%%%%%%

\usepackage{inputs/proof-dashed}


%%%%%%%%%%%%%%%%%%%%%%%%%%%%%%%%%%%%%%%%%%%%%%%%%%%%%%%%%%%%%%%%%

%% Put new macros in the macros.sty file
\usepackage{inputs/macros}

\usepackage[backend=bibtex]{biblatex}
\bibliography{inputs/refs.bib}

\begin{document}

\title[Difference Terms and Semilattice Terms]{%
  Notes on Deciding Existence of \\ Difference Terms and Semilattice Terms}
\date{\today}
\author[wjd]{DeMeo}
\address{University of Hawaii}
\email{williamdemeo@gmail.com}
\author[rsf]{Freese}
\email{ralph@math.hawaii.edu}
\author[ag]{Guillen}
\email{guillena@math.hawaii.edu}
\author[th]{Holmes}
\email{tristanh@hawaii.edu}
\author[wal]{Lampe}
\email{bill@math.hawaii.edu}
\author[jbn]{Nation}
\email{jb@math.hawaii.edu}


%% \thanks{The authors would like to extend special thanks to...}

\maketitle

%% \begin{abstract}\end{abstract}

\section{Introduction}
\label{sec:introduction}
The question motivating this effort is the following:
\begin{quote}
{\bf Problem 1:} Is there a polynomial-time algorithm to decide for a finite,
idempotent algebra $\bA$ if $V(\bA)$ has a difference term.
\end{quote}

Having a difference term is equivalent to $V(\bA)$ having a Taylor term
and no type-2 tails. Briefly: no 1's and no type-2 tails.
No 1's is poly-time decidable by Valeriote's subtype theorem.
Consider an analogous problem:
\begin{quote}
  {\bf Problem 2:}
  Is there a polynomial-time algorithm to decide for a finite,
  idempotent algebra $\bA$ if $V(\bA)$ is \ac{cm}?
\end{quote}
In~\cite{Freese:2009}, Freese and Valeriote showed that the answer is yes.
Congruence modularity is characterized by no 1's, no 5's and no tails.
Again no 1's and no 5's can be decided by the subtype theorem,
and in~\cite{Freese:2009} the authors prove that if there is
a tail in $V(\bA)$, there is a tail ``near the bottom.''
More precisely, if $\bA$ is finite and idempotent, and $V(\bA)$ has no
1's or 5's and has tails, then there is a tail in a 3-generated subalgebra of $\bA^2$.
Using this we see that deciding \cm is polynomial-time.

But the proof of the no tails part uses that in a variety with no 1's or 5's,
the congruence lattice modulo the solvability congruence is (join) semidistributive.
Now, restricting to just testing no type-2 tails (vs no tails of any type) is
not a problem. So, for example, there is a poly-time algorithm for testing if
$V(\bA)$ has no 1's, no 5's and no type-2 tails.  

Here is a related problem:
\begin{quote}
  {\bf Problem 3:}
  Is there an $\bA$, idempotent and having a Taylor term, no type-2 tail in 
  subalgebras of $\bA^k$, for $k < n$, but having a type-2 tail in a subalgebra
  of $\bA^n$. 
\end{quote}
Perhaps we can construct such an algebra using congruence lattice
representation techniques. 

%% Hobby and McKenzie give some info about the types in a $D_2$
%% embedded in $\Con (\bA)$. (See~\cite[Lemma 6.3]{HM:1988}).
%% Exercise 7 of that section considers 4-element
%% algebras whose congruence lattice is the concrete embedding of $D_2$
%% in $\Eq(4)$; the one with coatoms $01|23$, $02|13$, and $0|123$, and with atoms
%% $0|1|23$ and $0|2|13$. By~\cite[Lemma 6.3]{HM:1988}, the middle-top
%% interval must be type 5 (assuming a Taylor term). But all the others can be 5's,
%% or all the others can be 4's, or all the others can be 3's.
%% One might attempt to find an example where they are all 2's, but that not possible
%% since otherwise $0|123$ would be a solvable congruence,
%% which would imply the two atoms would permute.


\section{Background, definitions, and notations}
\label{sec:defin-notat}
Our starting point is the set of lemmas at the beginning of Section 3 in
the Freese-Valeriote paper~\cite{Freese:2009}.
We first review some of the basic \ac{tct}
that comes up in the proofs in that paper.
The main reference for \tct is of course the book by Hobby and McKenzie
\cite{HM:1988} which taught us that, for each covering 
$\alpha \prec \beta$ in the congruence lattice of a finite algebra
$\bA$, the local behaviour of the $\beta$-classes is captured by the
so-called $(\alpha, \beta)$-traces~\cite[Definition 2.15]{HM:1988};
modulo $\alpha$, the induced structure on the traces is limited to one
of five possible types:

\begin{enumerate}[(1)]
\item unary algebra whose basic operations are all permutations (unary type);
\item one-dimensional vector space over some finite field (affine type);
\item 2-element boolean algebra (boolean type);
\item 2-element lattice (lattice type);
\item 2-element semilattice (semilattice type).
\end{enumerate}

Thus to each covering $\alpha \prec \beta$
corresponds a type in $\{1,2,3,4,5\}$ (see~\cite[Definition 5.1]{HM:1988}),
denote $\typ(\alpha, \beta)$.
The set of all types that are realized by covering pairs of congruences of a
finite algebra $\bA$ is denoted by $\typ\{\bA\}$, and if $\sK$ is a class of algebras,
then $\typ\{\sK\}$ denotes the union of the typesets of all finite algebras in $\sK$.
The set of types is ordered by the lattice of types given here:

\newcommand{\dotsize}{0.8pt}
%% To create nodes of lattices in a uniform and consistent way, we define
\tikzstyle{lat} = [circle,draw,inner sep=\dotsize]
% To scale all diagrams uniformly, change this setting:
\begin{center}
  
\newcommand{\figscale}{.8}
\begin{tikzpicture}[scale=\figscale]
  \node[lat] (1) at (0,0) {};
  \node[lat] (2) at (-1,1.5) {};
  \node[lat] (3) at (0,3) {};
  \node[lat] (4) at (.8,2.1) {};
  \node[lat] (5) at (.8,.9) {};
  \draw (1) node [below] {$1$};
  \draw (2) node [left] {$2$};
  \draw (3) node [above] {$3$};
  \draw (4) node [right] {$4$};
  \draw (5) node [right] {$5$};
  \draw[semithick] 
  (1) -- (2) -- (3) -- (4) -- (5) -- (1);
\end{tikzpicture}

\end{center}

The following results demonstrate that, for a finite idempotent algebra $\bA$,
whether or not $V(\bA)$ omits one of the order ideals of the lattice of types can be
determined locally. An algebra is strictly simple if it is simple (i.e. has no non-trivial
congruences) and has no non-trivial subalgebras (i.e. has no proper subalgebras with
more than one element).

\begin{prop}[Proposition 2.1~\cite{Freese:2009}]
If A is a finite idempotent algebra and $i \in \typ(V(\bA))$ then there
is a finite strictly simple algebra $\bS$ of type $j$ for some $j \leq i$ in $\sansH \sansS (\bA)$.
If
\begin{enumerate}[(1)]
\item 
  $j = 1$ then $\bS$ is term equivalent to a 2-element set;
\item
  $j = 2$ then $\bS$ is term equivalent to the idempotent reduct of a module;
\item
  $j = 3$ then $\bS$ is functionally complete;
\item
  $j = 4$ then $\bS$ is polynomially equivalent to a 2-element lattice;
\item
  $j = 5$ then $\bS$ is term equivalent to a 2-element semilattice.
\end{enumerate}
\end{prop}
\begin{proof}
  This is a combination of~\cite[Proposition 3.1]{MR2504025} and~\cite[Theorem 6.1]{MR1191235}.
\end{proof}

\begin{corollary}[Corollary 2.2~\cite{Freese:2009}]
  \label{cor:2.2}
  Let $\bA$ be a finite idempotent algebra and $T$ an order ideal in the
  lattice of types. Then $V(\bA)$ omits $T$ if and only if $\sansS(\bA)$ does.
  In particular, $V(\bA)$ omits 1 and 2 if and only if $S(\bA)$ omits 1 and 2.
\end{corollary}

%% The following lemma ties in with the previous proposition and will be used
%% in Sec. 6.
%% \begin{lemma}[Lemma 2.3~\cite{Freese:2009}] 
%%   Let $\bA$ be a finite idempotent algebra and let $\bS \in \sansH \sansS(\bA)$
%%   be strictly simple. Then there are elements $a, b \in A$ such that, if
%%   $\bB = \Sg^{\bA} (a, b)$, then $1_B = \Cg^{\bB} (a, b)$ and is join irreducible
%%   with unique lower cover $\rho$ such that $\bS = \bB/\rho$.
%% \end{lemma}
%% \begin{proof}
%%   Choose $\bB \in \sansS (\bA)$ as small as possible having $\bS$ as a homomorphic image,
%%   say $\bS = \bB/\rho$. We claim that if $a, b \in B$ with
%%   $(a, b) \in \notin \rho$ then they generate $\bB$. To
%%   see this, let $\bB'= \Sg^{\bB} (a, b)$ and let $h$ be the quotient map from B to S with kernel
%%   ρ. Then h(B  ) is a non-trivial subuniverse of S and so must equal S. Thus B  = B.
%%   Now let a, b ∈ B with (a, b) ∈
%%   / ρ. Since the block of Cg B (a, b) containing a
%%   and b is a subuniverse of B then from the previous paragraph, we conclude that
%%   Cg B (a, b) = 1 B and that ρ is its unique lower cover.
%% \end{proof}

Corollary~\ref{cor:2.2} is the starting point of the development in~\cite{Freese:2009}
of a polynomial-time algorithm that determines if a given finite idempotent algebra
generates a \cm variety. According to the characterization
in~\cite[Chapter 8]{HM:1988} of locally finite congruence modular (distributive) varieties,
a finite algebra $\bA$ generates a congruence modular (distributive) variety $\sV$ if and
only if the typeset of $\sV$ is contained in $\{2, 3, 4\}$ ($\{3, 4\}$) and all
minimal sets of prime quotients of finite algebras in $\sV$ have empty
tails~\cite[Definition 2.15]{HM:1988}. Note that in the \cd
case the empty tails condition is equivalent to the minimal sets all having exactly
two elements.
It follows from Corollary~\ref{cor:2.2} that if $\bA$ is idempotent then one can
test the first condition, on omitting types 1 and 5 (or 1, 2, and 5) by searching
for a 2-generated subalgebra of $\bA$ whose typeset is not contained in
$\{2, 3, 4\}$ ($\{3, 4\}$). In~\cite[Sec. 6]{Freese:2009}, the authors prove that this
test can be performed in polynomial time, as a function of the size of $\bA$.
The following sequence of lemmas establishes that when A is finite, idempotent
and $\sV = V (\bA)$ omits types 1 and 5, then to test for the existence of tails in
$\sV$ we need only look for them in the 3-generated subalgebras of $\bA^2$.

Throughout the remainder of this section, let $\sS$ be a finite set of finite,
similar, idempotent algebras closed under the taking of subalgebras such that
$\sV = V (\sS)$ omits 1 and 5. We will suppose that some finite algebra
$\bB$ in $\sV$ has a prime quotient whose minimal sets have non-empty tails
and show that there is a 3-generated subalgebra of the product of two members
of $\sS$ with this property. Since $\sS$ is closed under the taking of subalgebras,
we may assume that the algebra $\bB$ from the previous paragraph is a subdirect
product of a finite number of members of $\sS$. Choose $n$ minimal such that for
some $\bA_i \in \sS$, the product $\prod_{i\leq n} \bA_i$
has a subdirect product $\bB$ that has a prime quotient with non-empty tails.
Under the assumption that $n > 1$ we will prove that $n = 2$.

For this $n$, select the $\bA_i$ and $\bB$ so that $|B|$ is as small as possible.
Let $\alpha \prec \beta$ be a prime quotient of $\bB$ with non-empty tails and choose
$\beta$ minimal with this property.
Let $U$ be an $(\alpha, \beta)$-minimal set and let $N$ be an
 $(\alpha, \beta)$-trace of $U$. Let 0 and 1 be
two distinct members of $N$ with $(0, 1) \notin \alpha$.

\begin{lemma}[Lemma 3.1~\cite{Freese:2009}]
Let $t$ be a member of the tail of $U$. Then $\beta$ is the congruence of $\bB$
generated by the pair $(0, 1)$ and $\bB$ is generated by $\{0, 1, t\}$.
\end{lemma}















%% \appendix
%% \section{Appendix Title}
%% This is the text of the appendix, if you need one.

%\bibliographystyle{amsplain} %% or amsalpha
%% \bibliographystyle{plain-url}
\printbibliography


\end{document}
