%% FILE: math_notes_template.tex
%% AUTHOR: William DeMeo
%% DATE: 25 July 2016
%% COPYRIGHT: (C) 2016 William DeMeo,

%%%%%%%%%%%%%%%%%%%%%%%%%%%%%%%%%%%%%%%%%%%%%%%%%%%%%%%%%%
%%                         BIBLIOGRAPHY FILE            %%
%%%%%%%%%%%%%%%%%%%%%%%%%%%%%%%%%%%%%%%%%%%%%%%%%%%%%%%%%%
%% The `filecontents` command will crete a file in the inputs directory called 
%% refs.bib containing the references in the document, in case this file does 
%% not exist already.
%% If you want to add a BibTeX entry, please don't add it directly to the
%% refs.bib file.  Instead, add it in this file between the
%% \begin{filecontents*}{refs.bib} and \end{filecontents*} lines
%% then delete the existing refs.bib file so it will be automatically generated 
%% again with your new entry the next time you run pdfaltex.
\begin{filecontents*}{inputs/refs.bib}
@article {MR3350327,
    AUTHOR = {Kozik, Marcin and Krokhin, Andrei and Valeriote, Matt and
              Willard, Ross},
     TITLE = {Characterizations of several {M}altsev conditions},
   JOURNAL = {Algebra Universalis},
  FJOURNAL = {Algebra Universalis},
    VOLUME = {73},
      YEAR = {2015},
    NUMBER = {3-4},
     PAGES = {205--224},
      ISSN = {0002-5240},
   MRCLASS = {08B05 (08A70 08B10)},
  MRNUMBER = {3350327},
MRREVIEWER = {David Hobby},
       DOI = {10.1007/s00012-015-0327-2},
       URL = {http://dx.doi.org/10.1007/s00012-015-0327-2},
}
@article {MR1358491,
    AUTHOR = {Kearnes, Keith A.},
     TITLE = {Varieties with a difference term},
   JOURNAL = {J. Algebra},
  FJOURNAL = {Journal of Algebra},
    VOLUME = {177},
      YEAR = {1995},
    NUMBER = {3},
     PAGES = {926--960},
      ISSN = {0021-8693},
     CODEN = {JALGA4},
   MRCLASS = {08B10 (08B05)},
  MRNUMBER = {1358491},
MRREVIEWER = {H. Peter Gumm},
       DOI = {10.1006/jabr.1995.1334},
       URL = {http://dx.doi.org/10.1006/jabr.1995.1334},
}
@book {MR2839398,
    AUTHOR = {Bergman, Clifford},
     TITLE = {Universal algebra},
    SERIES = {Pure and Applied Mathematics (Boca Raton)},
    VOLUME = {301},
      NOTE = {Fundamentals and selected topics},
 PUBLISHER = {CRC Press, Boca Raton, FL},
      YEAR = {2012},
     PAGES = {xii+308},
      ISBN = {978-1-4398-5129-6},
   MRCLASS = {08-02 (06-02 08A40 08B05 08B10 08B26)},
  MRNUMBER = {2839398 (2012k:08001)},
MRREVIEWER = {Konrad P. Pi{\'o}ro},
}
@article {MR0434928,
    AUTHOR = {Taylor, Walter},
     TITLE = {Varieties obeying homotopy laws},
   JOURNAL = {Canad. J. Math.},
  FJOURNAL = {Canadian Journal of Mathematics. Journal Canadien de
              Math\'ematiques},
    VOLUME = {29},
      YEAR = {1977},
    NUMBER = {3},
     PAGES = {498--527},
      ISSN = {0008-414X},
   MRCLASS = {08A25},
  MRNUMBER = {0434928 (55 \#7891)},
MRREVIEWER = {James B. Nation},
}
  @BOOK{HM:1988,
    AUTHOR = {Hobby, David and McKenzie, Ralph},
    TITLE = {The structure of finite algebras},
    SERIES = {Contemporary Mathematics},
    VOLUME = {76},
    PUBLISHER = {American Mathematical Society},
    ADDRESS = {Providence, RI},
    YEAR = {1988},
    PAGES = {xii+203},
    ISBN = {0-8218-5073-3},
    MRCLASS = {08A05 (03C05 08-02 08B05)},
    MRNUMBER = {958685 (89m:08001)},
    MRREVIEWER = {Joel Berman},
    note = {Available from:
      \href{http://math.hawaii.edu/~ralph/Classes/619/HobbyMcKenzie-FiniteAlgebras.pdf}{math.hawaii.edu}}
  }
  @article {Freese:2009,
    AUTHOR = {Freese, Ralph and Valeriote, Matthew A.},
    TITLE = {On the complexity of some {M}altsev conditions},
    JOURNAL = {Internat. J. Algebra Comput.},
    FJOURNAL = {International Journal of Algebra and Computation},
    VOLUME = {19},
    YEAR = {2009},
    NUMBER = {1},
    PAGES = {41--77},
    ISSN = {0218-1967},
    MRCLASS = {08B05 (03C05 08B10 68Q25)},
    MRNUMBER = {2494469 (2010a:08008)},
    MRREVIEWER = {Clifford H. Bergman},
    DOI = {10.1142/S0218196709004956},
    URL = {http://dx.doi.org/10.1142/S0218196709004956}
  }
@article {MR3076179,
    AUTHOR = {Kearnes, Keith A. and Kiss, Emil W.},
     TITLE = {The shape of congruence lattices},
   JOURNAL = {Mem. Amer. Math. Soc.},
  FJOURNAL = {Memoirs of the American Mathematical Society},
    VOLUME = {222},
      YEAR = {2013},
    NUMBER = {1046},
     PAGES = {viii+169},
      ISSN = {0065-9266},
      ISBN = {978-0-8218-8323-5},
   MRCLASS = {08B05 (08B10)},
  MRNUMBER = {3076179},
MRREVIEWER = {James B. Nation},
       DOI = {10.1090/S0065-9266-2012-00667-8},
       URL = {http://dx.doi.org/10.1090/S0065-9266-2012-00667-8},
}
@incollection {MR1404955,
    AUTHOR = {Kearnes, Keith A.},
     TITLE = {Idempotent simple algebras},
 BOOKTITLE = {Logic and algebra ({P}ontignano, 1994)},
    SERIES = {Lecture Notes in Pure and Appl. Math.},
    VOLUME = {180},
     PAGES = {529--572},
 PUBLISHER = {Dekker, New York},
      YEAR = {1996},
   MRCLASS = {08B05 (06F25 08A05 08A30)},
  MRNUMBER = {1404955 (97k:08004)},
MRREVIEWER = {E. W. Kiss},
}
@misc{william_demeo_2016_53936,
  author       = {DeMeo, William and Freese, Ralph},
  title        = {AlgebraFiles v1.0.1},
  month        = May,
  year         = 2016,
  doi          = {10.5281/zenodo.53936},
  url          = {http://dx.doi.org/10.5281/zenodo.53936}
}
@article{FreeseMcKenzie2016,
	Author = {Freese, Ralph and McKenzie, Ralph},
	Date-Added = {2016-08-22 19:43:56 +0000},
	Date-Modified = {2016-08-22 19:45:50 +0000},
	Journal = {Algebra Universalis},
	Title = {Mal'tsev families of varieties closed under join or Mal'tsev product},
	Year = {to appear}
}
@article {MR2333368,
    AUTHOR = {Kearnes, Keith A. and Tschantz, Steven T.},
     TITLE = {Automorphism groups of squares and of free algebras},
   JOURNAL = {Internat. J. Algebra Comput.},
  FJOURNAL = {International Journal of Algebra and Computation},
    VOLUME = {17},
      YEAR = {2007},
    NUMBER = {3},
     PAGES = {461--505},
      ISSN = {0218-1967},
   MRCLASS = {08A35 (08B20 20B25)},
  MRNUMBER = {2333368},
MRREVIEWER = {Giovanni Ferrero},
       DOI = {10.1142/S0218196707003615},
       URL = {http://dx.doi.org/10.1142/S0218196707003615},
}
@article {MR2504025,
    AUTHOR = {Valeriote, Matthew A.},
     TITLE = {A subalgebra intersection property for congruence distributive
              varieties},
   JOURNAL = {Canad. J. Math.},
  FJOURNAL = {Canadian Journal of Mathematics. Journal Canadien de
              Math\'ematiques},
    VOLUME = {61},
      YEAR = {2009},
    NUMBER = {2},
     PAGES = {451--464},
      ISSN = {0008-414X},
     CODEN = {CJMAAB},
   MRCLASS = {08B10 (08A30 08B05)},
  MRNUMBER = {2504025},
MRREVIEWER = {Jarom{\'{\i}}r Duda},
       DOI = {10.4153/CJM-2009-023-2},
       URL = {http://dx.doi.org/10.4153/CJM-2009-023-2},
}
@misc{UACalc,
	Author = {Ralph Freese and Emil Kiss and Matthew Valeriote},
	Date-Added = {2014-11-20 01:52:20 +0000},
	Date-Modified = {2014-11-20 01:52:20 +0000},
	Note = {Available at: {\verb+www.uacalc.org+}},
	Title = {Universal {A}lgebra {C}alculator},
	Year = {2011}
}
@article{Freese2008,
	Author = {Freese, Ralph},
	Date-Added = {2016-08-29 01:31:23 +0000},
	Date-Modified = {2016-08-29 01:32:09 +0000},
	Journal = {Alg. Univ.},
	Pages = {337--343},
	Title = {Computing congruences efficiently},
	Volume = {59},
	Year = {2008}
}	
@incollection {MR1191235,
    AUTHOR = {Szendrei, {\'A}gnes.},
     TITLE = {A survey on strictly simple algebras and minimal varieties},
 BOOKTITLE = {Universal algebra and quasigroup theory ({J}adwisin, 1989)},
    SERIES = {Res. Exp. Math.},
    VOLUME = {19},
     PAGES = {209--239},
 PUBLISHER = {Heldermann, Berlin},
      YEAR = {1992},
   MRCLASS = {08-02 (08A40 08B05)},
  MRNUMBER = {1191235 (93h:08001)},
MRREVIEWER = {Ivan Chajda},
}
@unpublished{Bergman-DeMeo,
    AUTHOR = {Bergman, Clifford and DeMeo, William},
    TITLE = {Universal Algebraic Methods for Constraint Satisfaction Problems:
      with applications to commutative idempotent binars},
    YEAR = {2016},
    NOTE = {unpublished notes; soon to be available online},
    URL = {https://github.com/UniversalAlgebra/algebraic-csp}
}
\end{filecontents*}
%:biblio
\documentclass[12pt]{amsart}
% The following \documentclass options may be useful:
% preprint      Remove this option only once the paper is in final form.
% 10pt          To set in 10-point type instead of 9-point.
% 11pt          To set in 11-point type instead of 9-point.
% numbers       To obtain numeric citation style instead of author/year.

\usepackage{setspace}\onehalfspacing
\usepackage[normalem]{ulem} % for \sout (strikeout)
\usepackage{amsmath}
\usepackage{amscd,amssymb,amsthm} %, amsmath are included by default
\usepackage{latexsym,stmaryrd,mathrsfs,enumerate,scalefnt,ifthen}
\usepackage{mathtools}
\usepackage[mathcal]{euscript}
\usepackage[colorlinks=true,urlcolor=black,linkcolor=black,citecolor=black]{hyperref}
\usepackage{url}
\usepackage{scalefnt}
\usepackage{tikz}
\usepackage{color}
%% \usepackage[margin=1in]{geometry}
\usepackage{geometry}
\usepackage{marginnote}
\usepackage{scrextend}

%%////////////////////////////////////////////////////////////////////////////////
%% Theorem styles
\numberwithin{equation}{section}
\theoremstyle{plain}
\newtheorem{thm}{Theorem}[section]
\newtheorem{lem}[thm]{Lemma}
%% \newtheorem{proposition}[theorem]{Proposition}
\newtheorem{prop}[thm]{Proposition}
\theoremstyle{definition}
\newtheorem{claim}[thm]{Claim}
\newtheorem{cor}[thm]{Corollary}
\newtheorem{definition}[thm]{Definition}
\newtheorem{notation}[thm]{Notation}
\newtheorem{fact}[thm]{Fact}
\newtheorem{question}{Question}
\newtheorem{prob}{Problem}
\newtheorem{example}[thm]{Example}
\newtheorem{examples}[thm]{Examples}
\newtheorem{exercise}{Exercise}
%% \newtheorem*{lem}{Lemma}
%% \newtheorem*{cor}{Corollary}
\newtheorem*{rem}{Remark}
\newtheorem*{rems}{Remarks}
%% \newtheorem*{obs}{Observation}


%%%%%%%%%%%%%%%%%%%%%%%%%%%%%%%%%%%%%%%%%%%%%%%%%%%%%%%%%%%%%%%%%

\usepackage{inputs/proof-dashed}


%%%%%%%%%%%%%%%%%%%%%%%%%%%%%%%%%%%%%%%%%%%%%%%%%%%%%%%%%%%%%%%%%

%% Put new macros in the macros.sty file
\usepackage{inputs/macros}

%% \usepackage[backend=bibtex]{biblatex}
%% \bibliography{inputs/refs.bib}

\begin{document}

\title[Difference Terms and Semilattice Terms]{%
  Notes on Complexity of Deciding Existence of \\ Difference Terms and Semilattice Terms}
\date{\today}
\author{DeMeo}
\address{University of Hawaii}
\email{williamdemeo@gmail.com}
\author{Freese}
\email{ralph@math.hawaii.edu}
\author{Guillen}
\email{guillena@math.hawaii.edu}
\author{Holmes}
\email{tristanh@hawaii.edu}
\author{Lampe}
\email{bill@math.hawaii.edu}
\author{Nation}
\email{jb@math.hawaii.edu}


%% \thanks{The authors would like to extend special thanks to...}

\maketitle

%% \begin{abstract}\end{abstract}

\section{Introduction}
\label{sec:introduction}
The question motivating this effort is the following:
\begin{prob}
  \label{prob:1}
  Is there a polynomial-time algorithm to decide for a finite,
  idempotent algebra $\bA$ if $\bbV(\bA)$ has a difference term.
\end{prob}
Kearnes proved in~\cite{MR1358491} that $\bA$ has a difference term iff
$\bbV(\bA)$ has a Taylor term and no type-2 tails
(equivalently, $\bbV(\bA)$ has no 1's and no type-2 tails).
No 1's is poly-time decidable by Valeriote's subtype theorem.
In~\cite{Freese:2009}, Freese and Valeriote solved an analogous problem, by
giving a positive answer to the following
\begin{prob}
  \label{prob:2}
  Is there a polynomial-time algorithm to decide for a finite,
  idempotent algebra $\bA$ if $\bbV(\bA)$ is \ac{cm}?
\end{prob}

Congruence modularity is characterized by no 1's, no 5's and no tails.
Again no 1's and no 5's can be decided by the subtype theorem,
and in~\cite{Freese:2009} the authors prove that if there is
a tail in $\bbV(\bA)$, there is a tail ``near the bottom.''
More precisely, if $\bA$ is finite and idempotent, and $\bbV(\bA)$ has no
1's and no 5's and has tails, then there is a tail in a 3-generated subalgebra of $\bA^2$.
Using this it is proved that deciding \cm is polynomial-time.

But the proof of the no tails part uses that in a variety with no 1's or 5's,
the congruence lattice modulo the {\it solvability congruence} (defined below)
is (join) semidistributive.
Now, restricting to just testing no type-2 tails (vs no tails of any type) is
not a problem. So, for example, there is a poly-time algorithm for testing if
$\bbV(\bA)$ has no 1's, no 5's and no type-2 tails.  

Here is a related problem.
\begin{prob}
  \label{prob:3}
  Is there an $\bA$, idempotent and having a Taylor term, no type-2 tail in 
  subalgebras of $\bA^k$, for $k < n$, but having a type-2 tail in a subalgebra
  of $\bA^n$. 
\end{prob}
Perhaps we could construct such an algebra using congruence lattice
representation techniques. 

%% Hobby and McKenzie give some info about the types in a $D_2$
%% embedded in $\Con (\bA)$. (See~\cite[Lemma 6.3]{HM:1988}).
%% Exercise 7 of that section considers 4-element
%% algebras whose congruence lattice is the concrete embedding of $D_2$
%% in $\Eq(4)$; the one with coatoms $01|23$, $02|13$, and $0|123$, and with atoms
%% $0|1|23$ and $0|2|13$. By~\cite[Lemma 6.3]{HM:1988}, the middle-top
%% interval must be type 5 (assuming a Taylor term). But all the others can be 5's,
%% or all the others can be 4's, or all the others can be 3's.
%% One might attempt to find an example where they are all 2's, but that not possible
%% since otherwise $0|123$ would be a solvable congruence,
%% which would imply the two atoms would permute.


\section{Background, definitions, and notations}
\label{sec:defin-notat}
Our starting point is the set of lemmas at the beginning of Section 3 in
the Freese-Valeriote paper~\cite{Freese:2009}.
We first review some of the basic \ac{tct}
that comes up in the proofs in that paper. (In fact, most of this section 
is copied from the nice presentation of \tct background that appears
in~\cite[Sec.~2]{Freese:2009}.)

The reference for \tct is the book by Hobby and McKenzie
\cite{HM:1988}, according to which,
for each covering $\alpha \prec \beta$ in the congruence lattice of a finite
algebra $\bA$, the local behavior of the $\beta$-classes is captured by the
so-called $(\alpha, \beta)$-traces~\cite[Def.~2.15]{HM:1988}.
Modulo $\alpha$, the induced structure on the traces is limited to one
of five possible types:

\begin{enumerate}[(1)]
\item unary algebra whose basic operations are all permutations (unary type);
\item one-dimensional vector space over some finite field (affine type);
\item 2-element boolean algebra (boolean type);
\item 2-element lattice (lattice type);
\item 2-element semilattice (semilattice type).
\end{enumerate}

Thus to each covering $\alpha \prec \beta$
corresponds a ``\tct type'' in $\{1,2,3,4,5\}$ (see~\cite[Def.~5.1]{HM:1988}),
denoted by $\typ(\alpha, \beta)$, called the \emph{typeset} of $\bA$.
The set of all \tct types that are realized by covering pairs of congruences of a
finite algebra $\bA$ is denoted by $\typ\{\bA\}$, and if $\sK$ is a class of algebras,
then $\typ\{\sK\}$ denotes the union of the typesets of all finite algebras in $\sK$.
\tct types are ordered according to the following ``lattice of types:''

\newcommand{\dotsize}{0.8pt}
%% To create nodes of lattices in a uniform and consistent way, we define
\tikzstyle{lat} = [circle,draw,inner sep=\dotsize]
% To scale all diagrams uniformly, change this setting:
\begin{center}
\newcommand{\figscale}{.7}
\begin{tikzpicture}[scale=\figscale]
  \scalefont{.8}
  \node[lat] (1) at (0,0) {};
  \node[lat] (2) at (-1,1.5) {};
  \node[lat] (3) at (0,3) {};
  \node[lat] (4) at (.8,2.1) {};
  \node[lat] (5) at (.8,.9) {};
  \draw (1) node [below] {$1$};
  \draw (2) node [left] {$2$};
  \draw (3) node [above] {$3$};
  \draw (4) node [right] {$4$};
  \draw (5) node [right] {$5$};
  \draw[semithick] 
  (1) -- (2) -- (3) -- (4) -- (5) -- (1);
\end{tikzpicture}
\end{center}
Whether or not $\bbV(\bA)$ omits one of the order ideals of the lattice of types can be
determined locally.  This is spelled out for us in the next proposition.
(A \defn{strictly simple} algebra is a simple
algebra with no non-trivial subalgebras.)
%% ; i.e.~no proper subalgebras with
%% more than one element.)


\begin{prop}[Prop.~2.1~\cite{Freese:2009}]
  \label{prop:2.1}
If A is a finite idempotent algebra and $i \in \typ(\bbV(\bA))$ then there
is a finite strictly simple algebra $\bS$ of type $j$ for some $j \leq i$ in $\sansH \sansS (\bA)$.
If
\begin{enumerate}[(1)]
\item 
  $j = 1$ then $\bS$ is term equivalent to a 2-element set;
\item
  $j = 2$ then $\bS$ is term equivalent to the idempotent reduct of a module;
\item
  $j = 3$ then $\bS$ is functionally complete;
\item
  $j = 4$ then $\bS$ is polynomially equivalent to a 2-element lattice;
\item
  $j = 5$ then $\bS$ is term equivalent to a 2-element semilattice.
\end{enumerate}
\end{prop}
\begin{proof}
  This is a combination of~\cite[Prop.~3.1]{MR2504025} and~\cite[Thm.~6.1]{MR1191235}.
\end{proof}

Table~\ref{tab:1} is from~\cite{MR3350327} and gives another characterization of
omitting types.
\begin{center}
  \begin{table}
    \caption{\cite{MR3350327}.}
    \label{tab:1}
    \begin{tabular}{|l|l|}
      \hline
      Omitting Class &  Equivalent Property\\
      \hline
      $\sM_{\{1\}}$ & satisfies a nontrivial idempotent \malcev condition \\
      \hline
      $\sM_{\{1,5\}}$ & satisfies a nontrivial congruence identity\\ % (see~\cite{MR3076179})\\
      \hline
      $\sM_{\{1,4,5\}}$ & congruence n-permutable, for some $n > 1$ \\
      \hline
      $\sM_{\{1,2\}}$ & congruence meet semidistributive \\
      \hline
      $\sM_{\{1,2,5\}}$ & congruence join semidistributive\\ % (see~\cite{MR3076179})\\
      \hline
      $\sM_{\{1,2,4,5\}}$ & congruence $n$-permutable for some $n$ and\\
      &congruence join semidistributive\\
      \hline
    \end{tabular}
  \end{table}
\end{center}

In Section~\ref{sec:freese-valer-lemm}, the following result will be useful.
\begin{cor}[Cor.~2.2~\cite{Freese:2009}]
  \label{cor:2.2}
  Let $\bA$ be a finite idempotent algebra and $T$ an order ideal in the
  lattice of types. Then $\bbV(\bA)$ omits $T$ if and only if $\sansS(\bA)$ does.
  %% In particular, $\bbV(\bA)$ omits 1 and 2 if and only if $\sansS(\bA)$ omits 1 and 2.
\end{cor}

\subsection{Equivalent conditions for existence of a difference term}
%% Let $\sV$ be a variety.  
In~\cite{MR1358491} Kearnes proved that
a locally finite variety has a difference term
iff it has a Taylor term and no type-2 tails.
Let $\sV$ be a variety and let $\bF = \bF_{\sV}(2)$ denote the 2-generated
free algebra in $\sV$.
Then the assumption that $\sV$ be locally finite can be weakened
to the hypothesis that $\bF$ is finite. This was observed in~\cite{MR1358491} 
by showing that $\sV$ has a difference term if and only if $\sansH \sansS \sansP(\bF)$
has a difference term.
The forward implication of this claim is trivial. 
The argument for the converse goes as follows:
assume that $d(x, y, z)$ is a difference term for $\sansH \sansS \sansP(\bF)$.
Choose $\bA \in \sV$ and $a, b \in A$. Let $\bB = \Sg^{\bA} (\{a, b\})$.
Since $\bB$ is 2-generated, $B \in \sansH \sansS \sansP (\bF)$.
Hence $d(x, y, z)$ interprets as a difference term in $\bB$. This means that
$d^{\bA} (a, a, b) = d^{\bB} (a, a, b) = b$.
Furthermore,
\[
d^{\bA} (a, b, b) = d^{\bB} (a, b, b) \mathrel{[\Cg^{\bB} (a, b), \Cg^{\bB} (a, b)]} a.
\]
But $[\Cg^{\bB} (a, b), \Cg^{\bB} (a,b)]\subseteq [\theta, \theta]$ for any congruence
$\theta \in \Con \bA$ for which $(a, b) \in \theta$. Consequently
$d^{\bA} (a, b, b) \mathrel{[\theta, \theta]} a$ as desired.

For the purposes of the present project,
it would be helpful if we could extend this observation and
prove that the existence (or nonexistence) of a difference term in $\sV$
is equivalent to the  existence (or nonexistence) of a difference term
operation for a specific algebra in $\sV$.  In fact, this is possible, as we
now demonstrate.

\begin{thm}
  \label{thm:F}
Let $\sV$ be a variety and $\bF = \bF_{\sV}(2)$ the 2-generated
free algebra in $\sV$. The following are equivalent:
\begin{enumerate}[(i)]
\item \label{item:1}
  $\sV$ has a difference term;
\item \label{item:2}
  $\sansH \sansS \sansP (\bF)$ has a difference term;
\item \label{item:3}
  $\bF$ has a difference term operation.
\end{enumerate}
\end{thm}
\begin{proof}
  The implications
  (\ref{item:1}) $\Rightarrow$  (\ref{item:2}) $\Rightarrow$  (\ref{item:3}) are
  obvious. We prove
  (\ref{item:3}) $\Rightarrow$  (\ref{item:1}) by contraposition.
  Suppose $\sV$ has no difference term. (We show $\bF$ has no difference term
  operation.)
  Let $d(x,y,z)$ be a ternary term of $\sV$.  Let $\bA\in \sV$ be such that
  $d^{\bA}(x,y,z)$ is not a difference term operation in $\bA$.
  Choose $a, b \in A$ witnessing this fact.  Then either
  \begin{enumerate}
  \item\label{item:4} $d^{\bA}(a,a,b) \neq b$, or
  \item\label{item:5} $(d^{\bA}(a,b,b), a) \notin [\Cg^{\bA} (a, b), \Cg^{\bA} (a,b)]$.
  \end{enumerate}
  Let $\bB = \Sg^{\bA} (\{a, b\})$.  In case
  (\ref{item:4}), 
  $d^{\bB}(a,a,b) = d^{\bA}(a,a,b) \neq b$, so $d^{\bB}(x,y,z)$ is not a difference
  term operation for $\bB$.
  In case (\ref{item:5}), observe that
  the pair $(d^{\bB}(a,b,b), a)$ is equal to the pair $(d^{\bA}(a,b,b), a)$ which
  does not belong to $[\Cg^{\bA} (a, b), \Cg^{\bA} (a,b)]$.
  But 
  $[\Cg^{\bB} (a, b), \Cg^{\bB} (a,b)] \subseteq[\Cg^{\bA} (a, b), \Cg^{\bA} (a,b)]$, so
  \[(d^{\bB}(a,b,b), a) \notin [\Cg^{\bB} (a, b), \Cg^{\bB} (a,b)],\]
  and again we conclude that $d^{\bB}(x,y,z)$ is not a difference term operation for $\bB$.
  Now, since there is a surjective homomorphism from $\bF$ to $\bB$,
  it follows that $d^{\bF}(x,y,z)$ cannot be a difference term operation for $\bF$.
  Finally, recall that $d(x,y,z)$ was an arbitrary ternary term of $\sV$, so
  $\bF$ has no difference term operation whatsoever.
  %% , as we set out to prove.
\end{proof}


\section{The Freese-Valeriote Lemmas Revisited}
\label{sec:freese-valer-lemm}
In~\cite{Freese:2009}, Corollary~\ref{cor:2.2} is the starting point of the
development of a polynomial-time algorithm that determines if a given finite
idempotent algebra generates a \cm variety. 

%% The following lemma ties in with the previous proposition and will be used
%% in Sec. 6.
%% \begin{lemma}[Lemma 2.3~\cite{Freese:2009}] 
%%   Let $\bA$ be a finite idempotent algebra and let $\bS \in \sansH \sansS(\bA)$
%%   be strictly simple. Then there are elements $a, b \in A$ such that, if
%%   $\bB = \Sg^{\bA} (a, b)$, then $1_B = \Cg^{\bB} (a, b)$ and is join irreducible
%%   with unique lower cover $\rho$ such that $\bS = \bB/\rho$.
%% \end{lemma}
%% \begin{proof}
%%   Choose $\bB \in \sansS (\bA)$ as small as possible having $\bS$ as a homomorphic image,
%%   say $\bS = \bB/\rho$. We claim that if $a, b \in B$ with
%%   $(a, b) \in \notin \rho$ then they generate $\bB$. To
%%   see this, let $\bB'= \Sg^{\bB} (a, b)$ and let $h$ be the quotient map from B to S with kernel
%%   ρ. Then h(B  ) is a non-trivial subuniverse of S and so must equal S. Thus B  = B.
%%   Now let a, b ∈ B with (a, b) ∈
%%   / ρ. Since the block of Cg B (a, b) containing a
%%   and b is a subuniverse of B then from the previous paragraph, we conclude that
%%   Cg B (a, b) = 1 B and that ρ is its unique lower cover.
%% \end{proof}

According to the characterization
in~\cite[Ch.~8]{HM:1988} of locally finite congruence modular (resp.,
distributive) varieties, a finite algebra $\bA$ generates a congruence modular
(resp., distributive) variety $\sV$ if and only if the typeset of $\sV$ is
contained in $\{2, 3, 4\}$ (resp., $\{3, 4\}$) and all minimal sets of prime
quotients of finite algebras in $\sV$ have empty
tails~\cite[Def.~2.15]{HM:1988}. Note that in the distributive 
case the empty tails condition is equivalent to the minimal sets all having exactly
two elements.

It follows from Corollary~\ref{cor:2.2} and Proposiiton~\ref{prop:2.1}
that if $\bA$ is idempotent then one can
test the first condition, on omitting types 1 and 5 (or 1, 2, and 5) by searching
for a 2-generated subalgebra of $\bA$ whose typeset is not contained in
$\{2, 3, 4\}$ ($\{3, 4\}$). It is proved in~\cite[Sec.~6]{Freese:2009} that this
test can be performed in polynomial time---that is, the running time of the test
is bounded by a polynomial function of the size of $\bA$.
In~\cite[Sec.~3]{Freese:2009}, Freese and Valeriote prove a sequence of
lemmas to establish that, if $\bA$ is finite and idempotent, and if
$\sV = \bbV(\bA)$ omits types 1 and 5, then to test for the existence of tails
in $\sV$ it suffices to look for them 
in the 3-generated subalgebras of $\bA^2$.
%% More specifically, the authors assume that the type set of $\bbV(\bA)$ contains no 1's
%% and no 5's, and under this 
%% assumption they prove that non-empty tails either do not occur in $\bbV(\bA)$,
%% or they occur in 3-generated subalgebras of $\bA^2$.
In other words, either there are no non-empty tails
or else there are non-empty tails that are easy to find
(since they occur in 3-generated subalgebras of $\bA^2$).
It follows that Problem~\ref{prob:2} has a positive answer:
deciding whether or not a finite idempotent algebra generates a congruence
modular variety is tractable.\footnote{That is, there are positive integers
  $C, n$, and an algorithm that takes
  a finite idempotent algebra $\bA$ as input and decides
  in at most $C|\bA|^n$ steps whether $\bbV(\bA)$ is congruence modular.
  Here $|\bA|$ denotes the number of bits required to encode the algebra $\bA$.}
%% polynomial-time algorithm to decide, for a finite idempotent algebra $\bA$,
%% whether $\bbV(\bA)$ is congruence modular.}

Our goal is to use the same strategy to solve Problem~\ref{prob:1}.
As such, we revisit each lemma in Section 3 of \cite{Freese:2009},
and consider whether it can be proved under modified hypotheses.
Specifically, we continue to assume that the type set of $\bbV(\bA)$ contains no 1's,
but we will now drop the ``no 5's'' assumption.  We will attempt to prove that
either there are no \emph{type-2} tails in $\bbV(\bA)$, or else \emph{type-2}
tails can be found ``quickly,'' (e.g., in a 3-generated subalgebra of $\bA^2$).
We continue to quote~\cite{Freese:2009} where possible,
while modifying the assumptions and adjusting the arguments as necessary.


Throughout, we let $\nn$ denote the set $\{0,1,\dots, n-1\}$ and 
(at least for the rest of this section) we let $\sS$ be a finite set of finite,
similar, idempotent algebras, closed under the taking of subalgebras, such that
$\sV = \bbV(\sS)$ omits 1 (but may include type 5).
We will suppose that some finite algebra
$\bB$ in $\sV$ has a prime quotient whose minimal sets have non-empty
\emph{type-2} tails and show that there is a 3-generated subalgebra of the
product of two members of $\sS$ with this property.

Since $\sS$ is closed under the taking of subalgebras,
we may assume that the algebra $\bB$ from the previous paragraph is a subdirect
product of a finite number of members of $\sS$. Choose $n$ minimal such that for
some $\bA_0$, $\bA_1$, $\dots$, $\bA_{n-1}$ in $\sS$, there is a subdirect
product $\bB \sdp \prod_{\nn} \bA_i$
that has a prime quotient with non-empty type-2 tails.
Under the assumption that $n > 1$ we will attempt to prove that $n = 2$.

For this $n$, select the $\bA_i$ and $\bB$ so that $|B|$ is as small as possible.
Let $\alpha \prec \beta$ be a prime quotient of $\bB$ with non-empty
type-2 tails and choose $\beta$ minimal with this property.
Let $U$ be an $(\alpha, \beta)$-minimal set and let $N$ be an
 $(\alpha, \beta)$-trace of $U$. Let 0 and 1 be
two distinct members of $N$ with $(0, 1) \notin \alpha$.

\begin{lem}[Lem.~3.1~\cite{Freese:2009}]
  \label{lem:fv_3-1}
  Let $t$ be a member of the tail of $U$. Then $\beta$ is the congruence of $\bB$
  generated by the pair $(0, 1)$ and $\bB$ is generated by $\{0, 1, t\}$.
\end{lem}
It seems the proof of~\cite[Lem.~3.1]{Freese:2009} goes through with only minor
adjustments. % to prove Lemma~\ref{lem:fv_3-1}.
\begin{proof}
  {\bf TODO:} fill in proof of  Lemma~\ref{lem:fv_3-1}.
\end{proof}



%% For $i \leq n$, let $\pi_i$ be the projection homomorphism from $\bB$ onto
%% $\bA_i$ and let $\rho_i$ be the kernel of $\pi_i$, so $\bB \cong \bA_i/\rho_i$.
For $i \leq n$, %% let $\pi_i$ be the projection homomorphism from $\bB$ onto
%% $\bA_i$ and
let $\rho_i$ denote the kernel of the projection of $\bB$ onto $\bA_i$,
so $\bB \cong \bA_i/\rho_i$.
For a subset $\sigma \subseteq \nn$, define
\[
\rho_\sigma := \bigwedge_{j\in \sigma} \rho_j.
\]
%% If $\sigma$ is a singleton, say, $\sigma= \{j\}$, then
%% we write $\rho_j$ instead of $\rho_{\{j\}}$.
Consequently,
\marginnote{wjd: I don't see why join in (3.1) is $1_B$... it's probably wrong.}[3cm]
\begin{equation}
  \label{eq:2}
 \rho_{\nn} = \bigwedge_{j\in \nn}\rho_j = 0_{B}
 %% \rho_\sigma= \bigwedge_{j\in \sigma}\rho_j, 
 \quad \text{ and } \quad
\bigvee_{j\in \nn}\rho_j =1_B. %% \qquad
\end{equation}
By minimality of $n$ we know that the intersection of any  proper subset of the
$\rho_i$, $1 \leq i \leq n$ is strictly above $0_B$.  Thus,
%%  in addition to the standard identities in~(\ref{eq:2}), we also have 
$0_B < \rho_\sigma < 1_B$ for all 
$\emptyset \subset \sigma\subset \nn$.
(N.B.,  $\subset$ means \emph{proper} subset.)


\begin{lem}[Lem.~3.2~\cite{Freese:2009}]
  \label{lem:fv_3-2}
  If $\sigma \subset \nn$, then
  either $\beta \leq \rho_\sigma$ or $\alpha \join \rho_\sigma = 1_B$.
\end{lem}
\begin{proof}
  {\bf TODO:} fill in proof of  Lemma~\ref{lem:fv_3-2}.
\end{proof}

\begin{lem}[Lem.~3.3~\cite{Freese:2009}]
  \label{lem:fv_3-3}
  $\alpha \join \rho_i < 1_B$ for at least one $i$ and $\alpha \join \rho_j = 1_B$
  for at least one $j$.  
\end{lem}
\begin{proof}
  {\bf TODO:} fill in proof of  Lemma~\ref{lem:fv_3-3}.
\end{proof}

\begin{thm}[Thm.~3.4~\cite{Freese:2009}]
  \label{thm:fv_3-4}
  Let $\sV$ be the variety generated by some finite set $\sS$ of finite,
  idempotent algebras that is closed under taking subalgebras. If $\sV$
  omits type 1 \sout{and 5} and some finite member of $\sV$ has a prime quotient
  whose minimal sets have non-empty \underline{type-2} tails then there is some
  3-generated algebra B with this property that belongs to $\sS$ or is a subdirect
  product of two algebras from $\sS$. 
\end{thm}
\begin{proof}
  {\bf TODO:} fill in proof of Theorem~\ref{thm:fv_3-4}.
\end{proof}

\newpage

\section{Some Useful Tools}
In this section we review some properties of congruences, centralizers, and abelian
algebras that we have recently found very useful (see~\cite{Bergman-DeMeo}).
The results in this section seem relevant and perhaps useful for the purpose of
proving some of the results in the previous section.

In previous work, nonabelian algebras played the following useful role:
a theorem would begin with the assumption that a particular algebra $\bA$ is
nonabelian and then proceed to show that if the result to be proved were false,
then $\bA$ would have to be abelian.
Such arguments employ some basic facts about abelian algebras that we now review.


\subsection{Definitions}
Let $\bA = \<A, F^{\bA}\>$ be an algebra.
A reflexive, symmetric, compatible binary relation $T\subseteq A^2$ is called a
\defn{tolerance of $\bA$}.  
Given a pair $(\bu, \bv) \in A^m\times A^m$ of $m$-tuples of $A$, we write 
$\bu \mathrel{\bT} \bv$ just in case $\bu(i) \mathrel{T} \bv(i)$ for all $i\in \mm$. 
We state a number of definitions in this section using tolerance relations, but 
the definitions don't change when the tolerance in question happens to be
a congruence relation (i.e., a transitive tolerance).

Suppose $S$ and $T$ are tolerances on $\bA$.  An \defn{$S,T$-matrix} 
is a $2\times 2$ array of the form
\[
\begin{bmatrix*}[r] t(\ba,\bu) & t(\ba,\bv)\\ t(\bb,\bu)&t(\bb,\bv)\end{bmatrix*},
\]
where $t$, $\ba$, $\bb$, $\bu$, $\bv$ have the following properties:
\begin{enumerate}[(i)] %[label=(\roman*)]
\item $t\in \sansClo_{\ell + m}(\bA)$,
\item $(\ba, \bb)\in A^\ell\times A^\ell$ and $\ba \mathrel{\bS} \bb$,
\item $(\bu, \bv)\in A^m\times A^m$ and $\bu \mathrel{\bT} \bv$.
\end{enumerate}
Let $\delta$ be a congruence relation of $\bA$.
If the entries of every $S,T$-matrix satisfy
\begin{equation}
  \label{eq:22}
t(\ba,\bu) \mathrel{\delta} t(\ba,\bv)\quad \iff \quad t(\bb,\bu) \mathrel{\delta} t(\bb,\bv),
\end{equation}
then we say that $S$ \defn{centralizes $T$ modulo} $\delta$ and we write 
$\CC{S}{T}{\delta}$.
That is, $\CC{S}{T}{\delta}$  means that 
(\ref{eq:22}) holds \emph{for all}
$\ell$, $m$, $t$, $\ba$, $\bb$, $\bu$, $\bv$ satisfying properties (i)--(iii).

The \defn{commutator} of $S$ and $T$, denoted by $[S, T]$,
is the least congruence $\delta$ such that $\CC{S}{T}{\delta}$ 
holds.  
Note that $\CC{S}{T}{0_A}$ is equivalent to $[S,T] = 0_A$, and this
is sometimes called the \defn{$S, T$-term condition};
when it holds we say  that
$S$ \defn{centralizes} $T$, and write $\C{S}{T}$.
A tolerance $T$ is called \defn{abelian} if
$\C{T}{T}$ (i.e., $[T, T] = 0_A$).  
An algebra $\bA$ is called \defn{abelian} if $1_A$ is abelian
(i.e., $\C{1_A}{1_A}$).

\begin{rem}
  An algebra $\bA$ is abelian iff $\C{1_A}{1_A}$ iff
  \[
  \forall \ell, m \in \N,
  \quad \forall t\in \sansClo_{\ell + m}(\bA),
  \quad \forall (\ba, \bb)\in A^\ell\times A^\ell,
  \]
  \[
  \ker t(\ba, \cdot)=\ker t(\bb, \cdot).
  \]
\end{rem}

\subsection{Facts about centralizers and abelian congruences}
\label{sec:facts-about-centr}
We now collect some useful facts about centralizers of congruence relations.
We begin with a list of properties of the centralizer relation
that are well-known and not too hard to prove.
(For examples, see \cite[Prop~3.4]{HM:1988} and~\cite[Thm~2.19]{MR3076179}.)
\begin{lem}
\label{lem:centralizers}
Let $\bA$ be an algebra and suppose
$\bB$ is a subalgebra of $\bA$. 
Let $\alpha$, $\beta$, $\gamma$, $\delta$, $\alpha_i$
$\beta_j$, $\gamma_k$
be congruences of $\bA$, for some 
$i \in I$, $j\in J$, $k \in K$. Then the following hold:
\begin{enumerate}
\item \label{centralizing_over_meet}
  $\CC{\alpha}{\beta}{\alpha \meet \beta}$;
\item \label{centralizing_over_meet2}
  if $\CC{\alpha}{ \beta}{ \gamma_k}$ for all $k \in K$, then
  $\CC{\alpha}{ \beta}{ \Meet_{K}\gamma_k}$;
\item \label{centralizing_over_join1}
  if $\CC{\alpha_i}{ \beta}{ \gamma}$ for all $i\in I$, then
  $\CC{\Join_{I}\alpha_i}{ \beta}{\gamma}$;
\item \label{monotone_centralizers1}
  if $\CC{\alpha}{ \beta}{ \gamma}$ and $\alpha' \leq \alpha$, then 
  $\CC{\alpha'}{ \beta}{ \gamma}$;
\item \label{monotone_centralizers2}
  if $\CC{\alpha}{ \beta}{ \gamma}$ and $\beta' \leq \beta$, then
  $\CC{\alpha}{ \beta'}{ \gamma}$;
\item \label{centralizing_over_subalg}
  if $\CC{\alpha}{ \beta}{ \gamma}$ in $\bA$, 
  then $\CC{\alpha\cap B^2}{ \beta\cap B^2}{\gamma\cap B^2}$ in $\bB$;
\item \label{centralizing_factors}
  if $\gamma \leq \delta$, then $\CC{\alpha}{ \beta}{ \delta}$
  in $\bA$ if and only if $\CC{\alpha/\gamma}{ \beta/\gamma}{ \delta/\gamma}$
  in $\bA/\gamma$.
\end{enumerate}
\end{lem}


\begin{rem}
By (\ref{centralizing_over_meet}), 
if $\alpha \meet \beta = 0_{A}$,  
then $\C{\beta}{\alpha}$ and $\C{\alpha}{\beta}$.
\end{rem}

The next two lemmas turn out to be very useful.
The first identifies special conditions
under which certain quotient congruences are abelian.
The second gives fairly general conditions under which
quotients of abelian congruences are abelian.
\begin{lem}
  \label{lem:common-meets}
  Let $\alpha_0$, $\alpha_1$, $\beta$ be congruences of $\bA$ and suppose 
  $\alpha_0 \meet \beta = \delta = \alpha_1 \meet \beta$.
  Then $\CC{\alpha_0 \join \alpha_1}{ \beta}{ \delta}$.  If, in addition, 
  $\beta \leq \alpha_0 \join \alpha_1$, then 
  $\CC{\beta}{ \beta}{ \delta}$, so $\beta/\delta$ is an 
  abelian congruence of $\bA/\delta$.
\end{lem}
Lemma~\ref{lem:common-meets}
is an easy consequence
of items (\ref{centralizing_over_meet}), (\ref{centralizing_over_join1}),
(\ref{monotone_centralizers1}), and (\ref{centralizing_factors}) of
  Lemma~\ref{lem:centralizers}.
\begin{lem}
  \label{lem:abelian-quotients}
  Let $\sV$ be a locally finite variety with a Taylor term and let $\bA\in \sV$.
  Then $\CC{\beta}{\beta}{\gamma}$ for all $[\beta, \beta] \leq \gamma$.
\end{lem}
Lemma~\ref{lem:abelian-quotients} can be proved  by combining  
the next result, of David Hobby and Ralph McKenzie,
with a result of Keith Kearnes and Emil Kiss.
\begin{lem}[cf.~\protect{\cite[Thm~7.12]{HM:1988}}]
  \label{lem:HM-thm-7-12}
  A locally finite variety $\var{V}$ has a Taylor term if and only if it has a
  so called \defn{weak difference term}; that is, a term $d(x,y,z)$ satisfying
  the following conditions for all $\bA \in \var{V}$, all $a, b \in A$, and all
  $\beta \in \Con (\bA)$: 
  $d^{\bA}(a,a,b) \mathrel{[\beta, \beta]} b \mathrel{[\beta, \beta]} d^{\bA}(b,a,a)$,
  where $\beta = \Cg^{\bA}(a,b)$.
\end{lem}

\begin{lem}[\protect{\cite[Lem~6.8]{MR3076179}}]
  \label{lem:KK-lem-6-8}
  If $\bA$ belongs to a variety with a
  weak difference term and if $\beta$ and $\gamma$ are congruences of $\bA$
  satisfying $[\beta, \beta] \leq \gamma$, then $\CC{\beta}{\beta}{\gamma}$.
\end{lem}
\begin{rem}
  \label{rem:abelian-quotients}
  It follows immediately from Lemma~\ref{lem:abelian-quotients} that in a locally
  finite Taylor variety, $\var{V}$, quotients of abelian algebras are abelian, so the
  abelian members of $\var{V}$ form a subvariety.
  But this can also be derived from Lemma~\ref{lem:HM-thm-7-12},
  since $[\beta, \beta] = 0_A$ implies $d^{\bA}$ is a \malcev term operation on
  the blocks of $\beta$, so if $\bA$ is abelian---i.e., if
  $\CC{1_A}{1_A}{0_A}$---then Lemma~\ref{lem:HM-thm-7-12},
  implies that $\bA$ has a \malcev term operation.
  %% (This will be recorded below in~Theorem~\ref{thm:type2cp}.)
  It then follows that homomorphic images of $\bA$ are
  abelian. (See~\cite[Cor~7.28]{MR2839398} for more details). 
\end{rem}


\newpage

\appendix

%% \section{Facts about centralizers and abelian algebras}
%% \label{sec:proofs-elem-facts}

\section{More About Abelian Algebras}
\label{sec:abelian-algebras}
Here are some additional facts about abelian algebras that are sometimes useful.

\begin{lem}
If $\Clo(\bA)$ is trivial (i.e., generated by the projections),
then $\bA$ is abelian.
\end{lem}
In fact, it can be shown that $\bA$ is \emph{strongly abelian} in this case, but
we won't prove this stronger result. The proof that $\bA$ is abelian is
elementary is a nice and easy example of a standard proof technique---induction on
term height.\footnote{This proof would be a good one to try in a proof assistant
  like Coq, since such tools excel at inductive arguments like this one.}
\begin{proof}
We want to show $\sansC(1_A, 1_A)$.  Equivalently, we must show
that for all $t\in \Clo(\bA)$ (say, $(\ell+m)$-ary) 
and all $a, b \in A^\ell$, we have $\ker t(a,\cdot)=\ker t(b,\cdot)$.
We prove this by induction on the height of the term $t$.  Height-one terms are
projections and the result is obvious for these.  Let $n>1$ and assume the result
holds for all terms  of height less than
$n$.  Let $t$ be a term of height $n$, say, $k$-ary.  Then for some terms 
$g_1, \dots, g_k$ of height less than $n$ and for some $j\leq k$, we have
$t = p^k_j [g_1, g_2, \dots, g_k] = g_j$ and since $g_j$ has height less than
$n$, we have
\[
\ker t(a,\cdot)=\ker g_j(a,\cdot) = \ker g_j(b,\cdot)=\ker t(b,\cdot).
\]\end{proof}


\begin{lem}
 An algebra $\bA$ is abelian if and only if there is some 
 $\theta \in \Con (\bA^2)$ that has the diagonal $D(A):= \{(a,a): a \in A\}$ 
 as a congruence class.
\end{lem}
\begin{proof}
($\Leftarrow$) Assume $\Theta$ is such a congruence.  Fix 
  $k<\omega$,
  $t^{\bA}\in \Clo_{k+1}(\bA)$, 
  $u, v \in A$, and
  $\bx, \by \in A^k$.
  We will prove the implication~(\ref{eq:22}), which in the present context is
\begin{equation*}
t^\bA(\bx,u) = t^\bA(\by,u) \quad \Longrightarrow \quad 
t^{\bA}(\bx,v) = t^{\bA}(\by,v).
\end{equation*}
Since $D(A)$ is a class of $\Theta$, we have 
  $(u,u) \mathrel{\Theta} (v,v)$, and since $\Theta$ is a reflexive relation, we have
  $(x_i,y_i)  \mathrel{\Theta} (x_i,y_i)$ for all $i$.  Therefore,
\begin{equation}
  \label{eq:9}  
  t^{\bA\times \bA}((x_1,y_1), \dots, (x_k,y_k), (u,u))
  \mathrel{\Theta}
  t^{\bA\times \bA}((x_1,y_1), \dots, (x_k,y_k), (v,v)).
\end{equation}
  since $t^{\bA \times \bA}$ is a term operation of $\bA\times \bA$.
  Note that~(\ref{eq:9}) is equivalent to
  \begin{equation}
    \label{eq:13}
    (t^{\bA}(\bx, u), t^{\bA}(\by,u))
    \mathrel{\Theta}
    (t^{\bA}(\bx, v), t^{\bA}(\by, v)).
  \end{equation}
  If $t^{\bA}(\bx, u)= t^{\bA}(\by, u)$ then 
  the first pair in~(\ref{eq:13}) belongs to the $\Theta$-class
  $D(A)$, so the second pair must also belong this $\Theta$-class.
  That is, $t^{\bA}(\bx, v)= t^{\bA}(\by, v)$, as desired.

  \vskip2mm

  \noindent ($\Rightarrow$) Assume $\bA$ is abelian. We show
  $\Cg^{\bA^2}(D(A)^2)$ has $D(A)$ as a block.  Assume
  \begin{equation}
    \label{eq:16}
  ((x,x), (c,c')) \in \Cg^{\bA^2}(D(A)^2).
  \end{equation}
  It suffices to prove that $c=c'$.  Recall, \malcev's congruence generation
  theorem states that (\ref{eq:16}) holds iff
  %$(x,x) \theta (c,c') \in \Cg^{\bA^2}(D(A)^2)$ iff %% for $0\leq i \leq n$ and 
  %% $0\leq j \leq n-1$, there exist
  \begin{align*}
  \exists \,& (z_0,z_0'), (z_1,z_1'), \dots, (z_n,z_n') \in A^2\\
    \exists \,& ((x_0,x_0'), (y_0,y_0')), ((x_1,x_1'), (y_1,y_1')), \dots, 
    ((x_{n-1},x_{n-1}'), (y_{n-1},y_{n-1}')) \in D(A)^2\\
    \exists \, & f_0, f_1, \dots, f_{n-1}\in F^*_{\bA^2}
  \end{align*}
  %% \begin{align*}
  %% (z_i,z_i') &\in A^2\\
  %% ((x_j,x_j'), (y_j,y_j')) &\in D(A)^2\\
  %% f_j &\in F^*_{\bA^2}
  %% \end{align*}
  such that 
  \begin{align}
    \label{eq:7}
    \{(x, x),(z_1,z_1')\} &= \{f_0(x_0,x_0'), f_0(y_0,y_0')\}\\
\nonumber
     \{(z_1,z_1'),(z_2,z_2')\} &= \{f_1(x_1,x_1'), f_1(y_1,y_1')\}\\
\nonumber
     & \vdots\\
    \label{eq:8}
     \{(z_{n-1},z_{n-1}'),(c, c')\} &= \{f_{n-1}(x_{n-1},x_{n-1}'), f_{n-1}(y_{n-1},y_{n-1}')\}
 \end{align}
The notation $f_i\in F^*_{\bA^2}$ means 
\begin{align*}
f_i(x, x') &= g_i^{\bA^2}((a_1, a_1'), (a_2, a_2'), \dots, (a_k, a_k'), (x, x'))\\
&= (g_i^{\bA}(a_1, a_2, \dots, a_k, x), g_i^{\bA}(a_1', a_2', \dots, a_k', x')),
\end{align*}
for some $g_i^{\bA} \in \Clo_{k+1}(\bA)$ and some constants 
$\ba = (a_1, \dots, a_k)$ and $\ba' = (a_1', \dots, a_k')$ in $A^k$. 
Now, $((x_i,x_i'), (y_i,y_i'))\in D(A)^2$ implies 
$x_i=x_i'$, and $y_i=y_i'$, so in fact we have 
\[
     \{(z_i,z_i'),(z_{i+1},z_{i+1}')\} = \{f_i(x_i,x_i), f_i(y_i,y_i)\} \quad (0\leq i < n).
\]
Therefore, by Equation~(\ref{eq:7}) we have either 
\[
     (x,x)= (g_i^{\bA}(\ba, x_0), g_i^{\bA}(\ba', x_0)) \quad \text{ or } \quad 
     (x,x)= (g_i^{\bA}(\ba, y_0), g_i^{\bA}(\ba', y_0)).
\]
Thus, either $g_i^{\bA}(\ba, x_0) =  g_i^{\bA}(\ba', x_0)$ %\quad \text{ or } \quad 
or $g_i^{\bA}(\ba, y_0) =  g_i^{\bA}(\ba', y_0)$.
By the abelian assumption, if one of these equations holds, then so does the
other. This and and Equation (\ref{eq:7}) imply $z_1 = z_1'$.  Applying the same
argument inductively, we find that $z_i = z_i'$ for all $1\leq i < n$ and so, by
(\ref{eq:8}) and the abelian property, we have $c= c'$.
\end{proof}

\begin{lem}
Suppose $\rho\colon A_1 \to A_2$ is a bijection and suppose the graph
$\{(x, \rho x) \mid x \in A_1\}$ is a block of some congruence
$\beta \in \Con (A_1 \times A_2)$.  Then both $\bA_1$ and $\bA_2$ are abelian.
\end{lem}
\begin{proof}
  Define the relation $\alpha\subseteq (A_1\times A_1)^2$ as follows: for
  $((a,a'), (b,b')) \in (A_1\times A_1)^2$,
  \[
  (a,a')\mathrel{\alpha} (b,b')
  \quad \iff \quad
  (a, \rho a') \mathrel{\beta} (b, \rho b')
  \]
  We prove that the diagonal $D(A_1)$ is a block of $\alpha$ by showing that
  $(a, a) \mathrel{\alpha} (b,b')$ implies $b = b'$.
  Indeed, if $(a, a) \mathrel{\alpha} (b,b')$, then
  $(a, \rho a) \mathrel{\beta} (b, \rho b')$, which means that
  $(b, \rho b')$ belongs to the block and
  $(a, \rho a)/\beta = \{(x, \rho x): x\in A_1\}$.  Therefore,
  $\rho b  = \rho b'$, so $b = b'$ since $\rho$ is injective.
  This proves that $\bA_1$ is abelian.

  To prove $\bA_2$ is abelian, we reverse the roles of $A_1$ and $A_2$ in the
  foregoing argument.  
  If $\{(x, \rho x) \mid x \in A_1\}$ is a block of $\beta$,
  then 
  $\{(\rho^{-1}(\rho x), \rho x) \mid \rho x \in A_2\}$ is a block of $\beta$; that
  is, $\{(\rho^{-1} y, y) \mid y \in A_2\}$ is a block of $\beta$.  Define 
  the relation $\alpha\subseteq (A_2\times A_2)^2$ as follows: for
  $((a,a'), (b,b')) \in (A_2\times A_2)^2$,
  \[
  (a,a')\mathrel{\alpha} (b,b')
  \quad \iff \quad
  (\rho^{-1}a, \rho a') \mathrel{\beta} (\rho^{-1}b, \rho b').
  \]
  As above, we can prove that the diagonal $D(A_2)$ is a block of $\alpha$
  by using the injectivity of $\rho^{-1}$ to show that $(a, a) \mathrel{\alpha}
  (b,b')$
  implies $b = b'$.
\end{proof}

%\bibliographystyle{amsplain} %% or amsalpha
%% \bibliographystyle{alpha-url}
%% \printbibliography
\bibliographystyle{alphaurl}
\bibliography{inputs/refs}


\end{document}







\subsection{Definitions} %: tolerance, centralizing, abelian}
Let $\bA = \<A, F^{\bA}\>$ be an algebra.
A reflexive, symmetric, compatible binary relation $T\subseteq A^2$ is called a
\defn{tolerance of $\bA$}.  
As a compatible binary relation, a tolerance is a subuniverse of $\bA^2$.
If $T$ is a tolerance of $\bA$, and if $(\bu, \bv) \in A^m\times A^m$
is a pair of $m$-tuples of $A$, then we write 
$\bu \mathrel{\bT} \bv$ just in case $\bu(i) \mathrel{T} \bv(i)$ for all $i\in \mm$. 
(Here we have surreptitiously introduced another very convenient notation, namely,
$\mm := \{0, 1, 2, \dots, m-1\}$.)

Suppose $S$ and $T$ are tolerances on $\bA$.  An \defn{$S,T$-matrix} 
is a $2\times 2$ array of the form
\[
\begin{bmatrix*}[r] t(\ba,\bu) & t(\ba,\bv)\\ t(\bb,\bu)&t(\bb,\bv)\end{bmatrix*},
\]
where $t$, $\ba$, $\bb$, $\bu$, $\bv$ have the following properties:
\begin{enumerate}[(i)]
\item $t\in \Clo_{\ell + m}(\bA)$,
\item $(\ba, \bb)\in A^\ell\times A^\ell$ and $\ba \mathrel{\bS} \bb$,
\item $(\bu, \bv)\in A^m\times A^m$ and $\bu \mathrel{\bT} \bv$.
\end{enumerate}
Let $\delta$ be a congruence relation of $\bA$.
If the entries of every $S,T$-matrix satisfy
\begin{equation}
  \label{eq:22}
t(\ba,\bu) \mathrel{\delta} t(\ba,\bv)\quad \iff \quad t(\bb,\bu) \mathrel{\delta} t(\bb,\bv),
\end{equation}
then we say that $S$ \defn{centralizes $T$ modulo} $\delta$ and we write 
$\sansC(S, T; \delta)$.
That is, $\sansC(S, T; \delta)$ holds iff 
(\ref{eq:22}) holds \emph{for all}
$\ell$, $m$, $t$, $\ba$, $\bb$, $\bu$, $\bv$ satisfying properties (i)--(iii).
The condition $\sansC(S, T; 0_{A})$ is sometimes called the 
\defn{$S, T$-term condition}, and when it holds we say  that
$S$ \defn{centralizes} $T$, and write
$\sansC(S, T)$.
%% REMOVING THIS since I don't think we ever use the commutator notation again.
%% The \defn{commutator} of $\bS$ and $\bT$, denoted by $[\bS, \bT]$,
%% is the least congruence $\delta$ such that $\sansC(\bS, \bT; \delta)$ 
%% holds.  

A tolerance $T$ is called \defn{abelian} if
$\sansC(T, T)$. %(i.e., $[\bT, \bT] = 0_A$).  
An algebra $\bA$ is called \defn{abelian} if $1_A$ is abelian.
%% (i.e., $[1_A, 1_A] = 0_A$).
%% The \defn{centralizer of $\bT$ modulo $\delta$}, denoted by
%% $(\delta : \bT )$, is the largest congruence $\alpha$ on $\bA$ such that 
%% $\sansC(\alpha, \bT ; \delta)$ holds.

\begin{rem}
An algebra $\bA$ is abelian iff $\sansC(1_A, 1_A)$ iff
\[
\forall \ell \in \{0,1,2,\dots \}, 
\quad \forall m \in  \{1,2,\dots \},
\quad \forall t\in \Clo_{\ell + m}(\bA),
\quad \forall (a, b)\in A^\ell\times A^\ell,
\]
\[
\ker t(a, \cdot)=\ker t(b, \cdot).
\]
\end{rem}










\begin{lem}[Lemma \ref{lem:M3-abelian}]
If $\alpha_1$, $\alpha_2$, $\alpha_3 \in \Con(\bA)$ are pairwise complements,
then $\sansC(1_A, \alpha_i)$ for each $i=1,2,3$.  If, in addition, $\bA$ is
idempotent and has a Taylor term operation, then $\sansC(1_A, 1_A)$; that is, $\bA$ is abelian.
\end{lem}
\begin{proof}
  The goal is to prove $\sansC(1_A, 1_A)$.
  By Lemma~\ref{lem:centralizers}~(\ref{fact:centralizing_over_meet}), we have
  $\sansC(\alpha_1, \alpha_2; \alpha_1 \meet \alpha_2)$.  
  Since $\alpha_1 \meet \alpha_2= 0_A$, this means
  $\sansC(\alpha_1, \alpha_2)$.
  Similarly, $\sansC(\alpha_3, \alpha_2)$.  Therefore, by 
  Lemma~\ref{lem:centralizers}~(\ref{fact:centralizing_over_join1}), we have
  $\sansC(\alpha_1 \join \alpha_3, \alpha_2)$. This is equivalent to 
  $\sansC(1_A, \alpha_2)$, since $\alpha_1 \join \alpha_3 = 1_A$. 
  The same argument \emph{mutatis-mutandis} yields
  $\sansC(1_A,\alpha_1)$ and $\sansC(1_A,\alpha_3)$. 
  Before proceeding, note that $\sansC(\alpha_1, \alpha_1)$, by 
  Lemma~\ref{lem:centralizers}~(\ref{fact:monotone_centralizers1}).
  Now, if $\bA$ is idempotent and has a Taylor term operation, then
  by \ref{thm:kearnes-kiss-3.27} we have 
  $\sansC(\alpha_1 \join \alpha_2,\alpha_1 \join \alpha_2; \alpha_2)$.
  That is, $\sansC(1_A,1_A; \alpha_2)$.
  Similarly, $\sansC(1_A,1_A; \alpha_3)$.
  By~\ref{lem:centralizers}~(\ref{fact:centralizing_over_meet2}) then, 
  $\sansC(1_A,1_A; \alpha_2 \meet\alpha_3)$. 
  That is, $\sansC(1_A,1_A)$.
\end{proof}











































\noindent [{\it wjd: I'm not sure about item
    (\ref{fact:dual_monotone_centralizers}) in the next lemma.}]

\begin{lem}
\label{lem:centralizers}
Let $\bA$ be an algebra with congruences 
$\alpha, \beta, \gamma, \alpha', \beta' \in \Con(\bA)$, and let 
$\bB$ be a subalgebra of $\bA$. Then,
\begin{enumerate}
\item \label{fact:centralizing_over_meet}
  $\sansC(\alpha, \beta; \alpha \meet \beta)$;
\item \label{fact:centralizing_over_meet2}
  if $\sansC(\alpha, \beta; \gamma)$ and $\sansC(\alpha, \beta; \gamma')$, then
  $\sansC(\alpha, \beta; \gamma \meet \gamma')$;
\item \label{fact:centralizing_over_join1}
  if $\sansC(\alpha, \beta; \gamma)$ and $\sansC(\alpha', \beta; \gamma)$, then
  $\sansC(\alpha \join \alpha', \beta; \gamma)$;
%% \item \label{fact:centralizing_over_join2}
%%   if $\sansC(\alpha, \beta; \gamma)$ and $\sansC(\alpha, \beta'; \gamma)$, then
%%   $\sansC(\alpha, \beta \join \beta'; \gamma)$;
\item \label{fact:monotone_centralizers1}
  if $\sansC(\alpha, \beta; \gamma)$ and $\alpha' \leq \alpha$, then 
  $\sansC(\alpha', \beta; \gamma)$;
\item \label{fact:monotone_centralizers2}
  if $\sansC(\alpha, \beta; \gamma)$ and $\beta' \leq \beta$, then
  $\sansC(\alpha, \beta'; \gamma)$;
\item \label{fact:dual_monotone_centralizers}
  if $\sansC(\alpha, \beta; \gamma)$ and $\gamma \leq \gamma'$, then
  $\sansC(\alpha, \beta; \gamma')$;
\item \label{item:subalg}
  if $\sansC(\alpha, \beta; \delta)$ holds in $\bA$, 
  then $\sansC(\alpha\cap B^2, \beta\cap B^2;\delta\cap B^2)$ holds in $\bB$;
\item \label{item:factors}
  if $\delta' \leq \delta$, then $\sansC(\alpha, \beta; \delta)$ holds 
  in $\bA$ if and only if $\sansC(\alpha/\delta', \beta/\delta'; \delta/\delta')$
  holds in $\bA/\delta'$.
\end{enumerate}
\end{lem}

\noindent [{\it wjd: I don't remember what made me think (\ref{fact:dual_monotone_centralizers})
is true. Maybe it holds in the \cm case?}]

\begin{rem}
By (\ref{fact:centralizing_over_meet}), 
if $\alpha \meet \beta = 0_{\bA}$,  
then $\sansC(\beta, \alpha)$ and $\sansC(\alpha, \beta)$.
By (\ref{fact:dual_monotone_centralizers}),
if an algebra $\bA$ is abelian, 
then $\sansC(1_A, 1_A; \theta)$ for all $\theta \in \Con(\bA)$, so
in this case (\ref{item:factors}) implies that %$\sansC(1_{\bA/\theta}, 1_{\bA/\theta})$ 
$\bA/\theta$ is abelian for every $\theta \in \Con(\bA)$.
\end{rem}

We close this section with a collection of lemmas that can be useful for showing 
that an algebra is abelian.  Proofs of these facts appear in Appendix
Section~\ref{sec:proofs-elem-facts}.

We denote the diagonal of $A$ by $D(A) := \{(a,a): a \in A\}$. 
\begin{lem}
\label{lem:diagonal}
An algebra $\bA$ is abelian if and only if there is some $\theta \in \Con (\bA^2)$ that has
the diagonal set $D(A)$ as a congruence class.
\end{lem}

Lemma~\ref{lem:diagonal} can be used to prove the next result
which states that if there is a congruence of $\bA_1 \times \bA_2$ that has the
graph of a bijection between $A_1$ and $A_2$ as a block, then both $\bA_1$ and
$\bA_2$ are abelian algebras.

\begin{lem}
  \label{lem:bijection_abelian}
Suppose $\rho \colon A_0 \to A_1$ is a bijection and suppose the graph
$\{(x, \rho x) \mid x \in A_0\}$ is a block of some congruence
$\beta \in \Con (A_0 \times A_1)$.  Then both $\bA_0$ and $\bA_1$ are abelian.
\end{lem}

\begin{lem}
\label{lem:triv-clone-implies-abelian}
If $\Clo(\bA)$ is trivial (i.e., generated by the projections),
then $\bA$ is abelian.
\end{lem}

%% \begin{lem}\label{lem:M3-abelian}
%% If $\alpha_1$, $\alpha_2$, $\alpha_3 \in \Con(\bA)$ are pairwise complements,
%% then $\sansC(1_A, \alpha_i)$ for each $i=1,2,3$.  If, in addition, $\bA$ is
%% idempotent and has a Taylor term operation, then $\sansC(1_A, 1_A)$; that is, $\bA$ is abelian.
%% \end{lem}














\subsection{Alternative development}
Here is the development of similar notions from Hobby and McKenzie~\cite[Ch.~3]{HM:1988}.
Let $\alpha$, $\beta$, $\delta$ be congruences of an algebra $\bA$.
We use the formula $\sansC(\alpha, \beta; \delta)$
(in words, $\alpha$ centralizes $\beta$ modulo $\delta$) as an abbreviation for the
following property: For every $n > 1$, for every $f \in \Pol_n(\bA)$,
for all $(u,v)\in \alpha$, and for all $(x_i, y_i)\in \beta$ $(1< i<n)$, 
the following equivalence holds:
\begin{equation}
  \label{eq:1}
  f(u, x_1, \dots, x_{n-1}) \mathrel{\delta}  f(u, y_1, \dots, y_{n-1})
  \quad \iff \quad
  f(v, x_1, \dots, x_{n-1}) \mathrel{\delta}  f(v, y_1, \dots, y_{n-1}).
\end{equation}
