\documentclass[12pt,xcolor=dvipsnames%,mathserif%,handout
   ]{beamer}
% \PassOptionsToClass{beamer}{handout}
% \usepackage{pgfpages} 
% \pgfpagesuselayout{4 on 1}[letterpaper,landscape,border shrink=3mm]
\usepackage{amsmath}
\usepackage{helvet}
\usepackage{../notes/inputs/macros}
% \usepackage{../inputs/macros}
\usepackage{comment}
\usepackage{xspace}
\usepackage{pifont}
\usepackage[all,cmtip,arrow]{xy}  %for xy-pic fonts
%%macs
%For this presentation, I don't want citations
\renewcommand{\cite}[1]{\relax}
\renewcommand{\defn}[1]{\alert{#1}}
% \newcommand{\V}{\ensuremath{\mathcal V}\xspace}
% \newcommand{\mbold}[1]{\ensuremath{\mathbf{#1}}\xspace}
% \makecs{\mbold}{ABDP}
\newcommand{\exmpl}[1]{{\color{green!50!black} #1}}
\newcommand{\dsize}{\displaystyle}
\DeclareMathOperator{\End}{End}
\DeclareMathOperator{\Var}{Var}
\newcommand{\bigpause}{\pause\bigskip}
\newcommand{\medpause}{\pause\medskip}

% \newcommand{\reduc}{\leq_{\text{\textnormal{p}}}}
% \newcommand{\equivp}{\equiv_{\text{\textnormal{p}}}}
% \newcommand{\NP}{\ensuremath{\mathbb{NP}}\xspace}
% \renewcommand{\P}{\ensuremath{\mathbb{P}}\xspace}
\newcommand{\Blue}{\textcolor{blue!50!black}}
\newcommand{\Red}{\textcolor{violet!50!red}}
\newcommand{\Green}{\textcolor{green!55!black}}
\let\origemph=\emph 
\let\origtextbf=\textbf 
%%endmacs
%%
\parskip=10pt
%% Beamer Setup
% \mode<handout>{\beamertemplatesolidbackgroundcolor{black!5}}
\mode<presentation>{
  % \usetheme{Singapore}
   % \usetheme{Frankfurt}
   % \usetheme{Pittsburgh}
   \usetheme{Boadilla}
   % \usetheme{lankton-keynote}
   \usecolortheme{beaver}
  % \useinnertheme[shadow]{rounded}
  \setbeamertemplate{navigation symbols}[only frame symbol]{}
}
% \mode<beamer>
% {%
%   \let\emph=\alert}
%   \renewcommand{\textbf}[1]{{\usebeamercolor[fg]{example text}%
%      \origtextbf{#1}}}
% }
\mode<article>{\usepackage{fullpage}}
%%
%\title{\TeX\ and \LaTeX\ in the Mathematics Department}
%\author{Clifford Bergman}

\renewcommand{\phi}{\ensuremath{\varphi}}
\begin{document}
\title[Existence of Difference Terms]{Complexity of the Existence of Difference Terms}
\author[\url{williamdemeo@gmail.com}]
       {William DeMeo\\
{\small \url{williamdemeo@gmail.com}}
\\[5pt]{\small joint with}\\
{\small   Cliff Bergman and Ralph Freese}}

\date[11 Mar 2017]{AMS Special Session\\[10pt]
11 Mar 2017}

\frame[plain]{\titlepage}

\setbeamercolor{block body example}%
{parent=normal text,use=block title example,bg=block title example.bg!25!bg}

\begin{frame}
\frametitle{What is a difference term?}
Let $\sV$ be a variety. %% \onslide<3>{{\color{gray}(idempotent?)}}

A term $d(x,y,z)$ is called a \alert{difference term} for $\sV$ 
if it satisfies for all $\bA \in \sV$ and all $(a,b) \in \theta\in \Con\bA$
\begin{equation}
\label{eq:3}  
a \comr{\theta} d^{\bA}(a,b,b) 
\quad \text{ and } \quad
d^{\bA}(a,a,b) = b 
\end{equation}
where $[\cdot, \cdot]$ denotes the (term condition) commutator.

\pause

By monotonicity of $[\cdot, \cdot]$, could replace $\theta$ 
with $\Cg^{\bA}(a,b)$ in the def.

% \pause 

% If for all $a, b \in A$ the relations in (\ref{eq:3}) hold 
% with $\theta = \Cg^{\bA}(a,b)$, then we call
% $d^{\bA}$ a \defn{difference term operation} for $\bA$.
\end{frame}



\begin{frame}
\frametitle{Easy facts}

Recall, $d(x,y,z)$ is a \alert{difference term}
if for all $\bA \in \sV$ and all $(a,b) \in \theta\in \Con\bA$
\begin{equation}
\label{eq:4}  
a \comr{\theta} d(a,b,b) 
\quad \text{ and } \quad
d(a,a,b) = b 
\end{equation}

\begin{itemize}
\item<2->
If $d(a,a,b) \comr{\theta} b$ instead, $d$ is a \alert{weak difference term}.
\item<3->  We can replace $\theta$ 
with $\Cg^{\bA}(a,b)$ in the definition, 
by monotonicity of $[\cdot, \cdot]$.
\item<4->
  If $\bA$ is \alert{abelian} ($[1_A, 1_A] = 0_A$), then 
  $[\theta, \theta] = 0_A$ for all $\theta$, by
  monotonicity, in which case
  (\ref{eq:4}) says $d^{\bA}$ is Mal'tsev.
\item<5-> If $[\theta, \theta] = \theta$, then
  $a \comr{\theta} d^{\bA}(a,b,b)$ holds
  for all $(a, b) \in \theta$, in which case
the third projection $d^{\bA}(x,y,z) = z$ satisfies (\ref{eq:4}).
\end{itemize}

% \pause 

% If for all $a, b \in A$ the relations in (\ref{eq:3}) hold 
% with $\theta = \Cg^{\bA}(a,b)$, then we call
% $d^{\bA}$ a \defn{difference term operation} for $\bA$.
\end{frame}


\begin{frame}
  \frametitle{Motivation}
Difference terms are studied extensively in the literature.

\begin{itemize}


\item
  {\footnotesize (1988)
    Hobby and McKenzie, ``The structure of finite algebras,''~{\bf 76} of
    {\em Contemporary Mathematics}, {\em Amer. Math. Soc.} }
\item
  {\footnotesize (1995) 
    Kearnes, ``Varieties with a difference term,'' 
    {\em J. Algebra} {\bf 177}(3).}
\item
  {\footnotesize (1996)
    Lipparini, ``A characterization of varieties with a difference term.''
    {\em Canad. Math. Bull.} {\bf 39}(3).}
\item
  {\footnotesize (1998)
    Kearnes and Szendrei, ``The relationship between two commutators,''
    {\em Internat. J. Algebra Comput.} {\bf 8}(4). } 
%% \item
%%   {\footnotesize (2013)
%%     Kearnes and Kiss, ``The shape of congruence lattices,'' 
%%     {\em Mem. Amer. Math. Soc.} {\bf 222}(1046).}
\item
  {\footnotesize (2014)
    Valeriote and Willard, ``Idempotent {$n$}-permutable varieties,'' 
    {\em Bull. Lond. Math. Soc.} {\bf 46}(4).}
\item
  {\footnotesize (2016)
    Kearnes, Szendrei, and Willard, ``A finite basis theorem for
    difference-term varieties with a finite residual bound,'' 
    {\em Trans. Amer. Math. Soc.} {\bf 368}(3).}
\item
  {\footnotesize (201?)
  Kearnes, Szendrei and Willard, ``Simpler maltsev conditions for (weak)
  difference terms in locally finite varieties,'' to appear.}
\end{itemize}
\end{frame}

\begin{frame}
  \frametitle{Motivation}
  Knowing a variety has a difference term allows us to deduce useful
  properties of the algebras in that variety.

  Roughly speaking, having a difference term is slightly stronger than having
  a Taylor term and slightly weaker than having a Mal'tsev term.

  Diff Term $\Rightarrow$ Weak Diff Term $\Rightarrow$  Taylor
  
  For finite $\bA$, $\bbV(\bA)$ has a weak difference term iff
  $\bbV(\bA)$ has a  Taylor term.
  \onslide<2>{{\color{gray}(locally finite?)}}
\end{frame}

\begin{frame}
  \frametitle{Computational Problem}
  \begin{problem}[1]
  \label{prob:1}
  Is there a polynomial-time algorithm that takes a finite
  idempotent algebra $\bA$ as input and decides whether the variety generated by
  $\bA$ has a difference term?
\end{problem}

We solve an the following easier problem:
\begin{problem}[2]
  \label{prob:2}
  Is there a polynomial-time algorithm that takes a finite
  idempotent algebra $\bA$ as input and decides whether 
  $\bA$ has a difference term operation?
\end{problem}
\end{frame}

%%%% LOC DIFF TERMS %%%%%%%%%%%%%%%%%%%%%%%%%%%%%%%%
\begin{frame}{Local Difference Terms}

% In~\cite{MR3239624},
% M.~Valeriote and R.~Willard, Idempotent {$n$}-permutable varieties, {\em Bull.
%   Lond. Math. Soc.} {\bf 46}(4)  (2014)  870--880.
Valeriote and Willard recently defined %% an \defn{$\bA$-triple for $\bp$}
%% to be a triple $(a,b,i)$ such that $a, b \in A$ and
%% $p_i(a,b,b) = p_{i+1}(a,a,b)$. They use this to define 
a ``local Hagemann-Mitschke sequence'' that they use for
efficiently deciding, for a given $n$, whether an idempotent
variety is $n$-permutable. 

  \bigskip

  We devise a similar device we call a
  \emph{\alert{local difference term}}

  leading to a polynomial-time
  algorithm for deciding whether a given finite idempotent algebra
  has a difference term operation.

\end{frame}



%%%% LOC DIFF TERMS %%%%%%%%%%%%%%%%%%%%%%%%%%%%%%%%
\begin{frame}{Notation}

  If $\theta$ is a tolerance or congruence of $\bA$, we abreviate
  {\Large  \[
  \Com{\theta}= [\theta, \theta]
  \]}
\vfill
  \onslide<2->{%
    Similar to standard notation
    for the iterated commutator
      \[
        [\theta]^0 =  \theta, \quad
        [\theta]^1 =  [\theta, \theta],  \quad
        [\theta]^2 =  \bigl[[\theta, \theta],[\theta, \theta]\bigr],  \; \dots \;
        \]}
\end{frame}


%%%% LOC DIFF TERMS %%%%%%%%%%%%%%%%%%%%%%%%%%%%%%%%
\begin{frame}{Local Difference Terms}

  Let $\bA=\< A, \dots\>$ be an algebra, fix $a, b \in A$ and
  $i \in \{0,1\}$.

  A \defn{local difference term for $(a,b,i)$} is a term $d(x,y,z)$ satisfying
  \begin{align}
    %% \text{ if $i=0$, then } & a \comm{\Cg^{\bA}(a,b)}{\Cg^{\bA}(a,b)} d(a,b,b); \label{eq:diff-triple}\\
    \text{ if $i=0$, } \; & \; a \Comr{\Cg(a,b)} d(a,b,b); \label{eq:diff-triple-1}\\
    \text{ if $i=1$, } \; & \hskip1cm d(a,a,b) = b. \label{eq:diff-triple-2}
  \end{align}

  \bigpause
  If $d$ satisfies~(\ref{eq:diff-triple-1}) and~(\ref{eq:diff-triple-2}) for all triples
  in some subset $S\subseteq A \times A \times \{0,1\}$, then $d(x,y,z)$ is a

  \hskip1cm \defn{local difference term for the set $S$}.
\end{frame}

%%%% LOC DIFF TERMS %%%%%%%%%%%%%%%%%%%%%%%%%%%%%%%%
\begin{frame}{Local Difference Terms}
  Let 
  $\sS = A \times A \times \{0,1\}$ and
  suppose that every pair
  $((a_0, b_0, \chi_0), (a_1, b_1, \chi_1))$
  in $\sS^2$ has a local difference term.

  \bigpause
  That is, for each pair $((a_0, b_0, \chi_0), (a_1, b_1, \chi_1))$ 

  $\exists \, d(x,y,z)$ such that for each $i \in \{0,1\}$
  \begin{align}
    a_i \Comr{\Cg(a_i,b_i)} d(a_i,b_i,b_i) & \;
  \text{ if $\chi_i=0$;}  \label{eq:d-trip-i1}\\
  d(a_i,a_i,b_i) =b_i & \;
  \text{ if $\chi_i=1$.}\label{eq:d-trip-i2} %\\\nonumber
  \end{align}
  
  \medpause
  Under these hypothesis we prove that every subset of $\sS$
  has a local difference term.

  \vskip1mm
  In other words, $\exists$ a single term $d$ that works 
  for all $(a_i, b_i, \chi_i) \in S$.
\end{frame}

%%%% LOC DIFF TERMS %%%%%%%%%%%%%%%%%%%%%%%%%%%%%%%%
\begin{frame}{First Result}

%% \subsection{Main Results}
\begin{theorem}[\protect{{\small D. 2016}}]
  \label{thm:local-diff-terms}
  Let $\sV$ be an idempotent variety and
  $\bA \in \sV$. Define
  $\sS= A \times A \times \{0,1\}$
  and suppose that every pair
  $((a_0, b_0, \chi_0), (a_1, b_1, \chi_1)) \in \sS^2$
  has a local difference term.
  Then every subset $S \subseteq \sS$
  has a local difference term.
\end{theorem}

\begin{corollary}
  \label{cor:loc-diff-term}
  A finite idempotent algebra $\bA$ has a difference term operation if and
  only if each pair $((a,b,i), (a',b',i')) \in (A\times A \times \{0,1\})^2$ has a local
  difference term.
\end{corollary}


\begin{note}
  One direction is clear, since a difference term operation for $\bA$ is
  obviously a local difference term for the whole set 
  $A\times A \times \{0,1\}$.
  For the converse, suppose
  each pair in $(A\times A \times \{0,1\})^2$ has a local
  difference term. Then, by Theorem~\ref{thm:local-diff-terms},
  there is a single local difference term for the whole set $A\times A \times \{0,1\}$,
  and this is a difference term operation for $\bA$.  Indeed, if $d$ is a
  local difference term for $A\times A \times \{0,1\}$, then 
  for all $a, b \in A$, we have
  $a \Comr{\Cg(a,b)} d(a,b,b)$,
  since $d$ is a local difference term for $(a,b,0)$, and we have
  $d(a,a,b) = b$, since $d$ is also a local difference term for
  $(a,b,1)$.
\end{note}
\end{frame}



\begin{frame}
  
\begin{corollary}
  \label{cor:algor-1}
  There is a polynomial-time algorithm that takes as input
  a finite idempotent algebra $\bA$ and decides whether
  %% the variety $\bbV(\bA)$ that it generates
  $\bA$ has a difference term operation.
\end{corollary}

\begin{note}
  We describe an efficient algorithm for deciding,
  given a finite idempotent algebra $\bA$,
  whether every pair $((a,b,i), (a',b',i')) \in (A\times A \times \{0,1\})^2$ has a local
  difference term.  By Corollary~\ref{cor:loc-diff-term}, this will prove we
  can decide in polynomial-time whether $\bA$ has a difference term operation.

  Fix a pair
  $((a,b,i), (a',b',i'))$ in $(A\times A \times \{0,1\})^2$. If $i = i' = 0$,
  then the first projection is a local difference term. If $i = i' = 1$,  
    then the third projection is a local difference term. The two remaining cases to
    consider are (1) $i = 0$ and $i'=1$, and (2)
    $i = 1$ and $i'=0$. Since these are completely symmetric, we only handle the
    first case. Assume  the given pair of triples is
    $((a,b,0), (a',b',1))$.  By definition, a term $t$ is local difference term
    for this pair iff
    \[
    a\Comr{\Cg(a,b)} t^{\bA}(a,b,b) \; \text{ and } \;
    t^{\bA}(a',a',b') = b'.
    \]
    We can rewrite this condition more compactly by
    considering 
    \[t^{\bA\times \bA}((a,a'), (b,a'), (b,b')) =
    (t^{\bA}(a,b,b),t^{\bA}(a',a',b')).\]
    Clearly $t$ is a local difference term for
    $((a,b,0), (a',b',1))$ iff
    \[
    t^{\bA\times \bA}((a,a'), (b,a'), (b,b'))\in a/\delta \times \{b'\},
    \]
    where $\delta = \Com{\Cg(a,b)}$ and $a/\delta$ denotes the
    $\delta$-class containing $a$.
    (Observe that $a/\delta \times \{b'\}$ is a subalgebra of $\bA \times \bA$
    by idempotence.)
    It follows that the pair
    $((a,b,0), (a',b',1))$ has a local difference term iff
    the subuniverse of $\bA\times \bA$ generated by
    $\{(a,a'), (b,a'), (b,b')\}$ intersects nontrivially with the subuniverse
    $a/\delta \times \{b'\}$.

    Thus, the algorithm takes as input $\bA$ and, for each 
    $((a,a'), (b,a'), (b,b'))$ in $(A\times A)^3$, computes
    $\delta = \Com{\Cg(a,b)}$, computes the subalgebra
    $\bS$ of $\bA\times \bA$ generated by
    %% \Sg^{\bA\times \bA}\{(a,a'), (b,a'), (b,b')\}$, and then
    $\{(a,a'), (b,a'), (b,b')\}$, and then
    tests whether $S \cap (a/\delta \times \{b'\})$ is empty.
    If we find an empty intersection at any point, then
    $\bA$ has a difference term operation.
    Otherwise the algorithm halts without witnessing an empty
    intersection, in which case $\bA$ has a difference term operation.

    Most of the operations carried out by this algorithm are well known to be
    polynomial-time.  For example, that the running time of subalgebra generation is
    polynomial has been known for a long time (see~\cite{MR0455543}).
    The time complexity of congruence generation is also known to be polynomial
    (see~\cite{MR2470585}).  The only operation whose tractability might be 
    called into question is the commutator, but we have a straight-forward 
    algorithm for computing it that we describe in detail in 
    Appendix Section~\ref{sec:interlude:-an-easy}.
\end{note}
\end{frame}

\end{document}









%%%%%%%%%%%%%%%% F(2) stuff %%%%%%%%%%%%%%%%%%%%%%%%%%%%%%%%%%%%%%%%%

\begin{frame}
  \frametitle{A Useful Lemma}
  
\begin{lemma}
  \label{lem:equiv-cond-exist-1}
  Let $\bA$ be an algebra, $t(x,y,z)$ a ternary term,
  $\bF := \bF_{\bbV(\bA)}(x,y)$. 

  Consider the following statements:
  \begin{enumerate}[(A)]
  \item \label{item:6} $t^{\bA}$ is not a difference term operation for $\bA$.
  \item \label{item:7} There exists a 2-generated subalgebra $\bB \leq \bA$
    such that $t^{\bB}$ is not a difference term operation for $\bB$.
  \item \label{item:8} $t^{\bF}$ is not a difference term operation for $\bF$.
  \end{enumerate}
  Then (\ref{item:6}) $\Rightarrow$ (\ref{item:7}) and (\ref{item:7})  $\Rightarrow$ (\ref{item:8}).
\end{lemma}
\end{frame}


\begin{frame}[shrink=25]{Proof sketch}
  (\ref{item:6}) $\Rightarrow $ (\ref{item:7}):
  Suppose  $t^{\bA}$ fails to be a difference term operation for $\bA$ and let $a, b \in
  A$ witness this failure. Then either
  \begin{enumerate}
  \item[1.]\label{item:9} $d^{\bA}(a,a,b) \neq b$, or
  \item[2.]\label{item:10} $(d^{\bA}(a,b,b), a) \notin \com{\Cg^{\bA} (a, b)}$.
  \end{enumerate}

  \vskip2mm
  \pause 
  Let $\bB = \Sg^{\bA} (a, b)$.  

  \vskip2mm
  In case 1,
  $d^{\bB}(a,a,b) = d^{\bA}(a,a,b) \neq b$, so $d^{\bB}(x,y,z)$ is not a difference
  term operation for $\bB$.
  
\vskip2mm
\pause
In case 2,
  $(d^{\bB}(a,b,b), a) = (d^{\bA}(a,b,b), a)$ is not in $\com{\Cg^{\bA} (a, b)}$.
  So, assuming 
  \[\com{\Cg^{\bB} (a, b)} \subseteq \com{\Cg^{\bA} (a, b)},\]
  we have $(d^{\bB}(a,b,b), a) \notin \com{\Cg^{\bB} (a, b)}$, so
  $d^{\bB}(x,y,z)$ is not a difference term for $\bB$.
\note{  Inclusion~(\ref{eq:6}) holds because 
  $\CC{\beta}{\beta}{\delta}$ holds when
  $\beta := \Cg^{\bB} (a, b)$ and
  $\delta:=\com{\Cg^{\bA} (a, b)} \cap B^2$.}
  % (See Appendix~\ref{sec:details-omitted-from} for details.)\\[5pt]
  % (\ref{item:7}) $\Rightarrow$ (\ref{item:8}):

  Since there is a homomorphism from $\bF$ to $\bB$,
  % Lemma~\ref{lem:hom-image-diff-term} implies that 
  $d^{\bF}(x,y,z)$ is not a difference term operation for $\bF$.

\end{frame}

\begin{frame}[shrink=25]{A useful theorem}
\begin{theorem}[Kearnes (1998)]
  \label{thm:F}
  Let $\sV$ be a variety, and $\bF = \bF_{\sV}(2)$ the 2-generated
  free algebra in $\sV$. The following are equivalent:
  \begin{enumerate}[(i)]
  \item \label{item:1}
    $\sV$ has a difference term;
  \item \label{item:2}
    $\sansH \sansS \sansP (\bF)$ has a difference term;
  \item \label{item:3}
    $\bF$ has a difference term operation.
  \end{enumerate}
\end{theorem}
  The implications
  (\ref{item:1}) $\Rightarrow$  (\ref{item:2}) $\Rightarrow$  (\ref{item:3}) are
  obvious. 

  \vskip2mm
  \pause 
  We prove
  (\ref{item:3}) $\Rightarrow$  (\ref{item:1}) by contraposition.
  Suppose $\sV$ has no difference term. 

  \vskip2mm
  Let $d(x,y,z)$ be an arbitrary ternary term.
  
  \vskip2mm
  Let $\bA\in \sV$ be such that $d^{\bA}(x,y,z)$ is not a difference term
  operation. 

  \vskip2mm
  By the previous lemma % Lemma~\ref{lem:equiv-cond-exist-1},
  $d^{\bF}(x,y,z)$ is not a difference term operation for $\bF$.
  %% Since $d(x,y,z)$ is arbitrary, it follows that
  %% $\bF$ has no difference term operation whatsoever.
  %% , as we set out to prove.
\end{frame}




















\end{document}


Digital computers have turned out to be invaluable tools for exploring and
understanding algebras and the varieties they inhabit, and this is largely due
to the fact that researchers have found ingenious ways
to get computers to solve abstract decision problems---such as
whether a variety is 
congruence-modular (\cite{Freese:2009}) or
congruence-$n$-permutable (\cite{MR3239624})---and to do so efficiently.
%% , in a way that scales well with problem size.
The contribution of the present paper is to present a solution to the following:
\begin{prob}
  \label{prob:1}
  Is there a polynomial-time algorithm that takes a finite
  idempotent algebra $\bA$ as input and decides whether the variety generated by
  $\bA$ has a difference term?
\end{prob}
By solving Problem~\ref{prob:1} we complete the project started
in~\cite{DeMeo:2017}; in the latter, we solved the following easier problem:
\begin{prob}
  \label{prob:2}
  Is there a polynomial-time algorithm that takes a finite
  idempotent algebra $\bA$ as input and decides whether 
  $\bA$ has a difference term operation?
\end{prob}

A system of linear equations is a CSP
\begin{alignat*}4
  a_{11}x_1 &&+ a_{12}x_2 &&+ \cdots &&+ a_{1n}x_n &= b_1 \\
  a_{21}x_1 &&+ a_{22}x_2 &&+ \cdots &&+ a_{2n}x_n &= b_2 \\
  &&\;\vdots \\
  a_{m1}x_1 &&+ a_{m2}x_2 &&+ \cdots &&+ a_{mn}x_n &= b_m \\
\end{alignat*}
\end{frame}

\begin{frame}
  Also, a system of nonlinear equations is a CSP
\begin{alignat*}4
  a_{11}x_1^2x_3 &+ a_{12}x_2x_3x_7 &&+ \cdots &&+ a_{1n}x_4x_n^3 &&= b_1 \\
  a_{21}x_2x_5 &+ a_{22}x_2 &&+ \cdots &&+ a_{2n}x_4^3 &&= b_2 \\
 &&&\vdots \\
  a_{m1}x_3x_5x_8 &+ a_{m2}x_2 &&+ \cdots &&+ a_{mn}x_n &&= b_m \\
\end{alignat*}
\end{frame}

\begin{frame}
  For a fixed $k$, determining whether a graph is $k$-colorable is a CSP
  \begin{center}
    \includegraphics{../inputs/k-col}
%    \fbox{picture suggesting 3-colorability}
  \end{center}
\end{frame}


\begin{frame}
Determining whether a given formula $\phi(x_1,\dots,x_n)$ is satisfiable is a CSP
\bigskip
For example,
\begin{equation*}
\phi(x,y,z) = (x\join y \join \neg z) \meet (\neg x\join y \join \neg z)
\end{equation*}
\pause

is satisfiable. e.g., $(x,y,z) = (0,0,1)$
\end{frame}

\begin{frame}
  \frametitle{Algorithms}
  There is an \emph{efficient} algorithm (Gaussian elimination) for solving any
  linear system.
  That is

% \setbeamercolor{block body example}%
% {parent=normal text,use=block title example,bg=block title example.bg!25!bg}
  \begin{exampleblock}{}
    There is an algorithm that accepts as input a linear system
    and decides whether that system has a solution.

    \smallskip
    The running time of the algorithm is bounded by $f(s)$,
    a \emph{polynomial} in the size $s$ of the system.
  \end{exampleblock}
  \pause

  The \alert{input}, a particular system, is an \alert{instance} of the \alert{problem}
  LINEAR SYSTEM.
  
\end{frame}

\begin{frame}
  Similarly

  % \setbeamercolor{block body example}%
  % {parent=normal text,use=block title example,bg=block title example.bg!25!bg}
  \begin{exampleblock}{}
    There is an algorithm that accepts as input a graph and decides
    whether the graph is 2-colorable.

    \smallskip
    Running time bounded by $f(s)$, a \emph{polynomial} in 
    the size $s$ of the graph.
  \end{exampleblock}

  The \alert{input}, a particular graph, is an \alert{instance} of the \alert{problem} 
  2-COLORABILITY.
\end{frame}


\begin{frame}
  % \setbeamercolor{block body example}%
  % {parent=normal text,use=block title example,bg=block title example.bg!25!bg}
  \begin{exampleblock}{}
  There is an algorithm that accepts as input a formula, $\phi = \phi_1 \meet \phi_2 \meet \cdots \meet \phi_k$ (each $\phi_i$ bijunctive) and decides whether $\phi$ is satisfiable.
  
  \smallskip
  Running time bounded by $f(s)$, a \emph{polynomial} in the length $s$ of a string
  describing $\phi$.
  \end{exampleblock}
  
  The \alert{intput} formula $\phi$ is an \alert{instance} of the \alert{problem} 2-SAT.

  % \bigpause
  We say that all these algorithms run in \alert{polynomial time}.
\end{frame}

\begin{frame}
  No polynomial-time algorithm is known for, NONLINEAR SYSTEM,
  3-COLORABILITY, or 3-SAT.

  \bigpause
  However, any candidate solution to either of these problems can be
  checked in polynomial-time. 

  \bigpause
  Thus these problems are solvable in \alert{nondeterministic polynomial
    time.} 
\end{frame}

\begin{frame}
  Let $X$ and $Y$ be two problems. We write $X \reduc Y$ to indicate that
  $Y$ is at least as hard as $X$.

  \bigpause
  Somewhat more precisely: any algorithm for solving $Y$ can be
  transformed into an algorithm for $X$ without drastically increasing
  its running time.

  \bigpause
  It is possible for $X\reduc Y \reduc X$. In that case, write $X\equivp
  Y$. 

\end{frame}

\begin{frame}
  \P is the class of all problems solvable in polynomial time. Its
  members are called \alert{tractable.}

  \NP is the class of problems solvable in nondeterministic polynomial
  time. 

  \pause
  \begin{itemize}
  \item $\P \sseq \NP$
  \item Both $\P$ and $\NP$ are downsets, i.e., \\
    $Y\in \P \mathrel{\&} X\reduc Y \implies X\in \P$
  \end{itemize}

  \pause
  The maximal members of \NP are called \alert{\NP-complete.} 

  
  3-COLORABILITY, NONLINEAR SYSTEM, and 3-SAT are known to be
  \NP-complete. 
\end{frame}

\begin{frame}
  \frametitle{\$$2^{20}$ question: $\P \overset{?}{=} \NP$.}
  \includegraphics[scale=.25]{../inputs/cash.png}
  \pause
  \parbox[b][2in][c]{2in}{
      If $\P=\NP$ then all of the above distinctions go away. 
    
     \bigskip

      Most problems we care about can be solved
      efficiently. Just build bigger computers.
    }
      \bigskip
      \pause
      In particular, this talk becomes pointless. So assume $\P\neq \NP$.

  \pause
  \setbeamercolor{block body theorem}%
  {parent=normal text,use=block title theorem,bg=block title theorem.bg!25!bg}
  \begin{exampleblock}{Ladner, 1975 \cite{Ladner1975}} % [Ladner, 1975 \cite{Ladner1975}]
  % \begin{theorem}[Ladner, 1975 \cite{Ladner1975}]
    If $\P \neq \NP$, then there are problems in $\NP\setminus
    \P$ that are not \NP-complete.
  % \end{theorem}
  \end{exampleblock}
\end{frame}

\begin{frame}
  \includegraphics[scale=1.25]{../inputs/complexity}
  \qquad
  \parbox[b][2in][c]{1.5in}{If $\P\neq \NP$ then the pink area is nonempty.}
\end{frame}



%%%%%%%%%%%%%%%%%%%%%%%%%%%%%%%%%%%%%%%%%%%%%%%%%%%%%%%%%%%
%% 1: CSP Intro 1
\begin{frame}
  \frametitle{Oversimplified Definition of CSP}
  
  {\bf Input}
  \begin{itemize}
  \item \emph{variables:} $\sV = \{v_1, v_2, \dots\}$
  \item \emph{domain:}  $\sD$
  \item \emph{constraints:} $C_1, C_2, \dots$
  \end{itemize}
\pause
  \vskip.5cm
  % \underline{Output}
  {\bf Output}
  \begin{itemize}
  \item ``yes'' if there is a \emph{solution}   $f : \sV \rightarrow \sD$ 

    {\small (assigning values to variables and satisfying all $C_i$)}

  \item ``no'' otherwise
  \end{itemize}

\end{frame}





\begin{frame}
  \frametitle{Slightly more formally...}

  Let $D$ be a set, $n$ a positive integer

  An \emph{$n$-ary relation on $D$} is a subset of $D^n$

  \pause
  $\Rel_n(D)$ denotes the set of all $n$-ary relations on $D$

  $\Rel(D) = \bigcup\limits_{n< \omega} \Rel_n(D)$ is the set of all finitary
  relations on $D$
\end{frame}

\begin{frame}  
  \frametitle{Slightly more formal defintion of CSP}

  Let $D$ be a finite set and $\sR \sseq \Rel(D)$

  $\CSP(D,\sR)$ is the following decision problem:

  \textbf{Instance:}
\vskip-3mm
  \begin{itemize}
  \item 
  \alert{variables}: $V=\setof{v_1,\dots,v_n}$ (a finite set) 

\item  \alert{constraints}: $(C_1,\dots,C_m)$ (a finite list) 
  \end{itemize}
  Each $C_i$ is a pair $(\bs_i,\, R_i)$, 
  where \[\bs_i(j) \in V\quad \text{ and } \quad R_i\in \sR\]

  \pause
  \textbf{Question:} Does there exist a \alert{solution}? 

  an assignment $f \colon V\to D$ such that, for all $i\leq m$,
\[(f(\bs_i(1)),\dots,f(\bs_i(p)))\in R_i \]

    % \pause
    % $\CSP(D,\sR)$ always lies in \NP.

%    $\CSP\<D,\sR\>$ is \emph{finitary} if $\sR$ is finite. 
\end{frame}


\begin{frame}
  \frametitle{Example: 3-colorability}

  $D=\{\Red{r},\Green{g},\Blue{b}\}$, \quad $\sR=\{R\}$

  $R =\setof{(x,y)\in D\times D : x\neq y}$

  Then $\CSP(D,\sR)$ is the 3-colorability problem

  \bigskip

    \begin{columns}
      \begin{column}{1.5in}
        \includegraphics{../inputs/col_csp}
      \end{column}
      \begin{column}{2in}
        $V=\{v_1,\dots,v_6\}$\\
        $\bs_1 = (v_1,v_2)$\\
        $\bs_2 = (v_2,v_4)$\\
        % $\bs_3 = (v_1,v_4)$\\
        % $\bs_4=(v_2,v_4)$\\
        \qquad$\vdots$\\
        $\bs_m = (v_5,v_6)$\\[4pt]
        $R_i = R$ for every $i$.
      \end{column}
    \end{columns}
  \end{frame}

% \begin{comment}
  
\begin{frame}
\frametitle{Example: 2-SAT}
$D=\{0,1\}$\\
For a bijunctive clause $\phi(x,y)$, 
\begin{equation*}
R_\phi= \setof{\<a,b\>\in D^2: \phi(a,b)=1}
\end{equation*}

\pause
\begin{tabular}{cc|c}
 $x$ & $y$ &$x\join y'$  \\
 \hline
0 & 0 & 1\\
\visible<2>{0 & 1 & 0}\\
1 & 0 & 1\\
1 & 1 & 1
\end{tabular}
\uncover<3->{\qquad \parbox{2.5in}{So $R_{x\join y'} = \{\<00\>,\,\<10\>,\,\<11\>\}%=\\ \{0,1\}^2-\{\<01\>\}
$}}


\uncover<4->{$\displaystyle\sR=\{R_{x\join y},\, R_{x\join y'},\,R_{x'\join y},\, R_{x'\join y'}\}$}

\uncover<5->{
2-SAT is $\CSP(D,\sR)$}
\end{frame}

\begin{comment}
  
\begin{frame}
\frametitle{Example: Horn-SAT}
\alert{Horn formula:}\\
 $(x_0 \meet x_1 \meet \cdots \meet x_{i-1} \meet x_{i+1} \meet \cdots \meet x_{n-1}) \rightarrow x_i$\\[6pt]
\alert{Equivalently:}\\
 $x_0' \join x_1' \join \dots x_{i-1}' \join x_i \join \cdots \join x_{n-1}'$

\bigpause
Corresponding relation $\gamma^n_i = \{0,1\}^n-\{\<111\dots0\dots111\>\}$

\bigpause
Horn-SAT is $\CSP(D,\Gamma)$\\
$D=\{0,1\}$,\quad $\Gamma=\setof{\gamma^n_i : 0\leq i <n}$
\end{frame}

\end{comment}

\begin{frame}
  \frametitle{Schaefer's Dichotomy}

  \begin{exampleblock}{Schaefer, 1978 \cite{Schaefer1978}}
    Let $D=\{0,1\}$. There are six families $\sR_0,
  \dots, \sR_5$ such that
  \begin{equation*}
    \CSP(D,\sR) \in \P \iff \sR \sseq \sR_i, \text{some $i< 6$}
  \end{equation*}
  Otherwise $\CSP(D,\sR)$ is $\NP$-complete.
\end{exampleblock}
% \end{theorem}
\end{frame}

\begin{frame}
{\large The six families}
$\sR_0 = \setof{R : \<0,0,\dots,0\>\in R}$ (``All False'')\\[2pt]
$\sR_1 = \setof{R : \<1,1,\dots,1\>\in R}$ (``All True'')\\[2pt]
$\sR_2 = \{R_{x\join y},\, R_{x\join y'},\,R_{x'\join y},\, R_{x'\join y'}\}$ (bijunctive)\\[2pt]
$\sR_3 = \Gamma$ (Horn)\\[2pt]
$\sR_4 = \Gamma^\partial$ (dual-Horn)\\[2pt]
$\sR_5$ (affine, i.e., linear system over $\FF_2$)
\end{frame}


\begin{frame}
  \frametitle{Two Motivating Questions}

  \begin{enumerate}
  \item \alert{Dichotomy Conjecture}\\ Every $\CSP(D,\sR)$ either
    lies in \P\ or is $\NP$-complete.

    \pause

  \item \alert{Tractability Problem}\\ Characterize those CSPs that lie in \P.
  \end{enumerate}
  
  \pause
  What would a characterization look like? What language could we use?
\end{frame}
% \begin{comment}
  
\begin{frame}
\frametitle{Why is 2-SAT tractable, but 3-SAT is not?}

2-SAT: $\sR_2=\{R_{x\join y},\,R_{x\join y'},\, R_{x'\join y},\, R_{x'\join y'}\} $

\smallskip
3-SAT: $\Lambda=\{\lambda_0,\lambda_1,\dots, \lambda_7\}$

\pause
\begin{equation*}
M(x,y,z) = \begin{cases}
	0 \quad&\text{if at least 2 of $x,y,z$ equal 0}\\
	1 &\text{otherwise}
	\end{cases}
\end{equation*}
``Majority Operation''

\pause
$M$ preserves each $R\in \sR_2$:
\begin{equation*}
\begin{matrix}
\<a_1, &b_1\> &\in R\\
\<a_2, &b_2\> &\in R\\
\<a_3, &b_3\> &\in R \text{ implies}\\
\<M(a_1,a_2,a_3), &M(b_1,b_2,b_3)\> &\in R
\end{matrix}
\end{equation*}

\end{frame}

\begin{frame}
But $M$ fails to preserve each $\lambda\in \Lambda$

\medskip
For example, with $\lambda=\lambda_{x\join y\join z'}=\{0,1\}^3-\{\<001\>\}$ 

\begin{align*}
\<1,0,0\> &\in \lambda\\
\<0,0,1\> &\in \lambda\\
\<0,1,1\> &\in \lambda \text{ but}\\
\<0,0,1\> &\notin \lambda
\end{align*}

\end{frame}
% \end{comment}

\begin{frame}
  \frametitle{Polymorphisms}

  \begin{exampleblock}{Definition}
    Let $R \in \Rel_k(D)$ and $f\: D^n \to D$. We say 
    \emph{$f$ preserves $R$} if
    \begin{equation*}
      \begin{split}
        (a_{11}, \dots, a_{1k}),&\dots, (a_{n1},\dots, a_{nk}) \in
        R \implies\\ 
        &\bigl( f(a_{11},\dots, a_{n1}), \dots, f(a_{1k},\dots,a_{nk})
        \bigr) \in R
    \end{split}
    \end{equation*}
  \end{exampleblock}
% \end{definition}

  \begin{overprint}
  \onslide<2|handout:0>
  $f$ is an \emph{$n$-ary operation} on $D$.
  % 
  \onslide<3|handout:1>
    \begin{equation*}
      \newcommand\flab{\scriptstyle{f}}
      \begin{matrix} a_{11} & a_{12} & \dots & a_{1k} & \in & R \\
        a_{21} & a_{22} & \dots & a_{2k} & \in & R \\
        \vdots  & \vdots &       &\vdots  &     & \vdots \\
        a_{n1} & a_{n2} & \dots & a_{nk} & \in & R \\
        \downarrow\flab &\downarrow\flab &  & \downarrow\flab \\
        \star  & \star  & \dots & \star  & \in & R
      \end{matrix}
    \end{equation*}
  \end{overprint}
\end{frame}

\begin{frame}
  % \begin{definition}
  \begin{exampleblock}{Definition}
    Let $\sR$ be a set of relations on $D$.

    \bigskip
    \emph{$\Pol(\sR)$} is the set of operations preserving
    all members of $\sR$. 

    \bigskip
    These are the \alert{polymorphisms} of  $\sR$. 

    \bigpause

    Let $F$ be a set of operations on $D$. 

    \bigskip
    \emph{$\Inv(F)$} is the set of relations preserved by all operations in $F$.
  \end{exampleblock}
% \end{definition}
  
  \bigpause
Important point: \emph{$(D,\Pol(\sR))$ is an algebraic structure}

\end{frame}

\begin{frame}
  \begin{exampleblock}{Theorem}
  % \begin{theorem}
    Let $\sS, \sR \sseq \Rel(D)$. Then
    \begin{equation*}
      \Pol(\sS) \sseq \Pol(\sR) \implies \CSP(\sR) \reduc
      \CSP(\sS). 
    \end{equation*}
  \end{exampleblock}

\pause
Thus, the richer the algebraic structure, the easier the corresponding CSP
\end{frame}
% \begin{comment}
  
\begin{frame}
Schaefer proved that on $D=\{0,1\}$, there are 4 key polymorphisms:

\begin{center}
$M(x,y,z)$ (majority)\\
$x\meet y$\\
$x\join y$\\
$P(x,y,z) =x\oplus y \oplus z = x-y+z$
\end{center}

\bigskip
$(\{0,1\},\sR)$ is tractable iff one of these is a polymorphism of $\sR$

\bigpause
(Un)fortunately, things are more complicated when $\card{D}>2$.
\end{frame}

% \end{comment}

\begin{frame}
\frametitle{Galois Connection}
  One can go back and forth between relational and algebraic structures

  \begin{center}
    \begin{tabular}{ccc}
      \origtextbf{Relational} & &\origtextbf{Algebraic}  \\
      % \origtextbf{Structures} & &\origtextbf{Structures}  \\
      $(D,\sR)$ & $\longrightarrow$ & $(D,\Pol(\sR))$ \\[2pt]
      $(D, \Inv(F))$ & $\longleftarrow$ &$(D,F)$
    \end{tabular}
  \end{center}

  $\CSP(D,\sR) \equivp \CSP(D,\Inv(\Pol(\sR)))$

  \bigpause
  Perhaps expressive power of algebra can help classify CSPs.

  \end{frame}

% \begin{comment}
  
\begin{frame}
\frametitle{The Relational Clone}
For a set $\sR$ of relations on $D$, let $\<\sR\> =
\Inv\bigl(\Pol(\sR)\bigr)$.

$\<\sR\>$ is called the \alert{relational clone} generated by $\sR$.

It coincides with the set of relations definable from $\sR$ by\\
 \emph{primitive positive formulas}.

\pause
$\phi(x_1,\dots,x_n)=(\exists y_1)(\exists y_2)\cdots(\exists y_m)\bigl(R_1(z_{1_1},\dots,z_{1_k}) \meet \dots \meet R_t(z_{t_1},\dots,z_{t_j})\bigr)$

Here $R_1\dots,R_t \in \sR$ and every $z_{i_j} \in \{x_1,\dots,x_n,y_1,\dots,y_m\}$
\end{frame}
% \end{comment}



\begin{frame}
  \frametitle{Algebraic Facts}

  For an algebra $\bA = \<A, F\>$ define 

  \[\CSP(\A) = \CSP(A, \Inv(F))\]

  \pause

  Let \A\ and \B\ be algebras

  $\B \text{ a subalgebra of \A} \implies \CSP(\B) \reduc \CSP(\A)$.

  $\B \text{ a homomorphic image of \A}\implies \CSP(\B) \reduc
  \CSP(\A)$. 
  
  $\CSP(\A^n) \equivp \CSP(\A)$

\end{frame}


\begin{frame}
  \begin{exampleblock}{Bulatov, Jeavons, Krokhin, 2000
    \cite{BulatovKrokhinJeavons2000}}
  % \begin{theorem}[Bulatov, Jeavons, Krokhin, 2000
    % \cite{BulatovKrokhinJeavons2000}]
    If $(D,\sR)$ is a ``core'' and every polymorphism is essentially unary,
    then $\CSP(\sR)$ is \NP-complete.
  % \end{theorem}
  \end{exampleblock}

  $f$ is \emph{essentially unary} if $f(x_1,\dots,x_n) = g(x_j)$ for
  some unary $g$ and some $j\leq n$.

 \pause
  \begin{exampleblock}{Corollary}
 % \begin{corollary}
    3-COLORABILITY, NONLINEAR SYSTEM, and 3-SAT are \NP-complete.
  % \end{corollary}
  \end{exampleblock}

\end{frame}


\begin{frame}
 
  \textbf{Informal reformulation of the dichotomy conjecture}\\
  If \A\ has some
  kind of decent algebraic structure then $\CSP(\A) \in \P$ otherwise
  $\CSP(\A)$ is \NP-complete.
\end{frame}

\begin{frame}
  % \begin{definition}
  \begin{exampleblock}{Definition}
    Let $n>1$. An $n$-ary operation $f$ is called a \emph{weak
      near-unanimity operation} if 
       \begin{equation*}
       \begin{gathered}
       f(x,x,\dots,x) = x \text{ and}\\
    f(y,x,x,x,\dots,x) = f(x,y,x,x,\dots,x) = \cdots\\
    = f(x,x,\dots,x,y)
  \end{gathered}
  \end{equation*}
  % \end{definition}
\end{exampleblock}  
Note: no essentially unary operation is WNU
\end{frame}

\begin{frame}
  \begin{exampleblock}{Bulatov, Larose, Z\'adori, McKenzie, Mar\'oti}
  % \begin{theorem}[Bulatov, Larose, Z\'adori, McKenzie, Mar\'oti
  %   \cite{BulatovJeavonsKrokhin2005,LaroseZadori2003,%
  %   MarotiMcKenzie2008}]
    If $\sR$ is a core and $\Pol(\sR)$ has no WNU operation
    then $\CSP(\sR)$ is \NP-complete.    
  % \end{theorem}
  \end{exampleblock}
\end{frame}


\begin{frame}
  \frametitle{Reformuated Dichotomy Conjecture}

  Let $\sR$ be a core. Then $\CSP(\sR)$ is tractable if and only
  if it has a WNU polymorphism. Otherwise, it is \NP-complete. 

  \bigskip
  \begin{overprint}
    \onslide<2|handout:0> \centering{\includegraphics{../inputs/dichotomy1}}
    \onslide<3|handout:0> \centering{\includegraphics{../inputs/dichotomy2}}
    \onslide<4|handout:0> \centering{\includegraphics{../inputs/dichotomy3}}
    \onslide<5-|handout:1>
    \begin{exampleblock}{Supporting Examples}
      \begin{itemize}
      \item 2-SAT, 2-COLARABILITY, LINEAR SYSTEM have a WNU.
      
             \item<6-> Let \A\ be an abelian group, $n=|A|$. Choose
        integers $k, l$ with $kl \equiv 1 \pmod n$. Then
        \begin{equation*}
          f(x_1,\dots,x_k) = l(x_1+\cdots + x_k)
        \end{equation*}
        is a WNU operation.
      \end{itemize}
    \end{exampleblock}
  \end{overprint}
\end{frame}

\begin{frame}
\frametitle{Two General Techniques for Tractable Algorithms}
%\begin{enumerate}
\begin{exampleblock}{Method 1}
If $\Pol(\sR)$ contains a ``cube operation'' then $\CSP(\sR)\in \P$
%\end{enumerate}
\end{exampleblock}

\pause
Examples of cube operations: 

$P(x,y,z) = x-y+z$\\
$M(x,y,z) = \text{majority}$

Essentially a generalization of Gaussian elimination. 

Algebras with a cube operation possess ``few subpowers''. This algebraic property is used to prove that the algorithm terminates in polynomial time.

\end{frame}

\begin{frame}
\begin{exampleblock}{Method 2}
If $\Pol(\sR)$ contains WNU operations $v(x,y,z)$ and $w(x,y,z,u)$ satisfying $v(y,x,x)= w(y,x,x,x)$, then $\CSP(\sR)\in \P$.
\end{exampleblock}

\pause
Examples: majority, semilattice 

Algebras with these operations have a property called ``congruence meet-semidistributivity.'' 

\end{frame}

\begin{frame}
\frametitle{Current State of Affairs}
The two general techniques do not cover all cases of a WNU. What to do next? 

\medskip
Two possible directions:
\begin{enumerate}
\item Find a completely new algorithm.
\item Combine the two existing algorithms.
\end{enumerate}

I am exploring both  approaches.
\end{frame}


\end{document}



\begin{frame}
  \frametitle{Overcomplicated Definition of CSP}

  $\bA = (A, \sF)$ is a finite idempotent algebra,
  $\Sub(\bA)$ is all subuniverses of $\bA$.

  {\it In this talk} $\CSP(\bA)$ denotes the following decision problem:

   An \alert{instance of degree} $n$ of $\CSP(\bA)$ is the tuple $\<\sV, \sA, \sS, \sR\>$ 
   \begin{itemize}
   \item \emph{variables} $\sV = \{0, 1, \dots, n-1\}$;
   \item \emph{domains} $\sA = \{\bA_0, \bA_1, \dots, \bA_{n-1}\} \subset \Sub(\bA)$  (one for each variable)
   \item \emph{scope functions} $\sS = (\bs_0, \bs_1, \dots, \bs_{p-1})$ 
     with \emph{constraint arities} $\ar(\sS) = (m_0, m_1, \dots, m_{p-1})$
   \item \emph{constraint relations} $\sR = (\bR_0, \bR_1, \dots, \bR_{p-1})$, where
     \[\bR_i \leq \bA_{\bs_i(0)} \times \bA_{\bs_i(1)}\times \cdots \times \bA_{\bs_i(m_i-1)}.\]
   \end{itemize}
 \end{frame}  
 % \bigskip
  
%   A \alert{solution} to $\<\sV, \sA, \sS, \sR\>$ is an assignment
%   $f: \sV \to A$ of values to variables that satisfies all constraints. That is,
%   \[f\in \myprod_{\sV}A_j
%     \quad  \text{and} \quad 
%     \Proj_{\bs_i} f \in \bR_i, \;\text{ for each $0\leq i < p$.}
%   \]

%   \pause
%   {\bf Notation:} $\nn = \{0,1,\dots, n-1\}$, so the $i$-th scope
%   has type $\bs_i : \mm_i \to \nn$ and
%   \[ \Proj_{\bs_i} f  = f \circ \bs_i \]
% \end{frame}





%%% Local Variables: 
%%% mode: latex
%%% TeX-master: t
%%% End: 
