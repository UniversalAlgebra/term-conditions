%% FILE: diffTerm.tex
%% AUTHOR: William DeMeo, Ralph Freese, Matthew Valeriote
%% DATE: 13 April 2017
%% COPYRIGHT: (C) 2017 DeMeo, Freese, Valeriote

%%%%%%%%%%%%%%%%%%%%%%%%%%%%%%%%%%%%%%%%%%%%%%%%%%%%%%%%%%
%%                         BIBLIOGRAPHY FILE            %%
%%%%%%%%%%%%%%%%%%%%%%%%%%%%%%%%%%%%%%%%%%%%%%%%%%%%%%%%%%
%% The `filecontents` command will crete a file in the inputs directory called 
%% refs.bib containing the references in the document, in case this file does 
%% not exist already.
%% If you want to add a BibTeX entry, please don't add it directly to the
%% refs.bib file.  Instead, add it in this file between the
%% \begin{filecontents*}{refs.bib} and \end{filecontents*} lines
%% then delete the existing refs.bib file so it will be automatically generated 
%% again with your new entry the next time you run pdfaltex.
\begin{filecontents*}{inputs/refs.bib}
@article {MR1871085,
    AUTHOR = {Bergman, Clifford and Slutzki, Giora},
     TITLE = {Computational complexity of some problems involving
              congruences on algebras},
   JOURNAL = {Theoret. Comput. Sci.},
  FJOURNAL = {Theoretical Computer Science},
    VOLUME = {270},
      YEAR = {2002},
    NUMBER = {1-2},
     PAGES = {591--608},
      ISSN = {0304-3975},
     CODEN = {TCSDI},
   MRCLASS = {08A30 (05C85 08A35 68Q17)},
  MRNUMBER = {1871085 (2002i:08002)},
MRREVIEWER = {Radim B{\v{e}}lohl{\'a}vek},
       DOI = {10.1016/S0304-3975(01)00009-3},
       URL = {http://dx.doi.org/10.1016/S0304-3975(01)00009-3},
}
@article {MR1695293,
    AUTHOR = {Bergman, Clifford and Juedes, David and Slutzki, Giora},
     TITLE = {Computational complexity of term-equivalence},
   JOURNAL = {Internat. J. Algebra Comput.},
  FJOURNAL = {International Journal of Algebra and Computation},
    VOLUME = {9},
      YEAR = {1999},
    NUMBER = {1},
     PAGES = {113--128},
      ISSN = {0218-1967},
   MRCLASS = {68Q17 (08A70 68Q15)},
  MRNUMBER = {1695293 (2000b:68088)},
       DOI = {10.1142/S0218196799000084},
       URL = {http://dx.doi.org/10.1142/S0218196799000084},
}
@article {MR3449235,
    AUTHOR = {Kearnes, Keith and Szendrei, {\'A}gnes and Willard, Ross},
     TITLE = {A finite basis theorem for difference-term varieties with a
              finite residual bound},
   JOURNAL = {Trans. Amer. Math. Soc.},
  FJOURNAL = {Transactions of the American Mathematical Society},
    VOLUME = {368},
      YEAR = {2016},
    NUMBER = {3},
     PAGES = {2115--2143},
      ISSN = {0002-9947},
   MRCLASS = {03C05 (08B05 08B10)},
  MRNUMBER = {3449235},
       DOI = {10.1090/tran/6509},
       URL = {http://dx.doi.org/10.1090/tran/6509},
}
@article {MR1663558,
    AUTHOR = {Kearnes, Keith A. and Szendrei, {\'A}gnes},
     TITLE = {The relationship between two commutators},
   JOURNAL = {Internat. J. Algebra Comput.},
  FJOURNAL = {International Journal of Algebra and Computation},
    VOLUME = {8},
      YEAR = {1998},
    NUMBER = {4},
     PAGES = {497--531},
      ISSN = {0218-1967},
   MRCLASS = {08A05 (08A30)},
  MRNUMBER = {1663558},
MRREVIEWER = {M. G. Stone},
       DOI = {10.1142/S0218196798000247},
       URL = {http://dx.doi.org/10.1142/S0218196798000247},
}
@article{KSW,
title = {Simpler maltsev conditions for (weak) difference terms in locally finite varieties},
author = {Kearnes, Keith and Szendrei, \'{A}gnes and Willard, Ross},
note = {to appear}
}

@article {MR3239624,
    AUTHOR = {Valeriote, M. and Willard, R.},
     TITLE = {Idempotent {$n$}-permutable varieties},
   JOURNAL = {Bull. Lond. Math. Soc.},
  FJOURNAL = {Bulletin of the London Mathematical Society},
    VOLUME = {46},
      YEAR = {2014},
    NUMBER = {4},
     PAGES = {870--880},
      ISSN = {0024-6093},
   MRCLASS = {08A05 (06F99 68Q25)},
  MRNUMBER = {3239624},
       DOI = {10.1112/blms/bdu044},
       URL = {http://dx.doi.org/10.1112/blms/bdu044},
}
@article {MR3350327,
    AUTHOR = {Kozik, Marcin and Krokhin, Andrei and Valeriote, Matt and
              Willard, Ross},
     TITLE = {Characterizations of several {M}altsev conditions},
   JOURNAL = {Algebra Universalis},
  FJOURNAL = {Algebra Universalis},
    VOLUME = {73},
      YEAR = {2015},
    NUMBER = {3-4},
     PAGES = {205--224},
      ISSN = {0002-5240},
   MRCLASS = {08B05 (08A70 08B10)},
  MRNUMBER = {3350327},
MRREVIEWER = {David Hobby},
       DOI = {10.1007/s00012-015-0327-2},
       URL = {http://dx.doi.org/10.1007/s00012-015-0327-2},
}
@article {MR1358491,
    AUTHOR = {Kearnes, Keith A.},
     TITLE = {Varieties with a difference term},
   JOURNAL = {J. Algebra},
  FJOURNAL = {Journal of Algebra},
    VOLUME = {177},
      YEAR = {1995},
    NUMBER = {3},
     PAGES = {926--960},
      ISSN = {0021-8693},
     CODEN = {JALGA4},
   MRCLASS = {08B10 (08B05)},
  MRNUMBER = {1358491},
MRREVIEWER = {H. Peter Gumm},
       DOI = {10.1006/jabr.1995.1334},
       URL = {http://dx.doi.org/10.1006/jabr.1995.1334},
}
@book {MR2839398,
    AUTHOR = {Bergman, Clifford},
     TITLE = {Universal algebra},
    SERIES = {Pure and Applied Mathematics (Boca Raton)},
    VOLUME = {301},
      NOTE = {Fundamentals and selected topics},
 PUBLISHER = {CRC Press, Boca Raton, FL},
      YEAR = {2012},
     PAGES = {xii+308},
      ISBN = {978-1-4398-5129-6},
   MRCLASS = {08-02 (06-02 08A40 08B05 08B10 08B26)},
  MRNUMBER = {2839398 (2012k:08001)},
MRREVIEWER = {Konrad P. Pi{\'o}ro},
}
@article {MR0434928,
    AUTHOR = {Taylor, Walter},
     TITLE = {Varieties obeying homotopy laws},
   JOURNAL = {Canad. J. Math.},
  FJOURNAL = {Canadian Journal of Mathematics. Journal Canadien de
              Math\'ematiques},
    VOLUME = {29},
      YEAR = {1977},
    NUMBER = {3},
     PAGES = {498--527},
      ISSN = {0008-414X},
   MRCLASS = {08A25},
  MRNUMBER = {0434928 (55 \#7891)},
MRREVIEWER = {James B. Nation},
}
  @BOOK{HM:1988,
    AUTHOR = {Hobby, David and McKenzie, Ralph},
    TITLE = {The structure of finite algebras},
    SERIES = {Contemporary Mathematics},
    VOLUME = {76},
    PUBLISHER = {American Mathematical Society},
    ADDRESS = {Providence, RI},
    YEAR = {1988},
    PAGES = {xii+203},
    ISBN = {0-8218-5073-3},
    MRCLASS = {08A05 (03C05 08-02 08B05)},
    MRNUMBER = {958685 (89m:08001)},
    MRREVIEWER = {Joel Berman},
    note = {Available from:
      \href{http://math.hawaii.edu/~ralph/Classes/619/HobbyMcKenzie-FiniteAlgebras.pdf}{math.hawaii.edu}}
  }
@article {MR0455543,
    AUTHOR = {Jones, Neil D. and Laaser, William T.},
     TITLE = {Complete problems for deterministic polynomial time},
   JOURNAL = {Theoret. Comput. Sci.},
  FJOURNAL = {Theoretical Computer Science},
    VOLUME = {3},
      YEAR = {1976},
    NUMBER = {1},
     PAGES = {105--117 (1977)},
      ISSN = {0304-3975},
   MRCLASS = {68A20},
  MRNUMBER = {0455543},
MRREVIEWER = {Forbes D. Lewis},
       DOI = {10.1016/0304-3975(76)90068-2},
       URL = {http://dx.doi.org/10.1016/0304-3975(76)90068-2},
}
	
  @article {Freese:2009,
    AUTHOR = {Freese, Ralph and Valeriote, Matthew A.},
    TITLE = {On the complexity of some {M}altsev conditions},
    JOURNAL = {Internat. J. Algebra Comput.},
    FJOURNAL = {International Journal of Algebra and Computation},
    VOLUME = {19},
    YEAR = {2009},
    NUMBER = {1},
    PAGES = {41--77},
    ISSN = {0218-1967},
    MRCLASS = {08B05 (03C05 08B10 68Q25)},
    MRNUMBER = {2494469 (2010a:08008)},
    MRREVIEWER = {Clifford H. Bergman},
    DOI = {10.1142/S0218196709004956},
    URL = {http://dx.doi.org/10.1142/S0218196709004956}
  }
@article {MR3076179,
    AUTHOR = {Kearnes, Keith A. and Kiss, Emil W.},
     TITLE = {The shape of congruence lattices},
   JOURNAL = {Mem. Amer. Math. Soc.},
  FJOURNAL = {Memoirs of the American Mathematical Society},
    VOLUME = {222},
      YEAR = {2013},
    NUMBER = {1046},
     PAGES = {viii+169},
      ISSN = {0065-9266},
      ISBN = {978-0-8218-8323-5},
   MRCLASS = {08B05 (08B10)},
  MRNUMBER = {3076179},
MRREVIEWER = {James B. Nation},
       DOI = {10.1090/S0065-9266-2012-00667-8},
       URL = {http://dx.doi.org/10.1090/S0065-9266-2012-00667-8},
}
@incollection {MR1404955,
    AUTHOR = {Kearnes, Keith A.},
     TITLE = {Idempotent simple algebras},
 BOOKTITLE = {Logic and algebra ({P}ontignano, 1994)},
    SERIES = {Lecture Notes in Pure and Appl. Math.},
    VOLUME = {180},
     PAGES = {529--572},
 PUBLISHER = {Dekker, New York},
      YEAR = {1996},
   MRCLASS = {08B05 (06F25 08A05 08A30)},
  MRNUMBER = {1404955 (97k:08004)},
MRREVIEWER = {E. W. Kiss},
}
@misc{william_demeo_2016_53936,
  author       = {DeMeo, William and Freese, Ralph},
  title        = {AlgebraFiles v1.0.1},
  month        = May,
  year         = 2016,
  doi          = {10.5281/zenodo.53936},
  url          = {http://dx.doi.org/10.5281/zenodo.53936}
}
@article{FreeseMcKenzie2016,
	Author = {Freese, Ralph and McKenzie, Ralph},
	Date-Added = {2016-08-22 19:43:56 +0000},
	Date-Modified = {2016-08-22 19:45:50 +0000},
	Journal = {Algebra Universalis},
	Title = {Mal'tsev families of varieties closed under join or Mal'tsev product},
	Year = {to appear}
}
@article {MR2333368,
    AUTHOR = {Kearnes, Keith A. and Tschantz, Steven T.},
     TITLE = {Automorphism groups of squares and of free algebras},
   JOURNAL = {Internat. J. Algebra Comput.},
  FJOURNAL = {International Journal of Algebra and Computation},
    VOLUME = {17},
      YEAR = {2007},
    NUMBER = {3},
     PAGES = {461--505},
      ISSN = {0218-1967},
   MRCLASS = {08A35 (08B20 20B25)},
  MRNUMBER = {2333368},
MRREVIEWER = {Giovanni Ferrero},
       DOI = {10.1142/S0218196707003615},
       URL = {http://dx.doi.org/10.1142/S0218196707003615},
}
@article {MR2504025,
    AUTHOR = {Valeriote, Matthew A.},
     TITLE = {A subalgebra intersection property for congruence distributive
              varieties},
   JOURNAL = {Canad. J. Math.},
  FJOURNAL = {Canadian Journal of Mathematics. Journal Canadien de
              Math\'ematiques},
    VOLUME = {61},
      YEAR = {2009},
    NUMBER = {2},
     PAGES = {451--464},
      ISSN = {0008-414X},
     CODEN = {CJMAAB},
   MRCLASS = {08B10 (08A30 08B05)},
  MRNUMBER = {2504025},
MRREVIEWER = {Jarom{\'{\i}}r Duda},
       DOI = {10.4153/CJM-2009-023-2},
       URL = {http://dx.doi.org/10.4153/CJM-2009-023-2},
}
@misc{UACalc,
	Author = {Ralph Freese and Emil Kiss and Matthew Valeriote},
	Date-Added = {2014-11-20 01:52:20 +0000},
	Date-Modified = {2014-11-20 01:52:20 +0000},
	Note = {Available at: {\verb+www.uacalc.org+}},
	Title = {Universal {A}lgebra {C}alculator},
	Year = {2011}
}
@article{Freese2008,
	Author = {Freese, Ralph},
	Date-Added = {2016-08-29 01:31:23 +0000},
	Date-Modified = {2016-08-29 01:32:09 +0000},
	Journal = {Alg. Univ.},
	Pages = {337--343},
	Title = {Computing congruences efficiently},
	Volume = {59},
	Year = {2008}
}	
@article {MR2470585,
    AUTHOR = {Freese, Ralph},
     TITLE = {Computing congruences efficiently},
   JOURNAL = {Algebra Universalis},
  FJOURNAL = {Algebra Universalis},
    VOLUME = {59},
      YEAR = {2008},
    NUMBER = {3-4},
     PAGES = {337--343},
      ISSN = {0002-5240},
   MRCLASS = {08A30 (08A40 68W30 68W40)},
  MRNUMBER = {2470585 (2009j:08003)},
MRREVIEWER = {Clifford H. Bergman},
       DOI = {10.1007/s00012-008-2073-1},
       URL = {http://dx.doi.org/10.1007/s00012-008-2073-1},
}
@incollection {MR1191235,
    AUTHOR = {Szendrei, {\'A}gnes.},
     TITLE = {A survey on strictly simple algebras and minimal varieties},
 BOOKTITLE = {Universal algebra and quasigroup theory ({J}adwisin, 1989)},
    SERIES = {Res. Exp. Math.},
    VOLUME = {19},
     PAGES = {209--239},
 PUBLISHER = {Heldermann, Berlin},
      YEAR = {1992},
   MRCLASS = {08-02 (08A40 08B05)},
  MRNUMBER = {1191235 (93h:08001)},
MRREVIEWER = {Ivan Chajda},
}
@unpublished{Bergman-DeMeo,
    AUTHOR = {Bergman, Clifford and DeMeo, William},
    TITLE = {Universal Algebraic Methods for Constraint Satisfaction Problems:
      with applications to commutative idempotent binars},
    YEAR = {2016},
    NOTE = {unpublished notes; soon to be available online},
    URL = {https://github.com/UniversalAlgebra/algebraic-csp}
}
\end{filecontents*}
%:biblio
\documentclass[12pt]{amsart}


%% wjd added these packages vvvvvvvvvvvvvvvvvvvvvvvvv
\usepackage{url,amssymb,enumerate,tikz,scalefnt} 
\usepackage[normalem]{ulem} % for \sout (strikeout)   wjd: could remove this in final draft
\usepackage[colorlinks=true,urlcolor=black,linkcolor=black,citecolor=black]{hyperref}
%%  wjd ^^^^^^^^^^^^^^^^^^^^^^^^^^^^^^^^^^^^^^^^^^^^^

\usepackage{amsmath}
\usepackage{amsfonts}
\usepackage{amscd}
%% \usepackage{exers}
\usepackage{inputs/rflatexmacs}
\usepackage{inputs/wjdlatexmacs}

\usepackage[mathcal]{euscript}


\renewcommand{\th}[2]{#1\mathrel{\theta}#2}
\newcommand{\infixrel}[3]{#2\mathrel{#1}#3}

%\usepackage{hyperref}

%\renewcommand{\V}{\text{\textup{V}}}
%\renewcommand{\V}{\operatorname{V}}
%\renewcommand{\T}{\operatorname{T}}
%\newcommand{\U}{\text{U}}

\newtheorem{theorem}{Theorem}
\newtheorem{lemma}[theorem]{Lemma}
\newtheorem{corollary}[theorem]{Corollary}
\newtheorem{prop}[theorem]{Proposition}
\newtheorem{example}[theorem]{Example}
\newtheorem{conjecture}[theorem]{Conjecture}
\newtheorem{fact}[theorem]{Fact}
\newtheorem{prob}{Problem}

\title[A test for a difference term]{A polynomial time test for a
difference term in an idempotent variety}
\author{William DeMeo}
\address[William DeMeo]{
Department of Mathematics\\
University of Hawaii\\
Honolulu, Hawaii\\
96822 USA}
\email[William DeMeo]{demeo@math.hawaii.edu}
\author{Ralph Freese}
\address[Ralph Freese]{
Department of Mathematics\\
University of Hawaii\\
Honolulu, Hawaii\\
96822 USA}
\email[Ralph Freese]{ralph@math.hawaii.edu}
\author{Matthew Valeriote}
\address[Matthew Valeriote]{
Department of Mathematics\\
McMaster University\\
Hamilton, Ontario\\
xxxxx Canada}
\email[Mathhew Valeriote]{matt@math.mcmaster.ca}

\date{\today}

\begin{document}

\maketitle 

\begin{abstract}
We consider the following practical question: given a finite 
algebra $\alg{A}$ in a
finite language, can we efficiently decide whether the variety 
generated by $\alg{A}$
has a difference term?  We answer this question in the idempotent case
and then describe possible algorithms for constructing difference terms.
\end{abstract}

\section{Introduction}
\label{sec:introduction}

A \defn{difference term} for a variety $\sV$ is a ternary term $d$ in the
language of $\sV$ that satisfies the following:
if $\alg{A} = \<A, \dots \> \in \sV$, then for all $a, b \in A$ we have
\begin{equation}
\label{eq:3}  
d^{\alg{A}}(a,a,b) = b \quad \text{ and } \quad
d^{\alg{A}}(a,b,b) \comm \theta \theta a,
\end{equation}
where $\theta$ is any congruence %% of $\alg{A}$
containing $(a,b)$
and $[\cdot, \cdot]$ denotes the \defn{commutator}.
%% (see Section~\ref{sec:defin-notat}).
When the relations in (\ref{eq:3}) hold we call $d^{\alg{A}}$
a \defn{difference term operation} for $\alg{A}$.

Difference terms are studied extensively in the general algebra literature.
(See, for example, \cite{MR1358491,MR1663558,MR3076179,KSW,MR3449235}.)
There are many reasons to study difference terms, but
one obvious reason is because if we know that a variety 
has a difference term, this fact allows us to deduce many useful
properties of the algebras inhabiting that variety.
Very roughly speaking, having a difference term is slightly stronger than having
a Taylor term and slightly weaker than having a Mal'cev term.
(Note that if
$\alg{A}$ is an \defn{abelian} algebra, which means 
that $[1_A, 1_A] = 0_A$, then, by
the monotonicity of the commutator,
$[\theta, \theta] = 0_A$ for all $\theta \in \Con \alg{A}$, 
in which case $\textbf{A}$
(\ref{eq:3}) says that $d^{\alg{A}}$ is a Mal'tsev term operation.)

Digital computers have turned out to be invaluable tools for exploring and
understanding algebras and the varieties they inhabit, and this is largely due
to the fact that, over the last three decades,
researchers have found ingenious ways to get computers to
solve challenging abstract decision problems---such as whether a variety is
congruence $n$-permutable (\cite{MR3239624}), or
congruence modular (\cite{Freese:2009})---and to do so very quickly.
This paper contributes to this effort by finding an efficient
algorithm for deciding whether a locally finite idempotent variety has a
difference term.

The central question motivating this project is the following:
\begin{prob}
  \label{prob:1}
  Is there a polynomial-time algorithm to decide for a finite,
  idempotent algebra $\alg{A}$ if $\bbV(\alg{A})$ has a difference term.
\end{prob}
Kearnes proved in~\cite{MR1358491} that $\alg{A}$ has a difference term iff
$\bbV(\alg{A})$ has a Taylor term and no type-2 tails
(equivalently, $\bbV(\alg{A})$ has no 1's and no type-2 tails).
No 1's is poly-time decidable by Valeriote's subtype theorem.
In~\cite{Freese:2009}, Freese and Valeriote solved an analogous problem, by
giving a positive answer to the following
\begin{prob}
  \label{prob:2}
  Is there a polynomial-time algorithm to decide for a finite,
  idempotent algebra $\alg{A}$ if $\bbV(\alg{A})$ is \ac{cm}?
\end{prob}

Congruence modularity is characterized by no 1's, no 5's and no tails.
Again no 1's and no 5's can be decided by the subtype theorem,
and in~\cite{Freese:2009} the authors prove that if there is
a tail in $\bbV(\alg{A})$, there is a tail ``near the bottom.''
More precisely, if $\alg{A}$ is finite and idempotent, and $\bbV(\alg{A})$ has no
1's and no 5's and has tails, then there is a tail in a 3-generated subalgebra of $\alg{A}^2$.
Using this it is proved that deciding \cm is polynomial-time.

But the proof of the no tails part uses that in a variety with no 1's or 5's,
the congruence lattice modulo the {\it solvability congruence} (defined below)
is (join) semidistributive.
Now, restricting to just testing no type-2 tails (vs no tails of any type) is
not a problem. So, for example, there is a poly-time algorithm for testing if
$\bbV(\alg{A})$ has no 1's, no 5's and no type-2 tails.  

Here is a related problem.
\begin{prob}
  \label{prob:3}
  Is there an $\alg{A}$, idempotent and having a Taylor term, no type-2 tail in 
  subalgebras of $\alg{A}^k$, for $k < n$, but having a type-2 tail in a subalgebra
  of $\alg{A}^n$. 
\end{prob}
Perhaps we could construct such an algebra using congruence lattice
representation techniques. 

%% Hobby and McKenzie give some info about the types in a $D_2$
%% embedded in $\Con (\alg{A})$. (See~\cite[Lemma 6.3]{HM:1988}).
%% Exercise 7 of that section considers 4-element
%% algebras whose congruence lattice is the concrete embedding of $D_2$
%% in $\Eq(4)$; the one with coatoms $01|23$, $02|13$, and $0|123$, and with atoms
%% $0|1|23$ and $0|2|13$. By~\cite[Lemma 6.3]{HM:1988}, the middle-top
%% interval must be type 5 (assuming a Taylor term). But all the others can be 5's,
%% or all the others can be 4's, or all the others can be 3's.
%% One might attempt to find an example where they are all 2's, but that not possible
%% since otherwise $0|123$ would be a solvable congruence,
%% which would imply the two atoms would permute.

%% \draftbreak

\section{Background, definitions, and notation}
\label{sec:defin-notat}
Our starting point is the set of lemmas at the beginning of Section 3 in
the Freese-Valeriote paper~\cite{Freese:2009}.
We first review some of the basic \ac{tct}
that comes up in the proofs in that paper. (In fact, most of this section 
is copied from the nice presentation of \tct background that appears
in~\cite[Sec.~2]{Freese:2009}.)

The reference for \tct is the book by Hobby and McKenzie
\cite{HM:1988}, according to which,
for each covering $\alpha \prec \beta$ in the congruence lattice of a finite
algebra $\alg{A}$, the local behavior of the $\beta$-classes is captured by the
so-called $(\alpha, \beta)$-traces~\cite[Def.~2.15]{HM:1988}.
Modulo $\alpha$, the induced structure on the traces is limited to one
of five possible types:

\begin{enumerate}[(1)]
\item unary algebra whose basic operations are all permutations (unary type);
\item one-dimensional vector space over some finite field (affine type);
\item 2-element boolean algebra (boolean type);
\item 2-element lattice (lattice type);
\item 2-element semilattice (semilattice type).
\end{enumerate}

Thus to each covering $\alpha \prec \beta$
corresponds a ``\tct type'' in $\{1,2,3,4,5\}$ (see~\cite[Def.~5.1]{HM:1988}),
denoted by $\typ(\alpha, \beta)$, called the \emph{typeset} of $\alg{A}$.
The set of all \tct types that are realized by covering pairs of congruences of a
finite algebra $\alg{A}$ is denoted by $\typ\{\alg{A}\}$, and if $\sK$ is a class of algebras,
then $\typ\{\sK\}$ denotes the union of the typesets of all finite algebras in $\sK$.
\tct types are ordered according to the following ``lattice of types:''

\newcommand{\dotsize}{0.8pt}
%% To create nodes of lattices in a uniform and consistent way, we define
\tikzstyle{lat} = [circle,draw,inner sep=\dotsize]
% To scale all diagrams uniformly, change this setting:
\begin{center}
\newcommand{\figscale}{.7}
\begin{tikzpicture}[scale=\figscale]
  \scalefont{.8}
  \node[lat] (1) at (0,0) {};
  \node[lat] (2) at (-1,1.5) {};
  \node[lat] (3) at (0,3) {};
  \node[lat] (4) at (.8,2.1) {};
  \node[lat] (5) at (.8,.9) {};
  \draw (1) node [below] {$1$};
  \draw (2) node [left] {$2$};
  \draw (3) node [above] {$3$};
  \draw (4) node [right] {$4$};
  \draw (5) node [right] {$5$};
  \draw[semithick] 
  (1) -- (2) -- (3) -- (4) -- (5) -- (1);
\end{tikzpicture}
\end{center}
Whether or not $\bbV(\alg{A})$ omits one of the order ideals of the lattice of types can be
determined locally.  This is spelled out for us in the next proposition.
(A \defn{strictly simple} algebra is a simple
algebra with no non-trivial subalgebras.)
%% ; i.e.~no proper subalgebras with
%% more than one element.)


\begin{prop}[Prop.~2.1~\cite{Freese:2009}]
  \label{prop:2.1}
If A is a finite idempotent algebra and $i \in \typ(\bbV(\alg{A}))$ then there
is a finite strictly simple algebra $\bS$ of type $j$ for some $j \leq i$ in $\sansH \sansS (\alg{A})$.
If
\begin{enumerate}[(1)]
\item 
  $j = 1$ then $\bS$ is term equivalent to a 2-element set;
\item
  $j = 2$ then $\bS$ is term equivalent to the idempotent reduct of a module;
\item
  $j = 3$ then $\bS$ is functionally complete;
\item
  $j = 4$ then $\bS$ is polynomially equivalent to a 2-element lattice;
\item
  $j = 5$ then $\bS$ is term equivalent to a 2-element semilattice.
\end{enumerate}
\end{prop}
\begin{proof}
  This is a combination of~\cite[Prop.~3.1]{MR2504025} and~\cite[Thm.~6.1]{MR1191235}.
\end{proof}

Table~\ref{tab:1} is from~\cite{MR3350327} and gives another characterization of
omitting types.
\begin{center}
  \begin{table}
    \caption{\cite{MR3350327}.}
    \label{tab:1}
    \begin{tabular}{|l|l|}
      \hline
      Omitting Class &  Equivalent Property\\
      \hline
      $\sM_{\{1\}}$ & satisfies a nontrivial idempotent \malcev condition \\
      \hline
      $\sM_{\{1,5\}}$ & satisfies a nontrivial congruence identity\\ % (see~\cite{MR3076179})\\
      \hline
      $\sM_{\{1,4,5\}}$ & congruence n-permutable, for some $n > 1$ \\
      \hline
      $\sM_{\{1,2\}}$ & congruence meet semidistributive \\
      \hline
      $\sM_{\{1,2,5\}}$ & congruence join semidistributive\\ % (see~\cite{MR3076179})\\
      \hline
      $\sM_{\{1,2,4,5\}}$ & congruence $n$-permutable for some $n$ and\\
      &congruence join semidistributive\\
      \hline
    \end{tabular}
  \end{table}
\end{center}

In Section~\ref{sec:freese-valer-lemm}, the following result will be useful.
\begin{corollary}

[{\protect cf.~\cite[Lem.~3.3]{Freese:2009}}]
  [Cor.~2.2~\cite{Freese:2009}]
  \label{cor:2.2}
  Let $\alg{A}$ be a finite idempotent algebra and $T$ an order ideal in the
  lattice of types. Then $\bbV(\alg{A})$ omits $T$ if and only if $\sansS(\alg{A})$ does.
  %% In particular, $\bbV(\alg{A})$ omits 1 and 2 if and only if $\sansS(\alg{A})$ omits 1 and 2.
\end{corollary}



%% \draftsecskip
%%%%%%%%%%%%%%%%%%%%%%%%%%%%%%%%%%%%%%%%%%%%%%%%%%%%%%%%%%%%%%%%%%%%%%%%%
%% \draftbreak

\section{The Freese-Valeriote Lemmas Revisited}
\label{sec:freese-valer-lemm}
In~\cite{Freese:2009}, Corollary~\ref{cor:2.2} is the starting point of the
development of a polynomial-time algorithm that determines if a given finite
idempotent algebra generates a \cm variety. 

%% The following lemma ties in with the previous proposition and will be used
%% in Sec. 6.
%% \begin{lemma}[Lemma 2.3~\cite{Freese:2009}] 
%%   Let $\alg{A}$ be a finite idempotent algebra and let $\bS \in \sansH \sansS(\alg{A})$
%%   be strictly simple. Then there are elements $a, b \in A$ such that, if
%%   $\alg{B} = \Sg^{\alg{A}} (a, b)$, then $1_B = \Cg^{\alg{B}} (a, b)$ and is join irreducible
%%   with unique lower cover $\rho$ such that $\bS = \alg{B}/\rho$.
%% \end{lemma}
%% \begin{proof}
%%   Choose $\alg{B} \in \sansS (\alg{A})$ as small as possible having $\bS$ as a homomorphic image,
%%   say $\bS = \alg{B}/\rho$. We claim that if $a, b \in B$ with
%%   $(a, b) \in \notin \rho$ then they generate $\alg{B}$. To
%%   see this, let $\alg{B}'= \Sg^{\alg{B}} (a, b)$ and let $h$ be the quotient map from B to S with kernel
%%   ρ. Then h(B  ) is a non-trivial subuniverse of S and so must equal S. Thus B  = B.
%%   Now let a, b ∈ B with (a, b) ∈
%%   / ρ. Since the block of Cg B (a, b) containing a
%%   and b is a subuniverse of B then from the previous paragraph, we conclude that
%%   Cg B (a, b) = 1 B and that ρ is its unique lower cover.
%% \end{proof}

According to the characterization
in~\cite[Ch.~8]{HM:1988} of locally finite congruence modular (resp.,
distributive) varieties, a finite algebra $\alg{A}$ generates a congruence modular
(resp., distributive) variety $\sV$ if and only if the typeset of $\sV$ is
contained in $\{2, 3, 4\}$ (resp., $\{3, 4\}$) and all minimal sets of prime
quotients of finite algebras in $\sV$ have empty
tails~\cite[Def.~2.15]{HM:1988}. Note that in the distributive 
case the empty tails condition is equivalent to the minimal sets all having exactly
two elements.

It follows from Corollary~\ref{cor:2.2} and Proposition~\ref{prop:2.1}
that if $\alg{A}$ is idempotent then one can
test the first condition, on omitting types 1 and 5 (or 1, 2, and 5) by searching
for a 2-generated subalgebra of $\alg{A}$ whose typeset is not contained in
$\{2, 3, 4\}$ ($\{3, 4\}$). It is proved in~\cite[Sec.~6]{Freese:2009} that this
test can be performed in polynomial time---that is, the running time of the test
is bounded by a polynomial function of the size of $\alg{A}$.
In~\cite[Sec.~3]{Freese:2009}, Freese and Valeriote prove a sequence of
lemmas to establish that, if $\alg{A}$ is finite and idempotent, and if
$\mathcal V = \bbV(\alg{A})$ omits types 1 and 5, then to test for the existence of tails
in $\sV$ it suffices to look for them 
in the 3-generated subalgebras of $\alg{A}^2$.
%% More specifically, the authors assume that the type set of $\bbV(\alg{A})$ contains no 1's
%% and no 5's, and under this 
%% assumption they prove that non-empty tails either do not occur in $\bbV(\alg{A})$,
%% or they occur in 3-generated subalgebras of $\alg{A}^2$.
In other words, either there are no non-empty tails
or else there are non-empty tails that are easy to find
(since they occur in 3-generated subalgebras of $\alg{A}^2$).
It follows that Problem~\ref{prob:2} has a positive answer:
deciding whether or not a finite idempotent algebra generates a congruence
modular variety is tractable.\footnote{That is, there are positive integers
  $C, n$, and an algorithm that takes
  a finite idempotent algebra $\alg{A}$ as input and decides
  in at most $C|\alg{A}|^n$ steps whether $\bbV(\alg{A})$ is congruence modular.
  Here $|\alg{A}|$ denotes the number of bits required to encode the algebra $\alg{A}$.}
%% polynomial-time algorithm to decide, for a finite idempotent algebra $\alg{A}$,
%% whether $\bbV(\alg{A})$ is congruence modular.}

Our goal is to use the same strategy to solve Problem~\ref{prob:1}.
As such, we revisit each lemma in Section 3 of \cite{Freese:2009},
and consider whether it can be proved under modified hypotheses.
Specifically, we continue to assume that the type set of $\bbV(\alg{A})$ contains no 1's,
but we will now drop the ``no 5's'' assumption.  We will attempt to prove that
either there are no \emph{type-2} tails in $\bbV(\alg{A})$, or else \emph{type-2}
tails can be found ``quickly,'' (e.g., in a 3-generated subalgebra of $\alg{A}^2$).
We continue to quote~\cite{Freese:2009} where possible,
while modifying the assumptions and adjusting the arguments as necessary.


Throughout, we let $\nn$ denote the set $\{0,1,\dots, n-1\}$ and 
(at least for the rest of this section) we let $\sS$ be a finite 
set of finite,
similar, idempotent algebras, closed under the taking of 
subalgebras, such that
$\sV = \bbV(\sS)$ omits 1 (but may include type 5).
If there is a finite algebra in $\sV$ having a type 2 minimal set 
having a tail, then, by standard tame congruence theory, there
is such an algebra which is a subalgebra of a product 
of elements
in $\sS$ have a type~2 minimal set with a tail. So we
suppose that some finite algebra
$\alg{B}$ in $\sV$ has a prime quotient whose minimal sets 
have non-empty
\emph{type-2} tails and show that there is a 3-generated 
subalgebra of the
product of two members of $\sS$ with this property.

Since $\sS$ is closed under the taking of subalgebras,
we may assume that the algebra $\alg{B}$ from the previous paragraph is a subdirect
product of a finite number of members of $\sS$. Choose $n$ minimal such that for
some $\alg{A}_0$, $\alg{A}_1$, $\dots$, $\alg{A}_{n-1}$ in $\sS$, there is a subdirect
product $\alg{B} \sdp \prod_{\nn} \alg{A}_i$
that has a prime quotient with non-empty type-2 tails.
Under the assumption that $n > 1$ we will attempt to prove that $n = 2$.

For this $n$, select the $\alg{A}_i$ and $\alg{B}$ so that $|B|$ is as small as possible.
Let $\alpha \prec \beta$ be a prime quotient of $\alg{B}$ with non-empty
type-2 tails and choose $\beta$ minimal with this property.
Let $U$ be an $(\alpha, \beta)$-minimal set and let $N$ be an
 $(\alpha, \beta)$-trace of $U$. Let 0 and 1 be
two distinct members of $N$ with $(0, 1) \notin \alpha$.

\begin{lemma}[{\protect cf.~\cite[Lem.~3.1]{Freese:2009}}]
  \label{lem:fv_3-1}
  Let $t$ be a member of the tail of $U$. Then $\beta$ is the congruence of $\alg{B}$
  generated by the pair $(0, 1)$ and $\alg{B}$ is generated by $\{0, 1, t\}$.
\end{lemma}
It seems the proof of~\cite[Lem.~3.1]{Freese:2009} goes through with only minor
adjustments. % to prove Lemma~\ref{lem:fv_3-1}.
\begin{proof}
  {\bf TODO:} fill in proof of  Lemma~\ref{lem:fv_3-1}.
\end{proof}


For $i \leq n$, let $\rho_i$ denote the kernel of the projection of $\alg{B}$ onto $\alg{A}_i$,
so $\alg{B} \cong \alg{A}_i/\rho_i$.
For a subset $\sigma \subseteq \nn$, define
\[
\rho_\sigma := \bigwedge_{j\in \sigma} \rho_j.
\]
Consequently,
$\rho_{\nn} = \bigwedge_{j\in \nn}\rho_j = 0_{B}$
%% \marginnote{wjd: I don't see why join in (3.1) is $1_B$... it's probably wrong.}[3cm]
%% \begin{equation}
%%   \label{eq:2}
%%   \rho_{\nn} = \bigwedge_{j\in \nn}\rho_j = 0_{B} \quad \text{ and } \quad
%%   \bigvee_{j\in \nn}\rho_j =1_B. %% \qquad
%% \end{equation}
By minimality of $n$ we know that the intersection of any  proper subset of the
$\rho_i$, $1 \leq i \leq n$ is strictly above $0_B$.  Thus,
$0_B < \rho_\sigma < 1_B$ for all 
$\emptyset \subset \sigma\subset \nn$
(by $\subset$ we mean \emph{proper} subset).

\begin{lemma}[{\protect cf.~\cite[Lem.~3.2]{Freese:2009}}]
  \label{lem:fv_3-2}
  If $\sigma \subset \nn$, then
  either $\beta \leq \rho_\sigma$ or $\alpha \join \rho_\sigma = 1_B$.
\end{lemma}
\begin{proof}
  {\bf TODO:} fill in proof of  Lemma~\ref{lem:fv_3-2}.
\end{proof}

\begin{lemma}[{\protect cf.~\cite[Lem.~3.3]{Freese:2009}}]
  \label{lem:fv_3-3}
  \begin{enumerate}[(i)]
    \item \label{item:6} There exists $0\leq i< n $
      such that $\alpha \join \rho_i = 1_B$
    \item \label{item:7} There exists $i$ such that  
      $\alpha \join \rho_i < 1_B$.
  \end{enumerate}
\end{lemma}
\begin{proof}
  %% {\bf TODO:} fill in proof of  Lemma~\ref{lem:fv_3-3}.
  %%\begin{enumerate}[(i)]
  %%\item %\label{item:6} There exists $0\leq i< n $  such that $\alpha \join \rho_i = 1_B$
If item \eqref{item:6} failed, then we would 
have $\beta \leq \rho_i$ for all $i$, and that
would imply $\beta = 0_B$.
  %%\item %\label{item:7} There exists $i$ such that $\alpha \join \rho_i < 1_B$.
  
To see \eqref{item:7}, assume
\begin{equation} \label{eq:4}
\alpha \join \rho_i = 1_B \; \text{ for all $i$.}
\end{equation}
Take a nonempty proper subset $\sigma \subset \nn$ of indices and let 
$\rho_\sigma = \bigwedge_{j\in \sigma} \rho_j$.
Then $\alpha \join \rho_\sigma = 1_B$ since otherwise,
by Lemma~\ref{lem:fv_3-2}, we have
$\alpha \leq \beta \leq \rho_\sigma \leq \rho_i$ for $i \in \sigma$, so
$\alpha \join \rho_i = \rho_i < 1_B$, contradicting (\ref{eq:4}).
Therefore, $\beta \join \rho_\sigma = 1_B$.

\smallskip

\noindent {\bf Subclaim:}
For all nonempty proper subsets $\sigma \subset \nn$,
for all $c\in \body(U)$ and all $v\in B$, we have
$(c, v) \in \beta \circ \rho_\sigma \cap \rho_\sigma \circ \beta$.

\smallskip
    
\noindent To prove the subclaim, we first note that 
$\beta \join \rho_\sigma = 1_B$ implies
$\restr{\beta}{U} \join \restr{\rho_\sigma}{U} = 1_U$ since $U = e(B)$.
Now, for all $x, y \in U$, if $x\in \body(U)$ and $y\in \tail(U)$, then
$(x,y) \notin \beta$.  Therefore,
$(x, y) \in  1_U = \restr{\beta}{U} \join \restr{\rho_\sigma}{U}$ implies
there must be some $a \in \body(U)$ and $t\in \tail(U)$ such that 
$a \mathrel{\rho_\sigma} t$.  

%%%%%%

%Let $b\in \body(U)$ be such that
%$(a,b) \in \restr{\beta}{U} - \restr{\alpha}{U}$.  

%By Lemma~\ref{lem:fv_3-1},
%$\alg{B} = \Sg^{\alg{B}}(a, b, t)$.
%Let $v \in B$. Then $v = s^{\alg{B}}(a,b,t)$ 
%for some (idempotent) term $s$, so
%\[
%v = s^{\alg{B}}(a,b,t)
%\mathrel{\rho_\sigma} s^{\alg{B}}(a,b,a)
%\mathrel{\beta} s^{\alg{B}}(a,a,a) = a,
%\]
%and
%\[
%v = s^{\alg{B}}(a,b,t)
%\mathrel{\beta}  s^{\alg{B}}(a,a,t)
%\mathrel{\rho_\sigma}s^{\alg{B}}(a,a,a)  = a.
%\]
%Therefore, 
%$(v,a) \in  \beta \circ \rho_\sigma \cap \rho_\sigma \circ \beta$.

%%%%%%%%%%%%%%%%%

Now, let $d(x,y,z)$ be a pseudo-\malcev polynomial for $U$,
which exists by~\cite[Lem.~4.20]{HM:1988}.
Thus,
\begin{itemize}
\item $d(B,B,B) = U$
\item $d(x,x,x) = x$ for all $x\in U$
\item $d(x,x,y) = y = d(y,x,x)$ for all $x\in \body(U)$, $y \in U$.
\end{itemize}
Moreover, for all $c, d \in \body(U)$, the unary polynomials
$d(x,c,d)$, $d(c,x,d)$, and $d(c,d,x)$ are permutations on $U$.

Now let $c$ be an arbitrary element of $\body(U)$, and
let $p(x) = d(x,a,c)$. 
Then (see~\cite[Lem.~4.20]{HM:1988})
we have $p(U) = U$, $p(a) = d(a,a,c) = c \in \body(U)$ 
and $p(t)\in \tail(U)$,
since $t\in \tail(U)$. Let $t' = p(t)$. 
Since $(a,t) \in \rho_\sigma$, we have
$(c, t') = (p(a), p(t)) \in \rho_\sigma$. 

Since $c$ is in the body, there is an element $d$ in the body with
$(c,d) \in \beta - \alpha$.
By Lemma~\ref{lem:fv_3-1},
$\alg{B} = \Sg^{\alg{B}}(c, d, t')$.
Let $v \in B$ be arbitrary. Then $v = s^{\alg{B}}(c,d,t')$ for 
some (idempotent) term $s$, so
\[
v = s^{\alg{B}}(c,d,t')
\mathrel{\rho_\sigma} s^{\alg{B}}(c,d,c)
\mathrel{\beta} s^{\alg{B}}(c,c,c) = c,
\]
and
\[
v = s^{\alg{B}}(c,d,t')
\mathrel{\beta}  s^{\alg{B}}(c,c,t')
\mathrel{\rho_\sigma}s^{\alg{B}}(c,c,c)  = c.
\]
Therefore, 
$(v,c) \in  \beta \circ \rho_\sigma \cap \rho_\sigma \circ \beta$,
proving the subclaim.


%We conclude from the foregoing that,
%for all nonempty proper subsets $\sigma \subset n$,
%for all $a\in \body(U)$ and all $v\in B$, we have
%$(a, v) \in \beta \circ \rho_\sigma \cap \rho_\sigma \circ \beta$, 
%proving the subclaim.

Let $b \in \body(U)$ and $t \in \tail(U)$. By
\cite[Lemma~4.25]{HM:1988}, $(b, d(b,t,t)) \notin \beta$.
We will arrive at a contradiction by showing that 
$b = d(b,t,t)$. By the claim $(b,t) \in \beta \circ \rho$ so
there is an element $a \in B$ with 
$b\mathrel\beta a \mathrel\rho_i t$. By applying the idempotent
polynomial $e$ with $e(U) = U$ to this, we may assume $a \in U$
and, since $a\mathrel \beta b$, $a \in \body(U)$. So
\[
d(b,t,t) \mathrel\rho_i d(b,a,a) = b.
\]
Since this hold for every~$i$, $d(b,t,t) = b$.
\end{proof}


%\begin{lemma}[{\protect cf.~\cite[Lem.~3.3]{Freese:2009}}]
%\label{lem:fv_3-3x}
%There is exactly one~$i$ such that $\beta \not\le \rho_i$.
%\end{lemma}
%\begin{proof}
%If $n = 1$ then $\rho_1 = 0$ so the results holds. The lemma
%above shows there must there must be at least one $\rho_i \ge \beta$
%so let $\rho_1 \not\ge \beta$ and $\rho_2 \ge \beta$. If $n = 2$, 
%the result holds. Otherwise let $\rho = \rho_1 \meet \rho_2$ and
%note it is clearly not above $\beta$. Since $\beta$ is 
%join irreducible and $\alpha$ is its unique lower cover,
%$\rho \meet \beta \le \alpha$. But then $\alg B/\rho$ has a type~2
%tail, contradicting the minimality of~$n$. 
%\end{proof}


\begin{theorem}[Thm.~3.4~\cite{Freese:2009}]
  \label{thm:fv_3-4}
  Let $\sV$ be the variety generated by some finite set $\sS$ of finite,
  idempotent algebras that is closed under taking subalgebras. If $\sV$
  omits type 1 \sout{and 5} and some finite member of $\sV$ has a prime quotient
  whose minimal sets have non-empty \underline{type-2} tails then there is some
  3-generated algebra B with this property that belongs to $\sS$ or is a subdirect
  product of two algebras from $\sS$. 
\end{theorem}
\begin{proof}
  {\bf TODO:} fill in proof of Theorem~\ref{thm:fv_3-4}.
\end{proof}

\newpage   %% delete this

%% This next part didn't work: the proof required minimality 
%% to get that A was generated by any pair in \beta - \alpha
%% and any t in the tail. It also used idempotence.  


\section{Excluded Lattices}

\begin{theorem}
\label{thm:conjecture}
Let $\alg A$ be a finite algebra with a Taylor term. Then 
$\Con(\alg{A})$ does not have congruences 
$\alpha$, $\beta$, $\rho_0, \ldots, \rho_{n-1}$
 satisfying
\begin{enumerate}
\item
$\rho_i \ne 1_{\alg A}$, for all $i$,
\item
$\Meet \rho_i = 0_{\alg A}$,
\item
$\beta$ is join irreducible,
\item 
$\alpha \cov \beta$ is a prime quotient of type 2, and
\item
$\alpha \join \rho_i = 1_{\alg A}$, for all $i$.
\end{enumerate}
\end{theorem}

\begin{proof}
First note that it follows from (2), (3), and (4) that
\[
\{\rho_i\meet\beta, \alpha, \beta, \rho_i, 1_{\alg A}\}
\]
forms a pentagon, for each~$i$. 
Let $U$ be an $(\alpha, \beta)$-minimal set and 
let $e$ be an idempotent polynomial with $e(A) = U$. 
Then $e$ induces a lattice homomorphism of $\Con \alg A$ onto
$\Con \alg A |_{U}$ separating $\alpha$ from $\beta$, thus
leaving the pentagon intact. It 
follows that $U$ must have a tail, for otherwise, it would
be a type 2 minimal set with only body and the pseudo 
Maltsev operation would mean $\Con \alg A |_{U}$ is
modular, a contradiction. 

\todo{finish proof of Theorem~\ref{thm:conjecture}}

\end{proof}


%\bibliographystyle{amsplain} %% or amsalpha
%% \bibliographystyle{alpha-url}
%% \printbibliography
\bibliographystyle{alphaurl}
\bibliography{inputs/refs}






\end{document}
















With notation and assumptions as above.

\begin{theorem}[cf. Theorem 3.3 of \cite{FreeseValeriote2009}]
This is a test.
\end{theorem}






\bibliographystyle{rsfplain}
%\bibliography{\jobname}
\bibliography{/Users/ralph/tex/bib/Database/db}



A \emph{minority term} (for an algebra or variety) is a 
3-variable term $q(x,y,z)$ such if two of the variables
are equal, its value is the other one; that is, 
\[
q(x,x,y) \approx q(x,y,x) \approx q(y,x,x) \approx y.
\]
We are interested in an algebraic description of varieties having
a minority term. One possible conjecture is:
\begin{conjecture}
$\mathcal V$ has a minority term if and only if it is CP and its
ring has characteristic~1 or~2.
\end{conjecture}



\begin{fact}
If $\mathcal V$ has a minority term, then it is CP.
\end{fact}

\begin{proof}
Clearly a minority term is a Maltsev term.
\end{proof}

\begin{fact}
A CD variety $\mathcal V$ has a minority term if and
only if it is CP.
\end{fact}

\begin{proof}
A variety is CD and CP if and only if it has a Pixley term. 
If $p(x,y,z)$ is a Pixley term then 
\[
q(x,y,z) = p(p(x,y,z),x,p(x,z,y))
\]
is a minority term. The fact can be derived from these
observations.
\end{proof}

\begin{fact}
A variety of groups has a minority term if and only
the variety has exponent 2 (well, or 1).
\end{fact}

\begin{proof}
If a variety has exponent 2 then it is abelian and so,
using additive notation, $x + y + z$ is a minority term.

If the exponent is not 2 then the variety contains a group
which contains an element of order $n$, where $n > 2$ (or 
infinite). This element generates a cyclic group. In any
abelian algebra the Maltsev term operation is unique and so
must be $x - y + z$. But one easily checks that this is not
a minority term when $n > 2$.
\end{proof}

\begin{lemma}
Let $\mathcal V$ be a CP variety with a Maltsev term $p(x,y.z)$.
The the following are equivalent:
\begin{enumerate}
\item
The ring $R(\mathcal V)$ of $\mathcal V$ has characteristic~2.
\item
On every block of every abelian congruence, $p$ restricted to the 
block is a minority term; that is, satisfies $p(a,b,a) = b$.
\item
If $\theta = \textup{Cg}^{\alg F_{\mathcal V}(x,y)}(x,y)$, then,
on each block of $\theta/[\theta,\theta]$, $p$ is a minority term.
\end{enumerate}
If $\mathcal V$ has a minority term, then these conditions hold.
\end{lemma}

\begin{proof}
By commutator theory (give some specific refs, also that the ring
is det by the strucure of $\theta/[\theta,\theta]$) 
the Maltsev term operation on an
abelian algebra, or even a block of an abelian congruence, 
is unique and it is $p(x,y,z) = x - y + z$ for
some abelian group. It is easy that $x - y + z$ is a minority
if and only if the abelian group has exponent~2.
\end{proof}


\end{document}


Let $\alg A$ be an algebra and let $S$ and $T$ be tolerances
on $\alg A$. 
Let $M(S,T)$, or $M^{\alg A}(S,T)$ to emphasize $\alg A$,
be the set of all $2 \times 2$ matrices of the form
\begin{equation}\label{eq1}
\begin{bmatrix}
p&q\\
r&s
\end{bmatrix}
=
\begin{bmatrix}
f(\mathbf{a},\mathbf{u})&f(\mathbf{a},\mathbf{v})\\
f(\mathbf{b},\mathbf{u})&f(\mathbf{b},\mathbf{v})
\end{bmatrix}
\end{equation}
where $f(\mathbf{x},\mathbf{y})$ is an $(m+n)$-ary polynomial of
$\alg A$, $\mathbf{a} \mathrel{S} \mathbf{b}$, and 
$\mathbf{u} \mathrel{T} \mathbf{v}$
(componentwise, of course). The members of $M(S,T)$ are called
\emph{$S,T$-matrices}.

The first exercise gives an efficient way to find $M(S,T)$.

\section*{Exercises}

\begin{exercises}

\prob
Show that $M(S,T)$ is the subalgebra of $\alg A^4$ generated by
\[
\left\{
\begin{bmatrix}
a&a\\
b&b
\end{bmatrix} : a \mathrel{S} b\right\}
\union
\left\{
\begin{bmatrix}
c&d\\
c&d
\end{bmatrix} : c \mathrel{T} d\right\}
\]

\prob
Use the symmetry of $S$ and $T$ to show the matrix obtained from an
$S,T$-matrix by interchanging the rows or columns (or both) is also
in $M(S,T)$. 

\prob
$M(T,T)$ is closed under taking transposes. 

\end{exercises}

\section*{Centrality Relations}

We define four kinds of centrality, called centrality, strong
centrality, weak centrality, and strong rectularity. The is a fifth
centrality condition known as rectangularity which we will save for
later.

Let $\delta$ be a congruence and $S$ and $T$ be
tolerance relations on  $\alg A$. The above centrality relations
are denoted $\alg C(S,T;\delta)$ (centrality), 
$\alg S(S,T;\delta)$ (strong centrality), 
$\alg W(S,T;\delta)$ (weak centrality),  and
$\alg SR(S,T;\delta)$ (strong rectangularity). They hold if the
appropriate implication below holds for all 
\[
\begin{bmatrix}
p&q\\
r&s
\end{bmatrix} \in M(S,T)
\]
\begin{itemize}
\item centrality: 
$p \mathrel{\delta} q \implies r \mathrel{\delta} s$.
\item strong rectangularity: 
$p \mathrel{\delta} s \implies r \mathrel{\delta} s$.
\item weak centrality: 
$p \mathrel{\delta} q \mathrel{\delta} s \implies r \mathrel{\delta} s$.
\item strong centrality holds if both centrality and strong
rectangularity hold.
\end{itemize}

Using the exercises it is easy to see that the implication defining 
$\alg C(S,T;\delta)$ can be replaced by 
$r \mathrel{\delta} s \implies p \mathrel{\delta} q$ and this is
equivalent to 
\[
p \mathrel{\delta} q \Longleftrightarrow r \mathrel{\delta} s.
\]
Similar statements hold for the other conditions: weak centrality
is equivalent to saying that if any three of $p$, $q$, $r$ and $s$
are $\delta$ related, then they all are. And strong rectangularity
says that if the elements of the main diagonal, or of the sinister
diagonal, are $\delta$ related, then all four are.

The \emph{$S,T$-term condition} is the condition $\alg C(S,T,0)$,
usually expressed using the right-hand matrix in~\eqref{eq1}. 
Other kinds of term conditions are defined similarly. 

If $\alg C(S,T;\delta_i)$ holds for all $i \in I$, then
$\alg C(S,T;\Meet_{i\in I}\delta_i)$ holds. Similar statements hold
for the other centrality conditions. So there is a least $\delta$
such that $\alg C(S,T;\delta)$ holds. This $\delta$ is the 
\emph{commutator} of $S$ and $T$, and is denoted $[S,T]$. The
commutators for the other centrality relations are denoted
$[S,T]_{\alg S}$, $[S,T]_{\alg {SR}}$, and $[S,T]_{\alg W}$.

The properties of these centrality relations are coverered in
Theorem~2.19 and Theorem~3.4 of~\cite{KearnesKiss2013}. Much stronger
properties hold in congruence modular varieties;
see~\cite{FreeseMcKenzie1987}.

\section*{Exercises}

\begin{exercises}

\prob
As defined in \cite{HobbyMcKenzie1988}, $\beta$ is \emph{strongly
Abelian} over $\delta$ ($\delta \leq \beta$, both congruences on $\alg A$)
if the following implication holds for all polynomials $f$ and all
elements $x_0, \ldots, x_{n-1}$, $y_0, \ldots, y_{n-1}$, and 
$z_1, \ldots, z_{n-1}$ with $x_0 \mathrel\beta y_0$ and
$x_i \mathrel\beta y_i \mathrel\beta z_i$, $i = 1, \ldots n-1$.
\begin{align*}
f(x_0,\ldots,&x_{n-1}) \mathrel\delta f(y_0,\ldots,y_{n-1}) \\
&\implies
f(x_0, z_1,\ldots,z_{n-1}) \mathrel\delta f(y_0, z_1,\ldots,z_{n-1})
\end{align*}
Show that $\beta$ is strongly 
Abelian over $\delta$ if and only if $\alg S(\beta,\beta;\delta)$ 
holds, and also show this is in turn equivalent to
$\alg {SR}(\beta,\beta;\delta)$.
\end{exercises}



\bibliographystyle{rsfplain}
%\bibliography{\jobname}
\bibliography{/home/ralph/tex/bib/Database/db}


\end{document}
