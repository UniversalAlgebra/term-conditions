%% FILE: diffTerm.tex
%% AUTHOR: William DeMeo, Ralph Freese, Matthew Valeriote
%% DATE: 13 April 2017
%% COPYRIGHT: (C) 2017 DeMeo, Freese, Valeriote

%%%%%%%%%%%%%%%%%%%%%%%%%%%%%%%%%%%%%%%%%%%%%%%%%%%%%%%%%%
%%                         BIBLIOGRAPHY FILE            %%
%%%%%%%%%%%%%%%%%%%%%%%%%%%%%%%%%%%%%%%%%%%%%%%%%%%%%%%%%%
%% The `filecontents` command will crete a file in the inputs directory called 
%% refs.bib containing the references in the document, in case this file does 
%% not exist already.
%% If you want to add a BibTeX entry, please don't add it directly to the
%% refs.bib file.  Instead, add it in this file between the
%% \begin{filecontents*}{refs.bib} and \end{filecontents*} lines
%% then delete the existing refs.bib file so it will be automatically generated 
%% again with your new entry the next time you run pdfaltex.
\begin{filecontents*}{inputs/refs.bib}
@article {MR3350334,
    AUTHOR = {Horowitz, Jonah},
     TITLE = {Testing for edge terms is decidable},
   JOURNAL = {Algebra Universalis},
  FJOURNAL = {Algebra Universalis},
    VOLUME = {73},
      YEAR = {2015},
    NUMBER = {3-4},
     PAGES = {321--334},
      ISSN = {0002-5240},
   MRCLASS = {03C05 (03B25 08A40 08B05)},
  MRNUMBER = {3350334},
MRREVIEWER = {David Casperson},
       DOI = {10.1007/s00012-015-0325-4},
       URL = {http://dx.doi.org/10.1007/s00012-015-0325-4},
}
@article {MR3109457,
    AUTHOR = {Horowitz, Jonah},
     TITLE = {Computational complexity of various {M}al'cev conditions},
   JOURNAL = {Internat. J. Algebra Comput.},
  FJOURNAL = {International Journal of Algebra and Computation},
    VOLUME = {23},
      YEAR = {2013},
    NUMBER = {6},
     PAGES = {1521--1531},
      ISSN = {0218-1967},
   MRCLASS = {03C05 (08B05 68Q25)},
  MRNUMBER = {3109457},
MRREVIEWER = {Klaus Denecke},
       DOI = {10.1142/S0218196713500343},
       URL = {http://dx.doi.org/10.1142/S0218196713500343},
}
@ARTICLE{2017arXiv170302764D,
   author = {{DeMeo}, W.},
    title = "{The Commutator as Least Fixed Point of a Closure Operator}",
  journal = {ArXiv e-prints},
archivePrefix = "arXiv",
   eprint = {1703.02764},
 primaryClass = "math.LO",
 keywords = {Mathematics - Logic, Mathematics - Rings and Algebras},
     year = 2017,
    month = mar,
   adsurl = {http://adsabs.harvard.edu/abs/2017arXiv170302764D},
  adsnote = {Provided by the SAO/NASA Astrophysics Data System}
}
@article {MR1871085,
    AUTHOR = {Bergman, Clifford and Slutzki, Giora},
     TITLE = {Computational complexity of some problems involving
              congruences on algebras},
   JOURNAL = {Theoret. Comput. Sci.},
  FJOURNAL = {Theoretical Computer Science},
    VOLUME = {270},
      YEAR = {2002},
    NUMBER = {1-2},
     PAGES = {591--608},
      ISSN = {0304-3975},
     CODEN = {TCSDI},
   MRCLASS = {08A30 (05C85 08A35 68Q17)},
  MRNUMBER = {1871085 (2002i:08002)},
MRREVIEWER = {Radim B{\v{e}}lohl{\'a}vek},
       DOI = {10.1016/S0304-3975(01)00009-3},
       URL = {http://dx.doi.org/10.1016/S0304-3975(01)00009-3},
}
@article {MR1695293,
    AUTHOR = {Bergman, Clifford and Juedes, David and Slutzki, Giora},
     TITLE = {Computational complexity of term-equivalence},
   JOURNAL = {Internat. J. Algebra Comput.},
  FJOURNAL = {International Journal of Algebra and Computation},
    VOLUME = {9},
      YEAR = {1999},
    NUMBER = {1},
     PAGES = {113--128},
      ISSN = {0218-1967},
   MRCLASS = {68Q17 (08A70 68Q15)},
  MRNUMBER = {1695293 (2000b:68088)},
       DOI = {10.1142/S0218196799000084},
       URL = {http://dx.doi.org/10.1142/S0218196799000084},
}
@article {MR3449235,
    AUTHOR = {Kearnes, Keith and Szendrei, {\'A}gnes and Willard, Ross},
     TITLE = {A finite basis theorem for difference-term varieties with a
              finite residual bound},
   JOURNAL = {Trans. Amer. Math. Soc.},
  FJOURNAL = {Transactions of the American Mathematical Society},
    VOLUME = {368},
      YEAR = {2016},
    NUMBER = {3},
     PAGES = {2115--2143},
      ISSN = {0002-9947},
   MRCLASS = {03C05 (08B05 08B10)},
  MRNUMBER = {3449235},
       DOI = {10.1090/tran/6509},
       URL = {http://dx.doi.org/10.1090/tran/6509},
}
@article {MR1663558,
    AUTHOR = {Kearnes, Keith A. and Szendrei, {\'A}gnes},
     TITLE = {The relationship between two commutators},
   JOURNAL = {Internat. J. Algebra Comput.},
  FJOURNAL = {International Journal of Algebra and Computation},
    VOLUME = {8},
      YEAR = {1998},
    NUMBER = {4},
     PAGES = {497--531},
      ISSN = {0218-1967},
   MRCLASS = {08A05 (08A30)},
  MRNUMBER = {1663558},
MRREVIEWER = {M. G. Stone},
       DOI = {10.1142/S0218196798000247},
       URL = {http://dx.doi.org/10.1142/S0218196798000247},
}
@article{KSW,
title = {Simpler {M}altsev conditions for (weak) difference terms in locally finite varieties},
author = {Kearnes, Keith and Szendrei, \'{A}gnes and Willard, Ross},
note = {to appear}
}

@article {MR3239624,
    AUTHOR = {Valeriote, M. and Willard, R.},
     TITLE = {Idempotent {$n$}-permutable varieties},
   JOURNAL = {Bull. Lond. Math. Soc.},
  FJOURNAL = {Bulletin of the London Mathematical Society},
    VOLUME = {46},
      YEAR = {2014},
    NUMBER = {4},
     PAGES = {870--880},
      ISSN = {0024-6093},
   MRCLASS = {08A05 (06F99 68Q25)},
  MRNUMBER = {3239624},
       DOI = {10.1112/blms/bdu044},
       URL = {http://dx.doi.org/10.1112/blms/bdu044},
}
@article {MR3350327,
    AUTHOR = {Kozik, Marcin and Krokhin, Andrei and Valeriote, Matt and
              Willard, Ross},
     TITLE = {Characterizations of several {M}altsev conditions},
   JOURNAL = {Algebra Universalis},
  FJOURNAL = {Algebra Universalis},
    VOLUME = {73},
      YEAR = {2015},
    NUMBER = {3-4},
     PAGES = {205--224},
      ISSN = {0002-5240},
   MRCLASS = {08B05 (08A70 08B10)},
  MRNUMBER = {3350327},
MRREVIEWER = {David Hobby},
       DOI = {10.1007/s00012-015-0327-2},
       URL = {http://dx.doi.org/10.1007/s00012-015-0327-2},
}
@article {MR1358491,
    AUTHOR = {Kearnes, Keith A.},
     TITLE = {Varieties with a difference term},
   JOURNAL = {J. Algebra},
  FJOURNAL = {Journal of Algebra},
    VOLUME = {177},
      YEAR = {1995},
    NUMBER = {3},
     PAGES = {926--960},
      ISSN = {0021-8693},
     CODEN = {JALGA4},
   MRCLASS = {08B10 (08B05)},
  MRNUMBER = {1358491},
MRREVIEWER = {H. Peter Gumm},
       DOI = {10.1006/jabr.1995.1334},
       URL = {http://dx.doi.org/10.1006/jabr.1995.1334},
}
@book {MR2839398,
    AUTHOR = {Bergman, Clifford},
     TITLE = {Universal algebra},
    SERIES = {Pure and Applied Mathematics (Boca Raton)},
    VOLUME = {301},
      NOTE = {Fundamentals and selected topics},
 PUBLISHER = {CRC Press, Boca Raton, FL},
      YEAR = {2012},
     PAGES = {xii+308},
      ISBN = {978-1-4398-5129-6},
   MRCLASS = {08-02 (06-02 08A40 08B05 08B10 08B26)},
  MRNUMBER = {2839398 (2012k:08001)},
MRREVIEWER = {Konrad P. Pi{\'o}ro},
}
@article {MR0434928,
    AUTHOR = {Taylor, Walter},
     TITLE = {Varieties obeying homotopy laws},
   JOURNAL = {Canad. J. Math.},
  FJOURNAL = {Canadian Journal of Mathematics. Journal Canadien de
              Math\'ematiques},
    VOLUME = {29},
      YEAR = {1977},
    NUMBER = {3},
     PAGES = {498--527},
      ISSN = {0008-414X},
   MRCLASS = {08A25},
  MRNUMBER = {0434928 (55 \#7891)},
MRREVIEWER = {James B. Nation},
}
@BOOK{HM:1988,
    AUTHOR = {Hobby, David and McKenzie, Ralph},
    TITLE = {The structure of finite algebras},
    SERIES = {Contemporary Mathematics},
    VOLUME = {76},
    PUBLISHER = {American Mathematical Society},
    ADDRESS = {Providence, RI},
    YEAR = {1988},
    PAGES = {xii+203},
    ISBN = {0-8218-5073-3},
    MRCLASS = {08A05 (03C05 08-02 08B05)},
    MRNUMBER = {958685 (89m:08001)},
    MRREVIEWER = {Joel Berman},
    note = {Available from:
      \href{http://math.hawaii.edu/~ralph/Classes/619/HobbyMcKenzie-FiniteAlgebras.pdf}{math.hawaii.edu}}
  }
@article {MR0455543,
    AUTHOR = {Jones, Neil D. and Laaser, William T.},
     TITLE = {Complete problems for deterministic polynomial time},
   JOURNAL = {Theoret. Comput. Sci.},
  FJOURNAL = {Theoretical Computer Science},
    VOLUME = {3},
      YEAR = {1976},
    NUMBER = {1},
     PAGES = {105--117 (1977)},
      ISSN = {0304-3975},
   MRCLASS = {68A20},
  MRNUMBER = {0455543},
MRREVIEWER = {Forbes D. Lewis},
       DOI = {10.1016/0304-3975(76)90068-2},
       URL = {http://dx.doi.org/10.1016/0304-3975(76)90068-2},
}
	
@article{Freese:2009,
    AUTHOR = {Freese, Ralph and Valeriote, Matthew A.},
    TITLE = {On the complexity of some {M}altsev conditions},
    JOURNAL = {Internat. J. Algebra Comput.},
    FJOURNAL = {International Journal of Algebra and Computation},
    VOLUME = {19},
    YEAR = {2009},
    NUMBER = {1},
    PAGES = {41--77},
    ISSN = {0218-1967},
    MRCLASS = {08B05 (03C05 08B10 68Q25)},
    MRNUMBER = {2494469 (2010a:08008)},
    MRREVIEWER = {Clifford H. Bergman},
    DOI = {10.1142/S0218196709004956},
    URL = {http://dx.doi.org/10.1142/S0218196709004956}
  }

@ARTICLE{KearnesKiss1999,
    AUTHOR = {Kearnes, Keith A. and Kiss, Emil W.},
     TITLE = {Modularity prevents tails},
   JOURNAL = {Proc. Amer. Math. Soc.},
  FJOURNAL = {Proceedings of the American Mathematical Society},
    VOLUME = {127},
      YEAR = {1999},
    NUMBER = {1},
     PAGES = {11--19},
      ISSN = {0002-9939},
     CODEN = {PAMYAR},
   MRCLASS = {08A05 (08A30 08B10)},
  MRNUMBER = {99m:08003},
MRREVIEWER = {Branimir {\v{S}}e{\v{s}}elja},
}
  
  
  
@article {MR3076179,
    AUTHOR = {Kearnes, Keith A. and Kiss, Emil W.},
     TITLE = {The shape of congruence lattices},
   JOURNAL = {Mem. Amer. Math. Soc.},
  FJOURNAL = {Memoirs of the American Mathematical Society},
    VOLUME = {222},
      YEAR = {2013},
    NUMBER = {1046},
     PAGES = {viii+169},
      ISSN = {0065-9266},
      ISBN = {978-0-8218-8323-5},
   MRCLASS = {08B05 (08B10)},
  MRNUMBER = {3076179},
MRREVIEWER = {James B. Nation},
       DOI = {10.1090/S0065-9266-2012-00667-8},
       URL = {http://dx.doi.org/10.1090/S0065-9266-2012-00667-8},
}
@incollection {MR1404955,
    AUTHOR = {Kearnes, Keith A.},
     TITLE = {Idempotent simple algebras},
 BOOKTITLE = {Logic and algebra ({P}ontignano, 1994)},
    SERIES = {Lecture Notes in Pure and Appl. Math.},
    VOLUME = {180},
     PAGES = {529--572},
 PUBLISHER = {Dekker, New York},
      YEAR = {1996},
   MRCLASS = {08B05 (06F25 08A05 08A30)},
  MRNUMBER = {1404955 (97k:08004)},
MRREVIEWER = {E. W. Kiss},
}
@misc{william_demeo_2016_53936,
  author       = {DeMeo, William and Freese, Ralph},
  title        = {AlgebraFiles v1.0.1},
  month        = May,
  year         = 2016,
  doi          = {10.5281/zenodo.53936},
  url          = {http://dx.doi.org/10.5281/zenodo.53936}
}
@article{FreeseMcKenzie2016,
	Author = {Freese, Ralph and McKenzie, Ralph},
	Date-Added = {2016-08-22 19:43:56 +0000},
	Date-Modified = {2016-08-22 19:45:50 +0000},
	Journal = {Algebra Universalis},
	Title = {Mal'tsev families of varieties closed under join or Mal'tsev product},
	Year = {to appear}
}
@article {MR2333368,
    AUTHOR = {Kearnes, Keith A. and Tschantz, Steven T.},
     TITLE = {Automorphism groups of squares and of free algebras},
   JOURNAL = {Internat. J. Algebra Comput.},
  FJOURNAL = {International Journal of Algebra and Computation},
    VOLUME = {17},
      YEAR = {2007},
    NUMBER = {3},
     PAGES = {461--505},
      ISSN = {0218-1967},
   MRCLASS = {08A35 (08B20 20B25)},
  MRNUMBER = {2333368},
MRREVIEWER = {Giovanni Ferrero},
       DOI = {10.1142/S0218196707003615},
       URL = {http://dx.doi.org/10.1142/S0218196707003615},
}
@article {MR2504025,
    AUTHOR = {Valeriote, Matthew A.},
     TITLE = {A subalgebra intersection property for congruence distributive
              varieties},
   JOURNAL = {Canad. J. Math.},
  FJOURNAL = {Canadian Journal of Mathematics. Journal Canadien de
              Math\'ematiques},
    VOLUME = {61},
      YEAR = {2009},
    NUMBER = {2},
     PAGES = {451--464},
      ISSN = {0008-414X},
     CODEN = {CJMAAB},
   MRCLASS = {08B10 (08A30 08B05)},
  MRNUMBER = {2504025},
MRREVIEWER = {Jarom{\'{\i}}r Duda},
       DOI = {10.4153/CJM-2009-023-2},
       URL = {http://dx.doi.org/10.4153/CJM-2009-023-2},
}
@misc{UACalc,
	Author = {Ralph Freese and Emil Kiss and Matthew Valeriote},
	Date-Added = {2014-11-20 01:52:20 +0000},
	Date-Modified = {2014-11-20 01:52:20 +0000},
	Note = {Available at: {\verb+www.uacalc.org+}},
	Title = {Universal {A}lgebra {C}alculator},
	Year = {2011}
}
@article{Freese2008,
	Author = {Freese, Ralph},
	Date-Added = {2016-08-29 01:31:23 +0000},
	Date-Modified = {2016-08-29 01:32:09 +0000},
	Journal = {Alg. Univ.},
	Pages = {337--343},
	Title = {Computing congruences efficiently},
	Volume = {59},
	Year = {2008}
}	
@article {MR2470585,
    AUTHOR = {Freese, Ralph},
     TITLE = {Computing congruences efficiently},
   JOURNAL = {Algebra Universalis},
  FJOURNAL = {Algebra Universalis},
    VOLUME = {59},
      YEAR = {2008},
    NUMBER = {3-4},
     PAGES = {337--343},
      ISSN = {0002-5240},
   MRCLASS = {08A30 (08A40 68W30 68W40)},
  MRNUMBER = {2470585 (2009j:08003)},
MRREVIEWER = {Clifford H. Bergman},
       DOI = {10.1007/s00012-008-2073-1},
       URL = {http://dx.doi.org/10.1007/s00012-008-2073-1},
}
@incollection {MR1191235,
    AUTHOR = {Szendrei, {\'A}gnes.},
     TITLE = {A survey on strictly simple algebras and minimal varieties},
 BOOKTITLE = {Universal algebra and quasigroup theory ({J}adwisin, 1989)},
    SERIES = {Res. Exp. Math.},
    VOLUME = {19},
     PAGES = {209--239},
 PUBLISHER = {Heldermann, Berlin},
      YEAR = {1992},
   MRCLASS = {08-02 (08A40 08B05)},
  MRNUMBER = {1191235 (93h:08001)},
MRREVIEWER = {Ivan Chajda},
}
@unpublished{Bergman-DeMeo,
    AUTHOR = {Bergman, Clifford and DeMeo, William},
    TITLE = {Universal Algebraic Methods for Constraint Satisfaction Problems:
      with applications to commutative idempotent binars},
    YEAR = {2016},
    NOTE = {unpublished notes; soon to be available online},
    URL = {https://github.com/UniversalAlgebra/algebraic-csp}
}
\end{filecontents*}
%:biblio
%\documentclass[12pt]{amsart}
%% \documentclass[12pt]{amsart}
\documentclass[12pt]{au}


%% wjd added these packages vvvvvvvvvvvvvvvvvvvvvvvvv
\usepackage{url,amssymb,enumerate,tikz,scalefnt} 
\usepackage[normalem]{ulem} % for \sout (strikeout)   wjd: could remove this in final draft
\usepackage[colorlinks=true,urlcolor=blue,linkcolor=blue,citecolor=blue]{hyperref}
\usepackage{algorithm2e}

\newcommand{\mysetminus}{\ensuremath{-}}
%% uncomment the next line if we want to revert to the "set" minus notation
%% \renewcommand{\mysetminus}{\ensuremath{\setminus}}

%%  wjd ^^^^^^^^^^^^^^^^^^^^^^^^^^^^^^^^^^^^^^^^^^^^^



\usepackage{color}
\usepackage{amsmath}
\usepackage{amsfonts}
\usepackage{amscd}
%% \usepackage{exers}
\usepackage{inputs/rflatexmacs}
\usepackage{inputs/wjdlatexmacs}

\usepackage[mathcal]{euscript}
\usepackage{comment}

\renewcommand{\th}[2]{#1\mathrel{\theta}#2}
\newcommand{\infixrel}[3]{#2\mathrel{#1}#3}

%\usepackage{hyperref}

%\renewcommand{\V}{\text{\textup{V}}}
%\renewcommand{\V}{\operatorname{V}}
%\renewcommand{\T}{\operatorname{T}}
%\newcommand{\U}{\text{U}}

\newtheorem{theorem}{Theorem}
\newtheorem{lemma}[theorem]{Lemma}
\newtheorem{corollary}[theorem]{Corollary}
\newtheorem{prop}[theorem]{Proposition}
\newtheorem{conjecture}[theorem]{Conjecture}
\theoremstyle{definition}
\newtheorem{example}[theorem]{Example}
\newtheorem{fact}[theorem]{Fact}
\newtheorem{remark}{Remark}
\newtheorem*{remarks}{Remarks}
\newtheorem{prob}{Problem}

\title[A Test for a Difference Term]{A Polynomial-Time Test for a
Difference Term in an Idempotent Variety}
\author[W. DeMeo]{William DeMeo}
\email{williamdemeo@gmail.com}
\urladdr{http://williamdemeo.github.io}
\address{Department of Mathematics\\Iowa State University\\Ames 50010\\USA}
\author[R. Freese]{Ralph Freese}
\email{ralph@math.hawaii.edu}
\urladdr{http://www.math.hawaii.edu/~ralph/}
\address{Department of Mathematics\\University of Hawaii\\Honolulu 96822\\USA}
\author{Matthew Valeriote}
\email{matt@math.mcmaster.ca}
\urladdr{http://ms.mcmaster.ca/~matt/}
\address{Department of Mathematics\\McMaster University\\Hamilton, Ontario\\
Canada L8S 4K1}


\thanks{This research was supported by the National
Science Foundation under Grant No. 1500235}

\date{\today}

\begin{document}

\maketitle 

\begin{abstract}
We consider the following practical question: given a finite 
algebra $\alg{A}$ in a
finite language, can we efficiently decide whether the variety 
generated by $\alg{A}$
has a difference term?  We answer this question in the idempotent case
and then describe possible algorithms for constructing difference terms.
\end{abstract}



\end{document}



























With notation and assumptions as above.

\begin{theorem}[cf. Theorem 3.3 of \cite{FreeseValeriote2009}]
This is a test.
\end{theorem}






\bibliographystyle{rsfplain}
%\bibliography{\jobname}
\bibliography{/Users/ralph/tex/bib/Database/db}



A \emph{minority term} (for an algebra or variety) is a 
3-variable term $q(x,y,z)$ such if two of the variables
are equal, its value is the other one; that is, 
\[
q(x,x,y) \approx q(x,y,x) \approx q(y,x,x) \approx y.
\]
We are interested in an algebraic description of varieties having
a minority term. One possible conjecture is:
\begin{conjecture}
$\mathcal V$ has a minority term if and only if it is CP and its
ring has characteristic~1 or~2.
\end{conjecture}



\begin{fact}
If $\mathcal V$ has a minority term, then it is CP.
\end{fact}

\begin{proof}
Clearly a minority term is a Maltsev term.
\end{proof}

\begin{fact}
A CD variety $\mathcal V$ has a minority term if and
only if it is CP.
\end{fact}

\begin{proof}
A variety is CD and CP if and only if it has a Pixley term. 
If $p(x,y,z)$ is a Pixley term then 
\[
q(x,y,z) = p(p(x,y,z),x,p(x,z,y))
\]
is a minority term. The fact can be derived from these
observations.
\end{proof}

\begin{fact}
A variety of groups has a minority term if and only
the variety has exponent 2 (well, or 1).
\end{fact}

\begin{proof}
If a variety has exponent 2 then it is abelian and so,
using additive notation, $x + y + z$ is a minority term.

If the exponent is not 2 then the variety contains a group
which contains an element of order $n$, where $n > 2$ (or 
infinite). This element generates a cyclic group. In any
abelian algebra the Maltsev term operation is unique and so
must be $x - y + z$. But one easily checks that this is not
a minority term when $n > 2$.
\end{proof}

\begin{lemma}
Let $\mathcal V$ be a CP variety with a Maltsev term $p(x,y.z)$.
The the following are equivalent:
\begin{enumerate}
\item
The ring $R(\mathcal V)$ of $\mathcal V$ has characteristic~2.
\item
On every block of every abelian congruence, $p$ restricted to the 
block is a minority term; that is, satisfies $p(a,b,a) = b$.
\item
If $\theta = \textup{Cg}^{\alg F_{\mathcal V}(x,y)}(x,y)$, then,
on each block of $\theta/[\theta,\theta]$, $p$ is a minority term.
\end{enumerate}
If $\mathcal V$ has a minority term, then these conditions hold.
\end{lemma}

\begin{proof}
By commutator theory (give some specific refs, also that the ring
is det by the strucure of $\theta/[\theta,\theta]$) 
the Maltsev term operation on an
abelian algebra, or even a block of an abelian congruence, 
is unique and it is $p(x,y,z) = x - y + z$ for
some abelian group. It is easy that $x - y + z$ is a minority
if and only if the abelian group has exponent~2.
\end{proof}


\end{document}


Let $\alg A$ be an algebra and let $S$ and $T$ be tolerances
on $\alg A$. 
Let $M(S,T)$, or $M^{\alg A}(S,T)$ to emphasize $\alg A$,
be the set of all $2 \times 2$ matrices of the form
\begin{equation}\label{eq1}
\begin{bmatrix}
p&q\\
r&s
\end{bmatrix}
=
\begin{bmatrix}
f(\mathbf{a},\mathbf{u})&f(\mathbf{a},\mathbf{v})\\
f(\mathbf{b},\mathbf{u})&f(\mathbf{b},\mathbf{v})
\end{bmatrix}
\end{equation}
where $f(\mathbf{x},\mathbf{y})$ is an $(m+n)$-ary polynomial of
$\alg A$, $\mathbf{a} \mathrel{S} \mathbf{b}$, and 
$\mathbf{u} \mathrel{T} \mathbf{v}$
(componentwise, of course). The members of $M(S,T)$ are called
\emph{$S,T$-matrices}.

The first exercise gives an efficient way to find $M(S,T)$.

\section*{Exercises}

\begin{exercises}

\prob
Show that $M(S,T)$ is the subalgebra of $\alg A^4$ generated by
\[
\left\{
\begin{bmatrix}
a&a\\
b&b
\end{bmatrix} : a \mathrel{S} b\right\}
\union
\left\{
\begin{bmatrix}
c&d\\
c&d
\end{bmatrix} : c \mathrel{T} d\right\}
\]

\prob
Use the symmetry of $S$ and $T$ to show the matrix obtained from an
$S,T$-matrix by interchanging the rows or columns (or both) is also
in $M(S,T)$. 

\prob
$M(T,T)$ is closed under taking transposes. 

\end{exercises}

\section*{Centrality Relations}

We define four kinds of centrality, called centrality, strong
centrality, weak centrality, and strong rectularity. The is a fifth
centrality condition known as rectangularity which we will save for
later.

Let $\delta$ be a congruence and $S$ and $T$ be
tolerance relations on  $\alg A$. The above centrality relations
are denoted $\alg C(S,T;\delta)$ (centrality), 
$\alg S(S,T;\delta)$ (strong centrality), 
$\alg W(S,T;\delta)$ (weak centrality),  and
$\alg SR(S,T;\delta)$ (strong rectangularity). They hold if the
appropriate implication below holds for all 
\[
\begin{bmatrix}
p&q\\
r&s
\end{bmatrix} \in M(S,T)
\]
\begin{itemize}
\item centrality: 
$p \mathrel{\delta} q \implies r \mathrel{\delta} s$.
\item strong rectangularity: 
$p \mathrel{\delta} s \implies r \mathrel{\delta} s$.
\item weak centrality: 
$p \mathrel{\delta} q \mathrel{\delta} s \implies r \mathrel{\delta} s$.
\item strong centrality holds if both centrality and strong
rectangularity hold.
\end{itemize}

Using the exercises it is easy to see that the implication defining 
$\alg C(S,T;\delta)$ can be replaced by 
$r \mathrel{\delta} s \implies p \mathrel{\delta} q$ and this is
equivalent to 
\[
p \mathrel{\delta} q \Longleftrightarrow r \mathrel{\delta} s.
\]
Similar statements hold for the other conditions: weak centrality
is equivalent to saying that if any three of $p$, $q$, $r$ and $s$
are $\delta$ related, then they all are. And strong rectangularity
says that if the elements of the main diagonal, or of the sinister
diagonal, are $\delta$ related, then all four are.

The \emph{$S,T$-term condition} is the condition $\alg C(S,T,0)$,
usually expressed using the right-hand matrix in~\eqref{eq1}. 
Other kinds of term conditions are defined similarly. 

If $\alg C(S,T;\delta_i)$ holds for all $i \in I$, then
$\alg C(S,T;\Meet_{i\in I}\delta_i)$ holds. Similar statements hold
for the other centrality conditions. So there is a least $\delta$
such that $\alg C(S,T;\delta)$ holds. This $\delta$ is the 
\emph{commutator} of $S$ and $T$, and is denoted $[S,T]$. The
commutators for the other centrality relations are denoted
$[S,T]_{\alg S}$, $[S,T]_{\alg {SR}}$, and $[S,T]_{\alg W}$.

The properties of these centrality relations are coverered in
Theorem~2.19 and Theorem~3.4 of~\cite{KearnesKiss2013}. Much stronger
properties hold in congruence modular varieties;
see~\cite{FreeseMcKenzie1987}.

\section*{Exercises}

\begin{exercises}

\prob
As defined in \cite{HobbyMcKenzie1988}, $\beta$ is \emph{strongly
Abelian} over $\delta$ ($\delta \leq \beta$, both congruences on $\alg A$)
if the following implication holds for all polynomials $f$ and all
elements $x_0, \ldots, x_{n-1}$, $y_0, \ldots, y_{n-1}$, and 
$z_1, \ldots, z_{n-1}$ with $x_0 \mathrel\beta y_0$ and
$x_i \mathrel\beta y_i \mathrel\beta z_i$, $i = 1, \ldots n-1$.
\begin{align*}
f(x_0,\ldots,&x_{n-1}) \mathrel\delta f(y_0,\ldots,y_{n-1}) \\
&\implies
f(x_0, z_1,\ldots,z_{n-1}) \mathrel\delta f(y_0, z_1,\ldots,z_{n-1})
\end{align*}
Show that $\beta$ is strongly 
Abelian over $\delta$ if and only if $\alg S(\beta,\beta;\delta)$ 
holds, and also show this is in turn equivalent to
$\alg {SR}(\beta,\beta;\delta)$.
\end{exercises}



%%%%%%% wjd: old stuff (will delete eventually)%%%%%%%%%%%%%%%%%

\noindent \emph{TODO: the remaining sections about the algorithm will be completely
rewritten and mostly deleted.  Obviously we don't want to build up the 
algorithm in the way described.  Instead, we'll use recursion.}

\subsection{Sets of size three}
An \ld term operation for the set
\begin{equation*}
\{(a_0,b_0, 0), (a_1, b_1, 0), (a_2, b_2, 0)\}
\end{equation*}
is the first projection, $t(x,y,z) = x$.
An \ld term operation for the set
\begin{equation*}
\{(a_0,b_0,1), (a_1, b_1, 1), (a_2, b_2, 1)\}
\end{equation*}
is the third projection, $t(x,y,z) = y$.
There are two other forms of 3-sets to consider.
We label these $P_3$ and $Q_3$ and handle them with 
the following subroutines.
% They are
% \begin{align*}
% P &= \{(a_0, b_0, 0), (a_1, b_1, 0),  (a_2, b_2, 1)\} \text{ and }\\
% Q &= \{(a_0, b_0, 1), (a_1, b_1, 1), (a_2, b_2, 0)\}.
% \end{align*}
% $P = \{(a_0, b_0, 0), (a_1, b_1, 0),  (a_2, b_2, 1)\}$
% and $Q = \{(a_0, b_0, 1), (a_1, b_1, 1), (a_2, b_2, 0)\}$.


\noindent \underline{\textbf{Subroutine \ld-3.0}}\\[4pt]
To find an \ld term operation for
$P_3:=\{(a_0, b_0, 0), (a_1, b_1, 0),  (a_2, b_2, 1)\}$,
% $\{(a_0, b_0, 0), (a_1, b_1, 1),  (a_2, b_2, 0)\}$,
\begin{enumerate}
\item use Subroutine \ld-2 to compute an \ld term operation $s$ for
\begin{equation*}
\{(a_1, b_1, 0), (a_2, b_2, 1)\}; 
\end{equation*}
\item use Subroutine \ld-2 to compute an \ld term operation $t$ for
\begin{equation*}
\{(a_0, s(a_0, b_0, b_0), 0), (a_2, b_2, 1)\}.
\end{equation*}
\end{enumerate}
It is easy to check that
$d(x,y,z) = t(x, s(x,y,y), s(x,y,z))$
is then an \ld term operation for $P_3$ 
(see Case $\chi_0=0$ in the proof of Theorem~\ref{thm:local-diff-terms}).


\noindent \underline{\textbf{Subroutine \ld-3.1}}\\[4pt]
To find an \ld term operation for 
$Q_3 := \{(a_0, b_0, 1), (a_1, b_1, 1), (a_2, b_2, 0)\}$,
\begin{enumerate}
\item \label{item:001-1}
use Subroutine \ld-2 to compute an \ld term $s$ for the set
\begin{equation*}
\{(a_1, b_1, 1), (a_2, b_2, 0)\};
\end{equation*}
\item \label{item:001-2} use Subroutine \ld-2 to compute an \ld term $t$
for the set
\begin{equation*}
\{(s(a_0, a_0, b_0), b_0, 1),  (a_2,a_2,0)\}.
\end{equation*}
\end{enumerate}
Then 
%%%
$d(x,y,z) = t(s(x,y,z), s(y,y,z),z)$
%%%
is an \ld term  for $Q_3$ (see Case $\chi_0=1$ in the proof of Theorem~\ref{thm:local-diff-terms}).

\subsection{Sets of Size 4} We handle one more case before 
using induction to give a general recursive algorithm.
The nontrivial forms of 4-sets are
\begin{align*}
P_4 &:= \{(a_0, b_0, 0), (a_1, b_1, 0),  (a_2, b_2, 0),  (a_3, b_3, 1)\},\\ 
Q_4 &:= \{(a_0, b_0, 1), (a_1, b_1, 1), (a_2, b_2, 1), (a_3, b_3, 0)\},\\
R_4 &:= \{(a_0, b_0, 0), (a_1, b_1, 0),  (a_2, b_2, 1),  (a_3, b_3, 1)\}.
\end{align*}

\medskip

\noindent \underline{\textbf{Subroutine \ld-4.0}}\\[4pt]
To find an \ld term operation for a set like $P_4$,
% $\{(a_0, b_0, 0), (a_1, b_1, 1),  (a_2, b_2, 0)\}$,
\begin{enumerate}
\item use Subroutine \ld-3.0 to compute an \ld term operation $s$ for
\begin{equation*}
\{(a_1, b_1, 0),  (a_2, b_2, 0),  (a_3, b_3, 1)\};
\end{equation*}
\item use Subroutine \ld-2 to compute an \ld term operation $t$ for
\begin{equation*}
\{(a_0, s(a_0, b_0, b_0), 0), (a_3, b_3, 1)\}.
\end{equation*}
\end{enumerate}
It is easy to check that
$d(x,y,z) = t(x, s(x,y,y), s(x,y,z))$
is then an \ld term operation for $P_4$ 
(see Case $\chi_0=0$ in the proof of Theorem~\ref{thm:local-diff-terms}).

\medskip

\noindent \underline{\textbf{Subroutine \ld-4.1}}\\[4pt]
To find an \ld term operation for a set like $Q_4$
% $\{(a_0, b_0, 0), (a_1, b_1, 1),  (a_2, b_2, 1)\}$,
\begin{enumerate}
\item 
use Subroutine \ld-3.1 to compute an \ld term $s$ for the set
\begin{equation*}
\{(a_1, b_1, 1), (a_2, b_2, 1), (a_3, b_3, 0)\};
\end{equation*}
\item  use Subroutine \ld-2 to compute an \ld term $t$
for the set
\begin{equation*}
\{(s(a_0, a_0, b_0), b_0, 1),  (a_3,a_3,0)\}.
\end{equation*}
\end{enumerate}
Then 
%%%
$d(x,y,z) = t(s(x,y,z), s(y,y,z),z)$
%%%
is an \ld term  for $Q_4$ (see Case $\chi_0=1$ in the proof of Theorem~\ref{thm:local-diff-terms}).


\medskip

\noindent \underline{\textbf{Subroutine \ld-4.2}}\\[4pt]
To find an \ld term operation for a set like $R_4$
% $\{(a_0, b_0, 0), (a_1, b_1, 1),  (a_2, b_2, 1)\}$,
\begin{enumerate}
\item 
use Subroutine \ld-3.1 to compute an \ld term $s$ for the set
\begin{equation*}
\{(a_1, b_1, 0), (a_2, b_2, 1), (a_3, b_3, 1)\};
\end{equation*}
\item use Subroutine \ld-2 to compute an \ld term operation $t$ for
\begin{equation*}
\{(a_0, s(a_0, b_0, b_0), 0), (a_3, b_3, 1)\}.
\end{equation*}
\end{enumerate}
Then 
$d(x,y,z) = t(x, s(x,y,y), s(x,y,z))$
is an \ld term operation for $P_4$. 
