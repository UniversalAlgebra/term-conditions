%% FILE: diffTerm.tex
%% AUTHOR: William DeMeo, Ralph Freese, Matthew Valeriote
%% DATE: 13 April 2017
%% COPYRIGHT: (C) 2017 DeMeo, Freese, Valeriote

%%%%%%%%%%%%%%%%%%%%%%%%%%%%%%%%%%%%%%%%%%%%%%%%%%%%%%%%%%
%%                         BIBLIOGRAPHY FILE            %%
%%%%%%%%%%%%%%%%%%%%%%%%%%%%%%%%%%%%%%%%%%%%%%%%%%%%%%%%%%
%% The `filecontents` command will crete a file in the inputs directory called
%% refs.bib containing the references in the document, in case this file does
%% not exist already.
%% If you want to add a BibTeX entry, please don't add it directly to the
%% refs.bib file.  Instead, add it in this file between the
%% \begin{filecontents*}{refs.bib} and \end{filecontents*} lines
%% then delete the existing refs.bib file so it will be automatically generated
%% again with your new entry the next time you run pdfaltex.
\begin{filecontents*}{inputs/refs.bib}
@book {MR2637477,
    AUTHOR = {Willard, Ross David},
     TITLE = {Varieties having {B}oolean factor congruences},
      NOTE = {Thesis (Ph.D.)--University of Waterloo (Canada)},
 PUBLISHER = {ProQuest LLC, Ann Arbor, MI},
      YEAR = {1989},
     PAGES = {(no paging)},
      ISBN = {978-0315-49135-9},
   MRCLASS = {Thesis},
  MRNUMBER = {2637477}
}
@article {MR3350334,
    AUTHOR = {Horowitz, Jonah},
     TITLE = {Testing for edge terms is decidable},
   JOURNAL = {Algebra Universalis},
  FJOURNAL = {Algebra Universalis},
    VOLUME = {73},
      YEAR = {2015},
    NUMBER = {3-4},
     PAGES = {321--334},
      ISSN = {0002-5240},
   MRCLASS = {03C05 (03B25 08A40 08B05)},
  MRNUMBER = {3350334},
MRREVIEWER = {David Casperson},
       DOI = {10.1007/s00012-015-0325-4},
       URL = {http://dx.doi.org/10.1007/s00012-015-0325-4},
}
@article {MR3109457,
    AUTHOR = {Horowitz, Jonah},
     TITLE = {Computational complexity of various {M}al'cev conditions},
   JOURNAL = {Internat. J. Algebra Comput.},
  FJOURNAL = {International Journal of Algebra and Computation},
    VOLUME = {23},
      YEAR = {2013},
    NUMBER = {6},
     PAGES = {1521--1531},
      ISSN = {0218-1967},
   MRCLASS = {03C05 (08B05 68Q25)},
  MRNUMBER = {3109457},
MRREVIEWER = {Klaus Denecke},
       DOI = {10.1142/S0218196713500343},
       URL = {http://dx.doi.org/10.1142/S0218196713500343},
}
@ARTICLE{2017arXiv170302764D,
   author = {{DeMeo}, W.},
    title = "{The Commutator as Least Fixed Point of a Closure Operator}",
  journal = {ArXiv e-prints},
archivePrefix = "arXiv",
   eprint = {1703.02764},
 primaryClass = "math.LO",
 keywords = {Mathematics - Logic, Mathematics - Rings and Algebras},
     year = 2017,
    month = mar,
   adsurl = {http://adsabs.harvard.edu/abs/2017arXiv170302764D},
  adsnote = {Provided by the SAO/NASA Astrophysics Data System}
}
@article {MR1871085,
    AUTHOR = {Bergman, Clifford and Slutzki, Giora},
     TITLE = {Computational complexity of some problems involving
              congruences on algebras},
   JOURNAL = {Theoret. Comput. Sci.},
  FJOURNAL = {Theoretical Computer Science},
    VOLUME = {270},
      YEAR = {2002},
    NUMBER = {1-2},
     PAGES = {591--608},
      ISSN = {0304-3975},
     CODEN = {TCSDI},
   MRCLASS = {08A30 (05C85 08A35 68Q17)},
  MRNUMBER = {1871085 (2002i:08002)},
MRREVIEWER = {Radim B{\v{e}}lohl{\'a}vek},
       DOI = {10.1016/S0304-3975(01)00009-3},
       URL = {http://dx.doi.org/10.1016/S0304-3975(01)00009-3},
}
@article {MR1695293,
    AUTHOR = {Bergman, Clifford and Juedes, David and Slutzki, Giora},
     TITLE = {Computational complexity of term-equivalence},
   JOURNAL = {Internat. J. Algebra Comput.},
  FJOURNAL = {International Journal of Algebra and Computation},
    VOLUME = {9},
      YEAR = {1999},
    NUMBER = {1},
     PAGES = {113--128},
      ISSN = {0218-1967},
   MRCLASS = {68Q17 (08A70 68Q15)},
  MRNUMBER = {1695293 (2000b:68088)},
       DOI = {10.1142/S0218196799000084},
       URL = {http://dx.doi.org/10.1142/S0218196799000084},
}
@article {MR3449235,
    AUTHOR = {Kearnes, Keith and Szendrei, {\'A}gnes and Willard, Ross},
     TITLE = {A finite basis theorem for difference-term varieties with a
              finite residual bound},
   JOURNAL = {Trans. Amer. Math. Soc.},
  FJOURNAL = {Transactions of the American Mathematical Society},
    VOLUME = {368},
      YEAR = {2016},
    NUMBER = {3},
     PAGES = {2115--2143},
      ISSN = {0002-9947},
   MRCLASS = {03C05 (08B05 08B10)},
  MRNUMBER = {3449235},
       DOI = {10.1090/tran/6509},
       URL = {http://dx.doi.org/10.1090/tran/6509},
}
@article {MR1663558,
    AUTHOR = {Kearnes, Keith A. and Szendrei, {\'A}gnes},
     TITLE = {The relationship between two commutators},
   JOURNAL = {Internat. J. Algebra Comput.},
  FJOURNAL = {International Journal of Algebra and Computation},
    VOLUME = {8},
      YEAR = {1998},
    NUMBER = {4},
     PAGES = {497--531},
      ISSN = {0218-1967},
   MRCLASS = {08A05 (08A30)},
  MRNUMBER = {1663558},
MRREVIEWER = {M. G. Stone},
       DOI = {10.1142/S0218196798000247},
       URL = {http://dx.doi.org/10.1142/S0218196798000247},
}
@article{KSW,
title = {Simpler {M}altsev conditions for (weak) difference terms in locally finite varieties},
author = {Kearnes, Keith and Szendrei, \'{A}gnes and Willard, Ross},
note = {to appear}
}

@article {MR3239624,
    AUTHOR = {Valeriote, M. and Willard, R.},
     TITLE = {Idempotent {$n$}-permutable varieties},
   JOURNAL = {Bull. Lond. Math. Soc.},
  FJOURNAL = {Bulletin of the London Mathematical Society},
    VOLUME = {46},
      YEAR = {2014},
    NUMBER = {4},
     PAGES = {870--880},
      ISSN = {0024-6093},
   MRCLASS = {08A05 (06F99 68Q25)},
  MRNUMBER = {3239624},
       DOI = {10.1112/blms/bdu044},
       URL = {http://dx.doi.org/10.1112/blms/bdu044},
}
@article {MR3350327,
    AUTHOR = {Kozik, Marcin and Krokhin, Andrei and Valeriote, Matt and
              Willard, Ross},
     TITLE = {Characterizations of several {M}altsev conditions},
   JOURNAL = {Algebra Universalis},
  FJOURNAL = {Algebra Universalis},
    VOLUME = {73},
      YEAR = {2015},
    NUMBER = {3-4},
     PAGES = {205--224},
      ISSN = {0002-5240},
   MRCLASS = {08B05 (08A70 08B10)},
  MRNUMBER = {3350327},
MRREVIEWER = {David Hobby},
       DOI = {10.1007/s00012-015-0327-2},
       URL = {http://dx.doi.org/10.1007/s00012-015-0327-2},
}
@article {MR1358491,
    AUTHOR = {Kearnes, Keith A.},
     TITLE = {Varieties with a difference term},
   JOURNAL = {J. Algebra},
  FJOURNAL = {Journal of Algebra},
    VOLUME = {177},
      YEAR = {1995},
    NUMBER = {3},
     PAGES = {926--960},
      ISSN = {0021-8693},
     CODEN = {JALGA4},
   MRCLASS = {08B10 (08B05)},
  MRNUMBER = {1358491},
MRREVIEWER = {H. Peter Gumm},
       DOI = {10.1006/jabr.1995.1334},
       URL = {http://dx.doi.org/10.1006/jabr.1995.1334},
}
@book {MR2839398,
    AUTHOR = {Bergman, Clifford},
     TITLE = {Universal algebra},
    SERIES = {Pure and Applied Mathematics (Boca Raton)},
    VOLUME = {301},
      NOTE = {Fundamentals and selected topics},
 PUBLISHER = {CRC Press, Boca Raton, FL},
      YEAR = {2012},
     PAGES = {xii+308},
      ISBN = {978-1-4398-5129-6},
   MRCLASS = {08-02 (06-02 08A40 08B05 08B10 08B26)},
  MRNUMBER = {2839398 (2012k:08001)},
MRREVIEWER = {Konrad P. Pi{\'o}ro},
}
@article {MR0434928,
    AUTHOR = {Taylor, Walter},
     TITLE = {Varieties obeying homotopy laws},
   JOURNAL = {Canad. J. Math.},
  FJOURNAL = {Canadian Journal of Mathematics. Journal Canadien de
              Math\'ematiques},
    VOLUME = {29},
      YEAR = {1977},
    NUMBER = {3},
     PAGES = {498--527},
      ISSN = {0008-414X},
   MRCLASS = {08A25},
  MRNUMBER = {0434928 (55 \#7891)},
MRREVIEWER = {James B. Nation},
}
@BOOK{HM:1988,
    AUTHOR = {Hobby, David and McKenzie, Ralph},
    TITLE = {The structure of finite algebras},
    SERIES = {Contemporary Mathematics},
    VOLUME = {76},
    PUBLISHER = {American Mathematical Society},
    ADDRESS = {Providence, RI},
    YEAR = {1988},
    PAGES = {xii+203},
    ISBN = {0-8218-5073-3},
    MRCLASS = {08A05 (03C05 08-02 08B05)},
    MRNUMBER = {958685 (89m:08001)},
    MRREVIEWER = {Joel Berman},
    note = {Available from:
      \href{http://math.hawaii.edu/~ralph/Classes/619/HobbyMcKenzie-FiniteAlgebras.pdf}{math.hawaii.edu}}
  }
@article {MR0455543,
    AUTHOR = {Jones, Neil D. and Laaser, William T.},
     TITLE = {Complete problems for deterministic polynomial time},
   JOURNAL = {Theoret. Comput. Sci.},
  FJOURNAL = {Theoretical Computer Science},
    VOLUME = {3},
      YEAR = {1976},
    NUMBER = {1},
     PAGES = {105--117 (1977)},
      ISSN = {0304-3975},
   MRCLASS = {68A20},
  MRNUMBER = {0455543},
MRREVIEWER = {Forbes D. Lewis},
       DOI = {10.1016/0304-3975(76)90068-2},
       URL = {http://dx.doi.org/10.1016/0304-3975(76)90068-2},
}
	
@article{Freese:2009,
    AUTHOR = {Freese, Ralph and Valeriote, Matthew A.},
    TITLE = {On the complexity of some {M}altsev conditions},
    JOURNAL = {Internat. J. Algebra Comput.},
    FJOURNAL = {International Journal of Algebra and Computation},
    VOLUME = {19},
    YEAR = {2009},
    NUMBER = {1},
    PAGES = {41--77},
    ISSN = {0218-1967},
    MRCLASS = {08B05 (03C05 08B10 68Q25)},
    MRNUMBER = {2494469 (2010a:08008)},
    MRREVIEWER = {Clifford H. Bergman},
    DOI = {10.1142/S0218196709004956},
    URL = {http://dx.doi.org/10.1142/S0218196709004956}
  }

@ARTICLE{KearnesKiss1999,
    AUTHOR = {Kearnes, Keith A. and Kiss, Emil W.},
     TITLE = {Modularity prevents tails},
   JOURNAL = {Proc. Amer. Math. Soc.},
  FJOURNAL = {Proceedings of the American Mathematical Society},
    VOLUME = {127},
      YEAR = {1999},
    NUMBER = {1},
     PAGES = {11--19},
      ISSN = {0002-9939},
     CODEN = {PAMYAR},
   MRCLASS = {08A05 (08A30 08B10)},
  MRNUMBER = {99m:08003},
MRREVIEWER = {Branimir {\v{S}}e{\v{s}}elja},
}



@article {MR3076179,
    AUTHOR = {Kearnes, Keith A. and Kiss, Emil W.},
     TITLE = {The shape of congruence lattices},
   JOURNAL = {Mem. Amer. Math. Soc.},
  FJOURNAL = {Memoirs of the American Mathematical Society},
    VOLUME = {222},
      YEAR = {2013},
    NUMBER = {1046},
     PAGES = {viii+169},
      ISSN = {0065-9266},
      ISBN = {978-0-8218-8323-5},
   MRCLASS = {08B05 (08B10)},
  MRNUMBER = {3076179},
MRREVIEWER = {James B. Nation},
       DOI = {10.1090/S0065-9266-2012-00667-8},
       URL = {http://dx.doi.org/10.1090/S0065-9266-2012-00667-8},
}
@incollection {MR1404955,
    AUTHOR = {Kearnes, Keith A.},
     TITLE = {Idempotent simple algebras},
 BOOKTITLE = {Logic and algebra ({P}ontignano, 1994)},
    SERIES = {Lecture Notes in Pure and Appl. Math.},
    VOLUME = {180},
     PAGES = {529--572},
 PUBLISHER = {Dekker, New York},
      YEAR = {1996},
   MRCLASS = {08B05 (06F25 08A05 08A30)},
  MRNUMBER = {1404955 (97k:08004)},
MRREVIEWER = {E. W. Kiss},
}
@misc{william_demeo_2016_53936,
  author       = {DeMeo, William and Freese, Ralph},
  title        = {AlgebraFiles v1.0.1},
  month        = May,
  year         = 2016,
  doi          = {10.5281/zenodo.53936},
  url          = {http://dx.doi.org/10.5281/zenodo.53936}
}
@article{FreeseMcKenzie2017,
    AUTHOR = {Freese, Ralph and McKenzie, Ralph},
     TITLE = {Maltsev families of varieties closed under join or {M}altsev
              product},
   JOURNAL = {Algebra Universalis},
  FJOURNAL = {Algebra Universalis},
    VOLUME = {77},
      YEAR = {2017},
    NUMBER = {1},
     PAGES = {29--50},
      ISSN = {0002-5240},
   MRCLASS = {08B05 (08B10 08B25)},
  MRNUMBER = {3602782},
MRREVIEWER = {Joel Berman},
       DOI = {10.1007/s00012-016-0420-1},
       URL = {http://dx.doi.org/10.1007/s00012-016-0420-1},
}
@BOOK{FreeseMcKenzie1987,
    AUTHOR = {Freese, Ralph and McKenzie, Ralph},
     TITLE = {Commutator theory for congruence modular varieties},
    SERIES = {London Mathematical Society Lecture Note Series},
    VOLUME = {125},
 PUBLISHER = {Cambridge University Press, Cambridge},
      YEAR = {1987},
     PAGES = {iv+227},
      ISBN = {0-521-34832-3},
   MRCLASS = {08B10},
  MRNUMBER = {909290 (89c:08006)},
MRREVIEWER = {Sheila Oates-Williams},
      NOTE = 	{Online version available at:
                 {\texttt{http://www.math.hawaii.edu/$\sim$ralph/papers.html}}}
}
@article {MR2333368,
    AUTHOR = {Kearnes, Keith A. and Tschantz, Steven T.},
     TITLE = {Automorphism groups of squares and of free algebras},
   JOURNAL = {Internat. J. Algebra Comput.},
  FJOURNAL = {International Journal of Algebra and Computation},
    VOLUME = {17},
      YEAR = {2007},
    NUMBER = {3},
     PAGES = {461--505},
      ISSN = {0218-1967},
   MRCLASS = {08A35 (08B20 20B25)},
  MRNUMBER = {2333368},
MRREVIEWER = {Giovanni Ferrero},
       DOI = {10.1142/S0218196707003615},
       URL = {http://dx.doi.org/10.1142/S0218196707003615},
}
@article {MR2504025,
    AUTHOR = {Valeriote, Matthew A.},
     TITLE = {A subalgebra intersection property for congruence distributive
              varieties},
   JOURNAL = {Canad. J. Math.},
  FJOURNAL = {Canadian Journal of Mathematics. Journal Canadien de
              Math\'ematiques},
    VOLUME = {61},
      YEAR = {2009},
    NUMBER = {2},
     PAGES = {451--464},
      ISSN = {0008-414X},
     CODEN = {CJMAAB},
   MRCLASS = {08B10 (08A30 08B05)},
  MRNUMBER = {2504025},
MRREVIEWER = {Jarom{\'{\i}}r Duda},
       DOI = {10.4153/CJM-2009-023-2},
       URL = {http://dx.doi.org/10.4153/CJM-2009-023-2},
}
@misc{UACalc,
	Author = {Ralph Freese and Emil Kiss and Matthew Valeriote},
	Date-Added = {2014-11-20 01:52:20 +0000},
	Date-Modified = {2014-11-20 01:52:20 +0000},
	Note = {Available at: {\verb+www.uacalc.org+}},
	Title = {Universal {A}lgebra {C}alculator},
	Year = {2011}
}
@article{Freese2008,
	Author = {Freese, Ralph},
	Date-Added = {2016-08-29 01:31:23 +0000},
	Date-Modified = {2016-08-29 01:32:09 +0000},
	Journal = {Alg. Univ.},
	Pages = {337--343},
	Title = {Computing congruences efficiently},
	Volume = {59},
	Year = {2008}
}	
@article {MR2470585,
    AUTHOR = {Freese, Ralph},
     TITLE = {Computing congruences efficiently},
   JOURNAL = {Algebra Universalis},
  FJOURNAL = {Algebra Universalis},
    VOLUME = {59},
      YEAR = {2008},
    NUMBER = {3-4},
     PAGES = {337--343},
      ISSN = {0002-5240},
   MRCLASS = {08A30 (08A40 68W30 68W40)},
  MRNUMBER = {2470585 (2009j:08003)},
MRREVIEWER = {Clifford H. Bergman},
       DOI = {10.1007/s00012-008-2073-1},
       URL = {http://dx.doi.org/10.1007/s00012-008-2073-1},
}
@incollection {MR1191235,
    AUTHOR = {Szendrei, {\'A}gnes.},
     TITLE = {A survey on strictly simple algebras and minimal varieties},
 BOOKTITLE = {Universal algebra and quasigroup theory ({J}adwisin, 1989)},
    SERIES = {Res. Exp. Math.},
    VOLUME = {19},
     PAGES = {209--239},
 PUBLISHER = {Heldermann, Berlin},
      YEAR = {1992},
   MRCLASS = {08-02 (08A40 08B05)},
  MRNUMBER = {1191235 (93h:08001)},
MRREVIEWER = {Ivan Chajda},
}
@unpublished{Bergman-DeMeo,
    AUTHOR = {Bergman, Clifford and DeMeo, William},
    TITLE = {Universal Algebraic Methods for Constraint Satisfaction Problems:
      with applications to commutative idempotent binars},
    YEAR = {2016},
    NOTE = {unpublished notes; soon to be available online},
    URL = {https://github.com/UniversalAlgebra/algebraic-csp}
}
\end{filecontents*}
%:biblio
%\documentclass[12pt]{amsart}
\documentclass[12pt]{amsart}


%%%%%%% wjd: added these packages vvvvvvvvvvvvvvvvvvvvvvvvv
% PAGE GEOMETRY
% These settings are for letter format
\def\OPTpagesize{8.5in,11in}     % Page size
\def\OPTtopmargin{0.75in}     % Margin at the top of the page
\def\OPTbottommargin{0.75in}  % Margin at the bottom of the page
%% \def\OPTinnermargin{0.5in}    % Margin on the inner side of the page
\def\OPTinnermargin{1in}    % Margin on the inner side of the page
\def\OPTbindingoffset{0.35in} % Extra offset on the inner side
%% \def\OPToutermargin{0.75in}   % Margin on the outer side of the page
\def\OPToutermargin{1in}   % Margin on the outer side of the page
%% \usepackage[papersize={\OPTpagesize},
%%             twoside,
%%             includehead,
%%             top=\OPTtopmargin,
%%             bottom=\OPTbottommargin,
%%             inner=\OPTinnermargin,
%%             outer=\OPToutermargin,
%%             bindingoffset=\OPTbindingoffset]{geometry}
\usepackage{url,amssymb,enumerate,tikz,scalefnt}
\usepackage[normalem]{ulem} % for \sout (strikeout)   wjd: could remove this in final draft
\usepackage[colorlinks=true,urlcolor=blue,linkcolor=blue,citecolor=blue]{hyperref}
\usepackage{algorithm2e}

\newcommand{\mysetminus}{\ensuremath{-}}
%% uncomment the next line if we want to revert to the "set" minus notation
%% \renewcommand{\mysetminus}{\ensuremath{\setminus}}

\usepackage[yyyymmdd,hhmmss]{datetime}
\usepackage{background}
\backgroundsetup{
  position=current page.east,
  angle=-90,
  nodeanchor=east,
  vshift=-1cm,
  hshift=8cm,
  opacity=1,
  scale=1,
  contents={\textcolor{gray!80}{WORK IN PROGRESS.  DO NOT DISTRIBUTE. (compiled on \today\ at \currenttime)}}
}
%%%%%%  (end wjd addition of packages)


\usepackage{pdfcomment}
\usepackage{color}
\usepackage{amsmath}
\usepackage{amsfonts}
\usepackage{amscd}
%% \usepackage{exers}
\usepackage{inputs/rflatexmacs}
\usepackage{inputs/wjdlatexmacs}

\usepackage[mathcal]{euscript}
\usepackage{comment}

\renewcommand{\th}[2]{#1\mathrel{\theta}#2}
\newcommand{\infixrel}[3]{#2\mathrel{#1}#3}


\newtheorem{theorem}{Theorem}
\newtheorem{lemma}[theorem]{Lemma}
\newtheorem{corollary}[theorem]{Corollary}
\newtheorem{prop}[theorem]{Proposition}
\newtheorem{conjecture}[theorem]{Conjecture}
\theoremstyle{definition}
\newtheorem{example}[theorem]{Example}
\newtheorem{fact}[theorem]{Fact}
\newtheorem{remark}{Remark}
\newtheorem*{remarks}{Remarks}
\newtheorem*{rem}{Remark}
\newtheorem{prob}{Problem}

% \title[A test for a difference term]{A polynomial time test for a
% difference term in an idempotent variety}
% \author[DeMeo]{William DeMeo}
% \address[William DeMeo]{
% Department of Mathematics\\
% University of Hawaii\\
% Honolulu, Hawaii\\
% 96822 USA}
% \email[William DeMeo]{demeo@math.hawaii.edu}
% \author[Freese]{Ralph Freese}
% \address[Ralph Freese]{
% Department of Mathematics\\
% University of Hawaii\\
% Honolulu, Hawaii\\
% 96822 USA}
% \email[Ralph Freese]{ralph@math.hawaii.edu}


\title[A Test for a Difference Term]{A Polynomial-Time Test for a
Difference Term in an Idempotent Variety}
\author[W.~DeMeo]{William DeMeo}
\email{williamdemeo@gmail.com}
\urladdr{http://williamdemeo.github.io}
\address{University of Colorado\\Mathematics Dept\\Boulder 80309\\USA}

\author[R.~Freese]{Ralph Freese}
\email{ralph@math.hawaii.edu}
\urladdr{http://www.math.hawaii.edu/~ralph/}
\address{University of Hawaii\\Mathematics Dept\\Honolulu 96822\\USA}
\author[M.~Valeriote]{Matthew Valeriote}
\email{matt@math.mcmaster.ca}
\urladdr{http://ms.mcmaster.ca/~matt/}
\address{McMaster University\\Mathematics Dept\\Hamilton L8S 4K1\\
CAN}

\thanks{This research was supported by the National
Science Foundation under Grant No. 1500235}

\date{\today}

\begin{document}

\maketitle

\begin{abstract}
We consider the following practical question: given a finite
algebra $\alg{A}$ in a
finite language, can we efficiently decide whether the variety
generated by $\alg{A}$
has a difference term?  We answer this question (positively) in the
idempotent case and then describe algorithms for constructing difference
terms.
\end{abstract}

\section{Introduction}
\label{sec:introduction}

A \defn{difference term} for a variety $\sV$ is a ternary term $d$ in the
language of $\sV$ that satisfies the following:
if $\alg{A} = \<A, \dots \> \in \sV$, then for all $a, b \in A$ we have
\begin{equation}
\label{eq:3}
d^{\alg{A}}(a,a,b) = b \quad \text{ and } \quad
d^{\alg{A}}(a,b,b) \mathrel{\comm \theta \theta} a,
\end{equation}
where $\theta$ is any congruence %% of $\alg{A}$
containing $(a,b)$
and $[\cdot, \cdot]$ denotes the \defn{commutator}.
%% (see Section~\ref{sec:defin-notat}).
When the relations in (\ref{eq:3}) hold we call $d^{\alg{A}}$
a \defn{difference term operation} for $\alg{A}$.

Difference terms are studied extensively in the general algebra literature.
(See, for example, \cite{MR1358491,MR1663558,MR3076179,KSW,MR3449235}.)
There are many reasons to study difference terms, but
one obvious reason is because if we know that a variety
has a difference term, this fact allows us to deduce some useful
properties of the algebras inhabiting that variety.
Any variety that has a Mal'tsev term or for which the commutator operation satisfies $[\alpha, \beta] = \alpha \meet \beta$ has a difference term.
%Roughly speaking, having a difference term is slightly stronger than having
%a Taylor term and slightly weaker than having a \malcev term.
(Note that if
$\alg{A}$ is an \defn{abelian} algebra, which means
that $[1_A, 1_A] = 0_A$, then, by
the monotonicity of the commutator,
$[\theta, \theta] = 0_A$ for all $\theta \in \Con \alg{A}$,
in which case $\textbf{A}$
(\ref{eq:3}) says that $d^{\alg{A}}$ is a \malcev term operation.)

Difference terms also play a role in recent work of Keith Kearnes,
Agnes Szendrei, and Ross Willard.
In~\cite{MR3449235} these authors give a positive answer
J\'onsson's famous question---whether a variety of finite
residual bound must be finitely
axiomatizable---for the
special case in which the variety has a difference
term.\footnote{To say a variety has \emph{finite residual bound} is to say
  there is a finite bound on the size of the subdirectly irreducible
  members of the variety.}

Computers have become invaluable as a research tool and have helped to
broaden and deepen our understanding of algebraic structures and the
varieties they inhabit.  This is largely due to the efforts
of researchers who, over the last three decades, have found ingenious
ways to coax computers into solving challenging abstract algebraic
decision problems, and to do so very quickly.
To give a couple of examples related to our own work,
it is proved in~\cite{MR3239624} (respectively,~\cite{Freese:2009})
that deciding whether a finite idempotent algebra generates a variety that is congruence-$n$-permutable
(respectively, congruence-modular) is \emph{tractable}.\footnote{To
  say that the decision problem is \emph{tractable} is to say
  that there exists an algorithm for solvng the problem that ``scales
  well'' with respect to increasing input size, by which we mean that
  the number of operations required to reach a correct decision is
  bounded by a polynomial function of the input size.}
The present paper continues this effort by presenting an efficient
algorithm for deciding whether a locally finite idempotent variety has a
difference term.

The question that motivated us to begin this project, and
whose solution is the main subject of this paper, is the following:
\begin{prob}
  \label{prob:1}
  Is there a polynomial-time algorithm to decide for a finite,
  idempotent algebra $\alg{A}$ if $\bbV(\alg{A})$ has a difference term?
\end{prob}

We note that for arbitrary finite algebras $\alg{A}$, the problem of deciding if $\bbV(\alg{A})$ has a difference term is an EXP-time complete problem.  This follows from Theorem 9.2 of \cite{Freese:2009}.

The remainder of this introduction uses the language of \emph{tame congruence theory} \tct.  Many of the terms we use are defined and explained
in the next section.  For others, see~\cite{HM:1988}.

Our solution to Problem~\ref{prob:1} exploits the connection
between difference terms and \tct that was established
by Keith Kearnes in~\cite{MR1358491}.
%The following theorem makes this connection precise.

\begin{theorem}[{\protect\cite[Theorem 1.1]{MR1358491}}]
\label{thm:KearnesThm}
The variety $\sV = \bbV(\alg A)$ generated by a
finite algebra $\alg A$ has a difference  if and only if
$\sV$ omits \tct-type \utyp and, for all finite algebras
$\alg B \in \sV$,
the minimal sets of every type~\atyp\ prime interval in
$\op{Con}(\alg B)$ have empty tails.
\end{theorem}

%%% NOTATION FOR TCT TYPES
% \newcommand{\otyp}{\textbf{0}}
% \newcommand{\utyp}{\textbf{1}}
% \newcommand{\atyp}{\textbf{2}}
% \newcommand{\btyp}{\textbf{3}}
% \newcommand{\ltyp}{\textbf{4}}
% \newcommand{\styp}{\textbf{5}}
% \newcommand{\ityp}{\textbf{i}}
% \newcommand{\jtyp}{\textbf{j}}


It follows from an observation of Bulatov that the problem of deciding if a finite idempotent algebra generates a variety that omits \tct-type \utyp is tractable (see Proposition 3.1 of \cite{MR2504025}).
In~\cite{Freese:2009}, the second and third authors solve an
analogous problem by giving a positive answer to the following:
\begin{prob}
  \label{prob:2}
  Is there a polynomial-time algorithm to decide for a finite,
  idempotent algebra $\alg{A}$ if $\bbV(\alg{A})$ is congruence modular?
\end{prob}

Congruence modularity is characterized by omitting tails and
\tct-types \utyp and \styp.
Omitting \utyp's and \styp's can be decided by the subtype theorem.
The second and third authors also prove in~\cite{Freese:2009} that
if there is a
nonempty
tail in $\bbV(\alg{A})$, then there is a
nonempty
tail ``near the bottom.''
More precisely, suppose $\alg{A}$ is a finite idempotent algebra, and suppose
$\bbV(\alg{A})$ has nonempty tails but lacks \utyp's and \styp's.
Then a nonempty tail must occur in a 3-generated subalgebra of $\alg{A}^2$.
The authors use this to prove that congruence modularity is polynomial-time decidable.

However, proving lack of tails uses the fact that a variety omitting
\utyp's and \styp's has a congruence lattice that---modulo
the {\it solvability congruence} (defined below)---is (join) semidistributive.
Now, restricting to just testing whether $\bbV(\alg{A})$ omits
type-\atyp\ tails is not a problem. So, for example, there is a
polynomial-time algorithm for testing if
$\bbV(\alg{A})$ omits \utyp's, \styp's, and type-\atyp\ tails.

%Here is a related problem.
%\wjd{deleted the related problem; it no longer seems relevant.}
%
%\begin{comment}
%\begin{prob}
%  \label{prob:3}
%  Is there an $\alg{A}$, idempotent and having a Taylor term,
%  no type-\atyp tail in
%  subalgebras of $\alg{A}^k$, for $k < n$, but having a type-\atyp
%  tail in a subalgebra of $\alg{A}^n$.
%\end{prob}
%Perhaps we could construct such an algebra using congruence lattice
%representation techniques.
%\end{comment}

%% Hobby and McKenzie give some info about the types in a $D_2$
%% embedded in $\Con (\alg{A})$. (See~\cite[Lemma 6.3]{HM:1988}).
%% Exercise 7 of that section considers 4-element
%% algebras whose congruence lattice is the concrete embedding of $D_2$
%% in $\Eq(4)$; the one with coatoms $01|23$, $02|13$, and $0|123$, and with atoms
%% $0|1|23$ and $0|2|13$. By~\cite[Lemma 6.3]{HM:1988}, the middle-top
%% interval must be type 5 (assuming a Taylor term). But all the others can be 5's,
%% or all the others can be 4's, or all the others can be 3's.
%% One might attempt to find an example where they are all 2's, but that not possible
%% since otherwise $0|123$ would be a solvable congruence,
%% which would imply the two atoms would permute.

%% \draftbreak

\section{Background, definitions, and notation}
\label{sec:defin-notat}
Our starting point is the set of lemmas at the beginning of Section 3 of~\cite{Freese:2009}.
We first review some of the basic \ac{tct}
that comes up in the proofs in that paper. (In fact, most of this section
is lifted directly from~\cite[Section~2]{Freese:2009}.)

The seminal reference for \tct is the book by Hobby and McKenzie
\cite{HM:1988}, according to which,
for each covering $\alpha \prec \beta$ in the congruence lattice of a finite
algebra $\alg{A}$, the local behavior of the $\beta$-classes is captured by the
so-called $(\alpha, \beta)$-traces~\cite[Def.~2.15]{HM:1988}.
Modulo $\alpha$, the induced structure on the traces is limited to one
of five possible types:

\begin{enumerate}[{\bf 1}]
\item  (unary type) an algebra whose basic operations are permutations;
\item  (affine type) a one-dimensional vector space over some finite field;
\item  (boolean type) a 2-element boolean algebra;
\item  (lattice type) a 2-element lattice;
\item  (semilattice type) a 2-element semilattice.
\end{enumerate}

Thus to each covering $\alpha \prec \beta$
corresponds a ``\tct type,'' denoted by $\typ(\alpha, \beta)$,
belonging to the set
$\{\mathbf{1},\mathbf{2},\mathbf{3},\mathbf{4},\mathbf{5}\}$
(see~\cite[Def.~5.1]{HM:1988}).
The set of all \tct types that are realized by covering pairs of congruences of a
finite algebra $\alg{A}$ is denoted by $\typ\{\alg{A}\}$
and called the \emph{typeset} of $\alg{A}$.
If $\sK$ is a class of algebras, then $\typ\{\sK\}$ denotes the union of the typesets of all finite algebras in $\sK$.
\tct types are ordered according to the following ``lattice of types:''

\newcommand{\dotsize}{0.8pt}
%% To create nodes of lattices in a uniform and consistent way, we define
\tikzstyle{lat} = [circle,draw,inner sep=\dotsize]
% To scale all diagrams uniformly, change this setting:
\begin{center}
\newcommand{\figscale}{.7}
\begin{tikzpicture}[scale=\figscale]
  \scalefont{.8}
  \node[lat] (1) at (0,0) {};
  \node[lat] (2) at (-1,1.5) {};
  \node[lat] (3) at (0,3) {};
  \node[lat] (4) at (.8,2.1) {};
  \node[lat] (5) at (.8,.9) {};
  \draw (1) node [below] {$1$};
  \draw (2) node [left] {$2$};
  \draw (3) node [above] {$3$};
  \draw (4) node [right] {$4$};
  \draw (5) node [right] {$5$};
  \draw[semithick]
  (1) -- (2) -- (3) -- (4) -- (5) -- (1);
\end{tikzpicture}
\end{center}
Whether or not $\bbV(\alg{A})$ omits one of the order ideals of the lattice of types can be
determined locally.  This is spelled out for us in the next proposition.
(A \defn{strictly simple} algebra is a simple
algebra with no non-trivial subalgebras.)
%% ; i.e.~no proper subalgebras with
%% more than one element.)


\begin{prop}[{\protect \cite[Proposition~2.1]{Freese:2009}}]
  \label{prop:2.1}
If $\alg A$ is a finite idempotent algebra and
$\mathbf{i} \in \typ(\bbV(\alg{A}))$ then there
is a finite strictly simple algebra $\bS$ of
type~$\mathbf{j}$ for
some $\mathbf{j} \leq \mathbf{i}$ in $\sansH \sansS (\alg{A})$.
The possible cases are
% \begin{enumerate} %[(1)]
\begin{itemize}
\item[$\bullet\  \mathbf{j} = 1$] $\;\Rightarrow \;\bS$ is term equivalent to a 2-element set
\item[$\bullet\   \mathbf{j} = 2$] $\;\Rightarrow \;\bS$ is term equivalent to the idempotent reduct of a module
\item[$\bullet\   \mathbf{j} = 3$] $\;\Rightarrow \;\bS$ is functionally complete
\item[$\bullet\   \mathbf{j} = 4$] $\;\Rightarrow \;\bS$ is polynomially equivalent to a 2-element lattice
\item[$\bullet\   \mathbf{j} = 5$] $\;\Rightarrow \;\bS$ is term equivalent to a 2-element semilattice.
\end{itemize} %enumerate}
\end{prop}
\begin{proof}
  This is a combination of~\cite[Proposition~3.1]{MR2504025} and~\cite[Theorem~6.1]{MR1191235}.
\end{proof}

\begin{comment}
  Table~\ref{tab:1} is from~\cite{MR3350327} and gives another characterization of
omitting types.
\begin{center}
  \begin{table}
    \caption{\cite{MR3350327}.}
    \label{tab:1}
    \begin{tabular}{|l|l|}
      \hline
      Omitting Class &  Equivalent Property\\
      \hline
      $\sM_{\{1\}}$ & satisfies a nontrivial idempotent \malcev condition \\
      \hline
      $\sM_{\{1,5\}}$ & satisfies a nontrivial congruence identity\\ % (see~\cite{MR3076179})\\
      \hline
      $\sM_{\{1,4,5\}}$ & congruence n-permutable, for some $n > 1$ \\
      \hline
      $\sM_{\{1,2\}}$ & congruence meet semidistributive \\
      \hline
      $\sM_{\{1,2,5\}}$ & congruence join semidistributive\\ % (see~\cite{MR3076179})\\
      \hline
      $\sM_{\{1,2,4,5\}}$ & congruence $n$-permutable for some $n$ and\\
      &congruence join semidistributive\\
      \hline
    \end{tabular}
  \end{table}
\end{center}

\end{comment}



We conclude this section with a result that will be useful in Section~\ref{sec:freese-valer-lemm}.
\rsf{did not see how FV 09 lemma 3.3 was used; eliminating it}
\begin{corollary}[{\protect \cite[Corollary~2.2]{Freese:2009}}]
  \label{cor:2.2}
  Let $\alg{A}$ be a finite idempotent algebra and $T$ an order ideal in the
  lattice of types. Then $\bbV(\alg{A})$ omits $T$ if and only if $\sansS(\alg{A})$ does.
  %% In particular, $\bbV(\alg{A})$ omits 1 and 2 if and only if $\sansS(\alg{A})$ omits 1 and 2.
\end{corollary}



%% \draftsecskip
%%%%%%%%%%%%%%%%%%%%%%%%%%%%%%%%%%%%%%%%%%%%%%%%%%%%%%%%%%%%%%%%%%%%%%%%%
%% \draftbreak

\section{The Characterization}
\label{sec:freese-valer-lemm}
In~\cite{Freese:2009}, Corollary~\ref{cor:2.2} is the starting point of the
development of a polynomial-time algorithm that determines if a given finite
idempotent algebra generates a congruence modular variety.

%% The following lemma ties in with the previous proposition and will be used
%% in Sec. 6.
%% \begin{lemma}[Lemma 2.3~\cite{Freese:2009}]
%%   Let $\alg{A}$ be a finite idempotent algebra and let $\bS \in \sansH \sansS(\alg{A})$
%%   be strictly simple. Then there are elements $a, b \in A$ such that, if
%%   $\alg{B} = \Sg^{\alg{A}} (a, b)$, then $1_B = \Cg^{\alg{B}} (a, b)$ and is join irreducible
%%   with unique lower cover $\rho$ such that $\bS = \alg{B}/\rho$.
%% \end{lemma}
%% \begin{proof}
%%   Choose $\alg{B} \in \sansS (\alg{A})$ as small as possible having $\bS$ as a homomorphic image,
%%   say $\bS = \alg{B}/\rho$. We claim that if $a, b \in B$ with
%%   $(a, b) \in \notin \rho$ then they generate $\alg{B}$. To
%%   see this, let $\alg{B}'= \Sg^{\alg{B}} (a, b)$ and let $h$ be the quotient map from B to S with kernel
%%   ρ. Then h(B  ) is a non-trivial subuniverse of S and so must equal S. Thus B  = B.
%%   Now let a, b ∈ B with (a, b) ∈
%%   / ρ. Since the block of Cg B (a, b) containing a
%%   and b is a subuniverse of B then from the previous paragraph, we conclude that
%%   Cg B (a, b) = 1 B and that ρ is its unique lower cover.
%% \end{proof}

According to the characterization
in~\cite[Chapter~8]{HM:1988} of locally finite congruence modular (resp.,
distributive) varieties, a finite algebra $\alg{A}$ generates a congruence modular
(resp., distributive) variety $\sV$ if and only if the typeset
of $\sV$ is
contained in $\{\atyp, \btyp, \ltyp\}$ (resp., $\{\btyp, \ltyp\}$)
and all minimal sets of prime
quotients of finite algebras in $\sV$ have empty
tails~\cite[Def.~2.15]{HM:1988}. (In the distributive
case the empty tails condition is equivalent to the minimal sets all having exactly
two elements.)

It follows from Corollary~\ref{cor:2.2} and Proposition~\ref{prop:2.1}
that if $\alg{A}$ is idempotent then one can
test the first condition---omitting
types \utyp, \styp (resp., \utyp, \atyp, \styp)---by searching
for a 2-generated subalgebra of $\alg{A}$ whose typeset is
not contained in
$\{\atyp, \btyp, \ltyp\}$ (resp., $\{\btyp, \ltyp\}$). It is proved
in~\cite[Section~6]{Freese:2009} that this
test can be performed in polynomial-time---that is, the running
time of the test is bounded by a polynomial function of the size of $\alg{A}$.
The main tools developed to this end are presented
in~\cite[Section~3]{Freese:2009} as a sequence of
lemmas that enable the authors to prove the following:
if $\alg{A}$ is finite and idempotent, and if
$\mathcal V = \bbV(\alg{A})$ omits types \utyp and \styp,
then to test for the existence of nonempty tails
in $\sV$ it suffices to look for them
in the 3-generated subalgebras of $\alg{A}^2$.
%% More specifically, the authors assume that the type set of $\bbV(\alg{A})$ contains no 1's
%% and no 5's, and under this
%% assumption they prove that nonempty tails either do not occur in $\bbV(\alg{A})$,
%% or they occur in 3-generated subalgebras of $\alg{A}^2$.
In other words, either there are no nonempty tails
or else there are nonempty tails that are easy to find
(since they occur in a 3-generated subalgebra of $\alg{A}^2$).
It follows that Problem~\ref{prob:2} has a positive answer:
deciding whether or not a finite idempotent algebra generates a congruence
modular variety is tractable.%
%\footnote{That is, there are positive integers
%$C, n$, and an algorithm that takes
%a finite idempotent algebra $\alg{A}$ as input and decides
%in at most $C|\alg{A}|^n$ steps whether $\bbV(\alg{A})$ is congruence modular.
%Here $|\alg{A}|$ denotes the number of bits required to encode
%the algebra $\alg{A}$.}
% polynomial-time algorithm to decide,
%for a finite idempotent algebra $\alg{A}$,
% whether $\bbV(\alg{A})$ is congruence modular.}

Our goal is to use the same strategy to solve Problem~\ref{prob:1}.
As such, we revisit each of the lemmas in Section 3 of \cite{Freese:2009},
and consider whether an analogous result can be proved under
modified hypotheses.
Specifically, we retain the assumption that the type set of $\bbV(\alg{A})$
omits \utyp, but we drop the assumption that it omits \styp.
We will  prove that, under these circumstances, % MAV: dropped  "we will attempt to prove"
either there are no type-\atyp tails in $\bbV(\alg{A})$ (so the latter has a difference term),
or else type-\atyp tails can be found ``quickly,''
(e.g., in a 3-generated subalgebra of $\alg{A}^2$).
Where possible, we will relate our new results to
analogous results in~\cite{Freese:2009}.

\subsection{Notation}
Throughout we let $\nn$ denote the set $\{0,1,\dots, n-1\}$ and
we take $\sS$ to be a finite set of finite,
similar, idempotent algebras that is closed under the taking of
subalgebras, and we assume that the type set of
$\sV = \bbV(\sS)$ omits \utyp (but may include \styp).
If there exists a finite algebra in $\sV$ having a type-\atyp\ minimal
set with a nonempty tail---in which case we say that
``$\sV$ has type-\atyp tails''---then,
by standard results in \tct (see~\cite{HM:1988}),
at least one such algebra appears as a subalgebra of a product of
elements in $\sS$.
So we suppose that some finite algebra
$\alg{B}$ in $\sV$ has a prime quotient of type~\atyp with
minimal sets that have
nonempty tails and show that there is a 3-generated
subalgebra of the
product of two members of $\sS$ with this property (which can be found ``quickly'').

Since $\sS$ is closed under the taking of subalgebras,
we may assume that the algebra $\alg{B}$ from the previous paragraph is a subdirect
product of a finite number of members of $\sS$. Choose $n$ minimal such that for
some $\alg{A}_0$, $\alg{A}_1$, $\dots$, $\alg{A}_{n-1}$ in $\sS$, there is a subdirect
product $\alg{B} \sdp \prod_{\nn} \alg{A}_i$
that has a prime quotient of type~\atyp\ whose minimal sets have
nonempty tails.
Under the assumption that $n > 1$ we will prove that $n = 2$.

For this $n$, select the $\alg{A}_i$ and $\alg{B}$ so that $|B|$ is as small as possible.
Let $\alpha \prec \beta$ be a prime quotient of $\alg{B}$
of type~\atyp\ whose minimal sets have
nonempty tails, and choose $\beta$ minimal with respect to this property.
By~\cite[Lemma 6.2]{HM:1988}, this implies $\beta$ is join
irreducible and $\alpha$ is its unique subcover.
Let $U$ be an $\<\alpha, \beta\>$-minimal set.

For $i \leq n$, we let $\rho_i$ denote the kernel of the projection
of $\alg{B}$ onto $\alg{A}_i$, so $\alg{B}/\rho_i \cong \alg{A}_i$.
For a subset $\sigma \subseteq \nn$, define
\[
\rho_\sigma := \bigwedge_{j\in \sigma} \rho_j.
\]
Consequently,
$\rho_{\nn} = \bigwedge_{j\in \nn}\rho_j = 0_{B}$.
%% \marginnote{wjd: I don't see why join in (3.1) is $1_B$... it's probably wrong.}[3cm]
%% \begin{equation}
%%   \label{eq:2}
%%   \rho_{\nn} = \bigwedge_{j\in \nn}\rho_j = 0_{B} \quad \text{ and } \quad
%%   \bigvee_{j\in \nn}\rho_j =1_B. %% \qquad
%% \end{equation}
By minimality of $n$ we know that the intersection of any  proper subset of the
$\rho_i$, $1 \leq i \leq n$ is strictly above $0_B$.  Thus,
$0_B < \rho_\sigma < 1_B$ for all
$\emptyset \subset \sigma\subset \nn$.
(By $\subset$ we mean \emph{proper} subset.)

The next four lemmas assume the context above, which for convenience
we will denote by $\Gamma$; that is,
``\emph{Assume $\Gamma$}'' will mean ``Assume $\sS$, $\alg{B}$, $n$, $\{\alg{A}_i \colon i < n\}$,
$\alpha$, $\beta$, $U$, and $\rho_\sigma$ are as described above.''


\begin{lemma}[{\protect cf.~\cite[Lemma~3.1]{Freese:2009}}]
\label{lem:fv_3-1}
Assume $\Gamma$. If\/ $0$, $1 \in U$, if $(0,1) \in \beta \mysetminus \alpha$, and if
$t$ belongs to the tail of $U$, then $\beta$ is the congruence of $\alg{B}$
generated by the pair $(0,1)$, and $\alg{B}$ is generated by $\{0, 1, t\}$.
\end{lemma}
%and let $N$ be an
% $(\alpha, \beta)$-trace of $U$. Let 0 and 1 be
%two distinct members of $N$ with $(0, 1) \notin \alpha$.
This follows from the same argument used to prove~\cite[Lemma~3.1]{Freese:2009}, using the fact that since $\beta$ is abelian over $\alpha$ it follows that $\gamma$ over $\delta$ that appears in the proof will also have type 2.


\begin{comment}
\begin{proof}
Since $\beta$ is join irreducible with unique subcover $\alpha$, any
pair of elements in $\beta \mysetminus \alpha$ generates $\beta$.
Let $\alg C$ be the subalgebra of $\alg B$ generated by $\{0,1,t\}$.
We will obtain a contradiction under the assumption that $|C| < |B|$.
%% (wjd: removed the following line since it follows from minimality)
%% and the minimal sets of $\alg C$ which have type~\atyp\ all have empty tails.

Let $\beta'$ and $\alpha'$ be the restrictions of $\beta$ and $\alpha $ to $C$,
respectively. Then $\alpha' < \beta'$ since $(0,1) \in
\beta'\mysetminus \alpha'$ and so there are $\delta \covs \gamma$ in
$\con{\alg C}$ with $\alpha' \le \delta \cov \gamma \le \beta'$ and
such that $(0,1) \in \gamma\mysetminus \delta$.
Since $\<\alpha,\beta\>$ is type~\atyp, $\beta$ is abelian over $\alpha$.
This implies $\beta'$ is abelian over $\alpha'$
by \cite[Lemma 2.19(9)]{MR3076179},
which implies the
types of the prime quotients occurring between $\alpha'$ and $\beta'$
are~\utyp\ or \atyp.
But since we are assuming % $\alg B$ has a Taylor term,
that $\sV$ omits type~\utyp  and $\alg B \in \sV$ then
they are all of type~\atyp.
In particular, $\<\delta,\gamma\>$ has type~\atyp.

Now, if $|C| < |B|$, then all
% Suppose that $|C| < |B|$ and all $\langle \delta, \gamma \rangle$
$\langle \delta, \gamma \rangle$-minimal sets have empty tails
(since $\alg{B}$ is minimal among algebras with nonempty
type-\atyp tails).
Let $V$ be a $\langle \delta,\gamma \rangle$-minimal set and let
$p(x)$ be some polynomial of $\alg C$ with range $V$ and with
$(p(0) ,p(1))\notin \delta$. Such a polynomial exists
by~\cite[Theorem~2.8]{HM:1988},
since $(0,1) \in \gamma\mysetminus \delta$.

The polynomial $p(x)$ can be expressed in the form
$s^{\alg C}(x,0,1,t)$ for some term $s(x,y,z,w)$
of $\mathcal V$, so $p(x)$ extends to a polynomial
$p'(x) = s^{\alg B}(x,0,1,t)$ of $\alg B$.  Since $(p(0),
p(1)) \in \gamma\mysetminus \delta$, then
$(p'(0), p'(1)) \in \beta\mysetminus \alpha$, so $p'$ must map the
minimal set $U$ onto a polynomially isomorphic set $W$. In particular,
%, since $p'$ is a polynomial isomorphism from $U$ to~$W$.
$\{p'(0), p'(1)\} \subseteq \body (W)$ and $p'(t) \in \tail (W)$.

Since the type of $\langle \delta, \gamma \rangle$ is \atyp\
and $V$ has no tail,
$\alg C|_V$ has a \malcev polynomial. % $s(x,y,z)$.
This polynomial has an extension to a polynomial of $\alg B$ and, since
$\{p(0), p(1),p(t)\} \subseteq V$, it follows that there is a polynomial
$f(x,y,z)$ of $\alg B$ that satisfies the \malcev identities when
restricted to the set $\{p'(0), p'(1), p'(t)\} \subseteq W$. This
contradicts~\cite[Lemma~4.26]{HM:1988}, since $p'(0)$ and $p'(1)$
are in the body of $W$ and $p'(t)$ is in the tail.
\end{proof}
\end{comment}


\begin{lemma}[{\protect cf.~\cite[Lemma~3.2]{Freese:2009}}]
  \label{lem:fv_3-2}
Assume $\Gamma$. For every proper nonempty subset $\sigma$ of $\nn$,
  either $\beta \leq \rho_\sigma$ or $\alpha \join \rho_\sigma = 1_B$.
\end{lemma}
  This follows from the proof ~\cite[Lemma~3.2]{Freese:2009}.

\begin{comment}
\begin{proof}
Let $\rho = \rho_\sigma$.
Suppose that $\beta \not\le {\rho}$ (or equivalently $(0,1) \notin
\rho$). Since $\beta$ is join irreducible, $\beta\meet\rho \le
\alpha$ and so $\beta\meet \rho = \alpha \meet \rho$.  Furthermore,
$\alpha\join {\rho} = \beta \join {\rho}$, or else we can find a
prime quotient between these two congruences that is perspective
with $\langle \alpha, \beta \rangle$.  But then the algebra
$\alg B/{\rho}$ has a prime quotient of type~\atyp\ whose minimal sets have nonempty
tails.  Since this algebra is isomorphic to a subdirect product of
fewer than $n$ members of $\mathcal S$, we conclude, by the minimality
of~$n$, that indeed $\alpha\join {\rho} = \beta \join {\rho}$.

Thus the set
\[
\mathcal P = \{\beta\meet\rho, \rho, \alpha, \beta, \alpha\join\rho\}
\]
forms a pentagon in $\Con \alg B$. Let $C$ be the
$(\alpha\join\rho)$-class that contains $0$ and let $M = C\mathrel{\cap} U$.
Note that $C$ contains $1$ and, since $\alg B$ is idempotent,  that
$C$ is a subuniverse of $\alg B$. By \cite[Lemma 2.4]{HM:1988}, we
conclude that the restriction to $M$ is a surjective lattice
homomorphism from the interval $I[0_B,
\alpha\join\rho]$ in $\Con{\alg B}$ to the interval $I[0_M,
(\alpha\join\rho)|_M]$ in $\Con{\alg B}|_M$.  Note that since $(0,1) \in
\beta|_M \mysetminus \alpha|_M$, this restriction map separates
$\alpha$ and $\beta$.  Then, the image under the restriction map of
the pentagon $\mathcal P$ is a pentagon in $\Con{\alg B}|_M$.  This
implies that $M$ contains some elements of the tail of $U$, since
otherwise $\Con{\alg B}|_M$ has a \malcev term operation and hence
is modular.
Thus, there is some $t$ in the tail of $U$ with $(0,t) \in
\alpha\join \rho$. Using Lemma~\ref{lem:fv_3-1} we conclude that $C =
B$ since it contains $\{0,1,t\}$.  Thus, $\alpha \join \rho = 1_B$.
\end{proof}

\end{comment}

\begin{lemma}\label{lem:nearperm}
Assume $\Gamma$.   For every proper nonempty subset $\sigma$ of $\nn$,
  for all $v\in B$, and for all $b\in \body(U)$, we have
  $(v,b) \in \beta \circ \rho_\sigma \cap \rho_\sigma \circ \beta$.
\end{lemma}

%\smallskip
%% \newcommand\rhosig{\ensuremath{\rho_\sigma}}
\newcommand\rhosig{\ensuremath{\rho}}
\begin{proof}
\noindent Let $\rho = \rho_\sigma$. Note that
$\beta \join \rhosig = 1_B$ implies
$\restr{\beta}{U} \join \restr{\rhosig}{U} = \restr{1_B}{U} = 1_U$,
since $U = e(B)$, for some idempotent unary polynomial~$e$.
Now, for all $x$, $y \in U$, if $x\in \body(U)$ and $y\in \tail(U)$, then
$(x,y) \notin \beta$.  Therefore,
$(x, y) \in  1_U = \restr{\beta}{U} \join \restr{\rhosig}{U}$ implies
there must be some $b' \in \body(U)$ and $t'\in \tail(U)$ such that
$b' \mathrel{\rhosig} t'$.

Now, let $d(x,y,z)$ be a pseudo-\malcev polynomial for $U$,
which exists by~\cite[Lemma~4.20]{HM:1988}.
Thus,
\begin{itemize}
\item $d(B,B,B) = U$
\item $d(x,x,x) = x$ for all $x\in U$
\item $d(x,x,y) = y = d(y,x,x)$ for all $x\in \body(U)$, $y \in U$.
\end{itemize}
Moreover, for all $c$, $d \in \body(U)$, the unary polynomials
$d(x,c,d)$, $d(c,x,d)$, and $d(c,d,x)$ are permutations on $U$.
If we now fix an arbitrary element $b\in \body(U)$ and
let $p(x) = d(x,b',b)$, then (see~\cite[Lemma~4.20]{HM:1988})
\begin{itemize}
\item  $p(U) = U$, since $U$ is minimal,
\item $p(b') = d(b',b',b) = b \in \body(U)$, and
\item  $t:=p(t')\in \tail(U)$, since $t'\in \tail(U)$.
\end{itemize}
Since $(b',t') \in \rhosig$, we have $(b, t) = (p(b'), p(t')) \in \rhosig$.
Since $b$ is in the body, there is an element $b''$ in the body with
$(b,b'') \in \beta - \alpha$. By Lemma~\ref{lem:fv_3-1}, this implies
$\alg{B} = \Sg^{\alg{B}}(\{b, b'', t\})$.

Finally, if $v \in B$, then $v = s^{\alg{B}}(b,b'',t)$ for
some (idempotent) term $s$, so
\[
v = s^{\alg{B}}(b,b'',t)
\mathrel{\rhosig} s^{\alg{B}}(b,b'',b)
\mathrel{\beta} s^{\alg{B}}(b,b,b) = b,
\]
and
\[
v = s^{\alg{B}}(b,b'',t)
\mathrel{\beta}  s^{\alg{B}}(b,b,t)
\mathrel{\rhosig}s^{\alg{B}}(b,b, b)  = b.
\]
Therefore,
$(v,b) \in  \beta \circ \rhosig \cap \rhosig \circ \beta$.
Since $v \in B$ and $b\in \body(U)$ were aribitrary,
this completes the proof.
\end{proof}



\begin{lemma}[{\protect cf.~\cite[Lemma~3.3]{Freese:2009}}]\
  \label{lem:fv_3-3}
  Assume $\Gamma$.
  \begin{enumerate}[(i)]
    \item \label{item:6} There exists $i$
      such that $\alpha \join \rho_i = 1_B$
    \item \label{item:7} There exists $i$ such that
      $\alpha \join \rho_i < 1_B$.
  \end{enumerate}
\end{lemma}
\begin{proof}
  %% {\bf TODO:} fill in proof of  Lemma~\ref{lem:fv_3-3}.
  %%\begin{enumerate}[(i)]
  %%\item %\label{item:6} There exists $0\leq i< n $  such that $\alpha \join \rho_i = 1_B$
If item \eqref{item:6} failed, then by Lemma~\ref{lem:fv_3-2} we would
have $\beta \leq \rho_i$ for all $i$, and that
would imply $\beta = 0_B$.
  %%\item %\label{item:7} There exists $i$ such that $\alpha \join \rho_i < 1_B$.

To see \eqref{item:7}, assume
\begin{equation} \label{eq:4}
\alpha \join \rho_i = 1_B \; \text{ for all $i$.}
\end{equation}
Take a nonempty proper subset $\sigma \subset \nn$ of indices and let
$\rho_\sigma = \bigwedge_{j\in \sigma} \rho_j$.
Then $\beta \join \rho_\sigma = 1_B$. (Otherwise,
$\alpha \join \rho_\sigma \leq \beta \join \rho_\sigma < 1_B$,
and it would follow from Lemma~\ref{lem:fv_3-2} that
$\alpha \leq \beta \leq \rho_\sigma \leq \rho_i$ for all $i \in \sigma$,
but then $\alpha \join \rho_i = \rho_i < 1_B$, contradicting (\ref{eq:4}).)
% Therefore, $\beta \join \rho_\sigma = 1_B$.

%\smallskip


%We conclude from the foregoing that,
%for all nonempty proper subsets $\sigma \subset n$,
%for all $a\in \body(U)$ and all $v\in B$, we have
%$(a, v) \in \beta \circ \rho_\sigma \cap \rho_\sigma \circ \beta$,
%proving the subclaim.

Let $b \in \body(U)$ and $t \in \tail(U)$, and let $d(x,y,z)$
denote the pseudo-\malcev operation introduced in the proof of Lemma~\ref{lem:nearperm}. By
\cite[Lemma~4.25]{HM:1988},  $(b, d(b,t,t)) \notin \beta$.
We will arrive at a contradiction by showing that
$b = d(b,t,t)$. By Lemma~\ref{lem:nearperm}, for every $i \in \nn$,
$(b,t) \in \beta \circ \rho_i$ so
there is an element $a \in B$ satisfying
$b\mathrel\beta a \mathrel\rho_i t$. By applying the idempotent
polynomial $e$ with $e(U) = U$, we have
$b\mathrel\beta e(a) \mathrel\rho_i t$, so we may
assume $a \in U$. But this puts $a \in \body(U)$,
since $a\mathrel \beta b$. Therefore,
\[
d(b,t,t) \mathrel\rho_i d(b,a,a) = b.
\]
Since this hold for every~$i$, $d(b,t,t) = b$.
\end{proof}


%\begin{lemma}[{\protect cf.~\cite[Lem.~3.3]{Freese:2009}}]
%\label{lem:fv_3-3x}
%There is exactly one~$i$ such that $\beta \not\le \rho_i$.
%\end{lemma}
%\begin{proof}
%If $n = 1$ then $\rho_1 = 0$ so the results holds. The lemma
%above shows there must there must be at least one $\rho_i \ge \beta$
%so let $\rho_1 \not\ge \beta$ and $\rho_2 \ge \beta$. If $n = 2$,
%the result holds. Otherwise let $\rho = \rho_1 \meet \rho_2$ and
%note it is clearly not above $\beta$. Since $\beta$ is
%join irreducible and $\alpha$ is its unique lower cover,
%$\rho \meet \beta \le \alpha$. But then $\alg B/\rho$ has a type~2
%tail, contradicting the minimality of~$n$.
%\end{proof}


\begin{theorem}[{\protect cf.~\cite[Theorem 3.4]{Freese:2009}}]\label{thm:fv_3-4}
Let $\sV$ be the variety generated by some finite set $\sS$ of finite,
idempotent algebras that is closed under taking subalgebras. If\/ $\sV$
omits type~\utyp\ and some finite member of $\sV$ has a prime quotient
of type~\atyp\
whose minimal sets have nonempty tails, then there is some
3-generated algebra $\alg B$ with this property that belongs to $\sS$ or is a subdirect
product of two algebras from $\sS$.
\end{theorem}
\begin{proof}
Choose $n > 0$, $\alg A_i \in \sS$, for $0 \le i \le n-1$ and $\alg B$
as above. From Lemma \ref{lem:fv_3-1} we know that $\alg B$ is
3-generated. If $n > 1$ then by the previous lemma we can choose $i$
and $j < n$ with $\beta \le \rho_i$ and $\alpha \join \rho_j =
1_B$. If $n > 2$ then Lemma~\ref{lem:fv_3-2} applies to $\rho = \rho_i
\meet \rho_j$ and so we know that either $\beta \le \rho$ or $\alpha
\join \rho = 1_B$. This yields a contradiction as the former is not
possible, since $\beta \not\le \rho_j$ and the latter can't hold
since both $\alpha$ and $\rho$ are below $\rho_i$.

So, the minimality of $n$ forces $n\le 2$ and the result follows.
\end{proof}


%\mav{Improve Ex 9 (see comment below)}
\begin{example}\label{alg:nodiff}
%Let $\alg{A}$ be the algebra with universe $\{0,1,2\}$ that has the ternary idempotent operation $f(x,y,z)$ such that on $\{0,1\}$ $f(x,y,z) = x \oplus y \oplus z$, on $\{0,2\}$, $f(x,y,z) = \min\{x,y,z\}$ and is equal to 0 otherwise.  Then $f$ is a cyclic operation and so $\alg{A}$ generates a variety that omits type~\utyp.
Let $\alg A$ be the (idempotent) algebra with universe $\{0,1,2, 3\}$ with basic operation $x\cdot y$ defined by:
  \[
  \begin{array}{c|cccc}
  \cdot&0&1&2&3\\\hline
  0&0&2&1&3\\
  1&2&1&0&3\\
  2&1&0&2&3\\
  3&3&0&0&3
  \end{array}
  \]
It can be checked that $\alg A$ is a simple algebra of type~\btyp and that no subalgebra of $\alg A$ has a prime quotient of type~\atyp whose minimal sets have nonempty tails.  It can also be checked that the subalgebra of $\alg A^2$ generated by $\{(0,0), (1,0), (0,3)\}$ does have such a prime quotient.
This demonstrates that in general one must look for nonempty tails of minimal sets of type~\atyp in the square of a finite idempotent algebra and not just in the subalgebras of the algebra itself.

We note that, since $\alg A$ is simple of type~\btyp, the ternary projection operation $p(x,y,z ) = z$ is a difference term operation for $\alg A$.  We also note that the term operation $(x\cdot y) \cdot(y \cdot x)$ of $\alg A$ is commutative and so the variety generated by $\alg A$ omits type~\utyp.  So, this example also demonstrates that for finite idempotent algebras, having a difference term operation and generating a variety that omits type~\utyp does not guarantee the existence of a difference term for the variety.
\end{example}
%\mav{Find $\alg{A}$ st $\sV(\alg{A})$ omits 1, has no diff term, but $\alg{A}$ has a diff term op; or show $\exists$ is no such $\alg{A}$!}

\mav{added so that it can be cited later on.}
\begin{corollary}\label{cor:diffterm}
  Let $\alg A$ be a finite idempotent algebra.  Then $\bbV(\alg A)$ has a difference term if and only if
  \begin{itemize}
  \item $\sansH \sansS (\alg{A})$ does not contain an algebra that is term equivalent to the 2-element set and
      \item no 3-generated subalgebra of $\alg A^2$ has a prime quotient of type~\atyp whose minimal sets have nonempty tails.
  \end{itemize}
\end{corollary}

\begin{proof}
  This is just a combination of Proposition~\ref{prop:2.1} and Theorems~\ref{thm:KearnesThm} and~\ref{thm:fv_3-4}.
\end{proof}
%\textcolor{purple}{This is the third version of our theorem presenting
%our algorithm. Write an intro and explain K-K's idea. Define $\alpha_0$, $\alpha_1$.}

The next theorem essentially gives an algorithm to decide if a finitely
generated, idempotent variety has a difference term. In the next
section, we will show that the algorithm runs in polynomial-time.

In \cite{KearnesKiss1999}, Kearnes and Kiss show there is a close connection
between $\alpha \cov \beta$ being the critical interval of
a pentagon and $\la \alpha,\beta\ra$-minimal sets
having nonempty tails.
By \cite[Theorem 2.1]{KearnesKiss1999}, the minimal sets
of a prime critical interval of a pentagon have nonempty tails, provided
the type is not~\utyp. In the other direction, if the
$\la \alpha,\beta\ra$-minimal sets have nonempty tails,
then there is a pentagon in the congruence lattice of a subalgebra of
$\alg A^2$ with a prime critical interval of the same type.
This connection between minimal sets with tails and
pentagons is important for us: we do not have a polynomial time algorithm
for finding an $\la \alpha,\beta\ra$-minimal set.

If $\alg B$ is a subalgebra of $\alg A^2$ and $\theta$ is a congruence
of $\alg A$, then we define $\thetao \in \con (\alg B)$ by
$(x_0,x_1) \mathrel{\thetao} (y_0,y_1)$ iff
$x_0 \mathrel{\theta} y_0$. We define $\theta_1$ similarly.
In case $\theta = 0_{\alg A}$, the least congruence,
we use the notation $\rho_0$ and $\rho_1$ instead of
$0_0$ and $0_1$. Of course $\rho_0$ and $\rho_1$ are the kernels
of the first and second projections of $\alg B$ onto~$\alg A$.



\begin{theorem}\label{thm:algorithm}
Let $\alg A$ be a finite idempotent algebra and let $\sV$ be the variety
it generates. Then $\sV$ has a difference term if and only if the
following conditions hold:
\begin{enumerate}

   % $\alg A$ has a Taylor term.
   \item \label{it:1}$\sV$ omits TCT-type~\utyp.
  \item \label{it:2}
    There do not exist $a$, $b$, $c\in A$
    satisfying the following, where
    % $\alg B$ is the subalgebra of $\alg A$ generated
    % by $a$, $b$ and $c$,  $\beta = \Cga a b B$, and
    $\alg B := \Sg^{\alg A}(a, b, c)$ and
    % $\alg C$ is the subalgebra of $\alg B^2$ generated by
    % $(a,b)$, $(a,c)$, $(b,c)$ and the diagonal of $B$,
    $\alg C:=\Sg^{\alg{B}^2}\bigl(\{(a,b), (a,c), (b, c)\}
                        \cup 0_{\alg B}\bigr)$:
    \begin{enumerate}
      \item \label{it:2a}
        $\beta := \Cga a b B$ is join irreducible with lower cover $\alpha$,
      \item \label{it:2b}
        $((a,b),(b,b)) \notin (\alpha_0 \meet \alpha_1) \join \Cga {(a,c)} {(b,c)} C$,  and
      \item \label{it:2c}
        $[\beta,\beta] \le \alpha$.
    \end{enumerate}

  \item \label{it:3}
    There do not exist $x_0$, $x_1$, $y_0$, $y_1\in A$ satisfying
    the following, where $\alg B$ is the subalgebra of
    $\alg A \times \alg A$ generated by $0 := (x_0, x_1)$, $1 := (y_0,x_1)$,
    and $t := (x_0,y_1)$:
    % $\rho_0$ is the kernel of the first projection:
    \begin{enumerate}
      \item \label{it:3a}
        $\beta := \Cga01{B}$ is join irreducible with lower cover $\alpha$,
      \item \label{it:3b}
        $\rho_0 \join \alpha = 1_{\alg B}$, and
      \item \label{it:3c}
        the type of $\beta$ over  $\alpha$ is~\atyp.
    \end{enumerate}
  \end{enumerate}
\end{theorem}

\begin{proof}
First assume $\sV$ has a difference term.
Then~(\ref{it:1}) holds by Theorem~\ref{thm:KearnesThm}.
If~(\ref{it:2}) fails then there are $a$, $b$ and $c\in A$ such
that the conditions specified in~(\ref{it:2}) hold.
In particular,~(\ref{it:2c}) holds and implies that
$\typ\<\alpha, \beta\> \subseteq \{\utyp, \atyp\}$,
so by~(\ref{it:1}) the type of $\<\alpha, \beta\>$ is~\atyp.
Let
\begin{align*}
\delta &:= (\alpha_0 \meet \alpha_1) \join \Cga {(a,c)} {(b,c)} C,
            \text{ and }\\
\theta &:= \delta \join \Cga {(a,b)}{(b,b)} C.
\end{align*}
By its definition, $\delta \nleq \alpha_0$,
so by~(\ref{it:2b}), $\alpha_0\meet\alpha_1 < \delta < \theta \le \beta_0$.
Since $C$ contains $0_B$, the diagonal of $B$, the coordinate
projections are onto, so $\alpha_0 \cov \beta_0$ and this interval
has type~\atyp.
From this it follows that $\alpha_0 \join \delta = \beta_0$.
Since $\theta \le \alpha_1$, we have $\alpha_0 \meet \theta = \alpha_0\meet\alpha_1$.
Hence
\[
\{\alpha_0\meet\alpha_1, \delta, \theta, \alpha_0, \beta_0 \}
\]
forms a pentagon.
Since $[\beta_0,\beta_0] \leq \alpha_0$, we have
$[\theta,\theta] \leq \alpha_0\meet\alpha_1 < \delta$,
so there is a congruence $\delta'$ such that
$\delta\leq \delta' \cov \theta$ and $\<\delta', \theta\>$
has type~\atyp.
As mentioned in the discussion above, this implies the
$\la \delta',\theta\ra$-minimal sets have tails, contradicting
Theorem~\ref{thm:KearnesThm}.

Now suppose that~(\ref{it:3}) fails. Then the conditions imply
\[
\{0_{\alg B}, \alpha, \beta, \rho_0, 1_{\alg B} \}
\]
is a pentagon whose critical prime interval has type~\atyp. This
leads to a contradiction in the same manner as above.

For the converse assume that $\sV$ does not have a difference
term.
We want to show that~(\ref{it:1}), (\ref{it:2}) or~(\ref{it:3}) fails.
Assume all three hold.
By Theorem~\ref{thm:KearnesThm} there is a finite
algebra $\alg B \in \sV$ and a join irreducible
$\beta \in \op{Con}(\alg B)$ with lower cover
$\alpha$ such that the type of $\<\alpha, \beta\>$ is~\atyp\
and the $\la\alpha,\beta\ra$-minimal sets have
nonempty tails. Let $U$ be one of these minimal sets.


We may assume $\alg B$ is minimal in the same manner as with the
above lemmas (with $\sS$ being the subalgebras of $\alg A$).
By Lemma~\ref{lem:fv_3-1}
we have that $\alg B$ is generated by any $0$, $1$, and $t$ in
$U$ such
that $\beta = \Cga 01{B}$ and $t$ is in the tail. By
Theorem~\ref{thm:fv_3-4}, $\alg B$ is either
in $\sS$ or is a subdirect product of two members of $\sS$.

Assume $\alg B$ is a subalgebra of $\alg A$. Taking
$a=0$, $b=1$ and $c=t$, we claim the conditions specified
in~(\ref{it:2}) hold.
Since the type of $\beta$ over $\alpha$ is \atyp,~(\ref{it:2c})
holds and we already have~(\ref{it:2a}) holds.
(\ref{it:2b}) holds by~\cite[Theorem~2.4]{KearnesKiss1999}.
So this choice of $a$, $b$, and $c$ witness that~(\ref{it:2}) fails.




Now assume $\alg B$ is not in $\sS$ but is a subdirect
product of two members of $\sS$.
Then
by Lemma~\ref{lem:fv_3-3} we may
assume $\rho_0 \join \alpha = 1_{\alg B}$ and
$\rho_1 \join \alpha < 1_{\alg B}$. By Lemma~\ref{lem:fv_3-2}
we have $\rho_1 \ge \beta$.

This implies that $0$ and $1$ have the same second coordinate; that is,
$0 = (x_0,x_1)$ and $1 = (y_0,x_1)$ for some $x_0$, $y_0$ and $x_1\in A$.
By Lemma~\ref{lem:nearperm}, $(0,t) \in \rho_0 \circ \beta$
so $0 \mathrel {\rho_0} t' \mathrel{\beta} t$. Let $U = e(B)$
where $e$ is an idempotent polynomial. Then
$0 \mathrel {\rho_0} e(t') \mathrel{\beta} t$. This gives
that $e(t')$ is in the tail of $U$ and
$0 \mathrel{\rho_0} e(t')$. We can
replace $t$ by $e(t')$, and so assume that
$0 \mathrel{\rho_0} t$.
Since $0 = (x_0,x_1)$, $t = (x_0,y_1)$ for some $y_1\in A$.
Now
$x_0$, $y_0$, $x_1$ and $y_1$ witness that~(\ref{it:3}) fails.
\end{proof}



\section{The Algorithm and its Time Complexity}
\label{sec:algorithm-its-time}
If $\alg A$ is an algebra with underlying set (or universe) $A$,
we let $|\alg A| = |A|$ be the cardinality of
$A$ and $||\alg A||$ be the \emph{input size}; that is,
\[
||\alg A|| = \sum_{i=0}^r k_i n^i
\]
where, $k_i$ is the number of basic operations of arity~$i$ and $r$
is the largest arity. We let
\begin{align*}
n &= |\alg A|  \qquad m = ||\alg A|| \\
r &= \text{the largest arity of the operations of $\alg A$}
\end{align*}

%Throughout this section we let $c$ denote a constant independent of
%these parameters.

\begin{prop}\label{speedprop}
Let $\alg A$ be a finite algebra with the parameters above. Then
there is a constant $c$ independent of these parameters
such that:
\begin{enumerate}
\item \label{speed1} If $S$ is a subset of $A$,
then $\Sg^{\alg A}(S)$ can be computed
in time
\[
c\, r\,||\Sg^{\alg A}(S)|| \le c\, r\,||\alg A|| = crm
\]
\item \label{speed2} If $a$, $b \in A$, then $\Cga a b A$ can be
computed in
$c\, r\, ||\alg A|| = crm$ time.
\item \label{speed3}
If $\alpha$ and $\beta$ are congruences of $\alg A$,
then $[\alpha,\beta]$ can be computed in time $crm^4$.
If $\alg A$ has a Taylor term and $[\alpha,\beta] = [\beta,\alpha]$,
then $[\alpha,\beta]$ can be computed in time $c(rm^2 + n^5)$.
In particular, $[\beta,\beta]$ can be computed in this time.
\end{enumerate}
\end{prop}

\begin{proof}
For the first two parts see~\cite[Proposition~6.1]{Freese:2009}.
To see the third part we first describe a method to compute
$[\alpha,\beta]$.


Following the notation of~\cite{FreeseMcKenzie1987}, we write
elements of $\alg A^4$ as $2 \times 2$ matrices, and
let $M(\alpha,\beta)$ be the subalgebra of $\alg A^4$ generated by
the elements of the form
\[
\begin{bmatrix}
a & a\\
a' & a'\\
\end{bmatrix}
\quad \text{and} \quad
\begin{bmatrix}
b & b'\\
b & b'\\
\end{bmatrix}
\]
where $a \mathrel\alpha a'$ and $b \mathrel\beta b'$. Then
by definition
$[\alpha,\beta]$ is the least congruence $\gamma$ such that
\begin{equation}\label{commprop}\text{if }
\begin{bmatrix}
x & y\\
u & v\\
\end{bmatrix}
\text{ is in $M(\alpha,\beta)$ and $x \mathrel\gamma y$,
then $u \mathrel\gamma v$.}
\end{equation}

Let $\delta = [\alpha,\beta]$. Clearly, if
$\begin{bmatrix}
x & x\\
u & v\\
\end{bmatrix}$ is in  $M(\alpha,\beta)$, then $u \mathrel\delta v$.
Let $\delta_1$ be the congruence generated by the $(u,v)$'s so
obtained. Then $\delta_1 \le \delta$.





We can now define $\delta_2$ as the congruence generated by the pairs
$(u,v)$ arising as the second row of members of $M(\alpha,\beta)$,
where the elements of the first row are $\delta_1$ related.
More precisely, we inductively define $\delta_0 = 0_{\alg A}$ and
\begin{equation}\label{delta}
\delta_{i+1} = \operatorname{Cg}^{\alg A}\bigg(\bigg\{(u,v) :
\begin{bmatrix}
x & y \\
u & v\\
\end{bmatrix}
\in M(\alpha,\beta) \text{ and } (x,y) \in \delta_i\bigg\}\bigg)
\end{equation}
Clearly, $\delta_1 \le \delta_2 \le \cdots \le
\delta$ and so $\Join_i \delta_i \le \delta$.

Now $\Join_i \delta_i$ has the property~\eqref{commprop} of the
definition of $[\alpha,\beta]$,
and hence $\delta \le \Join_i \delta_i$, and thus they are equal.

So to find $[\alpha,\beta]$ when $\alg A$ is finite, we find
$M(\alpha,\beta)$ and then compute the $\delta_i$'s, stoping when
$\delta_i = \delta_{i+1}$, which will be $[\alpha,\beta]$ of course.
The time to compute $M(\alpha,\beta)$ is bounded by $crm^4$ by
part~\eqref{speed1}. Suppose we have computed $\delta_i$. To
compute $\delta_{i+1}$ we run through the at most $n^4$ matrices
in
$\begin{bmatrix}
x & y \\
u & v\\
\end{bmatrix} \in M(\alpha,\beta)$. If $x$ and $y$ are in the same
block, we join the block containing $u$ and the one containing~$v$
into a single block. By the techniques of \cite{Freese2008},
this can be done in constant time. Now we take the congruence generated
by this partition. So, by \eqref{speed2} the time to compute
$\delta_{i+1}$ from $\delta_i$ is $c(n^4 + rm)$. Since the congruence
lattice of $\alg A$ has length at most $n-1$, there are at most~$n$
passes. So the total time is at most a constant times
$rm^4 + n(n^4 + rm) = rm^4 + n^5 + nrm$.
But since the commutator is trivial in unary algebras, we may
assume $m \ge n^2$, and thus the time is bounded by a
constant times $rm^4$, proving the first part of~\eqref{speed3}.
This procedure for calculating the commutator is part
of Ross Willard's thesis~\cite{MR2637477}.

To see the other part,
let $\alg A(\alpha)$ the subalgebra
of $\alg A \times \alg A$ whose components are $\alpha$ related. If we
view the elements of $\alg A(\alpha)$ as column vectors, then the
elements of $M(\alpha,\beta)$ can be thought of as pairs of elements
of $\alg A(\alpha)$, that is, as a binary relation on $\alg A(\alpha)$.
Define $\Delta_{\alpha,\beta}$ to be the congruence on $\alg A(\alpha)$
generated by this relation, that is,
the transitive closure of this relation.
Clearly $M(\alpha,\beta) \subseteq
\Delta_{\alpha,\beta}$. So, if in the algorithm above we used
$\Delta_{\alpha,\beta}$ in place of $M(\alpha,\beta)$, the result
would be at least $\delta = [\alpha,\beta]$. So, if we knew that
\begin{equation}\label{deltaStatement}
\text{whenever
$\begin{bmatrix}
x & y \\
u & v\\
\end{bmatrix} \in \Delta_{\alpha,\beta}$ and $(x,y) \in \delta$,
then $(u,v) \in \delta$},
\end{equation}
then this modified procedure would also
compute~$\delta$.

While \eqref{deltaStatement} is not true in general even if
$\alg A$ has a Taylor term, it is true when
$\alg A$ has a Taylor term and $[\alpha,\beta] = [\beta,\alpha]$.
So assume $\alg A$ has a Taylor term and $[\alpha,\beta] = [\beta,\alpha]$.
Under these conditions the commutator agrees with the
linear commutator, that is, $[\alpha,\beta] = [\alpha,\beta]_\ell$ by
Corollary~4.5 of~\cite{MR1663558}.
Suppose
$
\begin{bmatrix}
a&b\\c&d
\end{bmatrix}\in \Delta_{\alpha,\beta}$ and $(a,b)\in \delta$.
Then, since
$\Delta_{\alpha,\beta}$ is the transitive closure of
$M(\alpha,\beta)$, there are
elements $a_i$ and $c_i$ in $A$,
$i = 0, \ldots, k$, with $a_0 =a$,  $c_0 = c$, $a_k = b$ and
$c_k = d$, such that
$
\begin{bmatrix}
a_i&a_{i+1}\\c_i&c_{i+1}
\end{bmatrix}\in M(\alpha,\beta)$.

Now the linear commutator is $[\alpha^*,\beta^*]\big\vert_{A}$,
where $\alpha^*$ and $\beta^*$ are congruences on an expansion
$\alg A^*$ of $\alg A$ such that $\alpha \subseteq \alpha^*$
and $\beta \subseteq \beta^*$; see Lemma~2.4 of~\cite{MR1663558}
and the surrounding discussion.

Moreover $M(\alpha,\beta) \subseteq M(\alpha^*,\beta^*)$, the latter
calculated in $\alg A^*$, because the generating matrices of
$M(\alpha^*,\beta^*)$ contain those of $M(\alpha,\beta)$, and
the operations of $\alg A$ are contained in the operations
of $\alg A^*$.  So
$
\begin{bmatrix}
a_i&a_{i+1}\\c_i&c_{i+1}
\end{bmatrix}\in M(\alpha^*,\beta^*)$.
By its definition $\alg A^*$ has a Maltsev term, and
consequently $M(\alpha^*,\beta^*)$ is transitive as
a relation on $\alg A(\alpha^*)$. Thus
$
\begin{bmatrix}
a&b\\c&d
\end{bmatrix}\in M(\alpha^*,\beta^*)$, and hence,
$(c,d) \in [\alpha^*,\beta^*]\big\vert_A = [\alpha,\beta]$.

Now, since $\Delta_{\alpha,\beta}$ is a congruence on
$\alg A(\alpha)$, it can be computed in time
$crm^2$ by part~\eqref{speed2}. The result follows.
\end{proof}


%The below needs fixing: and def of Delta is needs to be changed to the
%Comm book and the property here is from (3.3)(1).

%We claim that if in \eqref{delta} we
%we replace $M(\alpha,\beta)$ by
%$\Delta_{\alpha,\beta}$, the resulting $\delta_i$'s are unchanged.
%To see this let the $\delta_i$'s be defined by~\eqref{delta} and
%let the $\delta'_i$'s be defined similarly but using
%$\Delta_{\alpha,\beta}$ in place of $M(\alpha,\beta)$.
%Since $M(\alpha,\beta) \subseteq \Delta_{\alpha,\beta}$, $\delta_i \le %\delta'_i$.





% define \Delta, ref that it is the transitive closure of the matrices
% and that if we replace the matrices with Delta in the above process
% we get the same result.



\begin{theorem}\label{thm:time}
Let $\alg A$ be a finite idempotent algebra with parameters as
above.
Then one can determine if $\bbV(\alg A)$ has a difference
term in time $c(rn^4m^4 + n^{14})$.
\end{theorem}
\rsf{See if we can omit $n^{14}$.}

\begin{proof}
Theorem~\ref{thm:algorithm} gives a three-step
algorithm to test
if $\bbV (\alg A)$ has a difference term.
The first step is to test if $\bbV (\alg A)$ omits type~\utyp. This
can be done in time $crn^3m$
by~\cite[Theorem~6.3]{Freese:2009}.

Looking now at part (3) of Theorem~\ref{thm:algorithm},
there are several things that have to be constructed.
By Proposition~\ref{speedprop}, all
of these things can be constructed in time $crm^2$ and
parts~(a) and~(b) can be executed in this time or less.
For part~(c) we need to test if the type of $\beta$
over $\alpha$ is \atyp. Since at this point in the
algorithm we know that $\alg A$ omits type ~\utyp,
we can test if the type is \atyp\ by testing if
$[\beta,\beta] \le \alpha$. By Proposition~\ref{speedprop}
this can be done in time $c(rm^4 + n^{10})$.
Since we need to do
this for all $x_0$, $x_1$, $y_0$ and $y_1$, the total
time for this step is at most $crn^4m^4 + n^{14}$.

A similar analysis applies to part~(2) and shows that it
can be done in time $crn^3m^2$. Since $crn^4m^4$ dominates
the other terms, the bound of the theorem holds.
\end{proof}



%%%%%%%%%%%%%%%%%%%%%%%%%%%% wjd: NEW SECTION  %%%%%%%%%%%%%%%%%%%%%%%%%%%%%%
\section{Difference Term Operations}
Above we addressed the problem of deciding the existence of a difference term
for a given (idempotent, finitely generated) variety.  In this section we are
concerned with the practical problem of finding a difference term
\emph{operation} for a given (finite, idempotent) algebra.
We describe algorithms for
\begin{enumerate}
\item \label{item:a} deciding whether a given finite idempotent algebra
has a difference term operation, and
\item \label{item:b} finding a difference term operation
for a given finite idempotent algebra.
\end{enumerate}
Note that Theorem~\ref{thm:time} gives a polynomial-time algorithm
for deciding whether or not the variety $\bbV(\alg A)$ generated by a
finite idempotent algebra $\alg A$ has a difference term.
If we run that algorithm on input $\alg A$, and if the observed
output is ``Yes'', then of course we have a positive answer to decision
problem~(\ref{item:a}).  However, a negative answer returned by the
algorithm only tells us that $\bbV(\alg A)$ has no difference term.
It does not tell us whether or not $\alg A$ has a difference term operation.  Example~\ref{alg:nodiff} provides a finite idempotent algebra that has a difference term operation such that the variety that it generates does not have a difference term.

%\mav{Again, improve example (see Ex. 9).}
%\begin{example}
%  Let $\alg A$ be the idempotent algebra with universe $\{0,1,2\}$ with basic operation $x\cdot y$ defined by:
%  \[
%  \begin{array}{c|ccc}
%  \cdot&0&1&2\\\hline
%  0&0&0&1\\
%  1&0&1&1\\
%  2&0&2&2
%  \end{array}
%  \]
%  Then $\alg A$ is a simple algebra of type~\ltyp and so the ternary projection operation $p(x,y,z ) = z$ is a difference term operation for $\alg A$. On the other hand, since $\{1,2\}$ is a subuniverse of $\alg A$ and $x\cdot y = x$ on this subset then the variety generated by $\alg A$ cannot have a difference term.
%\end{example}

%\wjd{insert example: $\alg A$ has a diff term op, but
 % $\bbV(\alg A)$ has no diff term.}

\medskip

In this section we present solutions to problems~(\ref{item:a}) and~(\ref{item:b})
using different methods than those of the previous sections.
In Subsection~\ref{sec:algor-1} we give a polynomial-time algorithm
for deciding whether a given idempotent algebra $\alg A$ has a difference term operation.
In Subsection~\ref{sec:comp-diff-term} we address problem~(\ref{item:b})
by presenting an algorithm for constructing a difference term operation.
%% However, this algorithm does not run in polynomial-time and, at the time
%% of this writing, we do not know of a more efficient algorithm for constructing a
%% difference term operation, even when such an operation is known to exist.
%Also, we have not yet proved the existence of a tractible algorithm for constructing %difference term operations.

\subsection{Local Difference Terms}
\label{sec:local-diff-terms}
In~\cite{MR3239624},
Ross Willard and the third author define %% an \defin{$\bA$-triple for $\bp$}
%% to be a triple $(a,b,i)$ such that $a, b \in A$ and
%% $p_i(a,b,b) = p_{i+1}(a,a,b)$. They use this to define
a ``local Hagemann-Mitschke sequence'' which they use as the basis of
an efficient algorithm for deciding for a given $n$ whether an idempotent
variety is $n$-permutable.
In~\cite{MR3109457}, Jonah Horowitz introduced similar
local methods for deciding when a given variety satisfies
certain \malcev conditions.
Inspired by these works, we now define a ``local difference term
operation'' and use it to develop a polynomial-time algorithm for deciding
the existence of a difference term operation.

Start with a finite idempotent algebra, $\alg A =\< A, \dots\>$.
For elements $a, b, a_j, b_j \in A$, the following are some shorthands
we will use to denote the congruences generated by these elements:
\[
\thetaab:= \Cg^{\alg{A}}(a, b) \qquad
\thetai:= \Cg^{\alg{A}}(a_i, b_i).
\]
Let $i \in \{0,1\}$.
By a \defin{local difference term operation for $(a,b,i)$}
we mean a ternary term operation $t$ satisfying the following conditions:
\begin{align}
\text{ if $i=0$, then } & a \comr{\thetaab} t(a,b,b); \label{eq:diff-triple1}\\
\text{ if $i=1$, then } & t(a,a,b) = b. \label{eq:diff-triple2}
\end{align}
If $t$ satisfies conditions~(\ref{eq:diff-triple1}) and~(\ref{eq:diff-triple2})
for all triples in some subset $S\subseteq A^2 \times \{0,1\}$, then we call $t$
a \defin{local difference term operation for $S$}.
Throughout the remainder of the paper, we will
write ``\ldto'' as shorthand for
``local difference term operation.''

Often we will take $S$ to be a finite list of elements of
$A^2 \times \{0,1\}$, and $|S|$ will denote the length of
the list. The definition above of a \ldto for
a set has an obvious analog for a list.

\newcommand\dtrel{\ensuremath{\mathrel{\mathcal{D}}}}
\newcommand\dtr{\ensuremath{\mathcal{D}}}
A few more notational conventions will come in handy below.
We will use the symbol
$\dtr \subseteq A^2\times \{0,1\} \times \Clo_3(\alg{A})$
to denote the relation that connects $(a, b, i)\in A^2\times \{0,1\}$
with operations in $\Clo_3(\alg{A})$ that are \ldtos for
$(a, b, i)$.  That is,
\[
((a,b,i), t) \in \dtr \text{ iff conditions~(\ref{eq:diff-triple1}) and~(\ref{eq:diff-triple2})
  are satisfied.}\]

The relation $\dtr$ induces an obvious Galois connection
from subsets of $A^2\times \{0,1\}$ to subsets
of $\Clo_3(\alg{A})$.  Overloading the notation, 
we will take
\begin{align*}
\dtr &\colon \sP(A^2 \times \{0,1\}) \to \sP(\Clo_3(\alg{A})) \text{ and }\\
\breve{\dtr} &\colon \sP(\Clo_3(\alg{A})) \to \sP(A^2\times \{0,1\})
\end{align*}
to denote the maps defined as follows:
\begin{align*}
  \text{for }\;  & S \subseteq A^2 \times \{0,1\}\; \text{ and } \;T \subseteq \Clo_3(\alg{A}),
  \text{ let }\\
  \dtr S &= \{t \in \Clo_3(\alg{A}) \colon (s,t) \in \dtr \text{ for all } s \in S\}, \text{ and }\\
  \breve{\dtr} T &= \{s \in A^2 \times \{0,1\} \colon (s,t) \in \dtr \text{ for all } t \in T\}.
\end{align*}
In other words, $\dtr S$ is the set of \ldtos
for $S$, and $\breve{\dtr} T$ is the set of triples for which every $t\in T$
is a \ldto.  When the set $S$ is just a singleton, we write 
$\dtr (a,b,\chi)$ instead of $\dtr \{(a,b,\chi)\}$.

Now, suppose that every pair
$(s_0, s_1)\in (\AsqBool)^2$ %$\AsqBool$
has a \ldto.
Then every subset $S\subseteq \AsqBool$
has a \ldto, as we now prove.

\begin{theorem} %[\protect{cf.~\cite[Theorem 2.2]{MR3239624}}]
  \label{thm:local-diff-terms}
  Let $\sV$ be an idempotent variety and let
  $\alg A  \in \sV$. %% . Define
  %% $\sS= A \times A \times \{0,1\}$
  If every pair
  $(s_0, s_1) \in (A^2 \times \{0,1\})^2$
  has a local difference term operation, then
  every subset $S \subseteq \AsqBool$
  has a local difference term operation.
\end{theorem}


\begin{proof}

The proof is by induction on the size of $S$.  In the base case,
$|S| = 2$, the claim holds by assumption. Fix $n\geq 2$ and assume that
every subset of $\AsqBool$ of size $k \leq n$ has a \ldto. Let
\[
S := \{(a_0, b_0, \chi_0), (a_1, b_1, \chi_1), \dots,
        (a_{n}, b_{n},\chi_{n})\} \subseteq \AsqBool,
\]
so $|S| = n+1$.  We prove $S$ has a \ldto.

Since $|S| \geq 3$, % and $\chi_i \in \{0,1\}$ for all $i$,
there exist indices $k\neq j$ such that $\chi_k = \chi_j$.
Assume without loss of generality that one such index is $j=n$,
and define the set
\[
S' := S \mysetminus \{(a_n, b_n, \chi_n)\}.
\]
Since $|S'| < |S|$, there exists $p \in \dtr S'$.
We split the remainder of the proof into two cases.

\medskip

%--------------------------------------
\noindent \underline{Case $\chi_n = 0$}:
Without loss of generality, assume
\begin{equation*}
  \chi_0 = %% \chi_2 =
\cdots =\chi_{k-1} = 1 \quad \text{and} \quad
\chi_{k} = \cdots = \chi_{n} = 0.
\end{equation*}
If we define %% $T$ to be the set
\[S_0 := \{(a_0, b_0, 1), (a_1, b_1, 1),
\dots, (a_{k-1}, b_{k-1}, 1), (a_n, p(a_n, b_n, b_n), 0)\},\]
then $|S_0| < |S|$, so $S_0$ has a \ldto, $q \in \dtr S_0$.
We now show that
\[
d(x,y,z) := q(x, p(x,y,y), p(x,y,z))
\]
is a local difference term operation for $S$.

Since $\chi_n =0$, we must check that
$a_n \comr{\thetan} d(a_n,b_n,b_n)$.
If $\gamman := \Cg(a_n, p(a_n,b_n,b_n))$, then
\begin{equation}
    \label{eq:100000}
  d(a_n,b_n,b_n) =
  q(a_n, p(a_n,b_n,b_n), p(a_n,b_n,b_n))\comr{\gamman} a_n.
\end{equation}
The pair $(a_n, p(a_n,b_n,b_n))$ is equal to
$(p(a_n,a_n,a_n), p(a_n,b_n,b_n))$ and so
% belongs to $\theta_n:= \Cg^{\alg{A}}(a_n, b_n)$.
belongs to $\thetan$.
Therefore, $\gamman\leq \thetan$, so
$\com{\gamman} \leq \com{\thetan}$.
% by monotonicity of the commutator.
It follows from this and (\ref{eq:100000}) that
$a_n \comr{\thetan} d(a_n,b_n,b_n)$, as desired.

For indices $i < k$, we have $\chi_i =1$, so
$d(a_i,a_i,b_i) = b_i$ for such $i$. Indeed,
\[
  d(a_i,a_i,b_i) =
  q(a_i, p(a_i,a_i,a_i), p(a_i,a_i,b_i)) % \label{eq:200000}\\
  =q(a_i, a_i, b_i) % \label{eq:200001}\\
  =b_i. % \label{eq:200002}
\]
The first equation holds by definition of $d$, the second
because $p$ is an idempotent \ldto for
$S'$, and the third because $q \in \dtr S_0$.

The remaining triples in our original set $S$
have indices satisfying $k\leq j < n$ and $\chi_j = 0$.
Here, we have $a_j \comr{\thetaj} d(a_j,b_j,b_j)$. Indeed,
by definition,
\begin{equation}
  \label{eq:450000}
d(a_j,b_j,b_j) =q(a_j, p(a_j,b_j,b_j), p(a_j,b_j,b_j)),
\end{equation}
and, since $p \in \dtr S'$, we have
%% the pair $(p(a_j,b_j,b_j), a_j)$ belongs to $\com{\thetaj}$.
$a_j \comr{\thetaj} p(a_j,b_j,b_j)$,
so (\ref{eq:450000}) implies that
$a_j = q(a_j,a_j,a_j) \comr{\thetaj} d(a_j, b_j,b_j)$.

%% Finally, by idempotence of $q$ we have
%% $d(a_j,b_j,b_j)\comr{\thetaj} a_j$,
%% as desired.

\medskip
%--------------------------------------
\noindent \underline{Case $\chi_n = 1$}:
Without loss of generality, assume
\begin{equation*}
  \chi_0 =\cdots =\chi_{k-1} = 0
  \quad \text{and} \quad
\chi_{k} = \cdots = \chi_{n} = 1.
\end{equation*}
If
%% \begin{equation*}
$S_1 := \{(a_0, b_0, 0), (a_1, b_1 0), \dots, (a_{k-1}, b_{k-1}, 0),
        (p(a_n, a_n, b_n), b_n, 1)\}$,\\[3pt]
%% \end{equation*}
then $|S_1| < |S|$, so there exists $q \in \dtr S_1$.
We claim  that
\begin{equation*}
  d(x,y,z) := q(p(x,y,z), p(y,y,z), z)
\end{equation*}
is a \ldto for $S$. For $(a_n,b_n,\chi_n) \in S$ we have that
\[
d(a_n,a_n,b_n) = q(p(a_n,a_n,b_n), p(a_n,a_n,b_n), b_n) =b_n.
\]
% \end{equation*}
The last equality holds since $q \in \dtr S_1$.
%% $q$ is \ldto for $(p(a_n, a_n, b_n), b_n, 1)$.

If $1\leq i < k$, then $\chi_i =0$. For these indices we must prove
that $a_i$ is congruent to $d(a_i,b_i,b_i)$ modulo $\com{\thetai}$.
Again, starting from the definition of $d$ and using idempotence of $p$, we have
\begin{equation}
  \label{eq:40000}
  d(a_i,b_i,b_i) =
  q(p(a_i,b_i,b_i), p(b_i,b_i,b_i), b_i)=
  q(p(a_i,b_i,b_i), b_i, b_i).
\end{equation}
Next, since $p \in \dtr S'$,
\begin{equation}
  \label{eq:50000}
  q(p(a_i,b_i,b_i), b_i, b_i)
 \comr{\thetai}
 q(a_i, b_i, b_i).
\end{equation}
Since $q \in \dtr S_1$, we have
$q(a_i, b_i, b_i) \comr{\thetai} a_i$, so
(\ref{eq:40000}) and (\ref{eq:50000}) imply
$d(a_i,b_i,b_i) \comr{\thetai} a_i$, as desired.

The remaining elements of $S$
have indices satisfying $k\leq j < n$ and $\chi_j = 1$.
For these we want $d(a_j,a_j,b_j) = b_j$.
Since $p \in \dtr S'$, we have
$p(a_j,a_j,b_j) = b_j$, and this plus idempotence of $q$ yields
\begin{equation*}
 d(a_j,a_j,b_j) =  q(p(a_j,a_j,b_j), p(a_j,a_j,b_j), b_j)=  q(b_j, b_j, b_j) =b_j,
\end{equation*}
as desired.
\end{proof}

\begin{corollary}
  \label{cor:loc-diff-term}
  Let $\alg A$ be a finite idempotent algebra and suppose that
  every pair $(s, s') \in (\AsqBool)^2$ has a local difference term operation.
  Then $\dtr (\AsqBool) \neq \emptyset$,
  so $\alg{A}$ has a difference term operation.
\end{corollary}
\begin{proof}
  Letting $S := \AsqBool$ in Theorem~\ref{thm:local-diff-terms} establishes
  the existence of a \ldto $d$ for $S$.  That is, $d \in \dtr S$.
  It follows that $d$ is a difference
  term operation for $A$. Indeed, for all $a, b \in A$, we have that
  $a \comr{\thetaab} d(a,b,b)$, since $d \in \dtr S \subseteq \dtr (a,b,0)$,
  %$d$ is \ldto for $(a,b,0)$,
  and $d(a,a,b) = b$, since $d\in \dtr S \subseteq \dtr (a,b,1)$.
  %$d$ is \ldto for $(a,b,1)$.
\end{proof}
%
% \begin{corollary}
%
%   A finite idempotent algebra $\alg A $ has a difference term operation if and
%   only if each pair in $(A^2 \times \{0,1\})^2$
%   has a local difference term.
% \end{corollary}
% \begin{proof}
%   A difference term operation for $\alg A $ is clearly a \ldto for every pair in
%   $(A^2 \times \{0,1\})^2$, so one direction of the corollary is obvious.
%   For the converse, suppose
%   each pair in $(A^2 \times \{0,1\})^2$ has a \ldto.
%   Then, by Theorem~\ref{thm:local-diff-terms},
%   there is a single \ldto for the whole set $A^2 \times \{0,1\}$,
%   and this is a difference term operation for $\alg A $.
% \end{proof}

% \draftsecskip

\subsection{Test for existence of a difference term operation}
\label{sec:algor-1}
%% In this subsection we prove the following:
%% \subsection*{Algorithm 1: existence of a difference term operation}

Here are some practical consequences of Theorem~\ref{thm:local-diff-terms}.
\begin{corollary}
  \label{cor:algor-1}
  There is a polynomial-time algorithm that takes as input
  any finite idempotent algebra $\alg A $ and decides if
  %% the variety $\bbV(\alg A )$ that it generates
  $\alg A $ has a difference term operation.
\end{corollary}
%\mav{Add Cor showing how to test if V(A) has a dt. (See Keith's proof.)}

\begin{proof}
  %% and let  $\sV = \bbV(\alg A )$.
  We describe an efficient algorithm for deciding,
  given a finite idempotent algebra $\alg A $,
  whether every pair in $(A^2 \times \{0,1\})^2$
  has a \ldto.  By Corollary~\ref{cor:loc-diff-term}, this will prove we
  can decide in polynomial-time whether $\alg A $ has an difference term operation.
  %% We will then complete the
  %% proof by explaining why $\alg A $ has a difference term operation iff the variety
  %% it generates has a difference term.

  Fix a pair
  $((a,b,i), (a',b',i'))$ in $(A^2 \times \{0,1\})^2$. If $i = i' = 0$,
  then the first projection is a \ldto. If $i = i' = 1$,
  then the third projection is a \ldto. The two remaining cases
  occur when $i\neq i'$. Without loss of generality, assume $i = 0$ and $i'=1$,
  so the given pair of triples is of the form $((a,b,0), (a',b',1))$.
  By definition, $t \in \dtr \{(a,b,0), (a',b',1)\}$ iff
  \[
  a\comr{\thetaab} t^{\alg A }(a,b,b) \; \text{ and } \;
  t^{\alg A }(a',a',b') = b'.
  \]
  We can rewrite this condition more compactly by noting that
  \[t^{\alg{A} \times \alg{A} }((a,a'), (b,a'), (b,b')) =
  (t^{\alg A }(a,b,b),t^{\alg A }(a',a',b')),\]
  and that
  $t \in \dtr \{(a,b,0), (a',b',1)\}$ if and only if
  \[
  t^{\alg A \times \alg A }((a,a'), (b,a'), (b,b'))\in a/\delta \times \{b'\},
  \]
  where $\delta = \com{\thetaab}$ and $a/\delta$ denotes the
  $\delta$-class containing $a$.
  It follows that $\dtr \{(a,b,0), (a',b',1)\} \neq \emptyset$
  %has a \ldto
  iff the subuniverse of $\alg A \times \alg A $ generated by
  $\{(a,a'), (b,a'), (b,b')\}$ intersects nontrivially with the subuniverse
  $a/\delta \times \{b'\}$.

  Thus, we take as input a finite idempotent algebra $\alg A $ and,
  for each element $((a,a'), (b,a'), (b,b'))$ of $(A\times A)^3$,
  \begin{enumerate}
    \item compute $\thetaab$,
    \item compute $\delta = \com{\thetaab}$,
    \item compute $\bS = \Sg^{\alg A \times \alg A }\{(a,a'), (b,a'), (b,b')\}$,
    \item \label{item:a3} test whether $S \cap (a/\delta \times \{b'\})$ is empty.
  \end{enumerate}
  If ever we find an empty intersection in step (\ref{item:a3}), then
  $\alg A $ has no difference term operation.
  Otherwise the algorithm halts without witnessing an empty
  intersection, in which case $\alg A $ has a difference term operation.

  Finally, we analyze the time-complexity of the procedure just described,
  using the same notation and complexity bounds as those appearing in
  Section~\ref{sec:algorithm-its-time}.  Recall, $n = |A|$, and 
  $m=\|\alg A\| = \sum_{i=0}^r k_i n^i$, where $k_i$ is the number of basic 
  operations of arity~$i$, and $r$ is the largest arity of the basic 
  operations of $\alg{A}$. The following assertions follow from the 
  proof of Proposition~\ref{speedprop}: $\thetaab$,
  $\delta$, and $S$ are computable in time $O(rm)$, $O(rm^2 + n^5)$, 
  and $O(rm^2)$, respectively; 
  $\thetaab$ and $\delta$ are computed for each pair $(a,b) \in A^2$;
  $S$ is computed for each triple of the form 
  $((a,a'),(b,a'),(b,b'))\in (A\times A)^3$, and there are $n^4$ such triples.
  Thus, the computational complexity of the above procedure is
  $O(rm^2n^4 + n^7)$.
\end{proof}

The following proposition provides another avenue for constructing a polynomial-time algorithm to decide if a finite idempotent algebra generates a variety that has a difference term (cf.~Theorem~\ref{thm:time}).
\begin{prop}
  Let $\alg A$ be a finite idempotent algebra.  Then $\bbV (\alg A)$ has a difference term if and only if each 3-generated subalgebra of $\alg A^2$ has a difference term operation.
\end{prop}

\begin{proof}
  Of course if $\bbV (\alg A)$ has a difference term then each 3-generated subalgebra of $\alg A^2$ has a difference term operation. For the converse, we refer to the proof of Theorem~\ref{thm:KearnesThm}~\cite[Theorem 3.8]{MR1358491}.  One part of this theorem establishes that if an algebra $\alg B$ is in a variety that has a difference term, then all type~\atyp minimal sets of $\alg B$ have empty tails.  A careful reading of the proof of this fact shows that a weaker hypothesis will suffice, namely that all quotients of $\alg B$ have difference term operations.  This is equivalent to just $\alg B$ having a difference term operation, since this property is preserved under taking quotients.
  \mav{This is true, I think.  Should it be mentioned somewhere, as part of a general discussion of difference term operations?}

  To complete the proof of this proposition, we use Corollary~\ref{cor:diffterm}. Suppose that each 3-generated subalgebra of $\alg A^2$ has a difference term operation.  Then, in particular, each 2-generated subalgebra of $\alg A$ does.  This rules out that $\sansH \sansS (\alg{A})$ contains an algebra that is term equivalent to the 2-element set.  It also follows, from the previous paragraph, that no 3-generated subalgebra of $\alg A^2$ has a prime quotient of type~\atyp whose minimal sets have nonempty tails.
\end{proof}

\begin{corollary}
  There is a polynomial-time algorithm to decide if a finite idempotent algebra $\alg A$ generates a variety that has a difference term.
\end{corollary}

\begin{proof}
  By the previous proposition, it suffices to check whether each 3-generated subalgebra of 
  $\alg A^2$ has a difference term operation. This can be decided by applying the 
  algorithm from Corollary~\ref{cor:algor-1} at most $\binom{n^2}{3}$ ($\approx n^6$) times, 
  so the total running time of this decision procedure is $O(rm^2n^{10} + n^{13})$ (cf.~$O(rn^4m^4 + n^{14}),$ 
  the complexity of the difference term existence test of Theorem~\ref{thm:time}).
\end{proof}

% \mav{If this is correct, should we say anything about the complexity of this algorithm?  Is it more efficient than the one from earlier in the paper?}
\section{Efficiently computing difference term tables}
\label{sec:comp-diff-term}
In this section we present a polynomial-time algorithm that takes 
a finite idempotent algebra and constructs the Cayley tables of 
its difference term operations.
% (given that we know such an operation exists).\wjd{Does the alg ever use the fact that a \ldto exists?}
% \mav{It is the case that we can construct an operation table for a difference term operation in poly-time.  I think that we can essentially follow the steps from this subsection, but rather than recording the terms at each step, we record the tables for the corresponding operations.  Should we mention this?}
The method consists of three subroutines.
Algorithm~\ref{alg:ld-2} finds a \ldto for sets
of size 2, and Algorithms~\ref{alg:stream-ldt1} and~\ref{alg:stream-ldt2}
alternate and produce \ldtos for successively larger subsets of $A^2 \times \{0,1\}$.

Some new notation will be helpful here.  Suppose that $ u \in \Sg^{\alg{A}}(X)$.
Then there is a term $t$ of some arity $k$ that, when applied to a certain tuple of 
generators $(x_1, \dots, x_k)$, produces $u$.  
In this situation, we may ask for the operation table (or ``Cayley table'') 
for $t^{\alg{A}}$, which is an $A^k$-dimensional array whose $(a_1, \dots, a_k)$-entry
is the value $t^{\alg{A}}(a_1, \dots, a_k)$. 
If $\sanst$ is such a table, we will denote the 
entry at position $(a_1, \dots, a_k)$ of the table by $\sanst[a_1, \dots, a_k]$.
Alternatively, if we wish to emphasize the means by which 
we arrived at the term that the table represents, we may use
$\sanst_{x_1,\dots, x_k \mid u}$ to denote the table.

\newcommand{\triple}{\ensuremath{(a_0,a_0),(b_0,a_0),(b_0,b_0)}}

\subsection{Algorithm~\ref{alg:ld-2}: implementation and complexity}
\label{sec:cc-ld-2}

The algorithm searches for term operations of type 
$(A\times A)^3 \to A\times A$, whose value at 
$(\triple)$ belongs to the set
$a_0/\delta_0 \times \{b_0\} \cap \Sg^{\alg{A}\times \alg{A}} (\triple)$.
When such an operation is found, its Cayley table
$\sanst$---which is an $n^3$-dimensional array satisfying 
$a_0 \mathrel{\delta_0}\sanst[a_0,b_0, b_0]$ and  $\sanst[a_0,a_0,b_0] = b_0$---is 
included in the return set $\tabT$.
% returned by Algorithm~\ref{alg:ld-2} represents an 

\RestyleAlgo{boxruled}%\LinesNumbered
\begin{algorithm}%[ht]
  % \SetKwInOut{Input}{Input}
  % \SetKwInOut{Output}{Output}

  \KwIn{$S = \{(a_0,b_0,0), (a_0, b_0, 1)\}$}
  \KwOut{Cayley tables for all operations in $\dtr S$}

  compute $\delta_0 = \com{ \theta_0 }$ and form $C_0= (a_0/\delta_0) \times \{b_0\}$\;
  compute $S_0=\Sg^{\alg{A}\times \alg{A}} (\triple)$\;
  \ForAll{$(u,v)  \in S_0$} {
    compute the table $\sanst_{\triple\mid (u,v)}$\;
    \If{$(u,v) \in C_0 \cap S_0$} {
    include $\sanst_{\triple \mid (u,v)}$ in the set $\tabT$\;
    }
  }
  \Return $\tabT$.
  \caption{Generate the set of all Cayley tables of \ldtos for $\{(a_0,b_0,0), (a_0, b_0, 1)\}$}
  \label{alg:ld-2}
  % in the $((a_0,a_1),(b_0,a_1),(b_0,b_1))$ position

\end{algorithm}


Recall the notation introduced in Section~\ref{sec:algorithm-its-time}: 
\begin{align*}
  n &= |A|, \quad m=\|\alg A\| = \sum_{i=0}^r k_i n^i,\\
k_i&= \text{ the number of basic operations of arity~$i$},\\ 
r &= \text{ the largest arity of the basic operations of $\alg{A}$.}
\end{align*}
Also, $\sanst_{(a_0,a_0),(b_0,a_0),(b_0,b_0)|(u,v)}$ denotes the Cayley table of a
term operation that generates $(u,v)$ from the set $\{(a_0,a_0),(b_0,a_0),(b_0,b_0)\}$.

Algorithm~\ref{alg:ld-2} can be implemented as follows:
\begin{enumerate}
  \item compute $\thetao$, in time $O(rm)$;
  \item compute $C_0= a_0/\com{ \thetao } \times \{b_0\}$,
  in time $O(rm^2 + n^5)$;
  \item generate $S_0=\Sg^{\alg{A}\times \alg{A}} \{(a_0,a_0),(b_0,a_0),(b_0,b_0)\}$,
    in time $O(r m^2)$;\\
    for each newly generated $(u,v) \in S_0$, 
  \begin{itemize}
    \item construct and store the table
      $\sanst_{(a_0,a_0),(b_0,a_0),(b_0,b_0)|(u,v)}$;
    \item if $(u,v) \in C_0$, then add $\sanst_{(a_0,a_0),(b_0,a_0),(b_0,b_0)|(u,v)}$
     to the list of tables to be returned; 
  \end{itemize}
\end{enumerate}
  
Each element $(u,v)\in S_0$ is the result of applying some (say, $k$-ary)
basic operation $f$ to previously generated pairs $(u_1, v_1)$, $\dots$, $(u_k,v_k)$
from $S_0$, and the operation tables generating these pairs were
already stored (the first bullet of Step 3).  Thus, to compute the table
for the operation that produced $(u,v) = f((u_1,v_1),\dots, (u_k,v_k))$
we simply compose $f$ with previously stored operation tables.
Since all tables represent ternary operations, the time-complexity of this composition
is $|A|^3$-steps multiplied by $k$ reads per step; that is,
$kn^3 \leq rn^3$.
  
All told, the time-complexity of Algorithm~\ref{alg:ld-2} is %$O(rm + rm^2 + n^5 + r^2m^2n^3) = 
$O(r^2m^2n^3 + n^5)$.
We only need to run the algorithm once since, as soon as we find all the
\ldto tables for elements generated by the single triple, $((a_0,a_0),(b_0,a_0),(b_0,b_0))$,
we move on to Algorithms~\ref{alg:stream-ldt1} and~\ref{alg:stream-ldt2}.


\subsection{Algorithm~\ref{alg:stream-ldt1}: implementation details}
The following notation will be useful:
\begin{align*}
  \sA_0 &:=\dtr \{(a_0, b_0, 0), (a_0, b_0, 1)\},\\
\sA_1 &:= \dtr\{(a_0, b_0, 0), (a_0, b_0, 1), (a_1,b_1,0)\},\\
\sA_2 &:= \dtr\{(a_0, b_0, 0), (a_0, b_0, 1), (a_1,b_1,0), (a_1,b_1,1)\},\dots,
\end{align*}
% $\sA_0 := \{(a_0, b_0, 0), (a_0, b_0, 1)\}$, $\sA_1 := \{(a_0, b_0, 0), (a_0, b_0, 1), (a_1,b_1,0)\}$;
%$\sA_2 := \{(a_0, b_0, 0), (a_0, b_0, 1), (a_1,b_1,0), (a_1,b_1,1)\}$, $\dots$,
and for all $0< k < n^2-1$, $\sA_{2k} := \sA_{2k-1} \cap \dtr (a_k,b_k,1)$, 
and $\sA_{2k+1} := \sA_{2k} \cap \dtr (a_{k+1},b_{k+1},0)$.
Also, define
\begin{align*}
\eta_0 &:= \dtr \{(a_0, b_0, 0), (a_0, b_0, 1)\} \\
\eta_1 &:= \dtr (a_1, b_1, 1), \; \eta_2 := \dtr (a_2, b_2, 1), \dots, 
\eta_i := \dtr (a_i, b_i, 1), \dots,\\
\eta_{\kk} &:= \bigcap_{0\leq i < k}\eta_i
  =\dtr \{(a_0, b_0, 0),(a_0, b_0, 1), (a_1, b_1, 1), \dots, (a_{k-1}, b_{k-1}, 1) \}.
\end{align*}
Finally, for a set $\sS$ of operations, let $\tabT\sS$ denote the set of 
all Cayley tables that represent operations in $\sS$. 

The output of Algorithm~\ref{alg:ld-2} is the set $\tabT \sA_0$ of all 
Cayley tables of operations in $\sA_0$, the \ldtos for $\{(a_0, b_0, 0), (a_0, b_0, 1)\}$.
The set $\tabT \sA_0$ is input to Algorithm~\ref{alg:stream-ldt1},
% A second input will be the set of all tables of operations in 
% $\dtr (a_0, b_0, 1)$.  To form this set, we take all tables of ternary operations.
% (There are $|A|^3$ such tables.) From these, we extract only those whose 
% $(a_0, a_0, b_0)$-entry is $b_0$.
the output of which is the set $\tabT \sA_1$ of all tables of 
operations in $\sA_1$.
The latter is input to Algorithm~\ref{alg:stream-ldt2}, the result of which is
the set $\tabT \sA_2$ of tables for $\sA_2$.
Thereafter, the process alternates between
Algorithms~\ref{alg:stream-ldt1} and~\ref{alg:stream-ldt2}, and 
terminates once we find the tables for $\sA_{2n^2-1} = A^2 \times \{0,1\}$,
which are tables of a difference term operations for $\alg A$ (as noted
in Corollary~\ref{cor:loc-diff-term}).
  
Before Algorithm~\ref{alg:stream-ldt1} is called, Algorithm~\ref{alg:ld-2} returns
$\tabT \sA_0$, as well as $\tabT \eta_0$ (since these two sets are identical in the $k=0$ case).
For the first iteration of Algorithm~\ref{alg:stream-ldt1}, we let $k=1$, so 
$\tabT \eta_0$ and $\tabT \sA_0$ are input, and the algorithm computes
$\tabT \eta_1$ and $\tabT \sA_1$.
For larger values of $k$, given $\tabT \eta_{\kk}$ and $\tabT \sA_{2k}$, 
Algorithm~\ref{alg:stream-ldt1} returns $\tabT \eta_{\kplus}$ and $\tabT \sA_{2k+1}$.

\RestyleAlgo{boxruled}
%\LinesNumbered
\begin{algorithm}
  \KwIn{$\tabT \eta_{\kk}$ and $\tabT \sA_{2k}$}

  \KwOut{$\tabT \eta_{\kplus}$ and $\tabT \sA_{2k+1}$}
  
  \ForEach{$\sanst_i \in \tabT \sA_{2k}$} {
  
    $\tabT_i := \tabT \eta_{\kk} \cap \tabT \dtr (a_{k+1}, \sanst_i[a_{k+1}, b_{k+1}, b_{k+1}], 0)$;
 
    \ForEach{$\sanss_j \in \tabT_i$} {
      \ForAll {$x$, $y$, $z \in A$}  {
        %define the table $\sanss_{j_i}$ as follows:
        $\sansr_{ij}[x,y,z] :=  \sanss_j\bigl[x, \sanst_i[x,y,y], \sanst_i[x,y,z]\bigr]$;
      }
      include $\sansr_{ij}$ in the set $\tabT$;
    }
  }
%  Let $\tabT[\sA_{2k+1}] := \{\sanss_{ji}: i, j\}$ be the set of all tables so defined.\\[5pt]
  compute $\tabS  := \bigl\{\sanst \in \tabT \eta_{\kk} \colon \sanst[a_k,a_k,b_k]= b_k \bigr\}$;

  \Return $\tabS$ and $\tabT$.
  \caption{Generate the set of all Cayley tables of \ldtos for $\sA_{2k+1}$ \label{alg:stream-ldt1} {\small ($k\geq 0$)}}
\end{algorithm}
  

Building the set $\tabT \eta_{\kplus}$ is trivial.  We simply remove 
from $\tabT \eta_{\kk}$ every table whose $(a_k,a_k,b_k)$-entry is not $b_k$.
As for $\tabT \sA_{2k+1}$,
for each table $\sanst_i \in \tabT \sA_{2k}$,
we let 
$c_i:= \sanst_i[a_{k+1}, b_{k+1}, b_{k+1}]$, and compute
$\gamma := \Cg^{\alg{A}}(a_{k+1}, c_i)$ and $\delta := \com{\gamma}$. % (in $O(rm^2 + n^5)$ time).
Next we generate the tables of operations in the set
$\eta_{\kk}\cap \dtr (a_{k+1}, c_i, 0)$, according to the usual
definition: 
$t \in \dtr (a_{k+1},c_i,0)$ iff $a_{k+1} \comr{\gamma} t^{\alg{A}}(a_{k+1},c_i,c_i)$.
The set of such tables is
$\tabT_i = \{\sanst \in \tabT \eta_{\kk} \colon  a_{k+1} \mathrel{\delta} \sanst [a_{k+1}, c_i, c_i]\}$.
For each $\sanss_j \in \tabT_i$, define the operation table $\sansr_{ij}$ as follows:
for all $x$, $y$, $z \in A$, $\sansr_{ij}[x,y,z] :=  \sanss_j[x, \sanst_i[x,y,y], \sanst_i[x,y,z]]$. 
The set of all tables $\{\sansr_{ij} \colon i, j\}$ so defined is $\tabT \sA_{2k+1}$.



\subsection{Algorithm~\ref{alg:stream-ldt2}: implementation}
For $i\geq 0$, let
\begin{align*}
\zeta_i &:= \dtr (a_i, b_i, 0), \\
  \zeta_{\kk} &:= \bigcap_{0\leq i < k}\zeta_i
  =\dtr \{(a_0, b_0, 0), (a_1, b_1, 0), \dots, (a_{k-1}, b_{k-1}, 0)\}.
\end{align*}

Given $\tabT \zeta_{\kminus}$
and $\tabT \sA_{2k-1}$---the sets of tables representing operations in
$\zeta_{\kminus}$ and $\sA_{2k-1}$, respectively---Algorithm~\ref{alg:stream-ldt2} 
returns the sets $\tabT \zeta_{\kk}$ and $\tabT \sA_{2k}$.
To build $\tabT \sA_{2k}$, take each table $\sanst_i$ from $\tabT \sA_{2k-1}$
and set $c_i:= \sanst_i[a_k, a_k, b_k]$. Then, let $\tabT_i$ denote 
the set of tables in $\tabT \zeta_{\kminus}$ whose 
$(c_i, c_i, b_k)$-entry is equal to $b_k$. 
For each $\sanss_j \in \tabT_i$, define the operation table 
$\sansr_{ij}$ as follows:
for all $x$, $y$, $z \in A$, 
$\sansr_{ij}[x,y,z] :=  \sanss_{j}\bigl[\sanst_i[x,y,z], \sanst_i[y,y,z], z\bigr]$. 
The set of all tables so defined is $\tabT \sA_{2k}$.
Finally, to build the set $\tabT \zeta_{\kk}$, we 
compute $\gamma_{k-1} := \Cg^{\alg{A}}(a_{k-1}, b_{k-1})$ and $\delta := \com{\gamma_{k-1}}$,
and then remove from $\tabT \zeta_{\kminus}$ every table whose 
$(a_{k-1},b_{k-1},b_{k-1})$-entry is not congruent to $a_{k-1}$ modulo $\delta_{k-1}$.



\RestyleAlgo{boxruled}
%\LinesNumbered
\begin{algorithm}
  \KwIn{$\tabT \zeta_{\kminus}$ and $\tabT \sA_{2k-1}$}
  \KwOut{$\tabT \zeta_{\kk}$ and $\tabT \sA_{2k}$}
  
  \ForEach{$\sanst_i \in \tabT \sA_{2k-1}$} {
    
    $\tabT_i := \tabT \zeta_{\kminus}  \cap \tabT \dtr (\sanst_i[a_k, a_k, b_k], b_k, 1)$;
  
    \ForEach{$\sanss_j \in \tabT_i$} {
      \ForAll {$x$, $y$, $z \in A$}  {
        $\sansr_{ij}[x,y,z] :=  \sanss_{j}\bigl[\sanst_i[x,y,z], \sanst_i[y,y,z], z\bigr]$;
      }
      include $\sansr_{ij}$ in the set $\tabT$;
    }
  }
  
  compute $\tabS := \{\sanst \in \tabT \zeta_{\kminus} \colon a_{k-1} \mathrel{\delta_{k-1}} \sanst[a_{k-1},b_{k-1},b_{k-1}]\}$;
  
  \Return $\tabS$ and $\tabT$.
  
  \caption{Generate the set of all Cayley tables of \ldtos for $\sA_{2k}$ \label{alg:stream-ldt2} {\small ($k> 0$)}}
\end{algorithm}
  

\subsection{Complexity of Algorithms~\ref{alg:stream-ldt1} and~\ref{alg:stream-ldt2}}
We now show that Algorithm~\ref{alg:stream-ldt1} runs in polynomial-time.
Proof that Algorithm~\ref{alg:stream-ldt2} is also polynomial-time is similar.

To build the set $\tabT \eta_{\kplus}$, we remove 
from $\tabT \eta_{\kk}$ every table $\sanst$ such that $\sanst[a_k,a_k,b_k]\neq b_k$.
This requires one look-up for each table.  There are 
at most $|A|^2$ tables in the set $\tabT \eta_{\kk}$
(since $\tabT \eta_{\kk} \subseteq \tabT \eta_0 = \tabT \sA_0$, and 
the latter clearly contains no more than $|S_0| \leq |A|^2$ tables).
Therefore, reducing $\tabT \eta_{\kk}$ to $\tabT \eta_{\kplus}$ takes at most $n^2$ steps.

To build $\tabT \sA_{2k+1}$,
for each table $\sanst \in \tabT \sA_{2k}$ (there are at most $n^2$),
%look up the $(a_{k+1}, b_{k+1}, b_{k+1})$-element of the table and call it 
%$c_i:= \sanst_i[a_{k+1}, b_{k+1}, b_{k+1}]$
compute $\gamma := \Cg^{\alg{A}}\bigl(a_{k+1}, \sanst[a_{k+1}, b_{k+1}, b_{k+1}]\bigr)$ and 
$\delta := \com{\gamma}$, which takes
$O(rm^2 + n^5)$-time.  Next, generate
\[
  \tabS = \{\sanss \in \tabT \eta_{\kk} \colon  a_{k+1} \mathrel{\delta} 
  \sanss \bigl[a_{k+1},  \sanst[a_{k+1}, b_{k+1}, b_{k+1}],  \sanst[a_{k+1}, b_{k+1}, b_{k+1}]\bigr]\}.
\]
This is simply a filtering of the set $\tabT \eta_{\kk}$:
include the table $\sanss$ in $\tabS$ iff the value in position 
$\bigl(a_{k+1},  \sanst[a_{k+1}, b_{k+1}, b_{k+1}],  \sanst[a_{k+1}, b_{k+1}, b_{k+1}]\bigr)$
of $\sanss$ belongs to the class $a_{k+1}/\gamma$.
A test of membership in a set is (at worst) linear-time, and there are at 
most $n^2$ tables to test, so (given $\delta$) the complexity of building 
$\tabS$ is $O(n^3)$.
Finally, for each table $\sanss \in \tabT$, we define the new table
$\sansr$ as follows:
$\forall x, y, z \in A$, $\sansr[x,y,z] :=  \sanss[x, \sanst[x,y,y], \sanst[x,y,z]]$. 
This operation takes $O(n^3)$ per table, and there are at most $n^2$ tables, 
so the complexity of this step is $O(n^5)$, and we do this once for each of the
(at most $n^2$) tables in $\tabT \sA_{2k}$.  All told, computing 
$\tabT \sA_{2k+1}$ is $O(rm^2n^2 + n^7)$.

We run Algorithms~\ref{alg:stream-ldt1} and~\ref{alg:stream-ldt2} $n^2$ times each,
so the total time-complexity of the inductive stage of our procedure is
$O(rm^2n^4 + n^9)$. As this dominates the running time of the basis 
stage (Algorithm~\ref{alg:ld-2}), it provides an upper bound on the 
complexity of the whole procedure.


  
  %\bibliographystyle{amsplain} %% or amsalpha
  %% \bibliographystyle{alpha-url}
  %% \printbibliography
  \bibliographystyle{alphaurl}
  \bibliography{inputs/refs}

  


  \end{document}
  %%%%%%%%%%%%%%%%%%%%%%%%%%%%%%%%%%%%%%%%%%%%%%%%%%%%%%%%%%%%%%%%%%%%%%%%%%%%%%%%%%%%
  %%%%%%%%%%%%%%%%%%%%%%%%%%%%  END OF DOCUMENT %%%%%%%%%%%%%%%%%%%%%%%%%%%%%%%%%%%%%%
  %%%%%%%%%%%%%%%%%%%%%%%%%%%%%%%%%%%%%%%%%%%%%%%%%%%%%%%%%%%%%%%%%%%%%%%%%%%%%%%%%%%%
  
  
  
  
  
  
  
  
  
  
  
  
  
  
  \appendix  
    
    \section{Alternative approaches/algorithms/notes}
    \subsection{A straight-forward (but intractable) algorithm}
    Define $\mathbf{c}$, $\mathbf{g}_0$, $\mathbf{g}_1$, $\mathbf{g}_2 \in \alg{A}^{2n}$ 
    and $R \subseteq A^{2n}\times A^{2n}$ as follows:
    \begin{equation}
    \begin{array}{cccccccccc}
      \mathbf{g}_0 &= (a_0 & a_0 &a_1 &a_1 &a_2 &a_2 & \dots & a_{n-1} & a_{n-1}) \\
      \mathbf{g}_1 &= (b_0 & a_0 &b_1 &a_1 &b_2 &a_2 & \dots & b_{n-1} & a_{n-1}) \\
      \mathbf{g}_2 &= (b_0 & b_0 &b_1 &b_1 &b_2 &b_2 & \dots & b_{n-1} & b_{n-1})\\[4pt]
        R  & = (\delta_0 & =  & \delta_1 & = & \delta_2 & = & \dots  & \delta_{n-1} & = \; )\\ [4pt]
      \mathbf{c}            & = (a_0     & b_0       & a_1      & b_1      & a_2      & b_2      & \dots  & a_{n-1}, & b_{n-1}) \\
    \end{array}
    \end{equation}
    where $\delta_k= \com{\theta_k}$.  We want a term $t$ such that $t^{\alg{A}^{2n}}(\mathbf{g}_0, \mathbf{g}_1, \mathbf{g}_2)\mathrel{R} \mathbf{c}$.
    In other terms, we want that $t^{\alg{A}^{2n}}(\mathbf{g}_0, \mathbf{g}_1, \mathbf{g}_2)$ belongs to the set
    \[
    \mathbf{C} := a_0/\delta_0 \times \{b_0\} \times a_1/\delta_1 \times \{b_1\} \times   \dots \times a_{n-1}/\delta_{n-1} \times \{b_{n-1}\}
    \]
    
    \noindent \textbf{An {\small EXPTIME} algorithm.}
     \begin{enumerate}
       \item For $0\leq k < n$,
       \begin{enumerate}
         \item compute $\theta_k:= \Cg^{\alg{A}}(a_k, b_k)$ in time $O(rmn)$;
         \item compute $\com{ \theta_k }$ in time $O(rnm^2 + n^6)$;
       \end{enumerate}
    \item Generate $S=\Sg^{\alg{A}^{2n}} \{\mathbf{g}_0, \mathbf{g}_1, \mathbf{g}_2\}$,
      in time $c r m^{2n}$ {\tiny {\color{red} $\quad \leftarrow$ EXPTIME}}\\
    for each newly generated $\mathbf{u} \in S$, 
      \begin{itemize}
      \item construct/store the table
        $\sanst_{\mathbf{g}_0, \mathbf{g}_1, \mathbf{g}_2|\mathbf{u}}$;
      \item if $\mathbf{u} \in \mathbf{C}$, then return     $\sanst_{\mathbf{g}_0, \mathbf{g}_1, \mathbf{g}_2|\mathbf{u}}$;
      \end{itemize}
    \end{enumerate}
    Notice that, if $\mathbf{u} \in \mathbf{C}$, then $\sanst_{\mathbf{g}_0, \mathbf{g}_1, \mathbf{g}_2|\mathbf{u}}$  is the table of a difference term operation.
    
    \subsection{Can we make the {\small EXPTIME} algorithm tractable?}
    We could partially produce the subalgebra of $\alg{A}^{\alg{A}^3}$ generated by
    the three $|A|^3$-tuples,
    % $\mathbf{g}^+_0 = (a_0, a_1, \dots)$,
    % $\mathbf{g}^+_1 = (b_0, a_1, \dots)$,
    % $\mathbf{g}^+_2 = (b_0, b_1, \dots)$,
    \begin{align*}
    \mathbf{g}^+_0 &= (a_0, a_0, \dots) = \mathbf{g}_0 :: \dots,\\
    \mathbf{g}^+_1 &= (b_0, a_0, \dots) = \mathbf{g}_1 :: \dots,\\
    \mathbf{g}^+_2 &= (b_0, b_0, \dots) = \mathbf{g}_2 :: \dots.
    \end{align*}
    Each $\mathbf{g}_i$ $(i=0,1,2)$ is extended so that every 
    triple from $A^3$ is represented as $(\mathbf{g}^+_0(i), \mathbf{g}^+_1(i), \mathbf{g}^+_2(i))$ for some 
    unique $i$.  Producing the entire subuniverse will be expensive in 
    general, but we only need to pay attention to the first $2n$ coordinates. 
    At each step, we check for a tuple $\mathbf{u} = (u_0, u_1, u_2, \dots, u_{2n-1})\in \mathbf{C}$.  
    If we find one, then the tuple $\mathbf{u}$ stores
    the table for a difference term operation and we can stop.
    
    At the end of each of these closure steps, we check for 
    tuples $v = (v_0, v_1, \dots)$ for which the pair $(v_0, v_1)$ has not
    yet appeared.  If no new pairs of this form are found, the algorithm halts
    and we conclude that no \ldto exists for this pair.
      % The runtime of this procedure is bounded by $|A|^2$.)
    
  


















  \section{Other practical considerations}
  \subsection{Simplified description of Section 5 procedure}
  To start off the algorithm requires three initial steps.
  \begin{enumerate}
    \item Compute $\theta_0 := \Cg^{\alg{A}}(a_0, b_0)$ and $\delta_0 := \com{\theta_0}$.
    \item Construct the list $L_0$ of all Cayley tables of operations $p$ 
    such that $p(a_0, b_0, b_0) \mathrel{\delta_0} a_0$; that is, the 
    $(a_0, b_0, b_0)$-entry in each operation table is congruent to $a_0$ modulo $\com{\theta_0}$.
    \item Let $L_1$ be the list $L_0$ filtered on the predicate,
    $\mathsf{C}[a_0, a_0, b_0] = b_0$, that the $(a_0, a_0, b_0)$-entry of 
    the table is $b_0$.
  \end{enumerate}
    % \item Compute $\theta_1 := \Cg^{\alg{A}}(a_1, b_1)$ and $\delta_1 := \com{\theta_1}$.
    % \item Let $L_2$ be the list $L_1$ filtered on the predicate 
    % $\mathsf[a_1, b_1, b_1] \mathrel{\delta_1} a_1$.
    % \item Let $L_3$ be the list $L_2$ filtered on the predicate,
    % $\mathsf{C}[a_1, a_1, b_1] = b_1$.
    % \item Compute $\theta_2 := \Cg^{\alg{A}}(a_2, b_2)$ and $\delta_2 := \com{\theta_2}$.
    % \item Let $L_4$ be the list $L_3$ filtered on the predicate 
    % $\mathsf[a_2, b_2, b_2] \mathrel{\delta_2} a_2$.
    % \item Let $L_5$ be the list $L_4$ filtered on the predicate,
    % $\mathsf{C}[a_2, a_2, b_2] = b_2$, $\dots$
  Thereafter these three steps are iterated, for $k> 0$.
  \begin{enumerate}
    \item Compute $\theta_{k} := \Cg^{\alg{A}}(a_k, b_k)$ and $\delta_{k} := \com{\theta_{k}}$.
    \item Let $L_{2k} := \{\mathsf{C} \in L_{2k-1} \colon \mathsf{C}[a_k, b_k, b_k] \mathrel{\delta_{k}} a_k \}$.
    \item Let $L_{2k+1} := \mathbf{filter} \; L_{2k} \; \mathbf{with} \; \mathsf{C}[a_k, a_k, b_k] = b_k$.
  \end{enumerate}
  The second and third steps are simple filtering operations:
  \begin{itemize}
    \item Let $L_{2k} = \mathbf{filter}\; L_{2k-1} \; \mathbf{with} \; \mathsf{C}[a_k, b_k, b_k] \mathrel{\delta_{k}} a_k$.
    \item Let $L_{2k+1} = \mathbf{filter} \; L_{2k} \; \mathbf{with} \; \mathsf{C}[a_k, a_k, b_k] = b_k$.
  \end{itemize}
  
  This algorithm can be implemented as follows:
   \begin{enumerate}
  \item compute $\theta_0$ in time $crm$;
  \item compute $\com{ \theta_0 }$ in time $c(rm^2 + n^5)$;
  \item generate $S_0=\Sg^{\alg{A}\times \alg{A}} \{(a_0,a_1),(b_0,a_1),(b_0,b_1)\}$,
    in time $c r m^2$;\\
  for each newly generated $(u,v) \in S_0$, 
    \begin{itemize}
    \item construct and store the table
      $\sanst_{(a_0,a_1),(b_0,a_1),(b_0,b_1)|(u,v)}$;
    \item if $(u,v) \in C_0$, then add $\sanst_{(a_0,a_1),(b_0,a_1),(b_0,b_1)|(u,v)}$
     to the list of tables to be returned; 
    \end{itemize}
  \end{enumerate}
  
  
  
  
  \begin{comment}
  \noindent \underline{\textbf{Subroutine \ld-2'}}\\[4pt]
  To compute a \ldto for
  $((a_0,b_0,1), (a_1, b_1, 0))$, obviously this is symmetric to
  the situation handled in Subroutine LD2 and so the general algorithm
  is the same.  Nonetheless, we include a listing of the computational
  steps required so that later we can easily refer to this special case
  of the general algorithm.
  \begin{enumerate}[{\bf 1}]
  \item Compute $\delta_1=\com{\thetaone}$;
  \item form $C_1= \{b_0\}\times a_1/\delta_1 \leq \alg{A}\times\alg{A}$;
  \item compute
        $S_1=\Sg^{\alg{A}\times \alg{A}} ((a_0,a_1),(a_0,b_1),(b_0,b_1))$;
  \item find a term operation $t$ of $\alg{A}$ satisfying
  %\[t((a_0,a_1),(a_0,b_1),(b_0,b_1)) \in C \cap S.\]
  \[t^{\alg{A}\times\alg{A}}((a_0,a_1),(a_0,b_1),(b_0,b_1)) =
   (t^{\alg{A}}(a_0,a_0,b_0), t^{\alg{A}}(a_1,b_1,b_1)) \in C_1 \cap S_1.\]
  \end{enumerate}
  Then $t$ is a \ldto for
  $((a_0, b_0, 1), (a_1, b_1, 0))$.
  \end{comment}
  
  
  
  
  \begin{comment}
  In practice, there are a number of different ways we could structure this
  algorithm when implementing it in software. In principle, the ordering of the
  first two steps is inconsequential.
  For instance, we might first present
  $\Sg^{\alg{A}\times \alg{A}} ((a_0,a_1),(b_0,a_1),(b_0,b_1))$
  as a stream $S_0$, then compute $\delta_0 = \com{ \thetao }$, and finally
  filter $S_0$ against the predicate $s \in (a_0/\delta_0) \times \{b_1\}$;
  the result is a stream of \ldtos.  This would be sufficient if our goal were merely
  to decide whether a difference term exists.  However, the algorithm calls for
  returning the table of a \ldto $t$.
  Therefore,
  \end{comment}
  
  
  
  
  
  
  
  
  
  \newpage
  \appendix
  
  \section{Miscellaneous Proofs}
  \subsection{Difference term identity verifications}
  \label{app:dt-ids}
  
  \begin{lemma}[Verifies Subroutine \ild-0]
  If $p$ is a \ldto for
  \begin{equation*}
  P_{j-1} = \{(a_0, b_0, 1), (a_1, b_1, 0), \dots, (a_{j-1}, b_{j-1}, 0)\}.
  \end{equation*}
  and $t$ is a \ldto for
  \begin{equation*}
  \{(a_0, b_0, 1), (a_j, p(a_j, b_j, b_j), 0)\},
  \end{equation*}
  then $d(x,y,z) = t(x, p(x,y,y), p(x,y,z))$ is a \ldto for
  $P_{j} = P_{j-1}  \cup \{(a_j, b_j, 0)\}$.
  \end{lemma}
  
  \begin{proof} There are three sorts of cases to check.
  \begin{enumerate}[1.]
  \item $(a_0, b_0, 1)$:
  \begin{equation*}
  d(a_0, a_0, b_0) =
  t(a_0, p(a_0,a_0,a_0), p(a_0,a_0,b_0)) =
  t(a_0, a_0, b_0) = b_0
  \end{equation*}
  
  \item $(a_k, b_k, 0)$ $(1\leq k < j)$:
  \begin{equation*}
  d(a_k, b_k, b_k) =
  t(a_k, p(a_k,b_k,b_k), p(a_k,b_k,b_k))
  \comr{\thetak}
  t(a_k, a_k, a_k)  = a_k
  \end{equation*}
  
  \item $(a_j, b_j, 0)$:
  If
  $\thetaj := \Cg(a_j, b_j)$ and
  $\gammaj := \Cg(a_j, p(a_j, b_j, b_j))$, then
  \begin{equation*}
  d(a_j, b_j, b_j) =
  t(a_j, p(a_j,b_j,b_j), p(a_j,b_j,b_j))
  \comr{\gammaj} a_j,
  \end{equation*}
  so
  $d(a_j, b_j, b_j) \comr{\thetaj} a_j$,
  since $\gammaj \leq \thetaj$.
  \end{enumerate}
  \end{proof}
  
  
  \begin{lemma}[Verifies Subroutine \ild-1]
  If $q$ is a \ldto for
  \begin{equation*}
  Q_{j-1} = \{(a_0, b_0, 0), (a_1, b_1, 1), \dots, (a_{j-1}, b_{j-1}, 1)\}.
  \end{equation*}
  and $t$ is a \ldto for
  \begin{equation*}
  \{(a_0, b_0, 0), (q(a_j, a_j, b_j), b_j, 1)\},
  \end{equation*}
  then $d(x,y,z) = t(q(x,y,z), q(y,y,z), z)$ is a \ldto for
  $Q_{j} = Q_{j-1}  \cup \{(a_j, b_j, 1)\}$.
  \end{lemma}
  
  \begin{proof} There are three sorts of cases to check.
  \begin{enumerate}[1.]
  \item $(a_0, b_0, 0)$:
  \begin{equation*}
  d(a_0, b_0, b_0) =
  t(q(a_0,b_0,b_0), q(b_0,b_0,b_0), b_0)
  \comr{\thetao}
  t(a_0, b_0, b_0) \comr{\thetao}
  a_0
  \end{equation*}
  
  \item $(a_k, b_k, 1)$ $(1\leq k < j)$:
  \begin{equation*}
  d(a_k, a_k, b_k) =
  t(q(a_k,a_k,b_k), q(a_k,a_k,b_k), b_k)
  = t(b_k, b_k, b_k) = b_k
  \end{equation*}
  
  \item $(a_j, b_j, 1)$:
  \begin{equation*}
  d(a_j, a_j, b_j) =
  t(q(a_j,a_j,b_j), q(a_j,a_j,b_j), b_j) = b_j
  \end{equation*}
  \end{enumerate}
  \end{proof}
  
  
  
  
  
  
  
  
  
  
  \subsection{Algorithm for constructing a difference term operation}
  \label{sec:algor-2}
  We are now in a position to describe a complete procedure for constructing
  a difference term operation for a given algebra $\alg{A}$.
  
  In this section we work with \emph{lists}
  (aka sequences, aka tuples) of triples,
  such as $((a_0, b_0, 0), (a_1, b_1, 0))$, rather than \emph{sets} of triples,
  simply because for practical (computer implementation) purposes lists are a little
  easier to work with than sets.
  % we refer to an ``\ldto for a tuple $(s_0)$
  
  First, let
  % $\begin{equation}
  %% $as = ((a_0, b_0), (a_1, b_1), \dots, (a_k, b_k))$
  $((a_0, b_0), (a_1, b_1), \dots, (a_k, b_k))$
  % \end{equation}
  be a list of all pairs in the set $A\times A$.
  % \footnote{Actually, we only need
  % $(A \times A) \mysetminus \{(a,a) \mid a \in A\}$, but we will elide this
  % detail in our description of the algorithm since it doesn't impact
  % complexity.}
  Let
  \begin{equation*}
  xs := %as \; zip \; (0,0,\dots, 0) =
  ((a_0, b_0, 0), (a_1, b_1, 0), \dots, (a_k, b_k, 0)),
  \end{equation*}
  and let
  % $xs_1 = xs \; zip \; (1,1, \dots, 1)$.
  \begin{equation*}
  ys := %as \; zip \; (1, 1, \dots, 1) =
  ((a_0, b_0, 1), (a_1, b_1, 1), \dots, (a_k, b_k, 1)).
  \end{equation*}
  
  
  Using Alg~\ref{alg:ld-2} for the base case
  and Alg~\ref{alg:ild1}
  for the induction step,
  we compute a \ldto $s_0$ for
  \begin{equation*}
  ((a_0, b_0, 0), (a_1, b_1, 1), (a_2, b_2, 1), \dots, (a_k, b_k, 1)),
  \end{equation*}
  and a \ldto $t_1$ for
  \begin{equation*}
  ((s_0(a_0, a_0, b_0), b_0, 1), (a_0, b_0, 0),(a_1, b_1, 0)).
  \end{equation*}
  The operation $s_1(x,y,z) := t_1(s_0(x,y,z), s_0(y,y,z), z)$ will then be
  a \ldto for
  % \begin{equation*}
  $((a_0, b_0, 0), (a_1, b_1, 0)) :: ys$  (where $::$ denotes
  list concatenation).
  % \end{equation*}
  
  Next, using Alg~\ref{alg:ld-2} for the base case
  and Alg~\ref{alg:ild0} for the induction step,
  we compute a \ldto $t_2$ for
  \begin{equation*}
  ((s_1(a_0, a_0, b_0), b_0, 1), (a_0, b_0, 0),(a_1, b_1, 0),(a_2, b_2, 0)).
  \end{equation*}
  Then $s_2(x,y,z) := t_2(s_1(x,y,z), s_1(y,y,z), z)$ is a \ldto
  for the elements of
  % \begin{equation}
  $((a_0, b_0, 0), (a_1, b_1, 0), (a_2, b_2, 0)) :: ys$.
  % \end{equation}
  
  Proceeding inductively, for fixed $1<j\leq k$,
  if $s_{j-1}$ is a \ldto
  for the elements in the list
  \begin{equation*}
  ((a_0, b_0, 0), (a_1, b_1, 0), \dots, (a_{j-1}, b_{j-1}, 0)) :: ys,
  \end{equation*}
  and if we use Algorithms~\ref{alg:ild0} and~\ref{alg:ild1}
  to compute a \ldto $t_j$ for
  \begin{equation*}
  ((s_{j-1}(a_0, a_0, b_0), b_0, 1), (a_0, b_0, 0),(a_1, b_1, 0),\dots,
  (a_j, b_j, 0)),
  \end{equation*}
  then $s_j(x,y,z) := t_j(s_{j-1}(x,y,z), s_{j-1}(y,y,z), z)$ will be a
  \ldto for
  $((a_0, b_0, 0), (a_1, b_1, 0), \dots, (a_j, b_j, 0)) :: ys$.
  In the end we arrive at a \ldto for $xs :: ys$, namely,
  \begin{equation*}
  d(x,y,z) = t_k(s_{k-1}(x,y,z), s_{k-1}(y,y,z), z).
  \end{equation*}
  Since $xs :: ys$ contains all of the triples $(a,b,\chi)$, where
  $a, b \in A$ and $\chi \in \{0,1\}$,
  $d(x,y,z)$ is the desired difference term operation for $\alg{A}$.
  
  
  
  
  
  
  
  compute $\delta_0$ and $\delta_1$ (see~(\ref{eqn:notation1}) for definition);
  
  compute the stream $\sS_0$ of term operations $s$ that satisfy $s(a_0,a_0,b_0) = b_0$;
  
  find a term operation $p$ in $\sS_0$ satisfying $p(a_0, b_0, b_0) \mathrel{\delta_0} a_0$;
  
  compute $c_1 = p(a_1,b_1,b_1)$;
  
  find a term operation $q$ in $\sS_0$ satisfying $q(a_1, c_1, c_1) \mathrel{\delta_1} a_1$;
  
  return $(s_0, \sS_0)$ where $s_0(x,y,z) = q(x,p(x,y,y),p(x,y,z))$
  \end{algorithm}
  
  
  
  
  
  
  
  
  
  
  
  
  
  
  
  
  
  
  
  
  
  
  
  
  
  
  
  
  
  
  
  
  
  \subsection{A more efficient algorithm}
  \label{sec:algor-3}
  In this section we refine the algorithm from Section~\ref{sec:algor-2}
  for constructing a difference term operation for a given finite idempotent
  algebra $\alg{A}$.
  The point here is to describe a version of the algorithm that is more practical
  and resembles how we might realistically implement it in a computer
  programming language.
  
  In addition to notational conventions above, such as
  \begin{equation}
  \label{eqn:notation1}
  \thetai = \Cg^{\alg{A}}(a_i, b_i) \; \text{ and } \;
  \deltai=\com{\thetai},
  \end{equation}
  we also take
  $as = ((a_0, b_0), (a_1, b_1), \dots, (a_k, b_k))$
  to be the list of all pairs in the set $A\times A$.
  Also define
  \begin{align*}
  \sA_0 &= \{(a_0, b_0, 0), (a_0, b_0, 1)\},\\
  \sA_1 &= \{(a_0, b_0, 0), (a_0, b_0, 1), (a_1,b_1,0)\},\\
  \sA_2 &= \{(a_0, b_0, 0), (a_0, b_0, 1), (a_1,b_1,0), (a_1,b_1,1)\},\\
  &\vdots\\
  \sA_{2k} &= \{(a_0, b_0, 0), (a_0, b_0, 1), \dots, (a_k,b_k,0), (a_k,b_k,1)\}.
  \end{align*}
  That is,
  $\sA_{2k} = \sA_{2k-1} \cup \{(a_k,b_k,1)\}$ and
  $\sA_{2k+1} = \sA_{2k} \cup \{(a_{k+1},b_{k+1},0)\}$.
  
  \RestyleAlgo{boxruled}
  \LinesNumbered
  \begin{algorithm}%[ht]
    \SetKwInOut{Input}{Input}
    \SetKwInOut{Output}{Output}
  
    \caption{return \ldto for $\sA_1$
    \label{alg:stream-ldt1}  }
  % \Input{$as$ = the list of all pairs in $A \times A$}
  % \Output{a difference term operation for $\alg A$}
  
  compute $\delta_0$ and $\delta_1$ (see~(\ref{eqn:notation1}) for definition);
  
  compute the stream $\sS_0$ of term operations $s$ that satisfy $s(a_0,a_0,b_0) = b_0$;
  
  find a term operation $p$ in $\sS_0$ satisfying $p(a_0, b_0, b_0) \mathrel{\delta_0} a_0$;
  
  compute $c_1 = p(a_1,b_1,b_1)$;
  
  find a term operation $q$ in $\sS_0$ satisfying $q(a_1, c_1, c_1) \mathrel{\delta_1} a_1$;
  
  return $(s_0, \sS_0)$ where $s_0(x,y,z) = q(x,p(x,y,y),p(x,y,z))$
  \end{algorithm}
  
  
  % \RestyleAlgo{boxed}
  \RestyleAlgo{boxruled}
  \LinesNumbered
  \begin{algorithm}
    \SetKwInOut{Input}{Input}
    \SetKwInOut{Output}{Output}
  
    \caption{return \ldto for $\sA_2$
    \label{alg:stream-ldt2}  }
    \Input{$(s_0, \sS_0)$ (from Alg~\ref{alg:stream-ldt1})}
    \Output{a \ldto for $\sA_2$}
  
  compute the stream $\sS_1$ of term operations $p$
  that satisfy $p(s_0(a_1,a_1,b_1), s_0(a_1,a_1,b_1), b_1) = b_1$;
  
  find a term operation $p$ in $\sS_0 \cap \sS_1$ and compute $d_1 = p(a_1,b_1,b_1)$;
  
  find a term operation $q$ in $\sS_1$ satisfying $q(a_1, d_1, d_1) \mathrel{\delta_1} a_1$.
  
  return $(\sS_0, \sS_1, s_1)$ where $s_1(x,y,z) = q(x, p(x,y,y), p(x,y,z))$
  \end{algorithm}
  
  
  % \RestyleAlgo{boxed}
  \RestyleAlgo{boxruled}
  \LinesNumbered
  \begin{algorithm}
    \SetKwInOut{Input}{Input}
    \SetKwInOut{Output}{Output}
  
    \caption{return \ldto for $\sA_3$
    \label{alg:stream-ldt3}  }
    \Input{$(s_1, \sS_0, \sS_1)$ (from Alg~\ref{alg:stream-ldt2})}
    \Output{a \ldto for $\sA_3$}
  
  compute $\delta_2$;
  
  compute the substream $\sS_2 \subseteq \sS_0$ of term operations satisfying
   $p(a_1,a_1,b_1) = b_1$;
  
  compute $c_2 = s_1(a_2, b_2, b_2)$;
  
  find a term operation $p$ in $\sS_2$ satisfying
  $p(a_2, c_2, c_2) \mathrel{\delta_2} a_2$;
  
  return $(s_2, \sS_0, \sS_1, \sS_2)$ where $s_2(x,y,z) = p(x, s_1(x,y,y), s_1(x,y,z))$
  \end{algorithm}
  
  
  
  
  
  
  
  
  
  
  
  
  
  
  
  
  
  
  %%% OLD ALGORITHM June 2017
  
  \subsubsection{Induction steps}
  \label{sec:induct}
  
  The next ingredient we need for our construction of a
  difference term operation for $\alg{A}$
  is
  a way to produce a \ld
  term operation for a set of the form  (for $j>1$)
  \begin{equation}
  \label{eqn:Pj}
  P_j = \{(a_0, b_0, 1), (a_1, b_1, 0), (a_2, b_2, 0), \dots,
  (a_j, b_j, 0)\},
  \end{equation}
  assuming we're given a \ldto for $P_{j-1} = P_{j} \mysetminus \{(a_j, b_j, 0)\}$.
  
  
  \begin{comment}
  \noindent \underline{\bf Subroutine \ild-0}\\[4pt]
  The input, $p$,  is
  a \ldto for $P_{j-1} = P_{j} \mysetminus \{(a_j, b_j, 0)\}$.
  \begin{enumerate}[1.]
  \item
  Call Subroutine \ld-2 to
  compute a \ldto $t$ for the set
  \begin{equation*}
  \{(a_0, b_0, 1), (a_j, p(a_j, b_j, b_j), 0)\}.
  \end{equation*}
  \item Return the following
  \ldto for $P_j$:
  \begin{equation}
  d(x,y,z) = t(x, p(x,y,y), p(x,y,z)).
  \label{eq:idl0-dto}
  \end{equation}
  \end{enumerate}
  \qed
  \end{comment}
  
  
  
  
  % \RestyleAlgo{boxed}
  \RestyleAlgo{boxruled}
  \LinesNumbered
  \begin{algorithm}%[ht]
    \SetKwInOut{Input}{Input}
    \SetKwInOut{Output}{Output}
  
    % \caption{Given \ldto for $P_{j-1}$,
  \caption{Return a \ldto for the set $P_j$ defined in~(\ref{eqn:Pj})
  \label{alg:ild0}}
  %   $P_{j-1}=\{(a_0, b_0, 1), (a_1, b_1, 0), (a_2, b_2, 0), \dots,
  % (a_{j-1}, b_{j-1}, 0)\}$, compute a \ldto for
   % = P_{j-1} \cup \{(a_j, b_j, 0)\}$
  \Input{$p =$ a \ldto for $P_{j-1}$}
  \Output{$d =$ a \ldto for $P_j$}
  
  compute a \ldto $t$ for $\{(a_0, b_0, 1), (a_j, p(a_j, b_j, b_j), 0)\}$;
  
  return $d(x,y,z) = t(x, p(x,y,y), p(x,y,z))$
  
  \end{algorithm}
  
  
  We also need a way to produce a \ld
  term operation for a set of the form  (for $j>1$)
  \begin{equation}
  \label{eqn:Qj}
  Q_{j} = \{(a_0, b_0, 0), (a_1, b_1, 1), (a_2, b_2, 1), \dots,
  (a_{j}, b_{j}, 1)\},
  \end{equation}
  assuming we're given a \ldto for
  $Q_{j-1} = Q_j \mysetminus \{(a_j, b_j, 1)\}$.
  
  % \RestyleAlgo{boxed}
  \RestyleAlgo{boxruled}
  \LinesNumbered
  \begin{algorithm}%[ht]
    \SetKwInOut{Input}{Input}
    \SetKwInOut{Output}{Output}
    \caption{Return a \ldto for the set $Q_j$ defined in~(\ref{eqn:Qj})
    \label{alg:ild1}}
    \Input{$q =$ a \ldto for $Q_{j-1}$}
    \Output{$d =$ a \ldto for $Q_j$}
  
    compute a \ldto $t$ for $\{(a_0, b_0, 0), (q(a_j, a_j, b_j), b_j, 1)\}$;
  
    return $d(x,y,z) = t(q(x,y,z), q(y,y,z), z)$
  \end{algorithm}
  
  \begin{remarks}\
  \begin{enumerate}[1.]
  \item In each of the Algorithms~\ref{alg:ild0} and \ref{alg:ild1},
  the first step is a call to Algorithm~\ref{alg:ld-2}
  for computing a \ldto for a pair of triples.
  \item The easy verification that Algorithm~\ref{alg:ild0} returns
  a \ldto for $P_j$ appears in Appendix Section~\ref{app:dt-ids}.
  \textcolor{red}{(Decide whether to omit proof.)}
  \item The easy verification that Algorithm~\ref{alg:ild1} returns
  a \ldto for $Q_j$ appears in Appendix Section~\ref{app:dt-ids}.
  \textcolor{red}{(Decide whether to omit proof.)}
  \end{enumerate}
  \end{remarks}
  % (see Case $\chi_0=0$ in the proof of Theorem~\ref{thm:local-diff-terms}).
  
  
  
  
  
  
  
  
  
  
  
  
  
  
  
  
  
  
  
  
  
  
  
  
  
  
  With notation and assumptions as above.
  
  \begin{theorem}[cf. Theorem 3.3 of \cite{FreeseValeriote2009}]
  This is a test.
  \end{theorem}
  
  
  
  
  
  
  \bibliographystyle{rsfplain}
  %\bibliography{\jobname}
  \bibliography{/Users/ralph/tex/bib/Database/db}
  
  
  
  A \emph{minority term} (for an algebra or variety) is a
  3-variable term $q(x,y,z)$ such if two of the variables
  are equal, its value is the other one; that is,
  \[
  q(x,x,y) \approx q(x,y,x) \approx q(y,x,x) \approx y.
  \]
  We are interested in an algebraic description of varieties having
  a minority term. One possible conjecture is:
  \begin{conjecture}
  $\mathcal V$ has a minority term if and only if it is CP and its
  ring has characteristic~1 or~2.
  \end{conjecture}
  
  
  
  \begin{fact}
  If $\mathcal V$ has a minority term, then it is CP.
  \end{fact}
  
  \begin{proof}
  Clearly a minority term is a \malcev term.
  \end{proof}
  
  \begin{fact}
  A CD variety $\mathcal V$ has a minority term if and
  only if it is CP.
  \end{fact}
  
  \begin{proof}
  A variety is CD and CP if and only if it has a Pixley term.
  If $p(x,y,z)$ is a Pixley term then
  \[
  q(x,y,z) = p(p(x,y,z),x,p(x,z,y))
  \]
  is a minority term. The fact can be derived from these
  observations.
  \end{proof}
  
  \begin{fact}
  A variety of groups has a minority term if and only
  the variety has exponent 2 (well, or 1).
  \end{fact}
  
  \begin{proof}
  If a variety has exponent 2 then it is abelian and so,
  using additive notation, $x + y + z$ is a minority term.
  
  If the exponent is not 2 then the variety contains a group
  which contains an element of order $n$, where $n > 2$ (or
  infinite). This element generates a cyclic group. In any
  abelian algebra the \malcev term operation is unique and so
  must be $x - y + z$. But one easily checks that this is not
  a minority term when $n > 2$.
  \end{proof}
  
  \begin{lemma}
  Let $\mathcal V$ be a CP variety with a \malcev term $p(x,y.z)$.
  The the following are equivalent:
  \begin{enumerate}
  \item
  The ring $R(\mathcal V)$ of $\mathcal V$ has characteristic~2.
  \item
  On every block of every abelian congruence, $p$ restricted to the
  block is a minority term; that is, satisfies $p(a,b,a) = b$.
  \item
  If $\theta = \textup{Cg}^{\alg F_{\mathcal V}(x,y)}(x,y)$, then,
  on each block of $\theta/[\theta,\theta]$, $p$ is a minority term.
  \end{enumerate}
  If $\mathcal V$ has a minority term, then these conditions hold.
  \end{lemma}
  
  \begin{proof}
  By commutator theory (give some specific refs, also that the ring
  is det by the strucure of $\theta/[\theta,\theta]$)
  the \malcev term operation on an
  abelian algebra, or even a block of an abelian congruence,
  is unique and it is $p(x,y,z) = x - y + z$ for
  some abelian group. It is easy that $x - y + z$ is a minority
  if and only if the abelian group has exponent~2.
  \end{proof}
  
  
  
  Let $\alg A$ be an algebra and let $S$ and $T$ be tolerances
  on $\alg A$.
  Let $M(S,T)$, or $M^{\alg A}(S,T)$ to emphasize $\alg A$,
  be the set of all $2 \times 2$ matrices of the form
  \begin{equation}\label{eq1}
  \begin{bmatrix}
  p&q\\
  r&s
  \end{bmatrix}
  =
  \begin{bmatrix}
  f(\mathbf{a},\mathbf{u})&f(\mathbf{a},\mathbf{v})\\
  f(\mathbf{b},\mathbf{u})&f(\mathbf{b},\mathbf{v})
  \end{bmatrix}
  \end{equation}
  where $f(\mathbf{x},\mathbf{y})$ is an $(m+n)$-ary polynomial of
  $\alg A$, $\mathbf{a} \mathrel{S} \mathbf{b}$, and
  $\mathbf{u} \mathrel{T} \mathbf{v}$
  (componentwise, of course). The members of $M(S,T)$ are called
  \emph{$S,T$-matrices}.
  
  The first exercise gives an efficient way to find $M(S,T)$.
  
  \section*{Exercises}
  
  \begin{exercises}
  
  \prob
  Show that $M(S,T)$ is the subalgebra of $\alg A^4$ generated by
  \[
  \left\{
  \begin{bmatrix}
  a&a\\
  b&b
  \end{bmatrix} : a \mathrel{S} b\right\}
  \union
  \left\{
  \begin{bmatrix}
  c&d\\
  c&d
  \end{bmatrix} : c \mathrel{T} d\right\}
  \]
  
  \prob
  Use the symmetry of $S$ and $T$ to show the matrix obtained from an
  $S,T$-matrix by interchanging the rows or columns (or both) is also
  in $M(S,T)$.
  
  \prob
  $M(T,T)$ is closed under taking transposes.
  
  \end{exercises}
  
  \section*{Centrality Relations}
  
  We define four kinds of centrality, called centrality, strong
  centrality, weak centrality, and strong rectularity. The is a fifth
  centrality condition known as rectangularity which we will save for
  later.
  
  Let $\delta$ be a congruence and $S$ and $T$ be
  tolerance relations on  $\alg A$. The above centrality relations
  are denoted $\alg C(S,T;\delta)$ (centrality),
  $\alg S(S,T;\delta)$ (strong centrality),
  $\alg W(S,T;\delta)$ (weak centrality),  and
  $\alg SR(S,T;\delta)$ (strong rectangularity). They hold if the
  appropriate implication below holds for all
  \[
  \begin{bmatrix}
  p&q\\
  r&s
  \end{bmatrix} \in M(S,T)
  \]
  \begin{itemize}
  \item centrality:
  $p \mathrel{\delta} q \implies r \mathrel{\delta} s$.
  \item strong rectangularity:
  $p \mathrel{\delta} s \implies r \mathrel{\delta} s$.
  \item weak centrality:
  $p \mathrel{\delta} q \mathrel{\delta} s \implies r \mathrel{\delta} s$.
  \item strong centrality holds if both centrality and strong
  rectangularity hold.
  \end{itemize}
  
  Using the exercises it is easy to see that the implication defining
  $\alg C(S,T;\delta)$ can be replaced by
  $r \mathrel{\delta} s \implies p \mathrel{\delta} q$ and this is
  equivalent to
  \[
  p \mathrel{\delta} q \Longleftrightarrow r \mathrel{\delta} s.
  \]
  Similar statements hold for the other conditions: weak centrality
  is equivalent to saying that if any three of $p$, $q$, $r$ and $s$
  are $\delta$ related, then they all are. And strong rectangularity
  says that if the elements of the main diagonal, or of the sinister
  diagonal, are $\delta$ related, then all four are.
  
  The \emph{$S,T$-term condition} is the condition $\alg C(S,T,0)$,
  usually expressed using the right-hand matrix in~\eqref{eq1}.
  Other kinds of term conditions are defined similarly.
  
  If $\alg C(S,T;\delta_i)$ holds for all $i \in I$, then
  $\alg C(S,T;\Meet_{i\in I}\delta_i)$ holds. Similar statements hold
  for the other centrality conditions. So there is a least $\delta$
  such that $\alg C(S,T;\delta)$ holds. This $\delta$ is the
  \emph{commutator} of $S$ and $T$, and is denoted $[S,T]$. The
  commutators for the other centrality relations are denoted
  $[S,T]_{\alg S}$, $[S,T]_{\alg {SR}}$, and $[S,T]_{\alg W}$.
  
  The properties of these centrality relations are coverered in
  Theorem~2.19 and Theorem~3.4 of~\cite{KearnesKiss2013}. Much stronger
  properties hold in congruence modular varieties;
  see~\cite{FreeseMcKenzie1987}.
  
  \section*{Exercises}
  
  \begin{exercises}
  
  \prob
  As defined in \cite{HobbyMcKenzie1988}, $\beta$ is \emph{strongly
  Abelian} over $\delta$ ($\delta \leq \beta$, both congruences on $\alg A$)
  if the following implication holds for all polynomials $f$ and all
  elements $x_0, \ldots, x_{n-1}$, $y_0, \ldots, y_{n-1}$, and
  $z_1, \ldots, z_{n-1}$ with $x_0 \mathrel\beta y_0$ and
  $x_i \mathrel\beta y_i \mathrel\beta z_i$, $i = 1, \ldots n-1$.
  \begin{align*}
  f(x_0,\ldots,&x_{n-1}) \mathrel\delta f(y_0,\ldots,y_{n-1}) \\
  &\implies
  f(x_0, z_1,\ldots,z_{n-1}) \mathrel\delta f(y_0, z_1,\ldots,z_{n-1})
  \end{align*}
  Show that $\beta$ is strongly
  Abelian over $\delta$ if and only if $\alg S(\beta,\beta;\delta)$
  holds, and also show this is in turn equivalent to
  $\alg {SR}(\beta,\beta;\delta)$.
  \end{exercises}
  
  
  
  %%%%%%% wjd: old stuff (will delete eventually)%%%%%%%%%%%%%%%%%
  
  \noindent \emph{TODO: the remaining sections about the algorithm will be completely
  rewritten and mostly deleted.  Obviously we don't want to build up the
  algorithm in the way described.  Instead, we'll use recursion.}
  
  \subsection{Sets of size three}
  a \ldto for the set
  \begin{equation*}
  \{(a_0,b_0, 0), (a_1, b_1, 0), (a_2, b_2, 0)\}
  \end{equation*}
  is the first projection, $t(x,y,z) = x$.
  a \ldto for the set
  \begin{equation*}
  \{(a_0,b_0,1), (a_1, b_1, 1), (a_2, b_2, 1)\}
  \end{equation*}
  is the third projection, $t(x,y,z) = y$.
  There are two other forms of 3-sets to consider.
  We label these $P_3$ and $Q_3$ and handle them with
  the following subroutines.
  % They are
  % \begin{align*}
  % P &= \{(a_0, b_0, 0), (a_1, b_1, 0),  (a_2, b_2, 1)\} \text{ and }\\
  % Q &= \{(a_0, b_0, 1), (a_1, b_1, 1), (a_2, b_2, 0)\}.
  % \end{align*}
  % $P = \{(a_0, b_0, 0), (a_1, b_1, 0),  (a_2, b_2, 1)\}$
  % and $Q = \{(a_0, b_0, 1), (a_1, b_1, 1), (a_2, b_2, 0)\}$.
  
  
  \noindent \underline{\textbf{Subroutine \ld-3.0}}\\[4pt]
  To find a \ldto for
  $P_3:=\{(a_0, b_0, 0), (a_1, b_1, 0),  (a_2, b_2, 1)\}$,
  % $\{(a_0, b_0, 0), (a_1, b_1, 1),  (a_2, b_2, 0)\}$,
  \begin{enumerate}
  \item use Subroutine \ld-2 to compute a \ldto $s$ for
  \begin{equation*}
  \{(a_1, b_1, 0), (a_2, b_2, 1)\};
  \end{equation*}
  \item use Subroutine \ld-2 to compute a \ldto $t$ for
  \begin{equation*}
  \{(a_0, s(a_0, b_0, b_0), 0), (a_2, b_2, 1)\}.
  \end{equation*}
  \end{enumerate}
  It is easy to check that
  $d(x,y,z) = t(x, s(x,y,y), s(x,y,z))$
  is then a \ldto for $P_3$
  (see Case $\chi_0=0$ in the proof of Theorem~\ref{thm:local-diff-terms}).
  
  
  \noindent \underline{\textbf{Subroutine \ld-3.1}}\\[4pt]
  To find a \ldto for
  $Q_3 := \{(a_0, b_0, 1), (a_1, b_1, 1), (a_2, b_2, 0)\}$,
  \begin{enumerate}
  \item \label{item:001-1}
  use Subroutine \ld-2 to compute a \ldto $s$ for the set
  \begin{equation*}
  \{(a_1, b_1, 1), (a_2, b_2, 0)\};
  \end{equation*}
  \item \label{item:001-2} use Subroutine \ld-2 to compute a \ldto $t$
  for the set
  \begin{equation*}
  \{(s(a_0, a_0, b_0), b_0, 1),  (a_2,a_2,0)\}.
  \end{equation*}
  \end{enumerate}
  Then
  %%%
  $d(x,y,z) = t(s(x,y,z), s(y,y,z),z)$
  %%%
  is a \ldto  for $Q_3$ (see Case $\chi_0=1$ in the proof of Theorem~\ref{thm:local-diff-terms}).
  
  \subsection{Sets of Size 4} We handle one more case before
  using induction to give a general recursive algorithm.
  The nontrivial forms of 4-sets are
  \begin{align*}
  P_4 &:= \{(a_0, b_0, 0), (a_1, b_1, 0),  (a_2, b_2, 0),  (a_3, b_3, 1)\},\\
  Q_4 &:= \{(a_0, b_0, 1), (a_1, b_1, 1), (a_2, b_2, 1), (a_3, b_3, 0)\},\\
  R_4 &:= \{(a_0, b_0, 0), (a_1, b_1, 0),  (a_2, b_2, 1),  (a_3, b_3, 1)\}.
  \end{align*}
  
  \medskip
  
  \noindent \underline{\textbf{Subroutine \ld-4.0}}\\[4pt]
  To find a \ldto for a set like $P_4$,
  % $\{(a_0, b_0, 0), (a_1, b_1, 1),  (a_2, b_2, 0)\}$,
  \begin{enumerate}
  \item use Subroutine \ld-3.0 to compute a \ldto $s$ for
  \begin{equation*}
  \{(a_1, b_1, 0),  (a_2, b_2, 0),  (a_3, b_3, 1)\};
  \end{equation*}
  \item use Subroutine \ld-2 to compute a \ldto $t$ for
  \begin{equation*}
  \{(a_0, s(a_0, b_0, b_0), 0), (a_3, b_3, 1)\}.
  \end{equation*}
  \end{enumerate}
  It is easy to check that
  $d(x,y,z) = t(x, s(x,y,y), s(x,y,z))$
  is then a \ldto for $P_4$
  (see Case $\chi_0=0$ in the proof of Theorem~\ref{thm:local-diff-terms}).
  
  \medskip
  
  \noindent \underline{\textbf{Subroutine \ld-4.1}}\\[4pt]
  To find a \ldto for a set like $Q_4$
  % $\{(a_0, b_0, 0), (a_1, b_1, 1),  (a_2, b_2, 1)\}$,
  \begin{enumerate}
  \item
  use Subroutine \ld-3.1 to compute a \ldto $s$ for the set
  \begin{equation*}
  \{(a_1, b_1, 1), (a_2, b_2, 1), (a_3, b_3, 0)\};
  \end{equation*}
  \item  use Subroutine \ld-2 to compute a \ldto $t$
  for the set
  \begin{equation*}
  \{(s(a_0, a_0, b_0), b_0, 1),  (a_3,a_3,0)\}.
  \end{equation*}
  \end{enumerate}
  Then
  %%%
  $d(x,y,z) = t(s(x,y,z), s(y,y,z),z)$
  %%%
  is a \ldto  for $Q_4$ (see Case $\chi_0=1$ in the proof of Theorem~\ref{thm:local-diff-terms}).
  
  
  \medskip
  
  \noindent \underline{\textbf{Subroutine \ld-4.2}}\\[4pt]
  To find a \ldto for a set like $R_4$
  % $\{(a_0, b_0, 0), (a_1, b_1, 1),  (a_2, b_2, 1)\}$,
  \begin{enumerate}
  \item
  use Subroutine \ld-3.1 to compute a \ldto $s$ for the set
  \begin{equation*}
  \{(a_1, b_1, 0), (a_2, b_2, 1), (a_3, b_3, 1)\};
  \end{equation*}
  \item use Subroutine \ld-2 to compute a \ldto $t$ for
  \begin{equation*}
  \{(a_0, s(a_0, b_0, b_0), 0), (a_3, b_3, 1)\}.
  \end{equation*}
  \end{enumerate}
  Then
  $d(x,y,z) = t(x, s(x,y,y), s(x,y,z))$
  is a \ldto for $P_4$.
  The output of Algorithm~\ref{alg:ld-2} is the set $\tabT \sA_0$ of all 
Cayley tables of operations in $\sA_0$, the \ldtos for $\{(a_0, b_0, 0), (a_0, b_0, 1)\}$.
The set $\tabT \sA_0$ is input to Algorithm~\ref{alg:stream-ldt1}.
A second input will be the set of all tables of operations in 
$\dtr (a_0, b_0, 1)$.  To form this set, we take all tables of ternary operations.
(There are $|A|^3$ such tables.) From these, we extract only those whose 
$(a_0, a_0, b_0)$-entry is $b_0$.

the output of which is the set $\tabT \sA_1$ of all tables of 
operations in $\sA_1$.
The latter is input to Algorithm~\ref{alg:stream-ldt2}, the result of which is
the set $\tabT \sA_2$ of tables for $\sA_2$.
Thereafter, the process alternates between
Algorithms~\ref{alg:stream-ldt1} and~\ref{alg:stream-ldt2}, and 
terminates once we find the tables for $\sA_{2n^2-1} = A^2 \times \{0,1\}$,
which are tables of a difference term operations for $\alg A$ (as noted
in Corollary~\ref{cor:loc-diff-term}).

\subsubsection{Induction Stage 1}
Let $\eta_0 := \dtr \{(a_0, b_0, 0), (a_0, b_0, 1)\}$ and, for $0 < i < n^2$, let
\begin{align*}
  \eta_i &:= \dtr (a_i, b_i, 1),\\
  \eta_{\kk} &:= \bigcap_{0\leq i < k}\eta_i
  =\dtr \{(a_0, b_0, 0),(a_0, b_0, 1), (a_1, b_1, 1), \dots, (a_{k-1}, b_{k-1}, 1) \}.
\end{align*}

\RestyleAlgo{boxruled}
%\LinesNumbered
\begin{algorithm}
  \KwIn{$\tabT \eta_{\kk}$ and $\tabT \sA_{2k}$}

  \KwOut{$\tabT \eta_{\kplus}$ and $\tabT \sA_{2k+1}$}

  \ForEach{$\sanst_i \in \tabT \sA_{2k}$} {

    $\tabT_i := \tabT \eta_{\kk} \cap \tabT \dtr (a_{k+1}, \sanst_i[a_{k+1}, b_{k+1}, b_{k+1}], 0)$;

    \ForEach{$\sanss_j \in \tabT_i$} {
      \ForAll {$x$, $y$, $z \in A$}  {
        %define the table $\sanss_{j_i}$ as follows:
        $\sansr_{ij}[x,y,z] :=  \sanss_j[x, \sanst_i[x,y,y], \sanst_i[x,y,z]]$;
      }
      include $\sansr_{ij}$ in the set $\tabT$;
    }
  }
%  Let $\tabT[\sA_{2k+1}] := \{\sanss_{ji}: i, j\}$ be the set of all tables so defined.\\[5pt]
  compute $\tabS  := \bigl\{\sanst \in \tabT \eta_{\kk} \colon \sanst[a_k,a_k,b_k]= b_k \bigr\}$;

  \Return $\tabS$ and $\tabT$.
  \caption{Generate the set of all Cayley tables of \ldtos for $\sA_{2k+1}$ \label{alg:stream-ldt1} {\small ($k\geq 0$)}}
\end{algorithm}

\subsubsection{Implementation of Algorithm~\ref{alg:stream-ldt1}}
Given the sets $\tabT \eta_{\kk}$
and $\tabT \sA_{2k}$, of all tables of operations in 
$\eta_{\kk}$ and $\sA_{2k}$, respectively,
Algorithm~\ref{alg:stream-ldt1} returns the sets
$\tabT \eta_{\kplus}$ and $\tabT \sA_{2k+1}$.
Building the set $\tabT \eta_{\kplus}$ is trivial.  We simply remove 
from $\tabT \eta_{\kk}$ every table whose $(a_k,a_k,b_k)$-entry is not $b_k$.
As for $\tabT \sA_{2k+1}$,
for each table $\sanst_i \in \tabT \sA_{2k}$,
let 
$c_i:= \sanst_i[a_{k+1}, b_{k+1}, b_{k+1}]$, compute
$\gamma := \Cg^{\alg{A}}(a_{k+1}, c_i)$ and $\delta := \com{\gamma}$. % (in $O(rm^2 + n^5)$ time).
Next construct the tables of operations in the set
$\eta_{\kk}\cap \dtr (a_{k+1}, c_i, 0)$, according to the usual
definition: 
$t \in \dtr (a_{k+1},c_i,0)$ iff $a_{k+1} \comr{\gamma} t^{\alg{A}}(a_{k+1},c_i,c_i)$.
Thus 
$\tabT_i = \{\sanst \in \tabT \eta_{\kk} \colon  a_{k+1} \mathrel{\delta} \sanst [a_{k+1}, c_i, c_i]\}$.
For each $\sanss_j \in \tabT_i$, define the operation table $\sansr_{ij}$ as follows:
for all $x$, $y$, $z \in A$, $\sansr_{ij}[x,y,z] :=  \sanss_j[x, \sanst_i[x,y,y], \sanst_i[x,y,z]]$. 
The set of all tables $\{\sansr_{ij} \colon i, j\}$ so defined is $\tabT \sA_{2k+1}$.



\subsubsection{Induction Stage 2}
For $i\geq 0$, let
\begin{align*}
\zeta_i &:= \dtr (a_i, b_i, 0), \\
  \zeta_{\kk} &:= \bigcap_{0\leq i < k}\zeta_i
  =\dtr \{(a_0, b_0, 0), (a_1, b_1, 0), \dots, (a_{k-1}, b_{k-1}, 0)\}.
\end{align*}

\RestyleAlgo{boxruled}
%\LinesNumbered
\begin{algorithm}
  \KwIn{$\tabT \zeta_{\kminus}$ and $\tabT \sA_{2k-1}$}
  \KwOut{$\tabT \zeta_{\kk}$ and $\tabT \sA_{2k}$}

  \ForEach{$\sanst_i \in \tabT \sA_{2k-1}$} {
  
    $\tabT_i := \tabT \zeta_{\kminus}  \cap \tabT \dtr (\sanst_i[a_k, a_k, b_k], b_k, 1)$;

    \ForEach{$\sanss_j \in \tabT_i$} {
      \ForAll {$x$, $y$, $z \in A$}  {
        $\sansr_{ij}[x,y,z] :=  \sanss_{j}[\sanst_i[x,y,z], \sanst_i[y,y,z], z]$;
      }
      include $\sansr_{ij}$ in the set $\tabT$;
    }
  }

  compute $\tabS := \{\sanst \in \tabT \zeta_{\kminus} \colon a_{k-1} \mathrel{\delta_{k-1}} \sanst[a_{k-1},b_{k-1},b_{k-1}]\}$;

  \Return $\tabS$ and $\tabT$.

  \caption{Generate the set of all Cayley tables of \ldtos for $\sA_{2k}$ \label{alg:stream-ldt2} {\small ($k> 0$)}}
\end{algorithm}




\subsubsection{Implementation of Algorithm~\ref{alg:stream-ldt2}}
Given $\tabT \zeta_{\kminus}$
and $\tabT \sA_{2k-1}$---the sets of tables representing operations in
$\zeta_{\kminus}$ and $\sA_{2k-1}$, respectively---Algorithm~\ref{alg:stream-ldt2} 
returns the sets $\tabT \zeta_{\kk}$ and $\tabT \sA_{2k}$.
To build $\tabT \sA_{2k}$, take each table $\sanst_i$ from $\tabT \sA_{2k-1}$
and set $c_i:= \sanst_i[a_k, a_k, b_k]$. Then, let $\tabT_i$ denote 
the set of tables in $\tabT \zeta_{\kminus}$ whose 
$(c_i, c_i, b_k)$-entry is equal to $b_k$. 
For each $\sanss_j \in \tabT_i$, define the operation table 
$\sansr_{ij}$ as follows:
for all $x$, $y$, $z \in A$, 
$\sansr_{ij}[x,y,z] :=  \sanss_{j}[\sanst_i[x,y,z], \sanst_i[y,y,z], z]$. 
The set of all tables so defined is $\tabT \sA_{2k}$.
Finally, to build the set $\tabT \zeta_{\kk}$, we 
compute $\gamma_{k-1} := \Cg^{\alg{A}}(a_{k-1}, b_{k-1})$ and $\delta := \com{\gamma_{k-1}}$,
and then remove from $\tabT \zeta_{\kminus}$ every table whose 
$(a_{k-1},b_{k-1},b_{k-1})$-entry is not congruent to $a_{k-1}$ modulo $\delta_{k-1}$.






\subsection{Implementation and Complexity of Algorithm 1}
\label{sec:cc-ld-2}
% Here we discuss the implementation details and analyze the computational 
% cost of Algorithm~\ref{alg:ld-2}, using the notation introduced in
% Section~\ref{sec:algorithm-its-time}.  
Recall the notation introduced in Section~\ref{sec:algorithm-its-time}: 
\begin{align*}
  n &= |A|, \quad m=\|\alg A\| = \sum_{i=0}^r k_i n^i,\\
k_i&= \text{ the number of basic operations of arity~$i$},\\ 
r &= \text{ the largest arity of the basic operations of $\alg{A}$.}
\end{align*}
Also, $\Cayley_{(a_0,a_1),(b_0,a_1),(b_0,b_1)|(u,v)}$ denotes the Cayley table of a
term operation that generates $(u,v)$ from the set $\{(a_0,a_1),(b_0,a_1),(b_0,b_1)\}$.
% The complexity bounds in this section are easily derivable 
% from Proposition~\ref{speedprop}.

Algorithm~\ref{alg:ld-2} can be implemented as follows:
\begin{enumerate}
\item compute $\thetao$, in time $O(rm)$;
\item compute $C_0= a_0/\com{ \thetao } \times \{b_1\}$,
in time $O(rm^2 + n^5)$;
\item generate $S_0=\Sg^{\alg{A}\times \alg{A}} \{(a_0,a_1),(b_0,a_1),(b_0,b_1)\}$,
  in time $O(r m^2)$;\\
for each newly generated $(u,v) \in S_0$, 
  \begin{itemize}
  \item construct and store the table
    $\Cayley_{(a_0,a_1),(b_0,a_1),(b_0,b_1)|(u,v)}$;
  \item if $(u,v) \in C_0$, then add $\Cayley_{(a_0,a_1),(b_0,a_1),(b_0,b_1)|(u,v)}$
   to the list of tables to be returned; 
  \end{itemize}
\end{enumerate}

%% (There are at most $|A|^2$ ($\geq |S_0|$) such tables, and,
%% (Since the operations are ternary, each table has size $|A|^3$, so
%% the cost of storing a table at each step is $n^3$.) 
Each element $(u,v)\in S_0$ is the result of applying some (say, $k$-ary)
basic operation $f$ to previously generated pairs $(u_1, v_1)$, $\dots$, $(u_k,v_k)$
from $S_0$, and the operation tables generating these pairs were
already stored (the first bullet of Step 3).  Thus, to compute the table
for the operation that produced $(u,v) = f((u_1,v_1),\dots, (u_k,v_k))$
we simply compose $f$ with previously stored operation tables.
Since all tables represent ternary operations, the time-complexity of this composition
is $|A|^3$-steps multiplied by $k$ reads per step; that is,
$kn^3 \leq rn^3$.

All told, the computational complexity of a single run of
Algorithm~\ref{alg:ld-2} is $O(rm + rm^2 + n^5 + r^2m^2n^3) = O(n^5 + r^2m^2n^3)$.
We only need to run the algorithm once since, as soon as we find all the
\ldto tables for elements generated by one triple of the form $((a,a'),(b,a'),(b,b'))$,
we move on to Algorithms~\ref{alg:stream-ldt1} and~\ref{alg:stream-ldt2}.




\section{Notes on alternative approaches and algorithms}
\subsection{A straight-forward (but intractable) algorithm}
Define $\mathbf{c}$, $\mathbf{g}_0$, $\mathbf{g}_1$, $\mathbf{g}_2 \in \alg{A}^{2n}$ 
and $R \subseteq A^{2n}\times A^{2n}$ as follows:
\begin{equation}
\begin{array}{cccccccccc}
  \mathbf{g}_0 &= (a_0 & a_0 &a_1 &a_1 &a_2 &a_2 & \dots & a_{n-1} & a_{n-1}) \\
  \mathbf{g}_1 &= (b_0 & a_0 &b_1 &a_1 &b_2 &a_2 & \dots & b_{n-1} & a_{n-1}) \\
  \mathbf{g}_2 &= (b_0 & b_0 &b_1 &b_1 &b_2 &b_2 & \dots & b_{n-1} & b_{n-1})\\[4pt]
    R  & = (\delta_0 & =  & \delta_1 & = & \delta_2 & = & \dots  & \delta_{n-1} & = \; )\\ [4pt]
  \mathbf{c}            & = (a_0     & b_0       & a_1      & b_1      & a_2      & b_2      & \dots  & a_{n-1}, & b_{n-1}) \\
\end{array}
\end{equation}
where $\delta_k= \com{\theta_k}$.  We want a term $t$ such that $t^{\alg{A}^{2n}}(\mathbf{g}_0, \mathbf{g}_1, \mathbf{g}_2)\mathrel{R} \mathbf{c}$.
In other terms, we want that $t^{\alg{A}^{2n}}(\mathbf{g}_0, \mathbf{g}_1, \mathbf{g}_2)$ belongs to the set
\[
\mathbf{C} := a_0/\delta_0 \times \{b_0\} \times a_1/\delta_1 \times \{b_1\} \times   \dots \times a_{n-1}/\delta_{n-1} \times \{b_{n-1}\}
\]
\subsubsection{An {\small EXPTIME} algorithm}
 \begin{enumerate}
   \item For $0\leq k < n$,
   \begin{enumerate}
     \item compute $\theta_k:= \Cg^{\alg{A}}(a_k, b_k)$ in time $O(rmn)$;
     \item compute $\com{ \theta_k }$ in time $O(rnm^2 + n^6)$;
   \end{enumerate}
\item Generate $S=\Sg^{\alg{A}^{2n}} \{\mathbf{g}_0, \mathbf{g}_1, \mathbf{g}_2\}$,
  in time $c r m^{2n}$ {\tiny {\color{red} $\quad \leftarrow$ EXPTIME}}\\
for each newly generated $\mathbf{u} \in S$, 
  \begin{itemize}
  \item construct/store the table
    $\Cayley_{\mathbf{g}_0, \mathbf{g}_1, \mathbf{g}_2|\mathbf{u}}$;
  \item if $\mathbf{u} \in \mathbf{C}$, then return     $\Cayley_{\mathbf{g}_0, \mathbf{g}_1, \mathbf{g}_2|\mathbf{u}}$;
  \end{itemize}
\end{enumerate}
Notice that, if $\mathbf{u} \in \mathbf{C}$, then $\Cayley_{\mathbf{g}_0, \mathbf{g}_1, \mathbf{g}_2|\mathbf{u}}$  is the table of a difference term operation.

\subsection{Can we make the {\small EXPTIME} algorithm tractable?}
We could partially produce the subalgebra of $\alg{A}^{\alg{A}^3}$ generated by
the three $|A|^3$-tuples,
% $\mathbf{g}^+_0 = (a_0, a_1, \dots)$,
% $\mathbf{g}^+_1 = (b_0, a_1, \dots)$,
% $\mathbf{g}^+_2 = (b_0, b_1, \dots)$,
\begin{align*}
\mathbf{g}^+_0 &= (a_0, a_0, \dots) = \mathbf{g}_0 :: \dots,\\
\mathbf{g}^+_1 &= (b_0, a_0, \dots) = \mathbf{g}_1 :: \dots,\\
\mathbf{g}^+_2 &= (b_0, b_0, \dots) = \mathbf{g}_2 :: \dots.
\end{align*}
Each $\mathbf{g}_i$ $(i=0,1,2)$ is extended so that every 
triple from $A^3$ is represented as $(\mathbf{g}^+_0(i), \mathbf{g}^+_1(i), \mathbf{g}^+_2(i))$ for some 
unique $i$.  Producing the entire subuniverse will be expensive in 
general, but we only need to pay attention to the first $2n$ coordinates. 
At each step, we check for a tuple $\mathbf{u} = (u_0, u_1, u_2, \dots, u_{2n-1})\in \mathbf{C}$.  
If we find one, then the tuple $\mathbf{u}$ stores
the table for a difference term operation and we can stop.

At the end of each of these closure steps, we check for 
tuples $v = (v_0, v_1, \dots)$ for which the pair $(v_0, v_1)$ has not
yet appeared.  If no new pairs of this form are found, the algorithm halts
and we conclude that no \ldto exists for this pair.
  % The runtime of this procedure is bounded by $|A|^2$.)

\section{Other practical considerations}
\subsection{Simplified description of Section 5 procedure}
To start off the algorithm requires three initial steps.
\begin{enumerate}
  \item Compute $\theta_0 := \Cg^{\alg{A}}(a_0, b_0)$ and $\delta_0 := \com{\theta_0}$.
  \item Construct the list $L_0$ of all Cayley tables of operations $p$ 
  such that $p(a_0, b_0, b_0) \mathrel{\delta_0} a_0$; that is, the 
  $(a_0, b_0, b_0)$-entry in each operation table is congruent to $a_0$ modulo $\com{\theta_0}$.
  \item Let $L_1$ be the list $L_0$ filtered on the predicate,
  $\mathsf{C}[a_0, a_0, b_0] = b_0$, that the $(a_0, a_0, b_0)$-entry of 
  the table is $b_0$.
\end{enumerate}
  % \item Compute $\theta_1 := \Cg^{\alg{A}}(a_1, b_1)$ and $\delta_1 := \com{\theta_1}$.
  % \item Let $L_2$ be the list $L_1$ filtered on the predicate 
  % $\mathsf[a_1, b_1, b_1] \mathrel{\delta_1} a_1$.
  % \item Let $L_3$ be the list $L_2$ filtered on the predicate,
  % $\mathsf{C}[a_1, a_1, b_1] = b_1$.
  % \item Compute $\theta_2 := \Cg^{\alg{A}}(a_2, b_2)$ and $\delta_2 := \com{\theta_2}$.
  % \item Let $L_4$ be the list $L_3$ filtered on the predicate 
  % $\mathsf[a_2, b_2, b_2] \mathrel{\delta_2} a_2$.
  % \item Let $L_5$ be the list $L_4$ filtered on the predicate,
  % $\mathsf{C}[a_2, a_2, b_2] = b_2$, $\dots$
Thereafter these three steps are iterated, for $k> 0$.
\begin{enumerate}
  \item Compute $\theta_{k} := \Cg^{\alg{A}}(a_k, b_k)$ and $\delta_{k} := \com{\theta_{k}}$.
  \item Let $L_{2k} := \{\mathsf{C} \in L_{2k-1} \colon \mathsf{C}[a_k, b_k, b_k] \mathrel{\delta_{k}} a_k \}$.
  \item Let $L_{2k+1} := \mathbf{filter} \; L_{2k} \; \mathbf{with} \; \mathsf{C}[a_k, a_k, b_k] = b_k$.
\end{enumerate}
The second and third steps are simple filtering operations:
\begin{itemize}
  \item Let $L_{2k} = \mathbf{filter}\; L_{2k-1} \; \mathbf{with} \; \mathsf{C}[a_k, b_k, b_k] \mathrel{\delta_{k}} a_k$.
  \item Let $L_{2k+1} = \mathbf{filter} \; L_{2k} \; \mathbf{with} \; \mathsf{C}[a_k, a_k, b_k] = b_k$.
\end{itemize}

This algorithm can be implemented as follows:
 \begin{enumerate}
\item compute $\theta_0$ in time $crm$;
\item compute $\com{ \theta_0 }$ in time $c(rm^2 + n^5)$;
\item generate $S_0=\Sg^{\alg{A}\times \alg{A}} \{(a_0,a_1),(b_0,a_1),(b_0,b_1)\}$,
  in time $c r m^2$;\\
for each newly generated $(u,v) \in S_0$, 
  \begin{itemize}
  \item construct and store the table
    $\Cayley_{(a_0,a_1),(b_0,a_1),(b_0,b_1)|(u,v)}$;
  \item if $(u,v) \in C_0$, then add $\Cayley_{(a_0,a_1),(b_0,a_1),(b_0,b_1)|(u,v)}$
   to the list of tables to be returned; 
  \end{itemize}
\end{enumerate}

%\bibliographystyle{amsplain} %% or amsalpha
%% \bibliographystyle{alpha-url}
%% \printbibliography
\bibliographystyle{alphaurl}
\bibliography{inputs/refs}

\end{document}
%%%%%%%%%%%%%%%%%%%%%%%%%%%%%%%%%%%%%%%%%%%%%%%%%%%%%%%%%%%%%%%%%%%%%%%%%%%%%%%%%%%%
%%%%%%%%%%%%%%%%%%%%%%%%%%%%  END OF DOCUMENT %%%%%%%%%%%%%%%%%%%%%%%%%%%%%%%%%%%%%%
%%%%%%%%%%%%%%%%%%%%%%%%%%%%%%%%%%%%%%%%%%%%%%%%%%%%%%%%%%%%%%%%%%%%%%%%%%%%%%%%%%%%

















\begin{comment}
\noindent \underline{\textbf{Subroutine \ld-2'}}\\[4pt]
To compute a \ldto for
$((a_0,b_0,1), (a_1, b_1, 0))$, obviously this is symmetric to
the situation handled in Subroutine LD2 and so the general algorithm
is the same.  Nonetheless, we include a listing of the computational
steps required so that later we can easily refer to this special case
of the general algorithm.
\begin{enumerate}[{\bf 1}]
\item Compute $\delta_1=\com{\thetaone}$;
\item form $C_1= \{b_0\}\times a_1/\delta_1 \leq \alg{A}\times\alg{A}$;
\item compute
      $S_1=\Sg^{\alg{A}\times \alg{A}} ((a_0,a_1),(a_0,b_1),(b_0,b_1))$;
\item find a term operation $t$ of $\alg{A}$ satisfying
%\[t((a_0,a_1),(a_0,b_1),(b_0,b_1)) \in C \cap S.\]
\[t^{\alg{A}\times\alg{A}}((a_0,a_1),(a_0,b_1),(b_0,b_1)) =
 (t^{\alg{A}}(a_0,a_0,b_0), t^{\alg{A}}(a_1,b_1,b_1)) \in C_1 \cap S_1.\]
\end{enumerate}
Then $t$ is a \ldto for
$((a_0, b_0, 1), (a_1, b_1, 0))$.
\end{comment}




\begin{comment}
In practice, there are a number of different ways we could structure this
algorithm when implementing it in software. In principle, the ordering of the
first two steps is inconsequential.
For instance, we might first present
$\Sg^{\alg{A}\times \alg{A}} ((a_0,a_1),(b_0,a_1),(b_0,b_1))$
as a stream $S_0$, then compute $\delta_0 = \com{ \thetao }$, and finally
filter $S_0$ against the predicate $s \in (a_0/\delta_0) \times \{b_1\}$;
the result is a stream of \ldtos.  This would be sufficient if our goal were merely
to decide whether a difference term exists.  However, the algorithm calls for
returning the table of a \ldto $t$.
Therefore,
\end{comment}









\newpage
\appendix

\section{Miscellaneous Proofs}
\subsection{Difference term identity verifications}
\label{app:dt-ids}

\begin{lemma}[Verifies Subroutine \ild-0]
If $p$ is a \ldto for
\begin{equation*}
P_{j-1} = \{(a_0, b_0, 1), (a_1, b_1, 0), \dots, (a_{j-1}, b_{j-1}, 0)\}.
\end{equation*}
and $t$ is a \ldto for
\begin{equation*}
\{(a_0, b_0, 1), (a_j, p(a_j, b_j, b_j), 0)\},
\end{equation*}
then $d(x,y,z) = t(x, p(x,y,y), p(x,y,z))$ is a \ldto for
$P_{j} = P_{j-1}  \cup \{(a_j, b_j, 0)\}$.
\end{lemma}

\begin{proof} There are three sorts of cases to check.
\begin{enumerate}[1.]
\item $(a_0, b_0, 1)$:
\begin{equation*}
d(a_0, a_0, b_0) =
t(a_0, p(a_0,a_0,a_0), p(a_0,a_0,b_0)) =
t(a_0, a_0, b_0) = b_0
\end{equation*}

\item $(a_k, b_k, 0)$ $(1\leq k < j)$:
\begin{equation*}
d(a_k, b_k, b_k) =
t(a_k, p(a_k,b_k,b_k), p(a_k,b_k,b_k))
\comr{\thetak}
t(a_k, a_k, a_k)  = a_k
\end{equation*}

\item $(a_j, b_j, 0)$:
If
$\thetaj := \Cg(a_j, b_j)$ and
$\gammaj := \Cg(a_j, p(a_j, b_j, b_j))$, then
\begin{equation*}
d(a_j, b_j, b_j) =
t(a_j, p(a_j,b_j,b_j), p(a_j,b_j,b_j))
\comr{\gammaj} a_j,
\end{equation*}
so
$d(a_j, b_j, b_j) \comr{\thetaj} a_j$,
since $\gammaj \leq \thetaj$.
\end{enumerate}
\end{proof}


\begin{lemma}[Verifies Subroutine \ild-1]
If $q$ is a \ldto for
\begin{equation*}
Q_{j-1} = \{(a_0, b_0, 0), (a_1, b_1, 1), \dots, (a_{j-1}, b_{j-1}, 1)\}.
\end{equation*}
and $t$ is a \ldto for
\begin{equation*}
\{(a_0, b_0, 0), (q(a_j, a_j, b_j), b_j, 1)\},
\end{equation*}
then $d(x,y,z) = t(q(x,y,z), q(y,y,z), z)$ is a \ldto for
$Q_{j} = Q_{j-1}  \cup \{(a_j, b_j, 1)\}$.
\end{lemma}

\begin{proof} There are three sorts of cases to check.
\begin{enumerate}[1.]
\item $(a_0, b_0, 0)$:
\begin{equation*}
d(a_0, b_0, b_0) =
t(q(a_0,b_0,b_0), q(b_0,b_0,b_0), b_0)
\comr{\thetao}
t(a_0, b_0, b_0) \comr{\thetao}
a_0
\end{equation*}

\item $(a_k, b_k, 1)$ $(1\leq k < j)$:
\begin{equation*}
d(a_k, a_k, b_k) =
t(q(a_k,a_k,b_k), q(a_k,a_k,b_k), b_k)
= t(b_k, b_k, b_k) = b_k
\end{equation*}

\item $(a_j, b_j, 1)$:
\begin{equation*}
d(a_j, a_j, b_j) =
t(q(a_j,a_j,b_j), q(a_j,a_j,b_j), b_j) = b_j
\end{equation*}
\end{enumerate}
\end{proof}










\subsection{Algorithm for constructing a difference term operation}
\label{sec:algor-2}
We are now in a position to describe a complete procedure for constructing
a difference term operation for a given algebra $\alg{A}$.

In this section we work with \emph{lists}
(aka sequences, aka tuples) of triples,
such as $((a_0, b_0, 0), (a_1, b_1, 0))$, rather than \emph{sets} of triples,
simply because for practical (computer implementation) purposes lists are a little
easier to work with than sets.
% we refer to an ``\ldto for a tuple $(s_0)$

First, let
% $\begin{equation}
%% $as = ((a_0, b_0), (a_1, b_1), \dots, (a_k, b_k))$
$((a_0, b_0), (a_1, b_1), \dots, (a_k, b_k))$
% \end{equation}
be a list of all pairs in the set $A\times A$.
% \footnote{Actually, we only need
% $(A \times A) \mysetminus \{(a,a) \mid a \in A\}$, but we will elide this
% detail in our description of the algorithm since it doesn't impact
% complexity.}
Let
\begin{equation*}
xs := %as \; zip \; (0,0,\dots, 0) =
((a_0, b_0, 0), (a_1, b_1, 0), \dots, (a_k, b_k, 0)),
\end{equation*}
and let
% $xs_1 = xs \; zip \; (1,1, \dots, 1)$.
\begin{equation*}
ys := %as \; zip \; (1, 1, \dots, 1) =
((a_0, b_0, 1), (a_1, b_1, 1), \dots, (a_k, b_k, 1)).
\end{equation*}


Using Alg~\ref{alg:ld-2} for the base case
and Alg~\ref{alg:ild1}
for the induction step,
we compute a \ldto $s_0$ for
\begin{equation*}
((a_0, b_0, 0), (a_1, b_1, 1), (a_2, b_2, 1), \dots, (a_k, b_k, 1)),
\end{equation*}
and a \ldto $t_1$ for
\begin{equation*}
((s_0(a_0, a_0, b_0), b_0, 1), (a_0, b_0, 0),(a_1, b_1, 0)).
\end{equation*}
The operation $s_1(x,y,z) := t_1(s_0(x,y,z), s_0(y,y,z), z)$ will then be
a \ldto for
% \begin{equation*}
$((a_0, b_0, 0), (a_1, b_1, 0)) :: ys$  (where $::$ denotes
list concatenation).
% \end{equation*}

Next, using Alg~\ref{alg:ld-2} for the base case
and Alg~\ref{alg:ild0} for the induction step,
we compute a \ldto $t_2$ for
\begin{equation*}
((s_1(a_0, a_0, b_0), b_0, 1), (a_0, b_0, 0),(a_1, b_1, 0),(a_2, b_2, 0)).
\end{equation*}
Then $s_2(x,y,z) := t_2(s_1(x,y,z), s_1(y,y,z), z)$ is a \ldto
for the elements of
% \begin{equation}
$((a_0, b_0, 0), (a_1, b_1, 0), (a_2, b_2, 0)) :: ys$.
% \end{equation}

Proceeding inductively, for fixed $1<j\leq k$,
if $s_{j-1}$ is a \ldto
for the elements in the list
\begin{equation*}
((a_0, b_0, 0), (a_1, b_1, 0), \dots, (a_{j-1}, b_{j-1}, 0)) :: ys,
\end{equation*}
and if we use Algorithms~\ref{alg:ild0} and~\ref{alg:ild1}
to compute a \ldto $t_j$ for
\begin{equation*}
((s_{j-1}(a_0, a_0, b_0), b_0, 1), (a_0, b_0, 0),(a_1, b_1, 0),\dots,
(a_j, b_j, 0)),
\end{equation*}
then $s_j(x,y,z) := t_j(s_{j-1}(x,y,z), s_{j-1}(y,y,z), z)$ will be a
\ldto for
$((a_0, b_0, 0), (a_1, b_1, 0), \dots, (a_j, b_j, 0)) :: ys$.
In the end we arrive at a \ldto for $xs :: ys$, namely,
\begin{equation*}
d(x,y,z) = t_k(s_{k-1}(x,y,z), s_{k-1}(y,y,z), z).
\end{equation*}
Since $xs :: ys$ contains all of the triples $(a,b,\chi)$, where
$a, b \in A$ and $\chi \in \{0,1\}$,
$d(x,y,z)$ is the desired difference term operation for $\alg{A}$.







compute $\delta_0$ and $\delta_1$ (see~(\ref{eqn:notation1}) for definition);

compute the stream $\sS_0$ of term operations $s$ that satisfy $s(a_0,a_0,b_0) = b_0$;

find a term operation $p$ in $\sS_0$ satisfying $p(a_0, b_0, b_0) \mathrel{\delta_0} a_0$;

compute $c_1 = p(a_1,b_1,b_1)$;

find a term operation $q$ in $\sS_0$ satisfying $q(a_1, c_1, c_1) \mathrel{\delta_1} a_1$;

return $(s_0, \sS_0)$ where $s_0(x,y,z) = q(x,p(x,y,y),p(x,y,z))$
\end{algorithm}

































\subsection{A more efficient algorithm}
\label{sec:algor-3}
In this section we refine the algorithm from Section~\ref{sec:algor-2}
for constructing a difference term operation for a given finite idempotent
algebra $\alg{A}$.
The point here is to describe a version of the algorithm that is more practical
and resembles how we might realistically implement it in a computer
programming language.

In addition to notational conventions above, such as
\begin{equation}
\label{eqn:notation1}
\thetai = \Cg^{\alg{A}}(a_i, b_i) \; \text{ and } \;
\deltai=\com{\thetai},
\end{equation}
we also take
$as = ((a_0, b_0), (a_1, b_1), \dots, (a_k, b_k))$
to be the list of all pairs in the set $A\times A$.
Also define
\begin{align*}
\sA_0 &= \{(a_0, b_0, 0), (a_0, b_0, 1)\},\\
\sA_1 &= \{(a_0, b_0, 0), (a_0, b_0, 1), (a_1,b_1,0)\},\\
\sA_2 &= \{(a_0, b_0, 0), (a_0, b_0, 1), (a_1,b_1,0), (a_1,b_1,1)\},\\
&\vdots\\
\sA_{2k} &= \{(a_0, b_0, 0), (a_0, b_0, 1), \dots, (a_k,b_k,0), (a_k,b_k,1)\}.
\end{align*}
That is,
$\sA_{2k} = \sA_{2k-1} \cup \{(a_k,b_k,1)\}$ and
$\sA_{2k+1} = \sA_{2k} \cup \{(a_{k+1},b_{k+1},0)\}$.

\RestyleAlgo{boxruled}
\LinesNumbered
\begin{algorithm}%[ht]
  \SetKwInOut{Input}{Input}
  \SetKwInOut{Output}{Output}

  \caption{return \ldto for $\sA_1$
  \label{alg:stream-ldt1}  }
% \Input{$as$ = the list of all pairs in $A \times A$}
% \Output{a difference term operation for $\alg A$}

compute $\delta_0$ and $\delta_1$ (see~(\ref{eqn:notation1}) for definition);

compute the stream $\sS_0$ of term operations $s$ that satisfy $s(a_0,a_0,b_0) = b_0$;

find a term operation $p$ in $\sS_0$ satisfying $p(a_0, b_0, b_0) \mathrel{\delta_0} a_0$;

compute $c_1 = p(a_1,b_1,b_1)$;

find a term operation $q$ in $\sS_0$ satisfying $q(a_1, c_1, c_1) \mathrel{\delta_1} a_1$;

return $(s_0, \sS_0)$ where $s_0(x,y,z) = q(x,p(x,y,y),p(x,y,z))$
\end{algorithm}


% \RestyleAlgo{boxed}
\RestyleAlgo{boxruled}
\LinesNumbered
\begin{algorithm}
  \SetKwInOut{Input}{Input}
  \SetKwInOut{Output}{Output}

  \caption{return \ldto for $\sA_2$
  \label{alg:stream-ldt2}  }
  \Input{$(s_0, \sS_0)$ (from Alg~\ref{alg:stream-ldt1})}
  \Output{a \ldto for $\sA_2$}

compute the stream $\sS_1$ of term operations $p$
that satisfy $p(s_0(a_1,a_1,b_1), s_0(a_1,a_1,b_1), b_1) = b_1$;

find a term operation $p$ in $\sS_0 \cap \sS_1$ and compute $d_1 = p(a_1,b_1,b_1)$;

find a term operation $q$ in $\sS_1$ satisfying $q(a_1, d_1, d_1) \mathrel{\delta_1} a_1$.

return $(\sS_0, \sS_1, s_1)$ where $s_1(x,y,z) = q(x, p(x,y,y), p(x,y,z))$
\end{algorithm}


% \RestyleAlgo{boxed}
\RestyleAlgo{boxruled}
\LinesNumbered
\begin{algorithm}
  \SetKwInOut{Input}{Input}
  \SetKwInOut{Output}{Output}

  \caption{return \ldto for $\sA_3$
  \label{alg:stream-ldt3}  }
  \Input{$(s_1, \sS_0, \sS_1)$ (from Alg~\ref{alg:stream-ldt2})}
  \Output{a \ldto for $\sA_3$}

compute $\delta_2$;

compute the substream $\sS_2 \subseteq \sS_0$ of term operations satisfying
 $p(a_1,a_1,b_1) = b_1$;

compute $c_2 = s_1(a_2, b_2, b_2)$;

find a term operation $p$ in $\sS_2$ satisfying
$p(a_2, c_2, c_2) \mathrel{\delta_2} a_2$;

return $(s_2, \sS_0, \sS_1, \sS_2)$ where $s_2(x,y,z) = p(x, s_1(x,y,y), s_1(x,y,z))$
\end{algorithm}


















%%% OLD ALGORITHM June 2017

\subsubsection{Induction steps}
\label{sec:induct}

The next ingredient we need for our construction of a
difference term operation for $\alg{A}$
is
a way to produce a \ld
term operation for a set of the form  (for $j>1$)
\begin{equation}
\label{eqn:Pj}
P_j = \{(a_0, b_0, 1), (a_1, b_1, 0), (a_2, b_2, 0), \dots,
(a_j, b_j, 0)\},
\end{equation}
assuming we're given a \ldto for $P_{j-1} = P_{j} \mysetminus \{(a_j, b_j, 0)\}$.


\begin{comment}
\noindent \underline{\bf Subroutine \ild-0}\\[4pt]
The input, $p$,  is
a \ldto for $P_{j-1} = P_{j} \mysetminus \{(a_j, b_j, 0)\}$.
\begin{enumerate}[1.]
\item
Call Subroutine \ld-2 to
compute a \ldto $t$ for the set
\begin{equation*}
\{(a_0, b_0, 1), (a_j, p(a_j, b_j, b_j), 0)\}.
\end{equation*}
\item Return the following
\ldto for $P_j$:
\begin{equation}
d(x,y,z) = t(x, p(x,y,y), p(x,y,z)).
\label{eq:idl0-dto}
\end{equation}
\end{enumerate}
\qed
\end{comment}




% \RestyleAlgo{boxed}
\RestyleAlgo{boxruled}
\LinesNumbered
\begin{algorithm}%[ht]
  \SetKwInOut{Input}{Input}
  \SetKwInOut{Output}{Output}

  % \caption{Given \ldto for $P_{j-1}$,
\caption{Return a \ldto for the set $P_j$ defined in~(\ref{eqn:Pj})
\label{alg:ild0}}
%   $P_{j-1}=\{(a_0, b_0, 1), (a_1, b_1, 0), (a_2, b_2, 0), \dots,
% (a_{j-1}, b_{j-1}, 0)\}$, compute a \ldto for
 % = P_{j-1} \cup \{(a_j, b_j, 0)\}$
\Input{$p =$ a \ldto for $P_{j-1}$}
\Output{$d =$ a \ldto for $P_j$}

compute a \ldto $t$ for $\{(a_0, b_0, 1), (a_j, p(a_j, b_j, b_j), 0)\}$;

return $d(x,y,z) = t(x, p(x,y,y), p(x,y,z))$

\end{algorithm}


We also need a way to produce a \ld
term operation for a set of the form  (for $j>1$)
\begin{equation}
\label{eqn:Qj}
Q_{j} = \{(a_0, b_0, 0), (a_1, b_1, 1), (a_2, b_2, 1), \dots,
(a_{j}, b_{j}, 1)\},
\end{equation}
assuming we're given a \ldto for
$Q_{j-1} = Q_j \mysetminus \{(a_j, b_j, 1)\}$.

% \RestyleAlgo{boxed}
\RestyleAlgo{boxruled}
\LinesNumbered
\begin{algorithm}%[ht]
  \SetKwInOut{Input}{Input}
  \SetKwInOut{Output}{Output}
  \caption{Return a \ldto for the set $Q_j$ defined in~(\ref{eqn:Qj})
  \label{alg:ild1}}
  \Input{$q =$ a \ldto for $Q_{j-1}$}
  \Output{$d =$ a \ldto for $Q_j$}

  compute a \ldto $t$ for $\{(a_0, b_0, 0), (q(a_j, a_j, b_j), b_j, 1)\}$;

  return $d(x,y,z) = t(q(x,y,z), q(y,y,z), z)$
\end{algorithm}

\begin{remarks}\
\begin{enumerate}[1.]
\item In each of the Algorithms~\ref{alg:ild0} and \ref{alg:ild1},
the first step is a call to Algorithm~\ref{alg:ld-2}
for computing a \ldto for a pair of triples.
\item The easy verification that Algorithm~\ref{alg:ild0} returns
a \ldto for $P_j$ appears in Appendix Section~\ref{app:dt-ids}.
\textcolor{red}{(Decide whether to omit proof.)}
\item The easy verification that Algorithm~\ref{alg:ild1} returns
a \ldto for $Q_j$ appears in Appendix Section~\ref{app:dt-ids}.
\textcolor{red}{(Decide whether to omit proof.)}
\end{enumerate}
\end{remarks}
% (see Case $\chi_0=0$ in the proof of Theorem~\ref{thm:local-diff-terms}).


























With notation and assumptions as above.

\begin{theorem}[cf. Theorem 3.3 of \cite{FreeseValeriote2009}]
This is a test.
\end{theorem}






\bibliographystyle{rsfplain}
%\bibliography{\jobname}
\bibliography{/Users/ralph/tex/bib/Database/db}



A \emph{minority term} (for an algebra or variety) is a
3-variable term $q(x,y,z)$ such if two of the variables
are equal, its value is the other one; that is,
\[
q(x,x,y) \approx q(x,y,x) \approx q(y,x,x) \approx y.
\]
We are interested in an algebraic description of varieties having
a minority term. One possible conjecture is:
\begin{conjecture}
$\mathcal V$ has a minority term if and only if it is CP and its
ring has characteristic~1 or~2.
\end{conjecture}



\begin{fact}
If $\mathcal V$ has a minority term, then it is CP.
\end{fact}

\begin{proof}
Clearly a minority term is a \malcev term.
\end{proof}

\begin{fact}
A CD variety $\mathcal V$ has a minority term if and
only if it is CP.
\end{fact}

\begin{proof}
A variety is CD and CP if and only if it has a Pixley term.
If $p(x,y,z)$ is a Pixley term then
\[
q(x,y,z) = p(p(x,y,z),x,p(x,z,y))
\]
is a minority term. The fact can be derived from these
observations.
\end{proof}

\begin{fact}
A variety of groups has a minority term if and only
the variety has exponent 2 (well, or 1).
\end{fact}

\begin{proof}
If a variety has exponent 2 then it is abelian and so,
using additive notation, $x + y + z$ is a minority term.

If the exponent is not 2 then the variety contains a group
which contains an element of order $n$, where $n > 2$ (or
infinite). This element generates a cyclic group. In any
abelian algebra the \malcev term operation is unique and so
must be $x - y + z$. But one easily checks that this is not
a minority term when $n > 2$.
\end{proof}

\begin{lemma}
Let $\mathcal V$ be a CP variety with a \malcev term $p(x,y.z)$.
The the following are equivalent:
\begin{enumerate}
\item
The ring $R(\mathcal V)$ of $\mathcal V$ has characteristic~2.
\item
On every block of every abelian congruence, $p$ restricted to the
block is a minority term; that is, satisfies $p(a,b,a) = b$.
\item
If $\theta = \textup{Cg}^{\alg F_{\mathcal V}(x,y)}(x,y)$, then,
on each block of $\theta/[\theta,\theta]$, $p$ is a minority term.
\end{enumerate}
If $\mathcal V$ has a minority term, then these conditions hold.
\end{lemma}

\begin{proof}
By commutator theory (give some specific refs, also that the ring
is det by the strucure of $\theta/[\theta,\theta]$)
the \malcev term operation on an
abelian algebra, or even a block of an abelian congruence,
is unique and it is $p(x,y,z) = x - y + z$ for
some abelian group. It is easy that $x - y + z$ is a minority
if and only if the abelian group has exponent~2.
\end{proof}



Let $\alg A$ be an algebra and let $S$ and $T$ be tolerances
on $\alg A$.
Let $M(S,T)$, or $M^{\alg A}(S,T)$ to emphasize $\alg A$,
be the set of all $2 \times 2$ matrices of the form
\begin{equation}\label{eq1}
\begin{bmatrix}
p&q\\
r&s
\end{bmatrix}
=
\begin{bmatrix}
f(\mathbf{a},\mathbf{u})&f(\mathbf{a},\mathbf{v})\\
f(\mathbf{b},\mathbf{u})&f(\mathbf{b},\mathbf{v})
\end{bmatrix}
\end{equation}
where $f(\mathbf{x},\mathbf{y})$ is an $(m+n)$-ary polynomial of
$\alg A$, $\mathbf{a} \mathrel{S} \mathbf{b}$, and
$\mathbf{u} \mathrel{T} \mathbf{v}$
(componentwise, of course). The members of $M(S,T)$ are called
\emph{$S,T$-matrices}.

The first exercise gives an efficient way to find $M(S,T)$.

\section*{Exercises}

\begin{exercises}

\prob
Show that $M(S,T)$ is the subalgebra of $\alg A^4$ generated by
\[
\left\{
\begin{bmatrix}
a&a\\
b&b
\end{bmatrix} : a \mathrel{S} b\right\}
\union
\left\{
\begin{bmatrix}
c&d\\
c&d
\end{bmatrix} : c \mathrel{T} d\right\}
\]

\prob
Use the symmetry of $S$ and $T$ to show the matrix obtained from an
$S,T$-matrix by interchanging the rows or columns (or both) is also
in $M(S,T)$.

\prob
$M(T,T)$ is closed under taking transposes.

\end{exercises}

\section*{Centrality Relations}

We define four kinds of centrality, called centrality, strong
centrality, weak centrality, and strong rectularity. The is a fifth
centrality condition known as rectangularity which we will save for
later.

Let $\delta$ be a congruence and $S$ and $T$ be
tolerance relations on  $\alg A$. The above centrality relations
are denoted $\alg C(S,T;\delta)$ (centrality),
$\alg S(S,T;\delta)$ (strong centrality),
$\alg W(S,T;\delta)$ (weak centrality),  and
$\alg SR(S,T;\delta)$ (strong rectangularity). They hold if the
appropriate implication below holds for all
\[
\begin{bmatrix}
p&q\\
r&s
\end{bmatrix} \in M(S,T)
\]
\begin{itemize}
\item centrality:
$p \mathrel{\delta} q \implies r \mathrel{\delta} s$.
\item strong rectangularity:
$p \mathrel{\delta} s \implies r \mathrel{\delta} s$.
\item weak centrality:
$p \mathrel{\delta} q \mathrel{\delta} s \implies r \mathrel{\delta} s$.
\item strong centrality holds if both centrality and strong
rectangularity hold.
\end{itemize}

Using the exercises it is easy to see that the implication defining
$\alg C(S,T;\delta)$ can be replaced by
$r \mathrel{\delta} s \implies p \mathrel{\delta} q$ and this is
equivalent to
\[
p \mathrel{\delta} q \Longleftrightarrow r \mathrel{\delta} s.
\]
Similar statements hold for the other conditions: weak centrality
is equivalent to saying that if any three of $p$, $q$, $r$ and $s$
are $\delta$ related, then they all are. And strong rectangularity
says that if the elements of the main diagonal, or of the sinister
diagonal, are $\delta$ related, then all four are.

The \emph{$S,T$-term condition} is the condition $\alg C(S,T,0)$,
usually expressed using the right-hand matrix in~\eqref{eq1}.
Other kinds of term conditions are defined similarly.

If $\alg C(S,T;\delta_i)$ holds for all $i \in I$, then
$\alg C(S,T;\Meet_{i\in I}\delta_i)$ holds. Similar statements hold
for the other centrality conditions. So there is a least $\delta$
such that $\alg C(S,T;\delta)$ holds. This $\delta$ is the
\emph{commutator} of $S$ and $T$, and is denoted $[S,T]$. The
commutators for the other centrality relations are denoted
$[S,T]_{\alg S}$, $[S,T]_{\alg {SR}}$, and $[S,T]_{\alg W}$.

The properties of these centrality relations are coverered in
Theorem~2.19 and Theorem~3.4 of~\cite{KearnesKiss2013}. Much stronger
properties hold in congruence modular varieties;
see~\cite{FreeseMcKenzie1987}.

\section*{Exercises}

\begin{exercises}

\prob
As defined in \cite{HobbyMcKenzie1988}, $\beta$ is \emph{strongly
Abelian} over $\delta$ ($\delta \leq \beta$, both congruences on $\alg A$)
if the following implication holds for all polynomials $f$ and all
elements $x_0, \ldots, x_{n-1}$, $y_0, \ldots, y_{n-1}$, and
$z_1, \ldots, z_{n-1}$ with $x_0 \mathrel\beta y_0$ and
$x_i \mathrel\beta y_i \mathrel\beta z_i$, $i = 1, \ldots n-1$.
\begin{align*}
f(x_0,\ldots,&x_{n-1}) \mathrel\delta f(y_0,\ldots,y_{n-1}) \\
&\implies
f(x_0, z_1,\ldots,z_{n-1}) \mathrel\delta f(y_0, z_1,\ldots,z_{n-1})
\end{align*}
Show that $\beta$ is strongly
Abelian over $\delta$ if and only if $\alg S(\beta,\beta;\delta)$
holds, and also show this is in turn equivalent to
$\alg {SR}(\beta,\beta;\delta)$.
\end{exercises}



%%%%%%% wjd: old stuff (will delete eventually)%%%%%%%%%%%%%%%%%

\noindent \emph{TODO: the remaining sections about the algorithm will be completely
rewritten and mostly deleted.  Obviously we don't want to build up the
algorithm in the way described.  Instead, we'll use recursion.}

\subsection{Sets of size three}
a \ldto for the set
\begin{equation*}
\{(a_0,b_0, 0), (a_1, b_1, 0), (a_2, b_2, 0)\}
\end{equation*}
is the first projection, $t(x,y,z) = x$.
a \ldto for the set
\begin{equation*}
\{(a_0,b_0,1), (a_1, b_1, 1), (a_2, b_2, 1)\}
\end{equation*}
is the third projection, $t(x,y,z) = y$.
There are two other forms of 3-sets to consider.
We label these $P_3$ and $Q_3$ and handle them with
the following subroutines.
% They are
% \begin{align*}
% P &= \{(a_0, b_0, 0), (a_1, b_1, 0),  (a_2, b_2, 1)\} \text{ and }\\
% Q &= \{(a_0, b_0, 1), (a_1, b_1, 1), (a_2, b_2, 0)\}.
% \end{align*}
% $P = \{(a_0, b_0, 0), (a_1, b_1, 0),  (a_2, b_2, 1)\}$
% and $Q = \{(a_0, b_0, 1), (a_1, b_1, 1), (a_2, b_2, 0)\}$.


\noindent \underline{\textbf{Subroutine \ld-3.0}}\\[4pt]
To find a \ldto for
$P_3:=\{(a_0, b_0, 0), (a_1, b_1, 0),  (a_2, b_2, 1)\}$,
% $\{(a_0, b_0, 0), (a_1, b_1, 1),  (a_2, b_2, 0)\}$,
\begin{enumerate}
\item use Subroutine \ld-2 to compute a \ldto $s$ for
\begin{equation*}
\{(a_1, b_1, 0), (a_2, b_2, 1)\};
\end{equation*}
\item use Subroutine \ld-2 to compute a \ldto $t$ for
\begin{equation*}
\{(a_0, s(a_0, b_0, b_0), 0), (a_2, b_2, 1)\}.
\end{equation*}
\end{enumerate}
It is easy to check that
$d(x,y,z) = t(x, s(x,y,y), s(x,y,z))$
is then a \ldto for $P_3$
(see Case $\chi_0=0$ in the proof of Theorem~\ref{thm:local-diff-terms}).


\noindent \underline{\textbf{Subroutine \ld-3.1}}\\[4pt]
To find a \ldto for
$Q_3 := \{(a_0, b_0, 1), (a_1, b_1, 1), (a_2, b_2, 0)\}$,
\begin{enumerate}
\item \label{item:001-1}
use Subroutine \ld-2 to compute a \ldto $s$ for the set
\begin{equation*}
\{(a_1, b_1, 1), (a_2, b_2, 0)\};
\end{equation*}
\item \label{item:001-2} use Subroutine \ld-2 to compute a \ldto $t$
for the set
\begin{equation*}
\{(s(a_0, a_0, b_0), b_0, 1),  (a_2,a_2,0)\}.
\end{equation*}
\end{enumerate}
Then
%%%
$d(x,y,z) = t(s(x,y,z), s(y,y,z),z)$
%%%
is a \ldto  for $Q_3$ (see Case $\chi_0=1$ in the proof of Theorem~\ref{thm:local-diff-terms}).

\subsection{Sets of Size 4} We handle one more case before
using induction to give a general recursive algorithm.
The nontrivial forms of 4-sets are
\begin{align*}
P_4 &:= \{(a_0, b_0, 0), (a_1, b_1, 0),  (a_2, b_2, 0),  (a_3, b_3, 1)\},\\
Q_4 &:= \{(a_0, b_0, 1), (a_1, b_1, 1), (a_2, b_2, 1), (a_3, b_3, 0)\},\\
R_4 &:= \{(a_0, b_0, 0), (a_1, b_1, 0),  (a_2, b_2, 1),  (a_3, b_3, 1)\}.
\end{align*}

\medskip

\noindent \underline{\textbf{Subroutine \ld-4.0}}\\[4pt]
To find a \ldto for a set like $P_4$,
% $\{(a_0, b_0, 0), (a_1, b_1, 1),  (a_2, b_2, 0)\}$,
\begin{enumerate}
\item use Subroutine \ld-3.0 to compute a \ldto $s$ for
\begin{equation*}
\{(a_1, b_1, 0),  (a_2, b_2, 0),  (a_3, b_3, 1)\};
\end{equation*}
\item use Subroutine \ld-2 to compute a \ldto $t$ for
\begin{equation*}
\{(a_0, s(a_0, b_0, b_0), 0), (a_3, b_3, 1)\}.
\end{equation*}
\end{enumerate}
It is easy to check that
$d(x,y,z) = t(x, s(x,y,y), s(x,y,z))$
is then a \ldto for $P_4$
(see Case $\chi_0=0$ in the proof of Theorem~\ref{thm:local-diff-terms}).

\medskip

\noindent \underline{\textbf{Subroutine \ld-4.1}}\\[4pt]
To find a \ldto for a set like $Q_4$
% $\{(a_0, b_0, 0), (a_1, b_1, 1),  (a_2, b_2, 1)\}$,
\begin{enumerate}
\item
use Subroutine \ld-3.1 to compute a \ldto $s$ for the set
\begin{equation*}
\{(a_1, b_1, 1), (a_2, b_2, 1), (a_3, b_3, 0)\};
\end{equation*}
\item  use Subroutine \ld-2 to compute a \ldto $t$
for the set
\begin{equation*}
\{(s(a_0, a_0, b_0), b_0, 1),  (a_3,a_3,0)\}.
\end{equation*}
\end{enumerate}
Then
%%%
$d(x,y,z) = t(s(x,y,z), s(y,y,z),z)$
%%%
is a \ldto  for $Q_4$ (see Case $\chi_0=1$ in the proof of Theorem~\ref{thm:local-diff-terms}).


\medskip

\noindent \underline{\textbf{Subroutine \ld-4.2}}\\[4pt]
To find a \ldto for a set like $R_4$
% $\{(a_0, b_0, 0), (a_1, b_1, 1),  (a_2, b_2, 1)\}$,
\begin{enumerate}
\item
use Subroutine \ld-3.1 to compute a \ldto $s$ for the set
\begin{equation*}
\{(a_1, b_1, 0), (a_2, b_2, 1), (a_3, b_3, 1)\};
\end{equation*}
\item use Subroutine \ld-2 to compute a \ldto $t$ for
\begin{equation*}
\{(a_0, s(a_0, b_0, b_0), 0), (a_3, b_3, 1)\}.
\end{equation*}
\end{enumerate}
Then
$d(x,y,z) = t(x, s(x,y,y), s(x,y,z))$
is a \ldto for $P_4$.
