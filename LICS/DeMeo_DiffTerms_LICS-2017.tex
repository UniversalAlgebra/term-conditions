%% \begin{filecontents*}{inputs/refs.bib}
%% @article {MR3239624,
%%     AUTHOR = {Valeriote, M. and Willard, R.},
%%      TITLE = {Idempotent {$n$}-permutable varieties},
%%    JOURNAL = {Bull. Lond. Math. Soc.},
%%   FJOURNAL = {Bulletin of the London Mathematical Society},
%%     VOLUME = {46},
%%       YEAR = {2014},
%%     NUMBER = {4},
%%      PAGES = {870--880},
%%       ISSN = {0024-6093},
%%    MRCLASS = {08A05 (06F99 68Q25)},
%%   MRNUMBER = {3239624},
%%        DOI = {10.1112/blms/bdu044},
%%        URL = {http://dx.doi.org/10.1112/blms/bdu044},
%% }

%% bare_conf.tex
%% V1.4b
%% 2015/08/26
%% by Michael Shell
%% See:
%% http://www.michaelshell.org/
%% for current contact information.
%%
%% This is a skeleton file demonstrating the use of IEEEtran.cls
%% (requires IEEEtran.cls version 1.8b or later) with an IEEE
%% conference paper.
%%
%% Support sites:
%% http://www.michaelshell.org/tex/ieeetran/
%% http://www.ctan.org/pkg/ieeetran
%% and
%% http://www.ieee.org/

%%*************************************************************************
%% Legal Notice:
%% This code is offered as-is without any warranty either expressed or
%% implied; without even the implied warranty of MERCHANTABILITY or
%% FITNESS FOR A PARTICULAR PURPOSE! 
%% User assumes all risk.
%% In no event shall the IEEE or any contributor to this code be liable for
%% any damages or losses, including, but not limited to, incidental,
%% consequential, or any other damages, resulting from the use or misuse
%% of any information contained here.
%%
%% All comments are the opinions of their respective authors and are not
%% necessarily endorsed by the IEEE.
%%
%% This work is distributed under the LaTeX Project Public License (LPPL)
%% ( http://www.latex-project.org/ ) version 1.3, and may be freely used,
%% distributed and modified. A copy of the LPPL, version 1.3, is included
%% in the base LaTeX documentation of all distributions of LaTeX released
%% 2003/12/01 or later.
%% Retain all contribution notices and credits.
%% ** Modified files should be clearly indicated as such, including  **
%% ** renaming them and changing author support contact information. **
%%*************************************************************************


% *** Authors should verify (and, if needed, correct) their LaTeX system  ***
% *** with the testflow diagnostic prior to trusting their LaTeX platform ***
% *** with production work. The IEEE's font choices and paper sizes can   ***
% *** trigger bugs that do not appear when using other class files.       ***                          ***
% The testflow support page is at:
% http://www.michaelshell.org/tex/testflow/



\documentclass[proceedings,10pt]{IEEEtran}
% Some Computer Society conferences also require the compsoc mode option,
% but others use the standard conference format.
%
% If IEEEtran.cls has not been installed into the LaTeX system files,
% manually specify the path to it like:
% \documentclass[conference]{../sty/IEEEtran}





% Some very useful LaTeX packages include:
% (uncomment the ones you want to load)


% *** MISC UTILITY PACKAGES ***
%
%\usepackage{ifpdf}
% Heiko Oberdiek's ifpdf.sty is very useful if you need conditional
% compilation based on whether the output is pdf or dvi.
% usage:
% \ifpdf
%   % pdf code
% \else
%   % dvi code
% \fi
% The latest version of ifpdf.sty can be obtained from:
% http://www.ctan.org/pkg/ifpdf
% Also, note that IEEEtran.cls V1.7 and later provides a builtin
% \ifCLASSINFOpdf conditional that works the same way.
% When switching from latex to pdflatex and vice-versa, the compiler may
% have to be run twice to clear warning/error messages.






% *** CITATION PACKAGES ***
%
%\usepackage{cite}
% cite.sty was written by Donald Arseneau
% V1.6 and later of IEEEtran pre-defines the format of the cite.sty package
% \cite{} output to follow that of the IEEE. Loading the cite package will
% result in citation numbers being automatically sorted and properly
% "compressed/ranged". e.g., [1], [9], [2], [7], [5], [6] without using
% cite.sty will become [1], [2], [5]--[7], [9] using cite.sty. cite.sty's
% \cite will automatically add leading space, if needed. Use cite.sty's
% noadjust option (cite.sty V3.8 and later) if you want to turn this off
% such as if a citation ever needs to be enclosed in parenthesis.
% cite.sty is already installed on most LaTeX systems. Be sure and use
% version 5.0 (2009-03-20) and later if using hyperref.sty.
% The latest version can be obtained at:
% http://www.ctan.org/pkg/cite
% The documentation is contained in the cite.sty file itself.






% *** GRAPHICS RELATED PACKAGES ***
%
\ifCLASSINFOpdf
  % \usepackage[pdftex]{graphicx}
  % declare the path(s) where your graphic files are
  % \graphicspath{{../pdf/}{../jpeg/}}
  % and their extensions so you won't have to specify these with
  % every instance of \includegraphics
  % \DeclareGraphicsExtensions{.pdf,.jpeg,.png}
\else
  % or other class option (dvipsone, dvipdf, if not using dvips). graphicx
  % will default to the driver specified in the system graphics.cfg if no
  % driver is specified.
  % \usepackage[dvips]{graphicx}
  % declare the path(s) where your graphic files are
  % \graphicspath{{../eps/}}
  % and their extensions so you won't have to specify these with
  % every instance of \includegraphics
  % \DeclareGraphicsExtensions{.eps}
\fi
% graphicx was written by David Carlisle and Sebastian Rahtz. It is
% required if you want graphics, photos, etc. graphicx.sty is already
% installed on most LaTeX systems. The latest version and documentation
% can be obtained at: 
% http://www.ctan.org/pkg/graphicx
% Another good source of documentation is "Using Imported Graphics in
% LaTeX2e" by Keith Reckdahl which can be found at:
% http://www.ctan.org/pkg/epslatex
%
% latex, and pdflatex in dvi mode, support graphics in encapsulated
% postscript (.eps) format. pdflatex in pdf mode supports graphics
% in .pdf, .jpeg, .png and .mps (metapost) formats. Users should ensure
% that all non-photo figures use a vector format (.eps, .pdf, .mps) and
% not a bitmapped formats (.jpeg, .png). The IEEE frowns on bitmapped formats
% which can result in "jaggedy"/blurry rendering of lines and letters as
% well as large increases in file sizes.
%
% You can find documentation about the pdfTeX application at:
% http://www.tug.org/applications/pdftex





% *** MATH PACKAGES ***
%
%\usepackage{amsmath}
% A popular package from the American Mathematical Society that provides
% many useful and powerful commands for dealing with mathematics.
%
% Note that the amsmath package sets \interdisplaylinepenalty to 10000
% thus preventing page breaks from occurring within multiline equations. Use:
%\interdisplaylinepenalty=2500
% after loading amsmath to restore such page breaks as IEEEtran.cls normally
% does. amsmath.sty is already installed on most LaTeX systems. The latest
% version and documentation can be obtained at:
% http://www.ctan.org/pkg/amsmath





% *** SPECIALIZED LIST PACKAGES ***
%
%\usepackage{algorithmic}
% algorithmic.sty was written by Peter Williams and Rogerio Brito.
% This package provides an algorithmic environment fo describing algorithms.
% You can use the algorithmic environment in-text or within a figure
% environment to provide for a floating algorithm. Do NOT use the algorithm
% floating environment provided by algorithm.sty (by the same authors) or
% algorithm2e.sty (by Christophe Fiorio) as the IEEE does not use dedicated
% algorithm float types and packages that provide these will not provide
% correct IEEE style captions. The latest version and documentation of
% algorithmic.sty can be obtained at:
% http://www.ctan.org/pkg/algorithms
% Also of interest may be the (relatively newer and more customizable)
% algorithmicx.sty package by Szasz Janos:
% http://www.ctan.org/pkg/algorithmicx




% *** ALIGNMENT PACKAGES ***
%
%\usepackage{array}
% Frank Mittelbach's and David Carlisle's array.sty patches and improves
% the standard LaTeX2e array and tabular environments to provide better
% appearance and additional user controls. As the default LaTeX2e table
% generation code is lacking to the point of almost being broken with
% respect to the quality of the end results, all users are strongly
% advised to use an enhanced (at the very least that provided by array.sty)
% set of table tools. array.sty is already installed on most systems. The
% latest version and documentation can be obtained at:
% http://www.ctan.org/pkg/array


% IEEEtran contains the IEEEeqnarray family of commands that can be used to
% generate multiline equations as well as matrices, tables, etc., of high
% quality.




% *** SUBFIGURE PACKAGES ***
%\ifCLASSOPTIONcompsoc
%  \usepackage[caption=false,font=normalsize,labelfont=sf,textfont=sf]{subfig}
%\else
%  \usepackage[caption=false,font=footnotesize]{subfig}
%\fi
% subfig.sty, written by Steven Douglas Cochran, is the modern replacement
% for subfigure.sty, the latter of which is no longer maintained and is
% incompatible with some LaTeX packages including fixltx2e. However,
% subfig.sty requires and automatically loads Axel Sommerfeldt's caption.sty
% which will override IEEEtran.cls' handling of captions and this will result
% in non-IEEE style figure/table captions. To prevent this problem, be sure
% and invoke subfig.sty's "caption=false" package option (available since
% subfig.sty version 1.3, 2005/06/28) as this is will preserve IEEEtran.cls
% handling of captions.
% Note that the Computer Society format requires a larger sans serif font
% than the serif footnote size font used in traditional IEEE formatting
% and thus the need to invoke different subfig.sty package options depending
% on whether compsoc mode has been enabled.
%
% The latest version and documentation of subfig.sty can be obtained at:
% http://www.ctan.org/pkg/subfig




% *** FLOAT PACKAGES ***
%
%\usepackage{fixltx2e}
% fixltx2e, the successor to the earlier fix2col.sty, was written by
% Frank Mittelbach and David Carlisle. This package corrects a few problems
% in the LaTeX2e kernel, the most notable of which is that in current
% LaTeX2e releases, the ordering of single and double column floats is not
% guaranteed to be preserved. Thus, an unpatched LaTeX2e can allow a
% single column figure to be placed prior to an earlier double column
% figure.
% Be aware that LaTeX2e kernels dated 2015 and later have fixltx2e.sty's
% corrections already built into the system in which case a warning will
% be issued if an attempt is made to load fixltx2e.sty as it is no longer
% needed.
% The latest version and documentation can be found at:
% http://www.ctan.org/pkg/fixltx2e


%\usepackage{stfloats}
% stfloats.sty was written by Sigitas Tolusis. This package gives LaTeX2e
% the ability to do double column floats at the bottom of the page as well
% as the top. (e.g., "\begin{figure*}[!b]" is not normally possible in
% LaTeX2e). It also provides a command:
%\fnbelowfloat
% to enable the placement of footnotes below bottom floats (the standard
% LaTeX2e kernel puts them above bottom floats). This is an invasive package
% which rewrites many portions of the LaTeX2e float routines. It may not work
% with other packages that modify the LaTeX2e float routines. The latest
% version and documentation can be obtained at:
% http://www.ctan.org/pkg/stfloats
% Do not use the stfloats baselinefloat ability as the IEEE does not allow
% \baselineskip to stretch. Authors submitting work to the IEEE should note
% that the IEEE rarely uses double column equations and that authors should try
% to avoid such use. Do not be tempted to use the cuted.sty or midfloat.sty
% packages (also by Sigitas Tolusis) as the IEEE does not format its papers in
% such ways.
% Do not attempt to use stfloats with fixltx2e as they are incompatible.
% Instead, use Morten Hogholm'a dblfloatfix which combines the features
% of both fixltx2e and stfloats:
%
% \usepackage{dblfloatfix}
% The latest version can be found at:
% http://www.ctan.org/pkg/dblfloatfix




% *** PDF, URL AND HYPERLINK PACKAGES ***
%
%\usepackage{url}
% url.sty was written by Donald Arseneau. It provides better support for
% handling and breaking URLs. url.sty is already installed on most LaTeX
% systems. The latest version and documentation can be obtained at:
% http://www.ctan.org/pkg/url
% Basically, \url{my_url_here}.




% *** Do not adjust lengths that control margins, column widths, etc. ***
% *** Do not use packages that alter fonts (such as pslatex).         ***
% There should be no need to do such things with IEEEtran.cls V1.6 and later.
% (Unless specifically asked to do so by the journal or conference you plan
% to submit to, of course. )


% correct bad hyphenation here
\hyphenation{op-tical net-works semi-conduc-tor}




%% Put new macros in the macros.sty file
\usepackage{amsmath}
\usepackage{amscd,amssymb,amsthm} %, amsmath are included by default
\usepackage{latexsym,stmaryrd,mathrsfs,enumerate,scalefnt,ifthen}
\usepackage{mathtools}
\usepackage[mathcal]{euscript}
%% \usepackage[colorlinks=true,urlcolor=black,linkcolor=black,citecolor=black]{hyperref}
\usepackage{url}
\usepackage{scalefnt}
\usepackage{tikz}
\usepackage{color}
%% \usepackage[margin=1in]{geometry}
%% \usepackage{geometry}
%% \usepackage{marginnote}
\usepackage{scrextend}


\newboolean{draftpagebreak}
\setboolean{draftpagebreak}{true}
%% \setboolean{draftpagebreak}{false}

\newcommand\draftbreak{\ifthenelse{\boolean{draftpagebreak}}{\newpage}{}}


\newboolean{draftsecskip}
\setboolean{draftsecskip}{true}
%% \setboolean{draftpagebreak}{false}

\newcommand\draftsecskip{\ifthenelse{\boolean{draftsecskip}}{\medskip}{}}


%%////////////////////////////////////////////////////////////////////////////////
%% Theorem styles
\numberwithin{equation}{section}
\theoremstyle{plain}
\newtheorem{thm}{Theorem}[section]
\newtheorem{lem}[thm]{Lemma}
%% \newtheorem{proposition}[theorem]{Proposition}
\newtheorem{prop}[thm]{Proposition}
\theoremstyle{definition}
\newtheorem{claim}[thm]{Claim}
\newtheorem{cor}[thm]{Corollary}
\newtheorem{definition}[thm]{Definition}
\newtheorem{notation}[thm]{Notation}
\newtheorem{fact}[thm]{Fact}
\newtheorem{question}{Question}
\newtheorem{problem}{Problem}
\newtheorem*{prob}{Problem}
\newtheorem{example}[thm]{Example}
\newtheorem{examples}[thm]{Examples}
\newtheorem{exercise}{Exercise}
%% \newtheorem*{lem}{Lemma}
%% \newtheorem*{cor}{Corollary}
\newtheorem*{rem}{Remark}
\newtheorem*{rems}{Remarks}
%% \newtheorem*{obs}{Observation}


%%%%%%%%%%%%%%%%%%%%%%%%%%%%%%%%%%%%%%%%%%%%%%%%%%%%%%%%%%%%%%%%%

%% \usepackage{inputs/proof-dashed}


%%%%%%%%%%%%%%%%%%%%%%%%%%%%%%%%%%%%%%%%%%%%%%%%%%%%%%%%%%%%%%%%%

%% Put new macros in the macros.sty file
\usepackage{macros}


\begin{document}
%
% paper title
% Titles are generally capitalized except for words such as a, an, and, as,
% at, but, by, for, in, nor, of, on, or, the, to and up, which are usually
% not capitalized unless they are the first or last word of the title.
% Linebreaks \\ can be used within to get better formatting as desired.
% Do not put math or special symbols in the title.
\title{A Polynomial-time Algorithm for Deciding\\Existence of Difference Terms}

\date{\today}
\author{\IEEEauthorblockN{William DeMeo\\}
\IEEEauthorblockA{Department of Mathematics\\
  University of Hawaii\\
Honolulu, Hawaii 96822\\
williamdemeo@gmail.com}
}

% author names and affiliations
% use a multiple column layout for up to three different
% affiliations
%% \author{\IEEEauthorblockN{Michael Shell}
%% \IEEEauthorblockA{School of Electrical and\\Computer Engineering\\
%% Georgia Institute of Technology\\
%% Atlanta, Georgia 30332--0250\\
%% Email: http://www.michaelshell.org/contact.html}
%% \and
%% \IEEEauthorblockN{Homer Simpson}
%% \IEEEauthorblockA{Twentieth Century Fox\\
%% Springfield, USA\\
%% Email: homer@thesimpsons.com}
%% \and
%% \IEEEauthorblockN{James Kirk\\ and Montgomery Scott}
%% \IEEEauthorblockA{Starfleet Academy\\
%% San Francisco, California 96678--2391\\
%% Telephone: (800) 555--1212\\
%% Fax: (888) 555--1212}}

% conference papers do not typically use \thanks and this command
% is locked out in conference mode. If really needed, such as for
% the acknowledgment of grants, issue a \IEEEoverridecommandlockouts
% after \documentclass

% for over three affiliations, or if they all won't fit within the width
% of the page, use this alternative format:
% 
%\author{\IEEEauthorblockN{Michael Shell\IEEEauthorrefmark{1},
%Homer Simpson\IEEEauthorrefmark{2},
%James Kirk\IEEEauthorrefmark{3}, 
%Montgomery Scott\IEEEauthorrefmark{3} and
%Eldon Tyrell\IEEEauthorrefmark{4}}
%\IEEEauthorblockA{\IEEEauthorrefmark{1}School of Electrical and Computer Engineering\\
%Georgia Institute of Technology,
%Atlanta, Georgia 30332--0250\\ Email: see http://www.michaelshell.org/contact.html}
%\IEEEauthorblockA{\IEEEauthorrefmark{2}Twentieth Century Fox, Springfield, USA\\
%Email: homer@thesimpsons.com}
%\IEEEauthorblockA{\IEEEauthorrefmark{3}Starfleet Academy, San Francisco, California 96678-2391\\
%Telephone: (800) 555--1212, Fax: (888) 555--1212}
%\IEEEauthorblockA{\IEEEauthorrefmark{4}Tyrell Inc., 123 Replicant Street, Los Angeles, California 90210--4321}}




% use for special paper notices
%\IEEEspecialpapernotice{(Invited Paper)}




% make the title area
\maketitle

% As a general rule, do not put math, special symbols or citations
% in the abstract
\begin{abstract}
We consider the following practical question: given a finite algebra $\bA$ in a
finite language, can we efficiently decide whether the variety generated by A
has a difference term? We answer this question affirmatively in the idempotent case
by using recent work of Valeriote and Willard as a guide and defining a
``local difference term. '' We use this idea to prove a new theorem that we use
as the basis of a polynomial-time algorithm for deciding,
given finite idempotent algebra, whether the variety it generates has a difference term.
\end{abstract}

% no keywords




% For peer review papers, you can put extra information on the cover
% page as needed:
% \ifCLASSOPTIONpeerreview
% \begin{center} \bfseries EDICS Category: 3-BBND \end{center}
% \fi
%
% For peerreview papers, this IEEEtran command inserts a page break and
% creates the second title. It will be ignored for other modes.
\IEEEpeerreviewmaketitle



\section{Introduction}
\label{sec:introduction}

% no \IEEEPARstart

Let $\sV$ be a variety (equational class) of algebras.
A ternary term $d$ in the language of $\sV$ is called 
a \defn{difference term for $\sV$} if it satisfies the following:
for all $\bA = \<A, \dots \> \in \sV$ and $a, b \in A$ we have
\begin{equation}
\label{eq:3}  
d^{\bA}(a,a,b) = b \quad \text{ and } \quad
d^{\bA}(a,b,b) \comm \theta \theta a,
\end{equation}
where $\theta$ is any congruence %% of $\bA$
containing $(a,b)$
and $[\cdot, \cdot]$ denotes the (term condition) commutator.
(See~\cite{HM:1988} or~\cite{MR3076179} for definitions.)
When the relations in (\ref{eq:3}) hold we will call $d^{\bA}$
a \defn{difference term operation} for $\bA$.


Difference terms are studied extensively in the universal algebra literature.
(See, for example, \cite{HM:1988,MR1358491,MR3076179,MR1663558,MR3449235,KSW}.)
There are many reasons to study difference terms, but
one of the most obvious is that knowing a variety 
has a difference term allows us to deduce many useful
properties of the algebras inhabiting that variety.
(Very roughly speaking, having a difference term is slightly stronger than having
a Taylor term and slightly weaker than having a Mal'tsev term.
Note that if
$\bA$ is an \defn{abelian} algebra---that is, $[1_A, 1_A] = 0_A$---then by
the monotonicity of the commutator we have
$[\theta, \theta] = 0_A$ for all $\theta \in \Con \bA$, in which case
(\ref{eq:3}) says that $d^{\bA}$ is a Mal'tsev term operation.)

Digital computers have turned out to be invaluable tools for exploring and
understanding algebras and the varieties they inhabit, and this is largely due
to the fact that researchers have found ingenious ways
to get computers to solve abstract decision problems---such as
whether a variety is 
congruence-modular (\cite{Freese:2009}) or
congruence-$n$-permutable (\cite{MR3239624})---and to do so efficiently.
%% , in a way that scales well with problem size.
In the present paper we add to the algorithmic arsenal
by solving the following:
%% Difference terms are studied extensively in the general algebra literature.
%% See, for example, 
%% Roughly speaking, having a difference term is slightly stronger than having
%% a Taylor term and slightly weaker than having a Mal'cev term.  Note that if
%% $\bA$ is an \defn{abelian} algebra, which means that $[1_A, 1_A] = 0_A$, then, by
%% the monotonicity of the commutator,
%% $[\theta, \theta] = 0_A$ for all $\theta \in \Con \bA$, in which case
%% (\ref{eq:3}) says that $d^{\bA}$ is a Mal'cev term operation.
\begin{prob}
  \label{prob:1}
  Is there a polynomial-time algorithm that takes a finite
  idempotent algebra $\bA$ as input and decides whether the variety generated by
  $\bA$ has a difference term?
\end{prob}

%% \hfill January 8, 2017


Let $\bp = (p_0, p_1, \dots, p_n)$ be an $(n+1)$-tuple of ternary terms, where
$p_0(x,y,z) \approx x$ and $p_n(x,y,z) \approx z$, the first and third
ternary projections, respectively. 
Let $\bA=\< A, \dots\>$ be an algebra.
In~\cite{MR3239624},
Valeriote and Willard define an \defn{$\bA$-triple for $\bp$}
to be a triple $(a,b,i)$ such that $a, b \in A$ and
$p_i(a,b,b) = p_{i+1}(a,a,b)$. They use this to define 
a ``local Hagemann-Mitschke sequence'' on which they base an efficient algorithm
for deciding for a given $n$ whether an idempotent variety is $n$-permutable.
Taking this as our inspiration, we devise a similar construct, which we call
a ``local difference term,'' and use it to develop a polynomial-time
algorithm for deciding the existence of a (global) difference term.
%% , given a finite idempotent algebra $\bA$, whether the variety
%% generated by $\bA$ has a difference term.



\section{Local Difference Terms}


Let $\bp = (p_0, p_1, \dots, p_n)$ be an $(n+1)$-tuple of ternary terms, where
$p_0(x,y,z) \approx x$ and $p_n(x,y,z) \approx z$, the first and third
ternary projections, respectively. 
Let $\bA=\< A, \dots\>$ be an algebra.
In~\cite{MR3239624},
Valeriote and Willard define an \defn{$\bA$-triple for $\bp$}
to be a triple $(a,b,i)$ such that $a, b \in A$ and
$p_i(a,b,b) = p_{i+1}(a,a,b)$. They use this to derive methods for deciding
for a given $k$ whether an idempotent variety is $k$-permutable.
This inspired us to develop the concept of a ``local difference term'' that we
hope will be useful for the
purpose of deciding, given a finite idempotent algebra $\bA$, whether the variety
generated by $\bA$ has a difference term.

Let $\bA=\< A, \dots\>$ be an algebra, fix $a, b \in A$ and
$i \in \{0,1\}$.
An
\defn{$\bA$-local difference term for
  $(a,b,i)$} is a ternary term $p$ satisfying the following:
\begin{align}
\text{ if $i=0$, then } & a \comm{\Cg^{\bA}(a,b)}{\Cg^{\bA}(a,b)} p(a,b,b); \label{eq:diff-triple}\\
\text{ if $i=1$, then } &p(a,a,b) = b. \nonumber
\end{align}
When often drop the
$\bA$ when the algebra is clear from context.
For example, we write $\Cg$ in place of $\Cg^{\bA}$, and 
call the term $p$ above a local difference term for 
$(a,b,i)$.
If $p$ satisfies~(\ref{eq:diff-triple}) for all triples
in some subset $S\subseteq A \times A \times \{0,1\}$, then we call $p$
a \defn{local difference term for $S$}.

Let 
$\sS = A \times A \times \{0,1\}$ and
suppose that every pair
$((a_0, b_0, \chi_0), (a_1, b_1, \chi_1))$
in $\sS^2$ has a local difference term.
That is, for each pair $((a_0, b_0, \chi_0), (a_1, b_1, \chi_1))$, there exists
$p$ such that for each $i \in \{0,1\}$ we have
\begin{align}
  a_i \comm{\Cg(a_i,b_i)}{\Cg(a_i,b_i)} p(a_i,b_i,b_i), & \;
  \text{ if $\chi_i=0$, and }  \label{eq:d-trip-i1}\\
  p(a_i,a_i,b_i) =b_i, & \;
  \text{ if $\chi_i=1$.}\nonumber
\end{align}
Under these hypothesis we will prove that every subset $S\subseteq \sS$
% \[ S = \{(a_0, b_0, \chi_0), (a_1, b_1, \chi_1), \dots, (a_{n-1},
% b_{n-1},\chi_{n-1})\} \subseteq \sS, \]
has a local difference term.
That is, there is a single term $p$ that works (i.e., satisfies
(\ref{eq:d-trip-i1})) for all $(a_i, b_i, \chi_i) \in S$.
The statement and proof of this new result follows.

\begin{lem}
  Let $\sV$ be an idempotent variety and
  $\bA \in \sV$. Define
  $\sS= A \times A \times \{0,1\}$, %% = \{(a, b, \chi) \mid a, b \in A, \, \chi \in \{0,1\}\}$,
  and suppose that every pair
  $((a_0, b_0, \chi_0), (a_1, b_1, \chi_1)) \in \sS^2$
  has a local difference term.
  Then every subset $S \subseteq \sS$,
  has a local difference term.
\end{lem}
\begin{proof}
The proof is by induction on the size of $S$.  In the base case, $|S| = 2$,
the claim holds by assumption.
Fix $n>2$ and assume that every subset of $\sS$ of size $2\leq k \leq n$ has a local
difference term. Let
$S = \{(a_0, b_0, \chi_0), (a_1, b_1, \chi_1), \dots, (a_{n}, b_{n},\chi_{n})\} \subseteq \sS$,
so that $|S| = n+1$.  We prove $S$ has a local difference term.

Since $|S| \geq 3$ and $\chi_i \in \{0,1\}$ for all $i$, there must exist
indices $i\neq j$ such that $\chi_i = \chi_j$. Assume without loss of generality
that one of these indices is $j=0$.  Define
the set
$S' = S \setminus \{(a_0, b_0, \chi_0)\}$.
%% $S' = S - \{(a_0, b_0, \chi_0)\}$.
Since $|S'| < |S|$, the set $S'$ has a local difference term $p$.
We split the remainder of the proof into two cases. In the first case
$\chi_0 = 0$ and in the second
$\chi_0 = 1$.

\vskip3mm

%--------------------------------------
\noindent \underline{Case 1:} $\chi_0 = 0$.
\\[4pt]
Assume $\chi_0 = 0$ and, 
without loss of generality, suppose $\chi_1 = \chi_2 =\cdots =\chi_k = 1$,
and $\chi_{k+1} = \chi_{k+2} = \cdots = \chi_{n} = 0$. Define
\[
T = \{(a_0, p(a_0, b_0, b_0), 0),
(a_1, b_1, 1), (a_2, b_2 1), \dots, (a_k, b_k, 1)\}.
\]
Let $t$ be a local difference term for $T$.
Define
\[
d(x,y,z) = t(x, p(x,y,y), p(x,y,z)).
\]
Since $\chi_0 =0$, we want $d(a_0,b_0,b_0)\comm{\Cg(a_0,b_0)}{\Cg(a_0,b_0)} a_0$.
We have
\begin{equation}
    \label{eq:100000}
  d(a_0,b_0,b_0) =
  t(a_0, p(a_0,b_0,b_0), p(a_0,b_0,b_0))\comm{\tau}{\tau} a_0,
\end{equation}
where $\tau:=\Cg(a_0, p(a_0,b_0,b_0))$.
Notice that
\[
(a_0, p(a_0,b_0,b_0)) = (p(a_0,a_0,a_0), p(a_0,b_0,b_0)) \in \Cg(a_0, b_0),
\]
so $\tau\leq \Cg(a_0,b_0)$. Therefore, 
$\comm{\tau}{\tau} {\leq} \comm{\Cg(a_0,b_0)}{\Cg(a_0,b_0)}$,
by monotonicity of the commutator.
It follows from this and (\ref{eq:100000}) that
$d(a_0,b_0,b_0)\comm{\Cg(a_0,b_0)}{\Cg(a_0,b_0)} a_0$,
as desired.

For $1\leq i \leq k$ we have $\chi_i =1$, so we want  $d(a_i,a_i,b_i) = b_i$. Indeed,
\begin{equation}
  \label{eq:200000}
  d(a_i,a_i,b_i) =
  t(a_i, p(a_i,a_i,a_i), p(a_i,a_i,b_i))=
  t(a_i, a_i, b_i) =b_i.
\end{equation}
The first equality in~(\ref{eq:200000}) holds by definition of $d$;
the second, because $p$ is an idempotent local difference term for
$S'$; the third, because $t$ is a local difference term for $T$.

The remaining triples in our original set $S$
have index $j$ satisfying $k<j\leq n$ and $\chi_j = 0$.
Thus, for these triples we want
$d(a_j,b_j,b_j)\comm{\Cg(a_j,b_j)}{\Cg(a_j,b_j)} a_j$. Indeed,
since $p$ is a local difference term for $S'$, we have
$p(a_j,b_j,b_j))  \comm{\Cg(a_j,b_j)}{\Cg(a_j,b_j)} a_j$, so
\begin{equation*}
    %% \label{eq:300000}
  d(a_j,b_j,b_j) =
  t(a_j, p(a_j,b_j,b_j), p(a_j,b_j,b_j)) =
  \comm{\Cg(a_j,b_j)}{\Cg(a_j,b_j)} t(a_j,a_j,a_j) = a_j,
\end{equation*}
as desired.
\\[6pt]
%--------------------------------------
\underline{Case 2:}
$\chi_0 = 1$.
\\[4pt]
Assume $\chi_0 = 1$ and, 
without loss of generality, suppose $\chi_1 = \chi_2 =\cdots =\chi_k = 0$,
and $\chi_{k+1} = \chi_{k+2} = \cdots = \chi_{n} = 1$. Define
\[
T = \{(p(a_0, a_0, b_0), b_0, 1),
(a_1, b_1, 0), (a_2, b_2 0), \dots, (a_k, b_k, 0)\}.
\]
Let $t$ be a local difference term for $T$.
Define
\[d(x,y,z) = t(p(x,y,z), p(y,y,z), z).\] 
Since $\chi_0 =1$, we want $d(a_0,a_0,b_0) = b_0$. Indeed, by the definition of
$d$ in this case,
%% \comm{\theta}{\theta} a_0
\begin{equation*}
    %% \label{eq:100001}
  d(a_0,a_0,b_0) =
  t(p(a_0,a_0,b_0), p(a_0,a_0,b_0), b_0) =b_0.
\end{equation*}
The last equality holds since $t$ is a local difference term for $T$, thus,
for $(p(a_0, a_0, b_0), b_0, 1)$.

For $1\leq i \leq k$ we have $\chi_i =0$, so we want
$d(a_i,b_i,b_i) \comm{\Cg(a_i,b_i)}{\Cg(a_i,b_i)} a_i$.
Again, starting from the definition of $d$ and using idempotence of $p$, we have
\begin{equation*}
  d(a_i,b_i,b_i) =
  t(p(a_i,b_i,b_i), p(b_i,b_i,b_i), b_i)=
  t(p(a_i,b_i,b_i), b_i, b_i)
\end{equation*}
Next, since $p$ is a local difference term for $S'$, we have
\[
  t(p(a_i,b_i,b_i), b_i, b_i)
 \comm{\Cg(a_i,b_i)}{\Cg(a_i,b_i)}
 t(a_i, b_i, b_i).
 \]
Finally  $ t(a_i, b_i, b_i) \comm{\Cg(a_i,b_i)}{\Cg(a_i,b_i)} a_i$, as desired.
%% $p(a_i,b_i,b_i)\comm{\Cg(a_i,b_i)}{\Cg(a_i,b_i)} a_i$. Therefore,
%% \begin{equation*}
%%   d(a_i,b_i,b_i) =
%%   t(p(a_i,b_i,b_i), p(b_i,b_i,b_i), b_i)=
%%   t(p(a_i,b_i,b_i), b_i, b_i) \comm{\Cg(a_i,b_i)}{\Cg(a_i,b_i)}
%%     t(a_i, b_i, b_i) \comm{\Cg(a_i,b_i)}{\Cg(a_i,b_i)} a_i.
%% \end{equation*}

The remaining elements of our original set $S$
have indices $j$ satisfying $k<j\leq n$ and $\chi_j = 1$.
For these we want $d(a_j,a_j,b_j) = b_j$.
Indeed,
since $p$ is a local difference term for $S'$, we have
$p(a_j,a_j,b_j) = b_j$, so
\begin{equation*}
d(a_j,a_j,b_j) =
  t(p(a_j,a_j,b_j), p(a_j,a_j,b_j), b_i)=
  t(b_j, b_j, b_j) =b_j,
\end{equation*}
as desired.
\end{proof}


% An example of a floating figure using the graphicx package.
% Note that \label must occur AFTER (or within) \caption.
% For figures, \caption should occur after the \includegraphics.
% Note that IEEEtran v1.7 and later has special internal code that
% is designed to preserve the operation of \label within \caption
% even when the captionsoff option is in effect. However, because
% of issues like this, it may be the safest practice to put all your
% \label just after \caption rather than within \caption{}.
%
% Reminder: the "draftcls" or "draftclsnofoot", not "draft", class
% option should be used if it is desired that the figures are to be
% displayed while in draft mode.
%
%\begin{figure}[!t]
%\centering
%\includegraphics[width=2.5in]{myfigure}
% where an .eps filename suffix will be assumed under latex, 
% and a .pdf suffix will be assumed for pdflatex; or what has been declared
% via \DeclareGraphicsExtensions.
%\caption{Simulation results for the network.}
%\label{fig_sim}
%\end{figure}

% Note that the IEEE typically puts floats only at the top, even when this
% results in a large percentage of a column being occupied by floats.


% An example of a double column floating figure using two subfigures.
% (The subfig.sty package must be loaded for this to work.)
% The subfigure \label commands are set within each subfloat command,
% and the \label for the overall figure must come after \caption.
% \hfil is used as a separator to get equal spacing.
% Watch out that the combined width of all the subfigures on a 
% line do not exceed the text width or a line break will occur.
%
%\begin{figure*}[!t]
%\centering
%\subfloat[Case I]{\includegraphics[width=2.5in]{box}%
%\label{fig_first_case}}
%\hfil
%\subfloat[Case II]{\includegraphics[width=2.5in]{box}%
%\label{fig_second_case}}
%\caption{Simulation results for the network.}
%\label{fig_sim}
%\end{figure*}
%
% Note that often IEEE papers with subfigures do not employ subfigure
% captions (using the optional argument to \subfloat[]), but instead will
% reference/describe all of them (a), (b), etc., within the main caption.
% Be aware that for subfig.sty to generate the (a), (b), etc., subfigure
% labels, the optional argument to \subfloat must be present. If a
% subcaption is not desired, just leave its contents blank,
% e.g., \subfloat[].


% An example of a floating table. Note that, for IEEE style tables, the
% \caption command should come BEFORE the table and, given that table
% captions serve much like titles, are usually capitalized except for words
% such as a, an, and, as, at, but, by, for, in, nor, of, on, or, the, to
% and up, which are usually not capitalized unless they are the first or
% last word of the caption. Table text will default to \footnotesize as
% the IEEE normally uses this smaller font for tables.
% The \label must come after \caption as always.
%
%\begin{table}[!t]
%% increase table row spacing, adjust to taste
%\renewcommand{\arraystretch}{1.3}
% if using array.sty, it might be a good idea to tweak the value of
% \extrarowheight as needed to properly center the text within the cells
%\caption{An Example of a Table}
%\label{table_example}
%\centering
%% Some packages, such as MDW tools, offer better commands for making tables
%% than the plain LaTeX2e tabular which is used here.
%\begin{tabular}{|c||c|}
%\hline
%One & Two\\
%\hline
%Three & Four\\
%\hline
%\end{tabular}
%\end{table}


% Note that the IEEE does not put floats in the very first column
% - or typically anywhere on the first page for that matter. Also,
% in-text middle ("here") positioning is typically not used, but it
% is allowed and encouraged for Computer Society conferences (but
% not Computer Society journals). Most IEEE journals/conferences use
% top floats exclusively. 
% Note that, LaTeX2e, unlike IEEE journals/conferences, places
% footnotes above bottom floats. This can be corrected via the
% \fnbelowfloat command of the stfloats package.




%% \section{Conclusion}
%% TODO: the conclusion goes here.




% conference papers do not normally have an appendix


% use section* for acknowledgment
\section*{Acknowledgment}
The author would like to thank
Ralph Freese for proposing this project and for recommending 
the work of Valeriote and Willard.
Other helpful suggestions came from Cliff Bergman and the
following members of the Hawaii Universal Algebra 
Seminar: Alex Guillen, Tristan Holmes, Bill Lampe, and J.~B.~Nation.
Their contributions are gratefully acknowledged.

% trigger a \newpage just before the given reference
% number - used to balance the columns on the last page
% adjust value as needed - may need to be readjusted if
% the document is modified later
%\IEEEtriggeratref{8}
% The "triggered" command can be changed if desired:
%\IEEEtriggercmd{\enlargethispage{-5in}}

% references section

% can use a bibliography generated by BibTeX as a .bbl file
% BibTeX documentation can be easily obtained at:
% http://mirror.ctan.org/biblio/bibtex/contrib/doc/
% The IEEEtran BibTeX style support page is at:
% http://www.michaelshell.org/tex/ieeetran/bibtex/
%\bibliographystyle{IEEEtran}
% argument is your BibTeX string definitions and bibliography database(s)
%\bibliography{IEEEabrv,../bib/paper}
%
% <OR> manually copy in the resultant .bbl file
% set second argument of \begin to the number of references
% (used to reserve space for the reference number labels box)

\begin{thebibliography}{1}
  %% \begin{thebibliography}{KKVW15}
  %% \bibitem[VW14]{MR3239624}
\bibitem{MR2839398}
Clifford Bergman.
\newblock {\em Universal algebra}, volume 301 of {\em Pure and Applied
  Mathematics (Boca Raton)}.
\newblock CRC Press, Boca Raton, FL, 2012.
\newblock Fundamentals and selected topics.

\bibitem{Freese:2009}
Ralph Freese and Matthew~A. Valeriote.
\newblock On the complexity of some {M}altsev conditions.
\newblock {\em Internat. J. Algebra Comput.}, 19(1):41--77, 2009.
\newblock URL: \url{http://dx.doi.org/10.1142/S0218196709004956}.%% , \href
  %% {http://dx.doi.org/10.1142/S0218196709004956}
  %% {\path{doi:10.1142/S0218196709004956}}.

\bibitem{HM:1988}
David Hobby and Ralph McKenzie.
\newblock {\em The structure of finite algebras}, volume~76 of {\em
  Contemporary Mathematics}.
\newblock American Mathematical Society, Providence, RI, 1988.

\bibitem{MR1358491}
Keith~A. Kearnes.
\newblock Varieties with a difference term.
\newblock {\em J. Algebra}, 177(3):926--960, 1995.
\newblock URL: \url{http://dx.doi.org/10.1006/jabr.1995.1334}.%% , \href
  %% {http://dx.doi.org/10.1006/jabr.1995.1334}
  %% {\path{doi:10.1006/jabr.1995.1334}}.

\bibitem{MR3076179}
Keith~A. Kearnes and Emil~W. Kiss.
\newblock The shape of congruence lattices.
\newblock {\em Mem. Amer. Math. Soc.}, 222(1046):viii+169, 2013.
\newblock URL: \url{http://dx.doi.org/10.1090/S0065-9266-2012-00667-8}.%% , \href
  %% {http://dx.doi.org/10.1090/S0065-9266-2012-00667-8}
  %% {\path{doi:10.1090/S0065-9266-2012-00667-8}}.

\bibitem{MR1663558}
Keith~A. Kearnes and {\'A}gnes Szendrei.
\newblock The relationship between two commutators.
\newblock {\em Internat. J. Algebra Comput.}, 8(4):497--531, 1998.
\newblock URL: \url{http://dx.doi.org/10.1142/S0218196798000247}.%% , \href
  %% {http://dx.doi.org/10.1142/S0218196798000247}
  %% {\path{doi:10.1142/S0218196798000247}}.

\bibitem{MR3449235}
Keith Kearnes, {\'A}gnes Szendrei, and Ross Willard.
\newblock A finite basis theorem for difference-term varieties with a finite
  residual bound.
\newblock {\em Trans. Amer. Math. Soc.}, 368(3):2115--2143, 2016.
\newblock URL: \url{http://dx.doi.org/10.1090/tran/6509}.%% , \href
  %% {http://dx.doi.org/10.1090/tran/6509} {\path{doi:10.1090/tran/6509}}.


\bibitem{KSW}
Keith Kearnes, \'{A}gnes Szendrei, and Ross Willard.
\newblock Simpler maltsev conditions for (weak) difference terms in locally
  finite varieties.
\newblock to appear.


\bibitem{MR3350327}
Marcin Kozik, Andrei Krokhin, Matt Valeriote, and Ross Willard.
\newblock Characterizations of several {M}altsev conditions.
\newblock {\em Algebra Universalis}, 73(3-4):205--224, 2015.
\newblock URL: \url{http://dx.doi.org/10.1007/s00012-015-0327-2}.%% , \href
  %% {http://dx.doi.org/10.1007/s00012-015-0327-2}
  %% {\path{doi:10.1007/s00012-015-0327-2}}.


\bibitem{MR3239624}
M.~Valeriote and R.~Willard.
\newblock Idempotent {$n$}-permutable varieties.
\newblock {\em Bull. Lond. Math. Soc.}, 46(4):870--880, 2014.
\newblock URL: \url{http://dx.doi.org/10.1112/blms/bdu044}. %% \href
  %% {http://dx.doi.org/10.1112/blms/bdu044} {\path{doi:10.1112/blms/bdu044}}.


%% \bibitem{MR3239624}
%%   M.~Valeriote and R.~Willard.
%%   \newblock Idempotent {$n$}-permutable varieties.
%%   \newblock {\em Bull. Lond. Math. Soc.}, 46(4):870--880, 2014.
%%   \newblock URL: \url{http://dx.doi.org/10.1112/blms/bdu044}.%%  \href
%%             %% {http://dx.doi.org/10.1112/blms/bdu044} {\path{doi:10.1112/blms/bdu044}}.


%% \bibitem{IEEEhowto:kopka}
%% H.~Kopka and P.~W. Daly, \emph{A Guide to \LaTeX}, 3rd~ed.\hskip 1em plus
%%   0.5em minus 0.4em\relax Harlow, England: Addison-Wesley, 1999.

\end{thebibliography}




% that's all folks
\end{document}


