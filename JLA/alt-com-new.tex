For an algebra $\bA$ with congruence relations $\alpha$, $\beta\in \Con\bA$,
let $\bbeta$ denote the subalgebra of $\bA\times \bA$ with universe 
$\beta$, and let $0_A$ denote the least equivalence relation on $A$.
%% (note that $0_A$ is a subuniverse of $\bbeta$).
Thus, $0_A = \{(a,a) \mid a\in A\} \leq \beta$.
Denote by $D_\alpha$ the following subset of $\beta \times \beta$:
\begin{equation}
  \label{eq:9009}
D_\alpha =(\alpha \otimes \alpha) \cap (0_A \times 0_A)
= \{((a,a), (b,b)) \in (0_A \times 0_A) \mid a\alphar b\}.
\end{equation}
Let $\Delta_{\beta, \alpha} = \Cg^{\bbeta}(D_\alpha)$ denote the congruence relation
of $\bbeta$ generated by $D_\alpha$.
The condition $\CC{\alpha}{\beta}{\gamma}$
holds iff for all $a \alphar b$, for all $u_i \betar v_i$ ($1\leq i\leq n$), and for all 
$t\in \Pol_{n+1}(\bA)$, we have
$t(a,\bu) \mathrel{\gamma} t(a, \bv)$
iff $t(b,\bu) \mathrel{\gamma} t(b, \bv)$.
There are a number of different ways to define a commutator.
See, for example,~\cite{MR0432511,MR541622,MR590312,MR1145556,MR1663558,MR1257643}.
In this note, the commutator $[\alpha, \beta]$ is defined to be the least
congruence $\gamma$ such that $\CC{\alpha}{\beta}{\gamma}$ holds.
\begin{comment}
Occasionally it is more convenient to write such an equivalence as a (two-way) derivation tree,
as follows:
\[
\infer=[\CC{\alpha}{\beta}{\gamma}]{\Gamma \vdash t(b,\bu) \mathrel{\gamma} t(b, \bv)}{\Gamma \vdash t(a,\bu) \mathrel{\gamma} t(a, \bv)}\]
where $\Gamma$ is a context containing
$a \alphar b$, $u_i \betar v_i$ ($1\leq i\leq n$), and 
$t\in \Pol_{n+1}(\bA)$.
\end{comment}


%% \section{Alternate Description of the Commutator}
%% \label{sec:altern-descr}

We now %% In~\cite{com-fix-poi} we gave 
describe an alternate way to express the commutator---specifically,
it is the least fixed point of a certain closure operator.
This description was inspired by the one that is mentioned in passing by
Keith Kearnes in~\cite[p.~930]{MR1358491}.  Our objective here is
to prove that the description we present is correct (i.e., describes the commutator)
and to show that it leads to a simple, efficient procedure for computing the commutator.

Let $\Tol(A)$ denote the collection of all tolerances (reflexive symmetric relations)
on the set $A$,\footnote{Actually, a
  \emph{tolerance} of an algebra $\bA = \<A, \dots\>$
  is a reflexive symmetric subalgebra of $\bA \times \bA$.
  Therefore, the set of all tolerances of $\bA$ forms an
  algebraic (hence complete) lattice.
  If we drop the operations and consider only the set $A$, then a tolerance relation on $A$
  is simply a reflexive symmetric binary relation.
}
and let 
$\Psi_{\beta, \alpha} \colon \Tol(A) \to \Tol(A)$ be the function defined
for each $T \in  \Tol(A)$ follows:
\begin{equation}
  \label{eq:7}
  \Psi_{\beta, \alpha}(T)
  = \{ (x,y) \in A\times A \mid
  (\exists\, (a,b) \in T)\,
 (a,b) \mathrel{\Delta_{\beta, \alpha}} (x,y)\},
\end{equation}
where
$\Delta_{\beta, \alpha} = \Cg^{\bbeta}(D_\alpha)$ and
$D_\alpha =(\alpha \otimes \alpha) \cap (0_A \times 0_A)$
(as in~(\ref{eq:9009})).

\begin{remarks}\
  \begin{enumerate}
\item
  It's easy to see that $\Psiba (T)$ is reflexive and symmetric
  whenever $T$ has these properties; similarly, $\Psiba (T)$ is compatible with the
  operations of $\bA$ whenever $T$ is. In other words $\Psiba$ maps tolerances of
  $A$ ($\bA$, resp.) to tolerances of $A$ ($\bA$, resp.).
  \item 
  Since $\Psiba$ is clearly a monotone increasing function on the complete
  lattice $\Tol(A)$, it is guaranteed to have a least fixed
  point---that is, there is a point $\tau\in \Tol(A)$ such that $\Psiba(\tau) = \tau$
  and $\tau \leq T$, for every $T \in \Tol(A)$
  satisfying $\Psiba(T) = T$.
\item
  Here are two ways the least fixed point of $\Psiba$ could be computed:
  \begin{equation}
    \label{eq:4}
  \tau = \Meet \{ T \in \Tol(A) \mid \Psiba(T) \leq T\}
  \quad \text{ and } \quad
     \tau = \Join_{k\geq 0} \Psiba^{k}(0_A).
  \end{equation}
  In Lemma~\ref{lem:fixed-point-comm} we will show that the least
  fixed point of $\Psiba$ is, in fact, the commutator,
  $\tau = [\alpha, \beta]$, so either
  expression in~(\ref{eq:4}) could potentially be used to compute it.
  However, Lemma~\ref{lem:fixed-point-comm} also shows that 
  $\Psiba$ is a closure operator; in particular, it is idempotent. Therefore,
  $\Psiba^{k}(0_A) = \Psiba(0_A)$ for all $k$, so we have 
  the following simple description of the commutator:
  \begin{align*}
    %% \label{eq:55}
          [\alpha, \beta] =
          \Psiba(0_A)
          &= \{ (x,y) \in A\times A \mid
          (\exists\, (a,b) \in 0_A)\, (a,b) \mathrel{\Delta_{\beta, \alpha}} (x,y)\}\\
          &= \{ (x,y) \in A\times A \mid
          (\exists a \in A)\, (a,a) \mathrel{\Delta_{\beta, \alpha}} (x,y)\}.
  \end{align*}
  \end{enumerate}
\end{remarks}


\subsection{Fixed Point Lemma}
\begin{lemma}
  \label{lem:fixed-point-comm}
  If $\alpha$, $\beta\in \Con(\bA)$ and 
  if $\Psi_{\beta, \alpha}$ is defined by~(\ref{eq:7}), then 
  \begin{enumerate}[(i)]
  \item $\Psiba$ is a closure operator on $\Tol(A)$;
  \item $[\alpha, \beta]$ is the least fixed point of $\Psiba$.
  \end{enumerate}
\end{lemma}
\begin{proof}\
  \begin{enumerate}[(i)]
  \item 
  %% In fact, $\Psi_{\beta, \alpha}$ is a closure operator on all
  %% of $\Tol(A)$ as we 
    To prove (i) we verify that
    $\Psi_{\beta, \alpha}$ has the three properties that define a closure
    operator---namely for all $T$, $T' \in \Tol(A)$,
  \begin{enumerate}[(c.1)]
  \item \label{item:c1} $T  \leq \Psiba(T )$;     
  \item \label{item:c2} $T  \leq T'  \Rightarrow \Psiba(T) \leq \Psiba(T')$;    
  \item \label{item:c3} $\Psiba(\Psiba(T))  = \Psiba(T)$. 
  \end{enumerate}

  \vskip2mm

  \noindent {\it Proof of (c.1):} $(a,b) \in T $
  implies $(a,b) \in \Psiba(T )$ because $(a,b)\mathrel{\Delta_{\beta, \alpha}} (a,b)$.\\[4pt]
  %% this proves~(c.\ref{item:c1}).
  \noindent {\it Proof of (c.2):} $(x,y) \in \Psiba(T )$ iff there exists
  $(a,b) \in T  \leq T'$ such that
  $(a,b) \mathrel{\Delta_{\beta, \alpha}} (x,y)$; this and $(a,b) \in T'$ implies
  $(x,y) \in \Psiba(T')$.\\[4pt]
  \noindent {\it Proof of (c.3):} $(x,y) \in \Psiba(\Psiba(T))$ if and only if
  there exists $(a,b) \in \Psiba(T)$ such that
  $(a,b) \Deltabar (x,y)$, and $(a,b) \in \Psiba(T)$ is in turn equivalent to 
  the existence of $(c,d) \in T $ such that
  $(c,d) \Deltabar (a,b)$. By transitivity of $\Deltaba$, we have that
  $(c,d) \Deltabar (a,b) \Deltabar (x,y)$ implies
  $(c,d) \Deltabar (x,y)$, proving that there exists $(c,d) \in T $ such that
  $(c,d) \Deltabar (x,y)$; equivalently, $(x,y) \in T $.

  \medskip

\item
  %% \noindent (ii) 
  As remarked above, from part (i) follows 
  $\Psiba^{k}(0_A) = \Psiba(0_A)$ for all $k$, so the least fixed point of
  $\Psiba$ that appears in the formula on the right in~(\ref{eq:4}) reduces
  to $\tau = \Psiba(0_A)$.  Therefore, to complete the proof it suffices to show
  $[\alpha, \beta] = \Psiba(0_A)$.

  We first prove $[\alpha, \beta]\leq \Psiba(0_A)$.
  Since $[\alpha, \beta]$ is the least congruence $\gamma$
  satisfying $\CC{\alpha}{\beta}{\gamma}$, it suffices to prove
    $\CC{\alpha}{\beta}{\Psiba(0_A)}$ holds.
    Suppose $a \alphar a'$ and $b_i \betar b_i'$ %% ($1\leq i \leq k$)
    and $t^{\bA} \in \Pol_{k+1}(\bA)$ satisfy
    $t^{\bA}(a, \bb) \mathrel{\Psiba(0_A)} t^{\bA}(a, \bb')$,
    where $\bb = (b_1, \dots, b_k)$ and $\bb' = (b_1', \dots, b_k')$.
    We must show $t(a', \bb) \mathrel{\Psiba(0_A)} t(a', \bb')$.  
    By definition of $\Psiba$,
    the antecedent $t^{\bA}(a, \bb) \mathrel{\Psiba(0_A)} t^{\bA}(a, \bb')$ is equivalent to    
    the existence of $c \in A$ such that $(c,c) \Deltabar (t^{\bA}(a, \bb), t^{\bA}(a, \bb'))$.
    Now
    \[
    (t^{\bA}(a, \bb), t^{\bA}(a, \bb')) = t^{\bbeta}((a,a),(b_1, b_1'), \dots,(b_k, b_k')),
    \]
    and since $a \alphar a'$, we have
    \[
    t^{\bbeta}((a,a),(b_1, b_1'), \dots,(b_k, b_k'))
    \Deltabar
    t^{\bbeta}((a',a'),(b_1, b_1'), \dots,(b_k, b_k')).
    \]
    The latter is equal to $(t^{\bA}(a', \bb), t^{\bA}(a', \bb'))$, and  it follows
    by transitivity of $\Deltaba$ that
    $(c,c) \Deltabar (t^{\bA}(a', \bb), t^{\bA}(a', \bb'))$.
    Therefore, $t(a', \bb) \mathrel{\Psiba(0_A)} t(a', \bb')$, as desired.  



  We now prove $\Psiba(0_A)\leq   [\alpha, \beta]$.
  If $(x,y)\in \Psiba(0_A)$ then there exists $a \in A$ such that 
  \begin{equation}
    \label{eq:1100}
    (a,a) \mathrel{\Delta_{\beta, \alpha}} (x,y).
  \end{equation}
  From the definition of $\Delta_{\beta, \alpha}$ and 
  \malcev's congruence generation theorem,~(\ref{eq:1100})
  holds if and only if for there exist
  $(z_i, z_i') \in \beta$ ($0\leq i \leq n+1$), and $(u_i, v_i) \in \alpha$,
  $f_i \in \Pol_1(\bbeta)$ ($0\leq i \leq n$), such that
  $(a, a) = (z_0,z_0')$ and $(x, y)=(z_{n+1},z'_{n+1})$ hold, and so do the
  following equations of sets: 
  \begin{align}
    \label{eq:001}
    \{(a, a),(z_1,z_1')\} &= \{f_0(u_0,u_0), f_0(v_0,v_0)\},\\
    \label{eq:011}
    \{(z_1,z_1'),(z_2,z_2')\} &= \{f_1(u_1,u_1), f_1(v_1,v_1)\},\\
    \nonumber
    &\; \; \vdots\\
    %% \label{eq:n-1}
    \nonumber
    \{(z_{n},z_{n}'),(x, y)\} &= \{f_{n}(u_{n},u_{n}), f_{n}(v_{n},v_{n})\}.
  \end{align}
  Now $f_i \in \Pol_1(\bbeta)$ for all $i$, so
  \newcommand\gA{\ensuremath{g^{\bA}}}%
  \[
  f_i(c, c') = g_i^{\bbeta}((c, c'), (b_1, b_1'), \dots, (b_k, b_k') )
  = (\gA_i(c, \bb), \gA_i(c', \bb')),%
  \]
  \renewcommand\gA{\ensuremath{g}}%
  for some $k$, some $(k+1)$-ary term $\gA_i$, and some constants
  $\bb = (b_1, \dots, b_k)$ and $\bb' = (b_1', \dots, b_k')$ satisfying
  $b_i \betar b_i'$ ($1\leq i\leq k$). 
  By~(\ref{eq:001}), either
  \[
  (a, a) = \bigl(\gA_0(u_0, \bb), \gA_0(u_0, \bb')\bigr)
  \quad \text{ and } \quad 
  (z_1,z_1')= \bigl(\gA_0(v_0, \bb), \gA_0(v_0, \bb')\bigr),
  \]
  or vice-versa. 
  We assumed $u_0 \alphar v_0$ and $b_i \betar b_i'$ ($1\leq i\leq k$),
  so the $\alpha,\beta$-term condition entails
  $\gA_0(u_0, \ba) \commr{\alpha}{\beta} \gA_0(u_0, \ba')$
  iff 
  $\gA_0(v_0, \ba) \commr{\alpha}{\beta} \gA_0(v_0, \ba')$.
  %% \[
  %%   \gA_0(u_0, \bb) \commr{\alpha}{\beta} \gA_0(u_0, \bb')
  %%   \quad \Longleftrightarrow \quad 
  %%   \gA_0(v_0, \bb) \commr{\alpha}{\beta} \gA_0(v_0, \bb').
  %%   \]
  From this and~(\ref{eq:001}) we deduce that 
  $(a,a)\in [\alpha, \beta]$ iff $(z_1,z_1')\in [\alpha, \beta]$.
  Similarly~(\ref{eq:011}) and $u_1 \alphar v_1$ imply
  $(z_1,z_1')\in [\alpha, \beta]$ iff
  $(z_2,z_2')\in [\alpha, \beta]$.  Inductively, and by transitivity of
  $[\alpha, \beta]$, we conclude $(a,a)\in [\alpha, \beta]$ iff
  $(x,y)\in [\alpha, \beta]$.
  Since $(a,a)\in [\alpha, \beta]$, we have $(x,y)\in [\alpha, \beta]$, as desired.

  \end{enumerate}
\end{proof}

\subsection{Computing the Commutator}
As a consequence of the description of the commutator given in the last section,
we now have the following simple method for computing it.

\smallskip

\noindent {\bf Input} \hskip2mm A finite algebra, $\bA = \<A, \dots\>$, and two congruence relations $\alpha$, $\beta \in \Con \bA$.

\smallskip
\noindent {\bf Procedure}
\begin{itemize}
\item {\bf Step 1} \hskip2mm Compute the congruence relation
  $\Deltaba = \Cg^{\bbeta}\bigl\{((a,a), (b,b)) \mid a \alphar b \bigr\}$.
\item {\bf Step 2} \hskip2mm Compute the commutator
  \[[\alpha, \beta] 
  = \bigl\{(x,y) \in A\times A \mid (\exists a \in A) \, (a,a) \Deltabar (x,y)\bigr\} 
  =\bigcup_{a\in A} (a,a)/\Deltaba
    \]
\end{itemize}
Note that $\Deltaba$ is a subalgebra of $\bA^2 \times \bA^2$ and such a congruence
can be computed in polynomial-time in the size of $\bA$. (See~\cite{MR2470585}.)
