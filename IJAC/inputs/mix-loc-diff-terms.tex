%%% NEW SECTION
\subsection{Mixed local difference terms}
\label{sec:mixed-local-diff}
Let $\sV$ be a variety and let $\bA_0=\< A_0, \dots\>$ and  $\bA_1=\< A_1, \dots\>$ be
algebras in $\sV$.  Fix $a_0, b_0 \in A_0$,  $a_1, b_1 \in A_i$,  and
$\chi_i \in \{0,1\}$.
%% An \defn{$\bA$-local difference term for
Define a \defn{mixed local difference term for
$((a_0, b_0,\chi_0), (a_1, b_1,\chi_1))$}
is a ternary term $d$ satisfying, for each $i \in \{0,1\}$,
\begin{align}
%% \text{ if $i=0$, then } & a \comm{\Cg^{\bA}(a,b)}{\Cg^{\bA}(a,b)} d(a,b,b); \label{eq:diff-triple}\\
\text{ if $\chi_i=0$, then } & a_i \comm{\Cg^{\bA_i}(a_i,b_i)}{\Cg^{\bA_i}(a_i,b_i)} d^{\bA_i}(a_i,b_i); \label{eq:mixed-diff-triple}\\
\text{ if $\chi_i=1$, then } &d^{\bA_i}(a_i,a_i,b_i) = b_i. \nonumber
\end{align}
If $d$ satisfies~(\ref{eq:mixed-diff-triple}) for all triples
in some subset $S$ of the disjoint union
\[
\sU(A_0, A_1)  := (A_0 \times A_0 \times \{0,1\}) + (A_1\times A_1 \times \{0,1\}),
\]
%% $(A_0^2 \times \{0,1\}) + (A_1^2 \times \{0,1\})$,
then we call $d$ a \defn{mixed local difference term for $S$}.
Elements of the disjoint union $\sU(A_0, A_1)$ will be denoted by
$\<(a, b, \chi), 0\>$ if $a, b \in A_0$ and
$\<(a, b, \chi), 1\>$ if $a, b \in A_1$.
We will use $\sU$ in place of $\sU(A_0, A_1)$ when the context renders the
universes involved either obvious or irrelevant.

Suppose that all pairs of triples 
in $\sU$ have mixed local difference terms.
%% That is, for each pair
%% $(\<(a, b, \chi), i\>, \<(a', b', \chi'), i'\>)$
%% there exists a term $d$ such that % for each $i \in \{0,1\}$ we have
%% \begin{align}
%%   a \comm{\Cg^{\bA_i}(a,b)}{\Cg^{\bA_i}(a,b)} d^{\bA_i}(a,b,b), & \;
%%   \text{ if $\chi_i=0$, and }  \label{eq:d-trip-i1}\\
%%   d^{\bA_i}(a,a,b) =b, & \;
%%   \text{ if $\chi_i=1$,}\label{eq:d-trip-i2} %\\\nonumber
%% \end{align}
%% and such that similar relations hold for $\<(a', b', \chi'), i'\>$.
Under this hypothesis we prove that every subset $S\subseteq \sU$
has a mixed local difference term.
That is, there is a single term $d$ that works (i.e., satisfies
%% (\ref{eq:mixed-trip-i1}) and (\ref{eq:mixed-trip-i2})) for all $\<(a, b, \chi), i\> \in S$.
the relations (\ref{eq:mixed-diff-triple})) for all $\<(a, b, \chi), i\> \in S$.
The statement and proof of this result follows.

\begin{thm} %[\protect{cf.~\cite[Theorem 2.2]{MR3239624}}]
  \label{thm:mixed-local-diff-terms}
  Let $\sV$ be an idempotent variety and let
  $\bA_0 = \<A_0, \dots\>$ and   $\bA_1 = \<A_1, \dots\>$ be algebras in $\sV$. Define
  $\sU  = (A_0^2 \times \{0,1\}) + (A_1^2\times \{0,1\})$
  and suppose that every pair
  $(\<(a, b, \chi), i\>, \<(a', b', \chi'), i'\>) \in \sU^2$
  has a mixed local difference term. Then every subset $S \subseteq \sU$,
  has a mixed local difference term.
\end{thm}

\begin{proof}
The proof is by induction on the size of $S$.  In the base case, $|S| = 2$,
the claim holds by assumption.
Fix $n\geq 2$ and assume that every subset of $\sU$ of size $2\leq k \leq n$ has
a mixed local
difference term. Let
\[
S =
\{\<(a_0, b_0, \chi_0), \iota_0\>, \<(a_1, b_1, \chi_1), \iota_1\>,
\dots, \<(a_{n}, b_{n},\chi_{n}), \iota_n\>\} \subseteq \sU,\]
so that $|S| = n+1$.  We prove $S$ has a mixed local difference term.

Since $|S| \geq 3$ and $\chi_i \in \{0,1\}$ for all $i$, there must exist
indices $i\neq j$ such that $\chi_i = \chi_j$. Assume without loss of generality
that one of these indices is $j=0$.  Define
the set
$S' = S \setminus \{\<(a_0, b_0, \chi_0), \iota_0\>\}$.
Since $|S'| < |S|$, the set $S'$ has a mixed local difference term $p$.
We split the remainder of the proof into two cases.
%% In the first case $\chi_0 = 0$ and in the second $\chi_0 = 1$.

\vskip3mm

%--------------------------------------
\noindent \underline{Case $\chi_0 = 0$}:
Without loss of generality, suppose that $\chi_1 = %% \chi_2 =
\cdots =\chi_k = 1$,
and $\chi_{k+1} %% = \chi_{k+2} 
= \cdots = \chi_{n} = 0$. Define %% $T$ to be the set
\[T = \{\<(a_0, p(a_0, b_0, b_0), 0), \iota_0\>,
\<(a_1, b_1, 1), \iota_1\>, %% (a_2, b_2, 1), \iota_2\>, 
\dots, \<(a_k, b_k, 1), \iota_k\>\},\] and 
note that $|T| < |S|$.
Let $t$ be a mixed local difference term for $T$.
Define
\[
d(x,y,z) = t(x, p(x,y,y), p(x,y,z)).
\]
We show that $d$ is a mixed local difference term for $S$.
Since $\chi_0 =0$, we need to show
that $(a_0, d^{\bA_{\iota_0}}(a_0,b_0,b_0))$ belongs to $\comm{\Cg^{\bA_{\iota_0}}(a_0,b_0)}{\Cg^{\bA_{\iota_0}}(a_0,b_0)}$.
We have
\begin{equation}
    \label{eq:100000}
  d(a_0,b_0,b_0) =
  t(a_0, p(a_0,b_0,b_0), p(a_0,b_0,b_0))\comm{\tau}{\tau} a_0,
\end{equation}
where we have used $\tau$ to denote $\Cg(a_0, p(a_0,b_0,b_0))$.
Note that
\[(a_0, p(a_0,b_0,b_0)) = (p(a_0,a_0,a_0), p(a_0,b_0,b_0)) \in \Cg(a_0, b_0),\]
%% $(a_0, p(a_0,b_0,b_0))$ is equal to $(p(a_0,a_0,a_0), p(a_0,b_0,b_0))$ which 
%% belongs to $\Cg(a_0, b_0)$,
so $\tau\leq \Cg(a_0,b_0)$. Therefore,
by monotonicity of the commutator we have
$\comm{\tau}{\tau} {\leq} \comm{\Cg(a_0,b_0)}{\Cg(a_0,b_0)}$.
It follows from this and (\ref{eq:100000}) that
%% $d(a_0,b_0,b_0)\comm{\Cg(a_0,b_0)}{\Cg(a_0,b_0)} a_0$,
\[d(a_0,b_0,b_0)\comm{\Cg(a_0,b_0)}{\Cg(a_0,b_0)} a_0,\]
as desired.

For the indices $1\leq i \leq k$ we have $\chi_i =1$, so we prove
$d(a_i,a_i,b_i) = b_i$ for such $i$. Observe,
\begin{align}
  d(a_i,a_i,b_i) &=
  t(a_i, p(a_i,a_i,a_i), p(a_i,a_i,b_i)) \label{eq:200000}\\
  &=t(a_i, a_i, b_i) \label{eq:200001}\\
  &=b_i. \label{eq:200002}
\end{align}
Equation~(\ref{eq:200000}) holds by definition of $d$,~(\ref{eq:200001})
because $p$ is an idempotent mixed local difference term for
$S'$, and~(\ref{eq:200002}) because $t$ is a mixed local difference term for $T$.

The remaining triples in our original set $S$
have indices satisfying $k<j\leq n$ and $\chi_j = 0$.
Thus, for these triples we want
$d(a_j,b_j,b_j)\comm{\Cg(a_j,b_j)}{\Cg(a_j,b_j)} a_j$.
By definition,
\begin{equation}
  \label{eq:450000}
d(a_j,b_j,b_j) =t(a_j, p(a_j,b_j,b_j), p(a_j,b_j,b_j)).  
\end{equation}
Since $p$ is a mixed local difference term for $S'$, %we have
the pair $(p(a_j,b_j,b_j), a_j)$ belongs to $[\Cg(a_j,b_j), \Cg(a_j,b_j)]$.
%% $(p(a_j,b_j,b_j), a_j)\in [\Cg(a_j,b_j), \Cg(a_j,b_j)]$.
This and 
(\ref{eq:450000}) imply
that 
$(d(a_j, b_j,b_j), t(a_j,a_j,a_j))$
belongs to
$\comm{\Cg(a_j,b_j)}{\Cg(a_j,b_j)}$.
Finally, by idempotence of $t$ we have
\[
d(a_j,b_j,b_j)\comm{\Cg(a_j,b_j)}{\Cg(a_j,b_j)} a_j,\]
as desired.
\\[6pt]
%--------------------------------------
\underline{Case $\chi_0 = 1$}:
%% \\[4pt]
%% Assume $\chi_0 = 1$ and, 
Without loss of generality, suppose $\chi_1 = \chi_2 =\cdots =\chi_k = 0$,
and $\chi_{k+1} = \chi_{k+2} = \cdots = \chi_{n} = 1$. Define 
\[
T = \{(p(a_0, a_0, b_0), b_0, 1),
(a_1, b_1, 0), (a_2, b_2 0), \dots, (a_k, b_k, 0)\},
\]
and note that $|T| < |S|$.
Let $t$ be a mixed local difference term for $T$ and
define
%% \[d(x,y,z) = t(p(x,y,z), p(y,y,z), z).\] 
$d(x,y,z) = t(p(x,y,z), p(y,y,z), z)$. 
Since $\chi_0 =1$, we want $d(a_0,a_0,b_0) = b_0$. By the definition of
$d$,
\begin{equation*}
  d(a_0,a_0,b_0) =
  t(p(a_0,a_0,b_0), p(a_0,a_0,b_0), b_0) =b_0.
\end{equation*}
The last equality holds since $t$ is a mixed local difference term for $T$, thus,
for $(p(a_0, a_0, b_0), b_0, 1)$.

If $1\leq i \leq k$, then $\chi_i =0$, so for these indices we prove
that $(a_i, d(a_i,b_i,b_i))$ belongs to $\comm{\Cg(a_i,b_i)}{\Cg(a_i,b_i)}$.
Again, starting from the definition of $d$ and using idempotence of $p$, we have
%% \begin{equation}
%%   \label{eq:40000}
%%   d(a_i,b_i,b_i) =
%%   t(p(a_i,b_i,b_i), p(b_i,b_i,b_i), b_i)=
%%   t(p(a_i,b_i,b_i), b_i, b_i).
%% \end{equation}
\begin{align}
  d(a_i,b_i,b_i) &=
  t(p(a_i,b_i,b_i), p(b_i,b_i,b_i), b_i)   \label{eq:40000}\\
  &=t(p(a_i,b_i,b_i), b_i, b_i). \nonumber
\end{align}
Next, since $p$ is a mixed local difference term for $S'$, we have
\begin{equation}
  \label{eq:50000}
  t(p(a_i,b_i,b_i), b_i, b_i)
 \comm{\Cg(a_i,b_i)}{\Cg(a_i,b_i)}
 t(a_i, b_i, b_i).
\end{equation}
Finally, since $t$ is a mixed local difference term for $T$, hence for
$(a_i, b_i, b_i)$,  %% $(1\leq i \leq k)$,
we have 
$t(a_i, b_i, b_i) \comm{\Cg(a_i,b_i)}{\Cg(a_i,b_i)} a_i$.
Combining this with (\ref{eq:40000}) and (\ref{eq:50000}) yields
$d(a_i,b_i,b_i) \comm{\Cg(a_i,b_i)}{\Cg(a_i,b_i)} a_i$,
as desired.

The remaining elements of our original set $S$
have indices $j$ satisfying $k<j\leq n$ and $\chi_j = 1$.
For these we want $d(a_j,a_j,b_j) = b_j$.
Since $p$ is a mixed local difference term for $S'$, we have
$p(a_j,a_j,b_j) = b_j$, and this along with idempotence of $t$ yields
%%\[ d(a_j,a_j,b_j) =  t(p(a_j,a_j,b_j), p(a_j,a_j,b_j), b_j)=  t(b_j, b_j, b_j) =b_j,\]
\begin{align*}
d(a_j,a_j,b_j) &=
t(p(a_j,a_j,b_j), p(a_j,a_j,b_j), b_j)\\
&=t(b_j, b_j, b_j) =b_j,
\end{align*}
as desired.
\end{proof}

\begin{cor}
  \label{cor:loc-diff-term}
  A finite idempotent algebra $\bA$ has a difference term operation if and
  only if each pair $((a,b,i), (a',b',i')) \in (A\times A \times \{0,1\})^2$ has a mixed local
  difference term.
\end{cor}
\begin{proof}
  One direction is clear, since a difference term operation for $\bA$ is
  obviously a mixed local difference term for the whole set 
  $A\times A \times \{0,1\}$.
  For the converse, suppose
  each pair in $(A\times A \times \{0,1\})^2$ has a mixed local
  difference term. Then, by Theorem~\ref{thm:mixed local-diff-terms},
  there is a single mixed local difference term for the whole set $A\times A \times \{0,1\}$,
  and this is a difference term operation for $\bA$.  Indeed, if $d$ is a
  mixed local difference term for $A\times A \times \{0,1\}$, then 
  for all $a, b \in A$, we have
  $a \comm{\Cg(a,b)}{\Cg(a,b)} d(a,b,b)$,
  since $d$ is a mixed local difference term for $(a,b,0)$, and we have
  $d(a,a,b) = b$, since $d$ is also a mixed local difference term for
  $(a,b,1)$.
\end{proof}
