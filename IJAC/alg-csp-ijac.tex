%% FILE: alg-csp-cib.tex
%% AUTHORS: Clifford Bergman, William DeMeo
%% DATE: 25 July 2016
%% COPYRIGHT: (C) 2016 Clifford Bergman, William DeMeo,

%%%%%%%%%%%%%%%%%%%%%%%%%%%%%%%%%%%%%%%%%%%%%%%%%%%%%%%%%%%%%%%%%%%%%%%%%%%%%%%
%%                         BIBLIOGRAPHY FILE                                 %%
%%%%%%%%%%%%%%%%%%%%%%%%%%%%%%%%%%%%%%%%%%%%%%%%%%%%%%%%%%%%%%%%%%%%%%%%%%%%%%%
%% The `filecontents` command will crete a file in the inputs directory called 
%% refs.bib containing the references in the document, in case this file does 
%% not exist already.
%% If you want to add a BibTeX entry, please don't add it directly to the
%% refs.bib file.  Instead, add it in this file between the
%% \begin{filecontents*}{refs.bib} and \end{filecontents*} lines
%% then delete the existing refs.bib file so it will be automatically generated 
%% again with your new entry the next time you run pdfaltex.
\begin{filecontents*}{inputs/refs.bib}
@article {MR2416347,
    AUTHOR = {Cohen, David and Jeavons, Peter and Gyssens, Marc},
     TITLE = {A unified theory of structural tractability for constraint
              satisfaction problems},
   JOURNAL = {J. Comput. System Sci.},
  FJOURNAL = {Journal of Computer and System Sciences},
    VOLUME = {74},
      YEAR = {2008},
    NUMBER = {5},
     PAGES = {721--743},
      ISSN = {0022-0000},
     CODEN = {JCSSBM},
   MRCLASS = {68Q25 (68T20 90B80)},
  MRNUMBER = {2416347},
       DOI = {10.1016/j.jcss.2007.08.001},
       URL = {http://dx.doi.org/10.1016/j.jcss.2007.08.001},
}
@article {MR1481313,
    AUTHOR = {Jeavons, Peter and Cohen, David and Gyssens, Marc},
     TITLE = {Closure properties of constraints},
   JOURNAL = {J. ACM},
  FJOURNAL = {Journal of the ACM},
    VOLUME = {44},
      YEAR = {1997},
    NUMBER = {4},
     PAGES = {527--548},
      ISSN = {0004-5411},
   MRCLASS = {68Q25 (68Q15 68R05 68T20)},
  MRNUMBER = {1481313 (99a:68089)},
MRREVIEWER = {Armin Cremers},
       DOI = {10.1145/263867.263489},
       URL = {http://dx.doi.org/10.1145/263867.263489},
}
  @unpublished{Barto-shanks,
    AUTHOR = {Libor Barto},
    ABSTRACT = {I will present a theorem saying (roughly) that a subdirect power of
      simple nonabelian Taylor algebras restricted to minimal absorbing subuniverses is
      the full product. Universal algebraic and CSP consequences will be discussed as
      well. This is joint work with Marcin Kozik.},
    TITLE= {A rectangularity theorem for simple Taylor algebras},
    YEAR = {2015},
    note= {Open Problems in Universal Algebra, a Shanks workshop at Vanderbilt University},
    MONTH = {May},
    ADDRESS = {Nashville, TN},
    URL= {http://www.math.vanderbilt.edu/~moorm10/shanks/}
  }
@book {MR2839398,
    AUTHOR = {Bergman, Clifford},
     TITLE = {Universal algebra},
    SERIES = {Pure and Applied Mathematics (Boca Raton)},
    VOLUME = {301},
      NOTE = {Fundamentals and selected topics},
 PUBLISHER = {CRC Press, Boca Raton, FL},
      YEAR = {2012},
     PAGES = {xii+308},
      ISBN = {978-1-4398-5129-6},
   MRCLASS = {08-02 (06-02 08A40 08B05 08B10 08B26)},
  MRNUMBER = {2839398 (2012k:08001)},
MRREVIEWER = {Konrad P. Pi{\'o}ro},
}
@unpublished{Bergman-DeMeo,
    AUTHOR = {Bergman, Clifford and DeMeo, William},
    TITLE = {Universal Algebraic Methods for Constraint Satisfaction Problems:
      with applications to commutative idempotent binars},
    YEAR = {2016},
    NOTE = {unpublished notes; available online},
    URL = {https://github.com/UniversalAlgebra/algebraic-csp}
}
@article {MR3350338,
    AUTHOR = {Bergman, Clifford and Failing, David},
    TITLE = {Commutative idempotent groupoids and the constraint
      satisfaction problem},
    JOURNAL = {Algebra Universalis},
    FJOURNAL = {Algebra Universalis},
    VOLUME = {73},
    YEAR = {2015},
    NUMBER = {3-4},
    PAGES = {391--417},
    ISSN = {0002-5240},
    MRCLASS = {08A70 (08B25 68Q25)},
    MRNUMBER = {3350338},
    DOI = {10.1007/s00012-015-0323-6},
    URL = {http://dx.doi.org/10.1007/s00012-015-0323-6},
  }
@article {MR2563736,
  AUTHOR = {Berman, Joel and Idziak, Pawe{\l} and Markovi{\'c}, Petar and
      McKenzie, Ralph and Valeriote, Matthew and Willard, Ross},
    TITLE = {Varieties with few subalgebras of powers},
    JOURNAL = {Trans. Amer. Math. Soc.},
    FJOURNAL = {Transactions of the American Mathematical Society},
    VOLUME = {362},
    YEAR = {2010},
    NUMBER = {3},
    PAGES = {1445--1473},
    ISSN = {0002-9947},
    CODEN = {TAMTAM},
    MRCLASS = {08B05 (08A30 08A70 08B10 68Q25 68Q32 68T20)},
    MRNUMBER = {2563736 (2010k:08010)},
    MRREVIEWER = {Ivan Chajda},
    DOI = {10.1090/S0002-9947-09-04874-0},
    URL = {http://dx.doi.org/10.1090/S0002-9947-09-04874-0}
  }
  @article{MR1630445,
    AUTHOR = {Feder, Tom{\'a}s and Vardi, Moshe Y.},
    TITLE = {The computational structure of monotone monadic {SNP} and
      constraint satisfaction: a study through {D}atalog and group
      theory},
    JOURNAL = {SIAM J. Comput.},
    FJOURNAL = {SIAM Journal on Computing},
    VOLUME = {28},
    YEAR = {1999},
    NUMBER = {1},
    PAGES = {57--104 (electronic)},
    ISSN = {0097-5397},
    MRCLASS = {68Q15 (03D15 68R05)},
    MRNUMBER = {1630445 (2000e:68063)},
    MRREVIEWER = {A. M. Dawes},
    DOI = {10.1137/S0097539794266766},
    URL = {http://dx.doi.org/10.1137/S0097539794266766}
  }
  @article{MR0464698,
    AUTHOR = {Ladner, Richard E.},
    TITLE = {On the structure of polynomial time reducibility},
    JOURNAL = {J. Assoc. Comput. Mach.},
    FJOURNAL = {Journal of the Association for Computing Machinery},
    VOLUME = {22},
    YEAR = {1975},
    PAGES = {155--171},
    ISSN = {0004-5411},
    MRCLASS = {68A20},
    MRNUMBER = {0464698 (57 \#4623)},
    MRREVIEWER = {Pavel Strnad}
  }

@incollection {MR2210131,
    AUTHOR = {Krokhin, Andrei and Bulatov, Andrei and Jeavons, Peter},
     TITLE = {The complexity of constraint satisfaction: an algebraic
              approach},
 BOOKTITLE = {Structural theory of automata, semigroups, and universal
              algebra},
    SERIES = {NATO Sci. Ser. II Math. Phys. Chem.},
    VOLUME = {207},
     PAGES = {181--213},
      NOTE = {Notes taken by Alexander Semigrodskikh},
 PUBLISHER = {Springer, Dordrecht},
      YEAR = {2005},
   MRCLASS = {68Q25 (68T20 90C60)},
  MRNUMBER = {2210131 (2006m:68053)},
MRREVIEWER = {Benoit Larose},
       DOI = {10.1007/1-4020-3817-8_8},
       URL = {http://dx.doi.org/10.1007/1-4020-3817-8_8},
}
  @article {MR2137072,
    AUTHOR = {Bulatov, Andrei and Jeavons, Peter and Krokhin, Andrei},
    TITLE = {Classifying the complexity of constraints using finite
      algebras},
    JOURNAL = {SIAM J. Comput.},
    FJOURNAL = {SIAM Journal on Computing},
    VOLUME = {34},
    YEAR = {2005},
    NUMBER = {3},
    PAGES = {720--742},
    ISSN = {0097-5397},
    CODEN = {SMJCAT},
    MRCLASS = {68T20 (08A70 68Q25)},
    MRNUMBER = {2137072 (2005k:68181)},
    MRREVIEWER = {Benoit Larose},
    DOI = {10.1137/S0097539700376676},
    URL = {http://dx.doi.org/10.1137/S0097539700376676}
  }
@article {MR0434928,
    AUTHOR = {Taylor, Walter},
     TITLE = {Varieties obeying homotopy laws},
   JOURNAL = {Canad. J. Math.},
  FJOURNAL = {Canadian Journal of Mathematics. Journal Canadien de
              Math\'ematiques},
    VOLUME = {29},
      YEAR = {1977},
    NUMBER = {3},
     PAGES = {498--527},
      ISSN = {0008-414X},
   MRCLASS = {08A25},
  MRNUMBER = {0434928 (55 \#7891)},
MRREVIEWER = {James B. Nation},
}
  @BOOK{HM:1988,
    AUTHOR = {Hobby, David and McKenzie, Ralph},
    TITLE = {The structure of finite algebras},
    SERIES = {Contemporary Mathematics},
    VOLUME = {76},
    PUBLISHER = {American Mathematical Society},
    ADDRESS = {Providence, RI},
    YEAR = {1988},
    PAGES = {xii+203},
    ISBN = {0-8218-5073-3},
    MRCLASS = {08A05 (03C05 08-02 08B05)},
    MRNUMBER = {958685 (89m:08001)},
    MRREVIEWER = {Joel Berman},
    note = {Available from:
      \href{http://math.hawaii.edu/~ralph/Classes/619/HobbyMcKenzie-FiniteAlgebras.pdf}{math.hawaii.edu}}
  }
  @article {Freese:2009,
    AUTHOR = {Freese, Ralph and Valeriote, Matthew A.},
    TITLE = {On the complexity of some {M}altsev conditions},
    JOURNAL = {Internat. J. Algebra Comput.},
    FJOURNAL = {International Journal of Algebra and Computation},
    VOLUME = {19},
    YEAR = {2009},
    NUMBER = {1},
    PAGES = {41--77},
    ISSN = {0218-1967},
    MRCLASS = {08B05 (03C05 08B10 68Q25)},
    MRNUMBER = {2494469 (2010a:08008)},
    MRREVIEWER = {Clifford H. Bergman},
    DOI = {10.1142/S0218196709004956},
    URL = {http://dx.doi.org/10.1142/S0218196709004956}
  }
  @incollection {MR2648455,
    AUTHOR = {Barto, Libor and Kozik, Marcin},
    TITLE = {Constraint satisfaction problems of bounded width},
    BOOKTITLE = {2009 50th {A}nnual {IEEE} {S}ymposium on {F}oundations of
              {C}omputer {S}cience ({FOCS} 2009)},
    PAGES = {595--603},
    PUBLISHER = {IEEE Computer Soc., Los Alamitos, CA},
    YEAR = {2009},
    MRCLASS = {68Q25 (68T20)},
    MRNUMBER = {2648455 (2011d:68051)},
    DOI = {10.1109/FOCS.2009.32},
    URL = {http://dx.doi.org/10.1109/FOCS.2009.32},
}
@incollection {MR2953899,
    AUTHOR = {Barto, Libor and Kozik, Marcin},
     TITLE = {New conditions for {T}aylor varieties and {CSP}},
 BOOKTITLE = {25th {A}nnual {IEEE} {S}ymposium on {L}ogic in {C}omputer
              {S}cience {LICS} 2010},
     PAGES = {100--109},
 PUBLISHER = {IEEE Computer Soc., Los Alamitos, CA},
      YEAR = {2010},
   MRCLASS = {08A70 (68Q25)},
  MRNUMBER = {2953899},
}
  @article {MR2893395,
    AUTHOR = {Barto, Libor and Kozik, Marcin},
    TITLE = {Absorbing subalgebras, cyclic terms, and the constraint
      satisfaction problem},
    JOURNAL = {Log. Methods Comput. Sci.},
    FJOURNAL = {Logical Methods in Computer Science},
    VOLUME = {8},
    YEAR = {2012},
    NUMBER = {1},
    PAGES = {1:07, 27},
    ISSN = {1860-5974},
    MRCLASS = {68Q17 (08A70)},
    MRNUMBER = {2893395},
    DOI = {10.2168/LMCS-8(1:7)2012},
    URL = {http://dx.doi.org/10.2168/LMCS-8(1:7)2012}
  }
  @article {Siggers:2010,
    author = {Mark Siggers}, 
    title ={A strong {M}al'cev condition for locally finite varieties omitting the
      unary type},
    journal = {Algebra Universalis},
    volume = {64},
    year = {2010}
  }
@article {MR2333368,
    AUTHOR = {Kearnes, Keith A. and Tschantz, Steven T.},
     TITLE = {Automorphism groups of squares and of free algebras},
   JOURNAL = {Internat. J. Algebra Comput.},
  FJOURNAL = {International Journal of Algebra and Computation},
    VOLUME = {17},
      YEAR = {2007},
    NUMBER = {3},
     PAGES = {461--505},
      ISSN = {0218-1967},
   MRCLASS = {08A35 (08B20 20B25)},
  MRNUMBER = {2333368},
MRREVIEWER = {Giovanni Ferrero},
       DOI = {10.1142/S0218196707003615},
       URL = {http://dx.doi.org/10.1142/S0218196707003615},
}
@article {MR3229955,
    AUTHOR = {Kearnes, Keith and Markovi{\'c}, Petar and McKenzie, Ralph},
    TITLE = {Optimal strong {M}al'cev conditions for omitting type 1 in
      locally finite varieties},
    JOURNAL = {Algebra Universalis},
    FJOURNAL = {Algebra Universalis},
    VOLUME = {72},
    YEAR = {2014},
    NUMBER = {1},
    PAGES = {91--100},
    ISSN = {0002-5240},
    MRCLASS = {08B05},
    MRNUMBER = {3229955},
    DOI = {10.1007/s00012-014-0289-9},
    URL = {http://dx.doi.org/10.1007/s00012-014-0289-9}
  }
@article {MR2926316,
    AUTHOR = {Markovi{\'c}, Petar and Mar{\'o}ti, Mikl{\'o}s and McKenzie,
              Ralph},
     TITLE = {Finitely related clones and algebras with cube terms},
   JOURNAL = {Order},
  FJOURNAL = {Order. A Journal on the Theory of Ordered Sets and its
              Applications},
    VOLUME = {29},
      YEAR = {2012},
    NUMBER = {2},
     PAGES = {345--359},
      ISSN = {0167-8094},
     CODEN = {ORDRE5},
   MRCLASS = {08A62 (08A99 08B05 08B10)},
  MRNUMBER = {2926316},
MRREVIEWER = {Anna Romanowska},
       DOI = {10.1007/s11083-011-9232-2},
       URL = {http://dx.doi.org/10.1007/s11083-011-9232-2},
}
@article {MR3076179,
    AUTHOR = {Kearnes, Keith A. and Kiss, Emil W.},
     TITLE = {The shape of congruence lattices},
   JOURNAL = {Mem. Amer. Math. Soc.},
  FJOURNAL = {Memoirs of the American Mathematical Society},
    VOLUME = {222},
      YEAR = {2013},
    NUMBER = {1046},
     PAGES = {viii+169},
      ISSN = {0065-9266},
      ISBN = {978-0-8218-8323-5},
   MRCLASS = {08B05 (08B10)},
  MRNUMBER = {3076179},
MRREVIEWER = {James B. Nation},
       DOI = {10.1090/S0065-9266-2012-00667-8},
       URL = {http://dx.doi.org/10.1090/S0065-9266-2012-00667-8},
}
@incollection {MR1404955,
    AUTHOR = {Kearnes, Keith A.},
     TITLE = {Idempotent simple algebras},
 BOOKTITLE = {Logic and algebra ({P}ontignano, 1994)},
    SERIES = {Lecture Notes in Pure and Appl. Math.},
    VOLUME = {180},
     PAGES = {529--572},
 PUBLISHER = {Dekker, New York},
      YEAR = {1996},
   MRCLASS = {08B05 (06F25 08A05 08A30)},
  MRNUMBER = {1404955 (97k:08004)},
MRREVIEWER = {E. W. Kiss},
}
  @article {MR3374664,
    AUTHOR = {Barto, Libor and Kozik, Marcin and Stanovsk{\'y}, David},
    TITLE = {Mal'tsev conditions, lack of absorption, and solvability},
    JOURNAL = {Algebra Universalis},
    FJOURNAL = {Algebra Universalis},
    VOLUME = {74},
    YEAR = {2015},
    NUMBER = {1-2},
    PAGES = {185--206},
    ISSN = {0002-5240},
    MRCLASS = {08B05 (08A05)},
    MRNUMBER = {3374664},
    DOI = {10.1007/s00012-015-0338-z},
    URL = {http://dx.doi.org/10.1007/s00012-015-0338-z},
  }  
@misc{william_demeo_2016_53936,
  author       = {William DeMeo and
                  Ralph Freese},
  title        = {AlgebraFiles v1.0.1},
  month        = May,
  year         = 2016,
  doi          = {10.5281/zenodo.53936},
  url          = {http://dx.doi.org/10.5281/zenodo.53936}
}
@article {MR2678065,
    AUTHOR = {Idziak, Pawe{\l} and Markovi{\'c}, Petar and McKenzie, Ralph
              and Valeriote, Matthew and Willard, Ross},
     TITLE = {Tractability and learnability arising from algebras with few
              subpowers},
   JOURNAL = {SIAM J. Comput.},
  FJOURNAL = {SIAM Journal on Computing},
    VOLUME = {39},
      YEAR = {2010},
    NUMBER = {7},
     PAGES = {3023--3037},
      ISSN = {0097-5397},
     CODEN = {SMJCAT},
   MRCLASS = {68Q25 (08A70 68T99)},
  MRNUMBER = {2678065},
       DOI = {10.1137/090775646},
       URL = {http://dx.doi.org/10.1137/090775646},
}
@article{FreeseMcKenzie2016,
	Author = {Freese, R. and McKenzie, R.},
	Date-Added = {2016-08-22 19:43:56 +0000},
	Date-Modified = {2016-08-22 19:45:50 +0000},
	Journal = {Algebra Universalis},
	Title = {Mal'tsev families of varieties closed under join or Mal'tsev product},
	Year = {to appear}
}
@article {MR2470592,
    AUTHOR = {Mar{\'o}ti, Mikl{\'o}s and McKenzie, Ralph},
     TITLE = {Existence theorems for weakly symmetric operations},
   JOURNAL = {Algebra Universalis},
  FJOURNAL = {Algebra Universalis},
    VOLUME = {59},
      YEAR = {2008},
    NUMBER = {3-4},
     PAGES = {463--489},
      ISSN = {0002-5240},
   MRCLASS = {08B05 (08A70)},
  MRNUMBER = {2470592},
MRREVIEWER = {James B. Nation},
       DOI = {10.1007/s00012-008-2122-9},
       URL = {http://dx.doi.org/10.1007/s00012-008-2122-9},
}
@article {MR2212000,
    AUTHOR = {Bulatov, Andrei A.},
     TITLE = {A dichotomy theorem for constraint satisfaction problems on a
              3-element set},
   JOURNAL = {J. ACM},
  FJOURNAL = {Journal of the ACM},
    VOLUME = {53},
      YEAR = {2006},
    NUMBER = {1},
     PAGES = {66--120},
      ISSN = {0004-5411},
   MRCLASS = {68Q25 (68T20)},
  MRNUMBER = {2212000},
MRREVIEWER = {Benoit Larose},
       DOI = {10.1145/1120582.1120584},
       URL = {http://dx.doi.org/10.1145/1120582.1120584},
}
@incollection {MR521057,
    AUTHOR = {Schaefer, Thomas J.},
     TITLE = {The complexity of satisfiability problems},
 BOOKTITLE = {Conference {R}ecord of the {T}enth {A}nnual {ACM} {S}ymposium
              on {T}heory of {C}omputing ({S}an {D}iego, {C}alif., 1978)},
     PAGES = {216--226},
 PUBLISHER = {ACM, New York},
      YEAR = {1978},
   MRCLASS = {68C25 (03B05 03D15 05C15 68E10)},
  MRNUMBER = {521057},
MRREVIEWER = {M. I. Dekhtyar},
}
@article {MR2504025,
    AUTHOR = {Valeriote, Matthew A.},
     TITLE = {A subalgebra intersection property for congruence distributive
              varieties},
   JOURNAL = {Canad. J. Math.},
  FJOURNAL = {Canadian Journal of Mathematics. Journal Canadien de
              Math\'ematiques},
    VOLUME = {61},
      YEAR = {2009},
    NUMBER = {2},
     PAGES = {451--464},
      ISSN = {0008-414X},
     CODEN = {CJMAAB},
   MRCLASS = {08B10 (08A30 08B05)},
  MRNUMBER = {2504025},
MRREVIEWER = {Jarom{\'{\i}}r Duda},
       DOI = {10.4153/CJM-2009-023-2},
       URL = {http://dx.doi.org/10.4153/CJM-2009-023-2},
}
@misc{UACalc,
	Author = {Ralph Freese and Emil Kiss and Matthew Valeriote},
	Date-Added = {2014-11-20 01:52:20 +0000},
	Date-Modified = {2014-11-20 01:52:20 +0000},
	Note = {Available at: {\verb+www.uacalc.org+}},
	Title = {Universal {A}lgebra {C}alculator},
	Year = {2011}
}
@article{Freese2008,
	Author = {Freese, R.},
	Date-Added = {2016-08-29 01:31:23 +0000},
	Date-Modified = {2016-08-29 01:32:09 +0000},
	Journal = {Alg. Univ.},
	Pages = {337--343},
	Title = {Computing congruences efficiently},
	Volume = {59},
	Year = {2008}
}	
@unpublished{Markovic:2011,
	Author = {B. Ba\v si \'c and J. Jovanovi\'c and P. Markovi\'c and others},
	Date-Added = {2016-09-14 22:50:27 +0000},
	Date-Modified = {2016-09-14 23:00:13 +0000},
	Month = {June},
	Title = {{CSP} on small templates},
	Note = {\url{http://2oal.tcs.uj.edu.pl}},
	Year = {2011}
}
@article {MR2900858,
    AUTHOR = {Kearnes, Keith A. and Szendrei, {\'A}gnes},
     TITLE = {Clones of algebras with parallelogram terms},
   JOURNAL = {Internat. J. Algebra Comput.},
  FJOURNAL = {International Journal of Algebra and Computation},
    VOLUME = {22},
      YEAR = {2012},
    NUMBER = {1},
     PAGES = {1250005, 30},
      ISSN = {0218-1967},
   MRCLASS = {08A40 (08A05)},
  MRNUMBER = {2900858},
MRREVIEWER = {Michael Pinsker},
       DOI = {10.1142/S0218196711006716},
       URL = {http://dx.doi.org/10.1142/S0218196711006716},
}
\end{filecontents*}
%:biblio
%%%%%%%%%%%%%%%%%%%%%%%%%%%%%%%%%%%%%%%%%%%%%%%%%%%%%%%%%%%%%%%%%%%%%%%%%%%%%%%%%%%%
%%                                     PREAMBLE                                   %%
%%%%%%%%%%%%%%%%%%%%%%%%%%%%%%%%%%%%%%%%%%%%%%%%%%%%%%%%%%%%%%%%%%%%%%%%%%%%%%%%%%%%
\documentclass{ws-ijac}
\usepackage{graphicx}
\usepackage{comment}
%%%%%%%%%%%%%%%%%%%%%%%%%%%%%%%%%%%%%%%%%%%%%%%
%% showkeys: just comment out in the final version
%\usepackage[notref,notcite]{showkeys}
%%%%%%%%%%%%%%%%%%%%%%%%%%%%%%%%%%%%%%%%%%%%%%%


%% \usepackage{setspace}\onehalfspacing

%% removed these for ijac
\usepackage{amsmath}
% \usepackage{amscd,amssymb,amsthm} %, amsmath are included by default
\usepackage{mathtools}
% \usepackage{scrextend}

\usepackage{bm}
\usepackage{latexsym,stmaryrd,mathrsfs,enumerate,scalefnt,ifthen}
\usepackage[mathscr]{euscript}
%% \usepackage{yfonts}
% \usepackage{eufrak}
\usepackage[colorlinks=true,urlcolor=black,linkcolor=black,citecolor=black]{hyperref}
\usepackage{url}
\usepackage{scalefnt}
\usepackage{tikz}
\usepackage{color}
%% \usepackage[margin=1.5in]{geometry}

%% \usepackage{cleveref} %[2012/02/15]% v0.18.4; 
% 0.16.1 of May 2010 would be sufficient, but what is the exact day?

\usepackage[multiple]{footmisc}  %% for multiple footnote marks on same text

%% \crefformat{footnote}{#2\footnotemark[#1]#3}

\newtheorem{Fact}{Fact}[section]

%%%%%%%%%%%%%%%%%%%%%%%%%%%%%%%%%%%%%%%%
% Acronyms
%%%%%%%%%%%%%%%%%%%%%%%%%%%%%%%%%%%%%%%%
%% \usepackage[acronym, shortcuts]{glossaries}
%\usepackage[smaller]{acro}
\usepackage[smaller]{acronym}
\usepackage{xspace}

%% \acs{CSP} -- short version of the acronym\\
%% \acl{CSP} -- expanded acronym without mentioning the acronym.\\
%% \acp{CSP} -- plurals.\\
%% \acfp{CSP} -- long forms into plurals.\\
%% \acsp{CSP} -- short form into a plural.\\
%% \aclp{CSP} -- long form into a plural.\\
%% \acfi{CSP} -- Full Name acronym in italics and abbreviated form in upshape.\\
%% \acsu{CSP} -- short form of the acronym and marks it as used.\\
%% \aclu{CSP} -- Prints the long form of the acronym and marks it as used.\\

%%%%%%%%%%%%%%%%%%%%%%%%%%%%%%%%%%%%%%%%%%%%%%%%%%%%%%%%%%%%%%%%%%
%%                      TODO items                              %%
%%%%%%%%%%%%%%%%%%%%%%%%%%%%%%%%%%%%%%%%%%%%%%%%%%%%%%%%%%%%%%%%%%
%% Use the \todo command to make notes of things to fix. 
%% Apply italics or bold or caps for emphasis as needed, e.g.,
%%
%%     \todo{(wjd) FIX THE NEXT LINE!}
%%     \todo{(wjd) \emph{Revise the paragraph above}}
%%
   \newboolean{todos}
   \setboolean{todos}{true}  % set to true to include TODO statements
%   \setboolean{todos}{false}  % set to false to exclude TODO statements

   %%% INTENDED USE OF THE arxiv AND extralong BOOLEAN VARIABLES
   %%% -- The brief journal version should have `arxiv` and `extralong` variables set to false.
   %%% -- The arxiv version should have `arxiv` set to true and `extralong` set to false.
   %%% -- The extralong version may contain notes intended for our own personal reference.
   %%%    For the extralong version, set both `arxiv` and `extralong` to true.   
   \newboolean{arxiv}
   \setboolean{arxiv}{true}  % set to true to include almost everything
   \setboolean{arxiv}{false}  % set to false for the brief version

   \newboolean{extralong}
   \setboolean{extralong}{true}  % set to true to include everything
   \setboolean{extralong}{false}  % set to false for the long (but not too long) version

   \newboolean{footnotes}
   \setboolean{footnotes}{true}  % set to true to include footnotes
   \setboolean{footnotes}{false}  % set to false for no footnotes

%%%% wjd: adding pagebreaks for ``draft mode'' to reduce printing costs
%%%%      To turn off these unnecessary page breaks, set `draft` to false:
   \newboolean{draft}
   \setboolean{draft}{true}  % set to true to include footnotes

%%
%% INCLUDE TODO ITEMS by commenting out `\setboolean{todos}{false}`
%% OMIT TODO ITEMS by uncommenting `\setboolean{todos}{false}`:
%%   \setboolean{todos}{false}
%%
   \newcommand{\todo}[1]{\ifthenelse{\boolean{todos}}{%
       \noindent {\bf To do:} #1}{}}
   %%

\newcommand{\longversionlink}{
  An extended version of this paper~\cite{Bergman-DeMeo} is available on
  the~\href{https://arxiv.org/find/math/1/au:+DeMeo_W/0/1/0/all/0/1}%
  {arXiv: http://tinyurl.com/jhw8ara}.}


%%%%%%%%%%%%%%%%%%%%%%%%%%%%%%%%%%%%%%%%%%%%%%%%%%%%%%%%%%%%%%%%%
\usepackage{inputs/proof-dashed}
\acrodef{lics}[LICS]{Logic in Computer Science}
\acrodef{sat}[SAT]{satisfiability}
\acrodef{nae}[NAE]{not-all-equal}
\acrodef{ctb}[CTB]{cube term blocker}
\acrodef{tct}[TCT]{tame congruence theory}
\acrodef{wnu}[WNU]{weak near-unanimity}
\acrodef{CSP}[CSP]{constraint satisfaction problem}
\acrodef{MAS}[MAS]{minimal absorbing subuniverse}
\acrodef{MA}[MA]{minimal absorbing}
\acrodef{cib}[CIB]{commutative idempotent binar}
\acrodef{sd}[SD]{semidistributive}
\acrodef{NP}[NP]{nondeterministic polynomial time}
\acrodef{P}[P]{polynomial time}
\acrodef{PeqNP}[P $ = $ NP]{P is NP}
\acrodef{PneqNP}[P $ \neq $ NP]{P is not NP}


%%%%%%%%%%%%%%%%%%%%%%%%%%%%%%%%%%%%%%%%%%%%%%%%%%%%%%%%%%%%%%%%%
%%% MY STUFF
\usepackage{inputs/macros}
%% END MY STUFF %%%%%%%%%%%%%%%%%%%%%%%%%%%%%
%%%CLIFF'S STUFF
%% For debugging we want to turn a few facilitites on
%For marginal notes. 
\usepackage{pifont}
\usepackage{bm}
\newcommand\mpar[1]{\mbox{}\marginpar{\raggedright\hspace{0pt}\small #1}}
%\newcommand\mpar[1]{\relax} %uncomment to turn off marginal notes
\newcommand{\towjd}[1]{\par \noindent \mpar{\ding{'120}}{{\bf chb:} \emph{#1}}\par}
\newcommand{\tochb}[1]{\par \noindent \mpar{\ding{'121}}{{\bf wjd:} \emph{#1}}\par}
\let\:\colon
\newcommand{\card}[1]{|#1|}
\DeclareMathOperator{\size}{size}
\newcommand{\sansV}{\ensuremath{\mathsf{V}}}
\newcommand{\Sl}{\ensuremath{Sl}} %I don't know what font to use
\newcommand{\Sq}[1]{\ensuremath{\mathbf{Sq}_{#1}}}
\newcommand{\sdm}{\sd-\meet\xspace}
\newcommand{\casespec}[1]{\medskip\noindent\textbf{#1}.\enspace}
%:Cliff's macs
%%% END OF CLIFF'S STUFF

%%% For pseudocode
\usepackage{algorithm}
\usepackage[noend]{algpseudocode}
%% If you are using ubuntu linux and you get the error
%%     LaTeX Error: File `algorithm.sty' not found.
%% then do `sudo apt-get install texlive-science`

%% \linenumbers

%%%%%%%%%%%%%%%%%%%%%%%%%%%%%%%%%%%%%%%%%%%%%%%%%%%%%%%%%%%%%%%%%%%%%%
%%                        FRONT MATTER                              %%
%%%%%%%%%%%%%%%%%%%%%%%%%%%%%%%%%%%%%%%%%%%%%%%%%%%%%%%%%%%%%%%%%%%%%%

\begin{document}

\markboth{Clifford Bergman and William DeMeo}
{Algebraic Methods for CSP}

%%%%%%%%%%%%%%%%%%%%% Publisher's Area please ignore %%%%%%%%%%%%%%%
%
\catchline{}{}{}{}{}
%
%%%%%%%%%%%%%%%%%%%%%%%%%%%%%%%%%%%%%%%%%%%%%%%%%%%%%%%%%%%%%%%%%%%%

\title{UNIVERSAL ALGEBRAIC METHODS FOR\\
CONSTRAINT SATISFACTION PROBLEMS\\
with applications to commutative idempotent binars}
%% \subtitle{with applications to commutative idempotent binars}

%///////////////////////////////////////////////////////////////////////////////////////
%% AUTHORS
\author{CLIFFORD BERGMAN}
\address{Department of Mathematics, Iowa State University\\Ames, IA 50010, USA\\
\email{cbergman@iastate.edu}}

\author{WILLIAM DEMEO}
\address{Department of Mathematics, Iowa State University\\Ames, IA 50010, USA\\
\email{williamdemeo@gmail.com}}


\maketitle

% ///////////////////////////////////////////////////////////////////////////////////////
%% Date
\begin{history}
\received{(Day Month Year)}
\accepted{(Day Month Year)}
\comby{[editor]}
\end{history}



\begin{abstract}
After substantial progress over the last 15
years, the ``algebraic \csp-dichotomy conjecture'' reduces to the following:
the \ac{CSP} associated with a finite idempotent algebra is tractable 
if and only if the algebra has a Taylor term operation.
Despite the tremendous achievements that have been made in this area,
there remain examples of small algebras with just a single binary operation
whose \csp resists classification as tractable or \NP-complete using known methods.   
In this paper we present some new methods for approaching this problem, 
with particular focus on those techniques that help us attack 
the class of finite algebras known as \emph{commutative idempotent binars} (\cibs).  
We demonstrate the utility of these methods by using them to prove 
that every \cib of cardinality at most~4 yields a tractable \csp. 
\end{abstract}


%///////////////////////////////////////////////////////////////////////////////////////
%% KEYWORDS
\keywords{%
  constraint satisfaction problem;
  Mal'tsev condition;
  absorption.
}



%///////////////////////////////////////////////////////////////////////////////////////
%% AMS SUBJECT CLASSIFICATION
%% see http://www.ams.org/msc Only one Primary. Possibly several Secondary. 
\ccode{Mathematics Subject Classification 2010: 
  Primary:       %
  Secondary:     %
}


%%%%%%%%%%%%%%%%%%%%%%%%%%%%%%%%%%%%%%%%%%%%%%%%%%%%%%%%%%%%%%%%%%%%%%%%%%%%%%%%%%
%%                           Main Matter                                        %%
%%%%%%%%%%%%%%%%%%%%%%%%%%%%%%%%%%%%%%%%%%%%%%%%%%%%%%%%%%%%%%%%%%%%%%%%%%%%%%%%%%


%/////////////////////////////////////////////////////////

% \thanks{25 July 2016 (draft 1)\\
%   The authors would like to extend special thanks to Libor Barto and Marcin Kozik for sharing
%   their ideas and results with us, to Alexander Kazda and Matthew Moore for organizing the special
%   2015 Shanks Workshop at Vanderbilt on ``Open Problems in Universal Algebra,'' and to the National
%   Science Foundation and Iowa State University for supporting this work.}



\section{Introduction}
\label{sec:introduction}
The ``\csp-dichotomy conjecture'' of Feder and Vardi asserts that every constraint satisfaction
problem (\csp) over a fixed finite constraint language is either \NP-complete or tractable.

A discovery of Jeavons, Cohen and Gyssens in~\cite{MR1481313}---later refined by Bulatov,
Jeavons and Krokhin in~\cite{MR2137072}---was the ability to transfer the question of the 
complexity of the \csp over a set of relations to a question of algebra. 
Specifically, these authors showed that the complexity of any particular \csp 
depends solely on the \emph{polymorphisms} of the constraint relations, 
that is, the functions preserving all the constraints. 
The transfer to universal algebra was made complete by Bulatov, Jeavons, and Krokhin 
in recognizing that to any set $\sR$ of constraint relations one can
associate an algebra $\bA(\sR)$ whose operations consist of the polymorphisms
of $\sR$. Following this, the \csp-dichotomy conjecture
of Feder and Vardi was recast as a universal algebra problem once it was
recognized that the conjectured sharp dividing line between those \csps that are
\NP-complete and those that are tractable was seen to depend upon
universal algebraic properties of the associated algebra. 
One such property is the existence of a Taylor term 
(defined in Section~\ref{ssec:term-ops}). Roughly speaking,
the ``algebraic \csp-dichotomy conjecture'' is the following:
The \csp associated with a finite idempotent algebra
is tractable if and only if the algebra has a Taylor term operation in its
clone. We state this more precisely as follows:

\begin{quote}
  {\bf Conjecture:} If $\bA$ is an algebra with a Taylor term it its
  clone and if $\sR$ is a finite set of relations compatible with $\bA$, then  
  $\CSP(\sR)$ is tractable.  Conversely, if $\bA$ is an idempotent algebra with
  no Taylor term in its clone, then there exists a finite set $\sR$ of relations
  such that $\CSP(\sR)$ is \NP-complete.
\end{quote}
The second sentence of the conjecture was already
established in~\cite{MR2137072}. One goal of this paper is to provide further
evidence in support of the first sentence.

%%%%%%%%%%%%%%%%%%%
%%Start of Cliff's Introduction%%%%
%%%%%%%%%%%%%%%%%%%
Algebraists have identified two quite different techniques for proving that a finite idempotent algebra is tractable. One, often called the ``local consistency algorithm,'' works for any finite algebra lying in an idempotent, congruence-meet-semidistributive variety (\sd-\meet\ for short). See~\cite{MR2648455} or~\cite{MR2893395} for details. The other, informally called the ``few subpowers technique,'' applies to any finite idempotent algebra possessing an edge term~\cite{MR2678065}. Definitions of these terms appear in Section~\ref{ssec:edge-sdm}. 

While these two algorithms cover a wide class of interesting algebras, they are not enough to resolve the above conjecture. So the question is, what is the way forward from here? One might hope for an entirely new algorithm, but none seem to be on the horizon. Alternately, we can try to combine the two existing approaches in a way that captures the outstanding cases. Several researchers have investigated this idea. In this paper, we do as well.

For example, suppose that $\bA$ is a finite, idempotent algebra possessing a congruence, $\theta$, such that $\bA/\theta$ lies in an \sd-\meet\ variety and every congruence class of $\theta$ has an edge term. It is not hard to show that $\bA$ has a Taylor term. Can local consistency be combined with few subpowers to prove that $\bA$ is tractable? 

We can formalize this idea as follows. Let $\var{V}$ and $\var{W}$ be idempotent varieties. The \defn{\malcev product} of $\var{V}$ and $\var{W}$ is the class
\begin{equation*}
\var{V} \circ \var{W} = \{\,\bA : (\exists\; \theta\in \Con(\bA))\;\;\bA/\theta \in \var{W} \mathrel{\&} (\forall a\in A)\;a/\theta \in \var{V}\,\}.
\end{equation*}
$\var{V}\circ \var{W}$ is always an idempotent quasivariety, but is generally not closed under homomorphic images. A long term goal would be to prove that if all finite members of $\var{V}$ and $\var{W}$ are tractable, then the same will hold for all members of $\var{V}\circ \var{W}$. Tellingly, Freese and McKenzie show in~\cite{FreeseMcKenzie2016} that a number of important properties are preserved by \malcev product.

\begin{theorem}\label{thm:robust}
Let $\var{V}$ and $\var{W}$ be idempotent varieties. For each of the following properties, $P$\!, if both $\var{V}$ and $\var{W}$ have $P$\!, then so does  $\sansH(\var{V}\circ \var{W})$.
\begin{enumerate}
\item Being idempotent;
\item
having a Taylor term;
\item
being \sd-\meet;
\item
having an edge term.
\end{enumerate}
\end{theorem}

It follows from the theorem that if both $\var{V}$ and $\var{W}$ are \sd-\meet, or both have an edge term, then every finite member of $\sansH(\var{V}\circ \var{W})$ will be tractable. So the next step would be to consider $\var{V}$ with one of these properties and $\var{W}$ with the other. 

But even if we could prove that tractability is preserved by \malcev product, that would not be enough. In particular, simple algebras would remain problematic. Interestingly, Barto and Kozik have recently developed an approach that seems particularly well-suited to simple algebras.
In \lics'10~(\cite{MR2953899}), these researchers proved a powerful ``Absorption Theorem'' 
for products of two absorption-free algebras in a Taylor variety.
At a more recent workshop~\cite{Barto-shanks}, Barto announced further joint work with Kozik on 
a general ``Rectangularity Theorem'' 
that says, roughly, a subdirect product of simple nonabelian algebras 
contains a full product of minimal absorbing subalgebras 

In Section~\ref{sec:tayl-vari-rect} of the present paper
we state and prove a version of Barto and Kozik's Rectangularity Theorem. 
% that (1) is \emph{relatively} easy to apply to certain \csp tractability problems and 
% (2) has a \emph{relatively} short proof. 
In Section~\ref{sec:csps-comm-idemp}, we apply this tool, together with some
techniqes involving \malcev products and other new decomposition strategies,
to prove that every commutative, idempotent binar of cardinality at most~4 is tractable. 

\section{Definitions and Notations}
\subsection{Notation for projections, scopes, and kernels}
\label{sec:proj-scop-kern}
Here we give a somewhat informal description of some definitions and notations
we use in the sequel.  Some of these will be reintroduced more carefully later on, as needed.

An operation $f : A^n \rightarrow A$ is called \defn{idempotent} provided 
$f(a, a, \dots, a) = a$ for all $a \in A$.
Examples of idempotent operations are the projection functions and these will 
play an important role in later sections, so we start by 
introducing a sufficiently general and flexible notation for them.

We define the natural numbers as usual and denote them as follows:
\[
\uzero := \emptyset, \quad
\uone := \{0\}, \quad
\utwo := \{0, 1\}, %\dots, %% \]%% \[
\quad \dots \quad \nn := \{0, 1, 2, \dots, n-1\}, \quad\dots
\]
Given sets $A_0$, $A_1$, $\dots$, $A_{n-1}$, their Cartesian product is
%% \[\myprod_{\nn} A_i := A_0 \times A_1 \times \cdots \times A_{n-1}.\] 
$\myprod_{\nn} A_i := A_0 \times % A_1 \times 
\cdots \times A_{n-1}$.

An element
$\ba \in \myprod_{\nn} A_i$ is an ordered $n$-tuple, which may be specified by
simply listing its values, 
as in $\ba = (\ba(0), \ba(1), \dots, \ba(n-1))$.
Thus, tuples are functions defined on a (finite) index set, and
this view may be emphasized sybolically as follows:
\[
\ba \colon \nn \to \bigcup_{i\in \nn} A_i; \;\; i\mapsto \ba(i) \in A_i.
\]

If $\sigma\colon \kk \to \nn$ is a $k$-tuple of numbers in 
$\nn$, then we can compose an $n$-tuple $\ba$ in $\myprod_{i\in \nn} A_i$
with $\sigma$ yielding the $k$-tuple $\ba\circ \sigma$ in 
$\myprod_{\kk}A_{\sigma(i)}$.
Generally speaking, we will try to avoid nonstandard notational conventions, 
but here are two exceptions: let
\[
\uA := \prod_{i\in \nn} A_i \quad \text{ and } \quad \uA_{\sigma} := \prod_{i\in \kk} A_{\sigma(i)}.
\]

Now, if the $k$-tuple $\sigma\colon \kk \to \nn$ happens to be one-to-one,
and if we let $p_\sigma$ denote the map $\ba \mapsto \ba\circ \sigma$, then $p_\sigma$
is the usual \emph{projection function} from $\uA$ onto $\uA_{\sigma}$.
Thus, $p_{\sigma}(\ba)$ is a $k$-tuple whose $i$-th component is 
$(\ba\circ \sigma)(i) = \ba(\sigma(i))$.
%$\myprod_{\kk}A_{\sigma(i)}$.
We will make frequent use of such projections, as well as
their images under the covariant and contravariant powerset functors
\ifthenelse{\boolean{extralong}}{
$\sP$ and $\overline{\sP}$. depicted in Figure \ref{fig:powersetfunctor}.
}{$\sP$ and $\overline{\sP}$.}
Indeed, we let
$\Proj_\sigma\colon \sP(\uA) \to \sP(\uA_\sigma)$
denote the \emph{projection set function} defined 
for each $R \subseteq \uA$ by
\[
\Proj_\sigma R = \sP(p_\sigma)(R) = \{p_\sigma(\bx) \mid \bx \in R\} = 
\{\bx\circ \sigma \mid \bx \in R\},
\]
and we let
$\Inj_\sigma\colon \overline{\sP}(\uA_\sigma) \to \overline{\sP}(\uA)$
denote the \emph{injection set function} defined for each
$S\subseteq \uA_\sigma$ by
\begin{equation}\label{eq:19}
  \Inj_\sigma S = \overline{\sP}(p_\sigma)(S) = \{\bx \mid p_\sigma(\bx) \in S\}
   = \{\bx \in \uA \mid \bx\circ \sigma \in S\}.
\end{equation}
Of course, $\Inj_\sigma S$ is nothing more than the inverse image of the set $S$ with respect to the
projection function $p_\sigma$.
We sometimes use the shorthand $R_\sigma = \Proj_\sigma R$ and
$S^{\overleftarrow{\sigma}} = \Inj_\sigma S$ for the projection and injection set functions, respectively.

%%% --- wjd: (2016.10.02) omitting figure from arxiv and journal version
\ifthenelse{\boolean{extralong}}{%
\begin{figure}[!h]
\centering
  \begin{tikzpicture}[node distance=2cm, auto, scale=2]
    \node (023) at (0,2.3)  {$\Set$};
    \node (123) at (1,2.3)  {$\Set$};
    \node (02) at (0,2)  {$\uA$};
    \node (12) at (1,2)  {$\sP (\uA)$};
    \node (01) at (0,1)  {$\uA_{\sigma}$};
    \node (11) at (1,1)  {$\sP(\uA_{\sigma})$};
    %% \node (middle) at (0.5,1.5)  {$\circlearrowleft$};
    \draw[->] (02) -- (01)  node[pos=.5,left] {$p_\sigma$};
    \draw[->] (12) -- (11)  node[pos=.5,right] {$\sP(p_\sigma)$};
    \draw[->] (023) -- (123)  node[pos=.5,above] {$\sP$};
  \end{tikzpicture}
  \hskip1cm
  \begin{tikzpicture}[node distance=2cm, auto, scale=2]
    \node (023) at (0,2.3)  {$\Set^{\op}$};
    \node (123) at (1,2.3)  {$\Set$};
    \node (02) at (0,2)  {$\uA_{\sigma}$};
    \node (11) at (1,1)  {$\overline{\sP} (\uA)$};
    \node (01) at (0,1)  {$\uA$};
    \node (12) at (1,2)  {$\overline{\sP}(\uA_{\sigma})$};
    \draw[->] (02) -- (01)  node[pos=.5,left] {$p_\sigma$};
    \draw[->] (12) -- (11)  node[pos=.5,right] {$\overline{\sP}(p_\sigma)$};
    \draw[->] (023) -- (123)  node[pos=.5,above] {$\overline{\sP}$};
  \end{tikzpicture}
  \caption{Projection under covariant (left) and contravariant (right) powerset functors.}
  \label{fig:powersetfunctor}
\end{figure}
}{}
%%% --- wjd: (7/24/2016) omitting this example from journal version
\ifthenelse{\boolean{arxiv}}{%
\begin{example}
To make clear why the term ``projection'' is reserved for the 
case when $\sigma$ is one-to-one, suppose 
$k=4$, $n=3$, and consider the 4-tuple $\sigma = (1, 0, 1, 1)$. 
Then $\sigma$ is the function 
$\sigma \colon \{0,1,2,3\} \to \{0,1,2\}$ given by 
$\sigma(0) = 1$,
$\sigma(1) = 0$, $\sigma(2) = 1$, $\sigma(3) = 1$, and so 
$a \mapsto a\circ \sigma$ is the function that takes
$(a_0, a_1, a_2)\in A_0 \times A_1 \times A_2$ to 
$(a_1, a_0, a_1, a_1) \in A_1 \times A_0 \times A_1 \times A_1$.
\end{example}
}{}
%%% ---

A one-to-one function $\sigma \colon \kk \to \nn$ is sometimes called a \emph{scope} or
\emph{scope function}.
If $R \subseteq \prod_{j \in \kk}A_{\sigma(j)}$
is a subset of the projection of $\uA$ onto coordinates
$(\sigma(0), \sigma(1), \dots, \sigma(k-1))$,
%  in the image of $\kk$ under $\sigma$
then we call $R$ a \emph{relation on $\uA$ with scope $\sigma$}.
The pair $(\sigma, R)$ is called a \emph{constraint}, and
$R^{\overleftarrow{\sigma}}$ % = \bigl\{\bx \in \uA \mid \bx \circ \sigma \in R\bigr\}$
is the set of tuples in $\uA$ that \emph{satisfy} % the constraint
$(\sigma, R)$.

By $\eta_\sigma$ we denote the \emph{kernel} of the projection function $p_\sigma$,
%% $\br \mapsto \br \circ \sigma$
which is the following equivalence relation on $\uA$:
% \begin{align} \label{eq:60}
%   \eta_\sigma 
%   &= \{(\ba,\ba') \in \uA^2 \mid p_\sigma(\ba) = p_\sigma(\ba') \}\nonumber\\
%   &= \{(\ba,\ba') \in \uA^2 \mid \ba \circ \sigma = \ba' \circ \sigma \}.
%   %% &= \{(\ba,\ba') \in \uA^2 \mid \forall j \in \im \sigma, \,\ba(j)= \ba'(j)  \}.\nonumber
% \end{align}
\begin{equation}
  \label{eq:60}
  \eta_\sigma 
  = \{(\ba,\ba') \in \uA^2 \mid p_\sigma(\ba) = p_\sigma(\ba') \}
  = \{(\ba,\ba') \in \uA^2 \mid \ba \circ \sigma = \ba' \circ \sigma \}.
  %% &= \{(\ba,\ba') \in \uA^2 \mid \forall j \in \im \sigma, \,\ba(j)= \ba'(j)  \}.\nonumber
\end{equation}

  More generally, if $\theta$ is an equivalence relation on the set
  $\myprod_{j\in \kk} A_{\sigma(j)}$, then we  define the equivalence relation $\theta_\sigma$ on
  the set $\uA = \myprod_{\nn} A_i$ as follows:
\begin{equation}
  \label{eq:17}
\theta_\sigma = %% (\Proj_\sigma)^{-1}(\theta)  = 
\{(\ba, \ba') \in \uA^2 \mid (\ba\circ \sigma) \mathrel{\theta} (\ba' \circ \sigma)\}.
\end{equation}
In other words,  $\theta_\sigma$ consists of all pairs in $\uA^2$ that land in $\theta$
when projected onto the scope $\sigma$. Thus,
if $0_{\uA}$ denotes the least equivalence relation on {\uA},
%% that is,  $0_{\uA}:= \{(\ba, \ba') \in \uA^2 \mid \ba = \ba'\}$.  
then $\eta_\sigma$ is shorthand for $(0_{\uA})_\sigma$.
%% --- wjd (2016.08.07) not sure if we ever use \theta_\sigma. Maybe we could remove it?


If the domain of $\sigma$ is a singleton, $\kk = \{0\}$, then of course
$\sigma$ is just a one-element list, say, $\sigma= (j)$. 
In such cases, we write $p_j$ instead of $p_{(j)}$. Similarly,
we write $\Proj_j$ instead of $\Proj_{(j)}$, $\eta_j$ instead of $\eta_{(j)}$, etc.  Thus, 
$p_j(\ba) = \ba(j)$, and $\eta_j = \{(\ba, \ba') \in \uA^2 \mid \ba(j) = \ba'(j)\}$,
and, if $\theta\in \Con \bA_j$, then
$\theta_j = \{(\ba, \ba') \in \uA^2 \mid \ba(j) \mathrel{\theta} \ba'(j)\}$.
Here are some obvious consequences of these definitions:
\[
\bigvee_{j\in \nn}\eta_j =\uA^2, \qquad
 \eta_\sigma= \bigwedge_{j\in \sigma}\eta_j, \qquad %\quad \text{ and } \quad
 \eta_{\nn} = \bigwedge_{j\in \nn}\eta_j = 0_{\uA}, \qquad
\theta_\sigma = \bigwedge_{j\in \kk}\theta_{\sigma(j)}.
\]
%% where $0_{\uA}$ denotes the least equivalence relation on {\uA}.
%% , that is, $0_{\uA}:= \{(\ba, \ba') \in \uA^2 \mid \ba = \ba'\}$.  

%%%%%%%%%%%%%%%%%%%%%%%%%%%%%%%%%%%%%%%%%%%%%%%%%%%%%%%%%%%%%%%%%%%%%%%%%%%%%%%%%%%%%%%
\subsection{Notation for algebraic structures}
This section introduces the notation we use for algebras and related concepts.
The reader should consult~\cite{MR2839398} for more details and background 
on general (universal) algebras and the varieties they inhabit.

\subsubsection{Product algebras}
Fix $n\in \N$, let $F$ be a set of operation symbols, and for each $i\in \nn$
let $\bA_i = \<A_i, F\>$ be an algebra of type $F$.
Let $\ubA = \bA_0 \times \bA_1 \times \cdots \times \bA_{n-1}= \myprod_{\nn} \bA_i$
denote the product algebra.
If $k \leq n$ and if $\sigma \colon \kk\to \nn$ is a one-to-one function, then
we denote by
$p_\sigma : \ubA \onto \myprod_{\kk}\bA_{\sigma(i)}$
%% $\Proj_\sigma : \ubA \onto \myprod_{\kk}\bA_{\sigma(i)}$
the \defn{projection} of $\ubA$ onto the 
``$\sigma$-factors'' % (factors with indices in $\im \sigma$)
of $\ubA$,
% The function $p_\sigma$ %% $\Proj_\sigma$
% so defined 
which is an algebra homomorphism; thus the kernel $\eta_\sigma$ defined
in~(\ref{eq:60}) is a congruence relation of $\ubA$.   


%%%%%%%%%%%%%%%%%%%%%%%%%%%%%%%%%%%%%%%%%%%%%%%%%%%%%%%%%%%%%%%%%%%%%%%%%%%%%%%%%%%%%%%
%% \subsection{Terms}
%% --- wjd 2016.08.11
%% After thinning out this subsection on terms, it now seems more
%% appropriate to fold it into the ``Product algebras'' subsection above.
%% ---
\subsubsection{Term operations}
\label{ssec:term-ops}
For a nonempty set $A$, we let $\sansO_A$ denote the set of all finitary
operations on $A$. That is, $\sansO_A = \bigcup_{n\in \N}\AAn$.
A \defn{clone of operations} on $A$ is a subset of $\sansO_A$ that contains all
projection operations and is closed under the (partial) operation of general
composition. 
If $\bA = \<A, F^\bA\>$ denotes the algebra with universe $A$ and set of basic
operations $F$, then $\sansClo (\bA)$ denotes the clone generated by
$F$, which is also known as the \defn{clone of term operations} of $\bA$.

Walter Taylor proved in~\cite{MR0434928} that a variety, $\var{V}$, satisfies some
nontrivial idempotent \malcev condition if and only if it satisfies one of the following
form: for some~$n$, $\var{V}$ has an idempotent $n$-ary term  $t$ such that
for each $i\in \nn$ there is an identity of the form
\[
t(\ast, \cdots, \ast, x, \ast, \cdots, \ast) \approx t(\ast, \cdots, \ast, y,
\ast, \cdots, \ast)
\]
true in $\var{V}$ where distinct variables $x$ and $y$ appear in the
$i$-th position on either side of the identity.  Such a term $t$ is now commonly
called a \defn{Taylor term}. 


Throughout this paper we assume all algebras are finite 
(though some results may apply more generally).
Starting in Section~\ref{sec:tayl-vari-rect},
we make the additional assumption that the algebras in
question come from a single \defn{Taylor variety} $\var{V}$, by which we mean that $\var{V}$
has a Taylor term and every term of $\var{V}$ is idempotent.

%%%%%%%%%%%%%%%%%%%%%%%%%%%%%%%%%%
%% \subsection{Subdirect products}
\subsubsection{Subdirect products}
If $k, n \in \N$, if $\sA = (A_0, A_1, \dots, A_{n-1})$ is a list
of sets, and if $\sigma \colon \kk \to \nn$ is a $k$-tuple, 
then a relation $R$ over $\sA$ with scope $\sigma$ is
a subset of the Cartesian product
$A_{\sigma(0)} \times A_{\sigma(1)} \times \cdots \times A_{\sigma(k-1)}$.
%%
%% --- wjd 2016.10.02: omitting the next sentence from the journal version
\ifthenelse{\boolean{arxiv}}{%
If $k, m\in \N$ and $R\subseteq \prod_{\kk}A_{\sigma(i)}$, and if
for each $j \in \kk$ we have $\bu_j \in A_{\sigma(j)}^m$,
then we write $(\bu_0, \bu_1, \dots, \bu_{k-1})\in \underline{R}$ 
precisely when $(\bu_0(i), \bu_1(i), \dots, \bu_{k-1}(i))\in R$ 
for all $i\in \mm$.
}{}
%% ---
Let $F$ be a set of operation symbols and for each $i\in \nn$
let $\bA_i = \<A_i, F\>$ be an algebra of type $F$.
If $\ubA = \myprod_{\nn}\bA_i$
is the product of these algebras, then a relation 
$R$ over $\sA$ with scope $\sigma$ is called 
\defn{compatible with $\ubA$} if 
it is closed under the basic operations in $F$.
In other words, $R$ is compatible if the induced algebra $\bR= \<R,F\>$ 
is a subalgebra of $\myprod_{\kk} \bA_{\sigma(j)}$.
If $R$ is compatible with the product algebra
and if the projection of $R$ onto each factor is surjective,
then $\bR$ is
called a \defn{subdirect product} of the algebras
in the list
$(\bA_{\sigma(0)}, \bA_{\sigma(1)}, \dots, \bA_{\sigma(k-1)})$; 
we denote this situation by writing
\ifthenelse{\boolean{arxiv}}{%
$\bR \sdp \myprod_{j\in \kk} \bA_{\sigma(j)}$.\footnote{Note 
  that even in the special case when $\Proj_j\bR = \bA_{\sigma(j)}$ for each
  $j\in \kk$ so that $\bR \sdp \myprod_{j\in \kk} \bA_{\sigma(j)}$, we 
  refrain from using $\Proj_\sigma \ubA$ to denote $\myprod_{j\in \kk} \bA_{\sigma(j)}$
  for the simple reason that $\sigma$ might not be one-to-one.  
  For example, we could have
  $\ubA = \bA_0 \times \bA_1$ and $\sigma = (1,0,1)$, in which case
  $\myprod_{j\in \kk} \bA_{\sigma(j)} = \bA_1 \times \bA_0 \times \bA_1$ and this is
  not the projection of $\ubA$ onto a subset of its factors.}}
           {$\bR \sdp \myprod_{\kk} \bA_{\sigma(j)}$.}

%%%%
%%%% \input{centralizers.tex}
%%%%
\section{Abelian Algebras}
In later sections nonabelian algebras will play the following role:
some of the theorems will begin with the assumption 
that a particular algebra $\bA$ is nonabelian and then proceed to show that 
if the result to be proved were false, then $\bA$ would have to be abelian. 
To prepare the way for such arguments, we review some basic facts about abelian
algebras. 
%% (Much of the notation we adopt here is similar to that used
%% in~\cite{MR3076179}.)


\subsection{Definitions} %: tolerance, centralizing, abelian}
Let $\bA = \<A, F^{\bA}\>$ be an algebra.
A reflexive, symmetric, compatible binary relation $T\subseteq A^2$ is called a
\defn{tolerance of $\bA$}.  
%% As a compatible binary relation, $T$ is a subuniverse of $\bA^2$.
Given a pair $(\bu, \bv) \in A^m\times A^m$ of $m$-tuples of $A$, we write 
$\bu \mathrel{\bT} \bv$ just in case $\bu(i) \mathrel{T} \bv(i)$ for all $i\in \mm$. 
We state a number of definitions in this section using tolerance relations, but 
the definitions don't change when the tolerance in question happens to be
a congruence relation (i.e., a transitive tolerance).

Suppose $S$ and $T$ are tolerances on $\bA$.  An \defn{$S,T$-matrix} 
is a $2\times 2$ array of the form
\[
\begin{bmatrix*}[r] t(\ba,\bu) & t(\ba,\bv)\\ t(\bb,\bu)&t(\bb,\bv)\end{bmatrix*},
\]
where $t$, $\ba$, $\bb$, $\bu$, $\bv$ have the following properties:
\begin{enumerate}[(i)]
\item $t\in \sansClo_{\ell + m}(\bA)$,
\item $(\ba, \bb)\in A^\ell\times A^\ell$ and $\ba \mathrel{\bS} \bb$,
\item $(\bu, \bv)\in A^m\times A^m$ and $\bu \mathrel{\bT} \bv$.
\end{enumerate}
Let $\delta$ be a congruence relation of $\bA$.
If the entries of every $S,T$-matrix satisfy
\begin{equation}
  \label{eq:22}
t(\ba,\bu) \mathrel{\delta} t(\ba,\bv)\quad \iff \quad t(\bb,\bu) \mathrel{\delta} t(\bb,\bv),
\end{equation}
then we say that $S$ \defn{centralizes $T$ modulo} $\delta$ and we write 
$\CC{S}{T}{\delta}$.
That is, $\CC{S}{T}{\delta}$  means that 
(\ref{eq:22}) holds \emph{for all}
$\ell$, $m$, $t$, $\ba$, $\bb$, $\bu$, $\bv$ satisfying properties (i)--(iii).

The \defn{commutator} of $S$ and $T$, denoted by $[S, T]$,
is the least congruence $\delta$ such that $\CC{S}{T}{\delta}$ 
holds.  
Note that $\CC{S}{T}{0_A}$ is equivalent to $[S,T] = 0_A$, and this
is sometimes called the \defn{$S, T$-term condition};
when it holds we say  that
$S$ \defn{centralizes} $T$, and write $\C{S}{T}$.  %% (equivalently, $[S,T] = 0_A$).
A tolerance $T$ is called \defn{abelian} if
$\C{T}{T}$ (i.e., $[T, T] = 0_A$).  
An algebra $\bA$ is called \defn{abelian} if $1_A$ is abelian
(i.e., $\C{1_A}{1_A}$). %, or $[1_A, 1_A] = 0_A$).
%% The \defn{centralizer of $\bT$ modulo $\delta$}, denoted by
%% $(\delta : \bT )$, is the largest congruence $\alpha$ on $\bA$ such that 
%% $\sansC(\alpha, \bT ; \delta)$ holds.

\ifthenelse{\boolean{extralong}}{%
\begin{remark}
An algebra $\bA$ is abelian iff $\C{1_A}{1_A}$ iff
\[
\forall \ell \in \{0,1,2,\dots \}, 
\quad \forall m \in  \{1,2,\dots \},
\quad \forall t\in \sansClo_{\ell + m}(\bA),
\quad \forall (a, b)\in A^\ell\times A^\ell,
\]
\[
\ker t(a, \cdot)=\ker t(b, \cdot).
\]
\end{remark}
}{}

\ifthenelse{\boolean{arxiv}}{%
\begin{remark}
An algebra $\bA$ is abelian iff $\C{1_A}{1_A}$ iff
\[
\forall \ell, m \in \N,
%% \quad \forall m \in  \{1,2,\dots \},\]
\quad \forall t\in \sansClo_{\ell + m}(\bA),
\quad \forall (\ba, \bb)\in A^\ell\times A^\ell,
\]
\[
\ker t(\ba, \cdot)=\ker t(\bb, \cdot).
\]
\end{remark}
}{}

\subsection{Facts about centralizers and abelian congruences}
We now collect some useful facts about centralizers of congruence relations
that are needed in Section~\ref{sec:tayl-vari-rect}.
The facts collected in the first lemma are well-known and easy to prove.
(For examples, see \cite[Prop~3.4]{HM:1988} and~\cite[Thm~2.19]{MR3076179}.)
\begin{lemma}
\label{lem:centralizers}
Let $\bA$ be an algebra and suppose
$\bB$ is a subalgebra of $\bA$. 
Let $\alpha$, $\beta$, $\gamma$, $\delta$, $\alpha_i$
$\beta_j$, $\gamma_k$
be congruences of $\bA$, for some 
$i \in I$, $j\in J$, $k \in K$. Then the following hold:
\begin{enumerate}
\item \label{centralizing_over_meet}
  $\CC{\alpha}{\beta}{\alpha \meet \beta}$;
\item \label{centralizing_over_meet2}
  if $\CC{\alpha}{ \beta}{ \gamma_k}$ for all $k \in K$, then
  $\CC{\alpha}{ \beta}{ \Meet_{K}\gamma_k}$;
\item \label{centralizing_over_join1}
  if $\CC{\alpha_i}{ \beta}{ \gamma}$ for all $i\in I$, then
  $\CC{\Join_{I}\alpha_i}{ \beta}{\gamma}$;
\item \label{monotone_centralizers1}
  if $\CC{\alpha}{ \beta}{ \gamma}$ and $\alpha' \leq \alpha$, then 
  $\CC{\alpha'}{ \beta}{ \gamma}$;
\item \label{monotone_centralizers2}
  if $\CC{\alpha}{ \beta}{ \gamma}$ and $\beta' \leq \beta$, then
  $\CC{\alpha}{ \beta'}{ \gamma}$;
\item \label{centralizing_over_subalg}
  if $\CC{\alpha}{ \beta}{ \gamma}$ in $\bA$, 
  then $\CC{\alpha\cap B^2}{ \beta\cap B^2}{\gamma\cap B^2}$ in $\bB$;
\item \label{centralizing_factors}
  if $\gamma \leq \delta$, then $\CC{\alpha}{ \beta}{ \delta}$
  in $\bA$ if and only if $\CC{\alpha/\gamma}{ \beta/\gamma}{ \delta/\gamma}$
  in $\bA/\gamma$.
\end{enumerate}
\end{lemma}

%%% It seems we never use this remark, so I'm omitting it from all but the extralong version.
\ifthenelse{\boolean{extralong}}{
\begin{remark}
By (\ref{centralizing_over_meet}), 
if $\alpha \meet \beta = 0_{A}$,  
then $\C{\beta}{\alpha}$ and $\C{\alpha}{\beta}$.
\end{remark}
}{}

The next two lemmas are essential in a number proofs below.
The first lemma identifies special conditions
under which certain quotient congruences are abelian.
The second gives fairly general conditions under which
quotients of abelian congruences are abelian.
\begin{lemma}
  % \label{lem:M3-abelian}
  \label{lem:common-meets}
  Let $\alpha_0$, $\alpha_1$, $\beta$ be congruences of $\bA$ and suppose 
  $\alpha_0 \meet \beta = \delta = \alpha_1 \meet \beta$.
  Then $\CC{\alpha_0 \join \alpha_1}{ \beta}{ \delta}$.  If, in addition, 
  $\beta \leq \alpha_0 \join \alpha_1$, then 
  $\CC{\beta}{ \beta}{ \delta}$, so $\beta/\delta$ is an 
  abelian congruence of $\bA/\delta$.
\end{lemma}
Lemma~\ref{lem:common-meets}
is an easy consequence
of items~(\ref{centralizing_over_meet}),~(\ref{centralizing_over_join1}),   
and~(\ref{centralizing_factors}) of Lemma~\ref{lem:centralizers}.
\begin{lemma}
  \label{lem:abelian-quotients}
  Let $\var{V}$ be a locally finite variety with a Taylor term and let $\bA\in \var{V}$.
  %% If $\CC{\beta}{\beta}{0_A}$, 
  Then $\CC{\beta}{\beta}{\gamma}$ for all $[\beta, \beta] \leq \gamma$.
\end{lemma}
Lemma~\ref{lem:abelian-quotients} can be proved  by combining  
the next result, of David Hobby and Ralph McKenzie,
    with a result of
    %% ~(Lemma~\ref{lem:HM-thm-7-12}),
    Keith Kearnes and Emil Kiss.
%% ~(Lemma~\ref{lem:KK-lem-6-8}):
\begin{lemma}[cf.~\protect{\cite[Thm~7.12]{HM:1988}}]
  \label{lem:HM-thm-7-12}
  A locally finite variety $\var{V}$ has a Taylor term if and only if it has a
  so called \defn{weak difference term}; that is, a term $d(x,y,z)$ satisfying
  the following conditions for all $\bA \in \var{V}$, all $a, b \in A$, and all
  $\beta \in \Con (\bA)$: 
  $d^{\bA}(a,a,b) \mathrel{[\beta, \beta]} b \mathrel{[\beta, \beta]} d^{\bA}(b,a,a)$,
  where $\beta = \Cg^{\bA}(a,b)$.
\end{lemma}

\begin{lemma}[\protect{\cite[Lem~6.8]{MR3076179}}]
  \label{lem:KK-lem-6-8}
  If $\bA$ belongs to a variety with a
  weak difference term and if $\beta$ and $\gamma$ are congruences of $\bA$
  satisfying $[\beta, \beta] \leq \gamma$, then $\CC{\beta}{\beta}{\gamma}$.
\end{lemma}
\begin{remark}
  \label{rem:abelian-quotients}
  It follows immediately from Lemma~\ref{lem:abelian-quotients} that in a locally
  finite Taylor variety, $\var{V}$, quotients of abelian algebras are abelian, so the
  abelian memebers of $\var{V}$ form a subvariety.
  But this already follows from Lemma~\ref{lem:HM-thm-7-12},
  since $[\beta, \beta] = 0_A$ implies $d^{\bA}$ is a \malcev term operation on
  the blocks of $\beta$, so if $\bA$ is abelian---i.e., if
  $\CC{1_A}{1_A}{0_A}$---then Lemma~\ref{lem:HM-thm-7-12},
  implies that $\bA$ has a \malcev term operation.
  (This is recorded as~Theorem~\ref{thm:type2cp} below.)
  Therefore (\cite[Cor~7.28]{MR2839398}), homomorphic images of $\bA$ are
  abelian. 
\end{remark}


%%%% wjd: adding pagebreak for ``draft mode'' to reduce printing costs
%%%%      To turn off these unnecessary page breaks, set `draft` to false
%%%%      near the top of this file.
\ifthenelse{\boolean{draft}}{\newpage}{}

%%%%%%%%%%%%%%%%%%%%%%%%%%%%%%%%%%%%%%%%%%%%%%%%%%%%%%%%%%%%%%%%%%%%%%%%%%%%%%%%%%%%%%%
\section{Absorption Theory}
\label{sec:absorption}
In this section we survey some of the theory related to 
an important concept called \emph{absorption}, which was invented
by Libor Barto and Marcin Kozik. After introducing the concept, we discuss some of the properties that make it
so useful. 
The main results in this section are not new. The only possibly novel contributions are some
straightforward observations in Section~\ref{sec:linking-lemmas} (which have likely been
observed by others). Our intention here is merely to collect and present these
known results in a way that makes them easy to apply in later sections and future work.

Let $\bA = \<A, F^{\bA}\>$ be a finite algebra in the Taylor variety $\var{V}$ 
and let $t\in \sansClo(\bA)$ be a $k$-ary term operation.
A subalgebra  $\bB = \<B, F^{\bB}\> \leq \bA$ is said to be 
\defn{absorbing in $\bA$ with respect to $t$} if
for all $1\leq j\leq k$ and for all 
$(b_1, \dots, b_{j-1}, a, b_{j+1}, \dots, b_k)\in B^{j-1}\times A \times B^{k-j}$
we have
\[
t^{\bA}(b_1, \dots, b_{j-1}, a, b_{j+1}, \dots, b_k)\in B.
\]
In other terms,
$t^{\bA}[B^{j-1}\times A \times B^{k-j}] \subseteq B$,
for all $1\leq j \leq k$.
%% (Cliff, I didn't edit this to add nonempty because I'm simply 
%% rephrasing the definition a little less formally.)
Here, $t^\bA[D]$ denotes the set $\{ t^\bA(x) \mid x\in D \}$.

The notation $\bB \absorbing \bA$ means that
$\bB$ is an absorbing subalgebra of $\bA$ with respect to some term.
If we wish to be explicit about the term, we write $\bB \absorbing_t \bA$.
We may also call $t$ an ``absorbing term'' for $\bB$ in this case.
The notation $\bB \minabsorbing \bA$ 
means that $\bB$ is a minimal absorbing subalgebra of $\bA$, that is, 
$B$ is minimal (with respect to set inclusion) among the absorbing subuniverses
of $\bA$. 
An algebra is said to be \defn{absorption-free} if it has no proper absorbing
subalgebras.

\subsection{Absorption theorem}
\label{sec:absorption-thm}
In later sections, we will make frequent use of a 
powerful tool of Libor Barto and Marcin Kozik called the 
``Absorption Theorem'' (Thm~2.3 of~\cite{MR2893395}).
This result concerns the special class of ``linked'' subdirect products.  
A subdirect product $\bR \sdp \bA_0 \times \bA_1$
is said to be \defn{linked} if it satisfies the following:
for $a, a' \in \Proj_0 R$ there 
exist elements $c_0, c_2, \dots, c_{2n} \in A_0$ and
$c_1, c_3, \dots, c_{2n+1} \in A_1$ such that $c_0 = a$, 
$c_{2n} = a'$, and for all $0\leq i<n$, 
$(c_{2i},c_{2i+1})\in R$ and $(c_{2i+2},c_{2i+1})\in R$.
Here is an easily proved fact that provides some equivalent ways to define ``linked.''
\begin{Fact}
Let $\bR \sdp \bA_0 \times \bA_1$, let $\etaR_i = \ker(\bR \onto \bA_i)$
denote the kernel of the projection of $\bR$ onto its $i$-th coordinate, and let 
$R^{-1} = \{(y,x) \in A_1 \times A_0 \mid (x,y) \in R\}$. 
Then the following are equivalent:
\begin{enumerate}
\item $\bR$ is linked;
\item $\etaR_0\join \etaR_1 = 1_R$;
\item if $a, a' \in \Proj_0 R$, then $(a,a')$ is in the transitive closure of $R\circ R^{-1}$.
\end{enumerate}
\end{Fact}
%% \todo{write out the proofs that these are equivalent and maybe insert them here
%%   (unless they are too simple and tedious).}
%%%% ALTERNATIVE (SHORTER) VERSION:
%% It is easy to see that if
%% $\bR \sdp \bA_0 \times \bA_1$ and $\etaR_i := \ker(\bR \onto \bA_i)$, then 
%% $\bR$ is linked if and only if
%% $\etaR_0\join \etaR_1 = 1_R$.

\begin{theorem}[Absorption Theorem \protect{\cite[Thm~2.3]{MR2893395}}]
\label{thm:absorption}
If $\var{V}$ is an idempotent locally finite variety, then the following are equivalent:
\begin{itemize}
\item $\var{V}$ is a Taylor variety;
\item if $\bA_0, \bA_1 \in \var{V}$ are finite absorption-free algebras, 
  and if $\bR \sdp \bA_0 \times \bA_1$ is linked, then $\bR = \bA_0 \times \bA_1$.
\end{itemize}
\end{theorem}

Before moving on to the next subsection, we need one more definition.  
If $f\colon A^\ell\to A$ and 
$g\colon A^m\to A$ are 
$\ell$-ary and $m$-ary operations on $A$,
then by $f \star g$ we mean the $\ell m$-ary operation 
%% $f \star g: A^{\ell m}\to A$ 
that maps each 
\ifthenelse{\boolean{arxiv}}{%
  \[
  \ba = (a_{1 1},\dots, a_{1 m}, a_{2 1}, \dots,  a_{2 m},\dots, a_{\ell m}) \in A^{\ell m}
  \]
to $(f\star g) \,\ba
= 
f(g(a_{1 1}, \dots, a_{1 m}), g(a_{2 1}, \dots, a_{2 m}), \dots,  g(a_{\ell 1}, \dots, a_{\ell m}))$.}{
$(a_{1 1},\dots, a_{1 m}, a_{2 1}, \dots,  a_{2 m},\dots, a_{\ell m}) \in A^{\ell m}$ to
$f(g(a_{1 1}, \dots, a_{1 m}), g(a_{2 1}, \dots, a_{2 m}), \dots,  g(a_{\ell 1}, \dots, a_{\ell m}))$.
}


%\medskip

%%%%%%%%%%%%%%%%%%%%%%%%%%%%%%%%%%%%%%%%%%%%%%% Some Observations %%%%%%%%%%%%%%%%%%%%%%%%%%%%


\subsection{Properties of absorption}
%% \subsection{General Facts}
In this section we prove some easy facts about algebras that hold
quite generally.  In particular, we don't assume the presence of Taylor terms or
other \malcev conditions in this section.  However, for the sake of
simplicity, we do assume that all algebras are finite and belong to a single
idempotent variety, though some of the results of this section hold for
infinite algebras as well.  
Almost all of the results in this section are well known.
The only exceptions are mild generalizations of facts that have already appeared in print.
We have restated or reworked some of the known results in a way that makes them easier
for us to apply. 

\newcommand\sseq{\ensuremath{\subseteq}}
\begin{lemma}
\label{lem:fact1}
Let $\bA$ be an idempotent algebra with terms $f$ and $g$. If
$\bB\absorbing \bA$ with respect to either $f$ or $g$, then
$\bB\absorbing \bA$ with respect to $f\star g$. 
\end{lemma}
\begin{proof}
In this proof, to keep the notation simple,
let $ABB\cdots B$ denote the usual Cartesian product 
$A\times B \times B \times \cdots \times B$, and
let $t=f\star g$.
We only handle the $j=1$ 
case of the definition in Section~\ref{sec:absorption}. 
(The general case is equally simple, but the notation becomes tedious.)
That is, we prove the following subset inclusion:
\[
%% t[A\times B \times B \times \cdots \times B]
%% = f\bigl[g[A\times B \times \cdots \times B] \times 
%% g[B\times \cdots \times B] \times \cdots \times
%% g[B\times \cdots \times B]\bigr]
 %% (A,B,\dots,B), \, g(B,B,\dots,B),\dots, g(B,B,\dots,B) \bigr).
t[ABB\cdots B]
= f\bigl[g[ABB\cdots B] \times 
g[BB\cdots B] \times \cdots
\]
\[\phantom{XXXXXXXXXXXXXX}\cdots \times
g[BB \cdots B]\bigr]\sseq B. 
\]
Suppose first that $f$ is the absorbing term. Since $B$ is a subalgebra,
$g[BB\cdots B]$ is contained in $B$. (This holds for any term.) Hence, 
$t[ABB\cdots B]$ is contained in $f[ABB\cdots B] \sseq B$. 
On the other hand, if $g$ is the absorbing term, then
$t[ABB \cdots B] \sseq f[BB \cdots B] \sseq B$, again because $B$ is a
subalgebra. 
\end{proof}
\begin{corollary}
  \label{cor:fact1gen}
  Fix $p <\omega$ and suppose $\{t_0, t_1, \dots, t_{p-1}\}$ is a set of terms in the variety $\var{V}$.
  If $\bA$ and $\bB$ are finite algebras in $\var{V}$ and if $\bB\absorbing_{t} \bA$ for some
  $t$ in $\{t_0, t_1, \dots, t_{p-1}\}$, then $\bB \absorbing_s \bA$, where 
  $s = t_0 \star t_1 \star \cdots \star t_{p-1}$.
\end{corollary}
\begin{proof}
  By induction on~$p$.
  \end{proof}

\begin{lemma}[\protect{\cite[Prop~2.4]{MR2893395}}] 
\label{lem:bk-prop-2-4}
  Let $\bA$ be an algebra.
  \begin{itemize}
  \item If $\bC \absorbing \bB \absorbing \bA$, then $\bC \absorbing \bA$.
  \item If $\bB \absorbing_f \bA$ and $\bC \absorbing_g \bA$ and $B \cap C\neq \emptyset$, then 
    %% $\bB \cap \bC \absorbing_t \bA$, where $t = f\star g$.
    $B \cap C$ is an absorbing subuniverse of $\bA$ with respect to $t = f\star g$.
  \end{itemize}
\end{lemma}

\begin{corollary}
\label{cor:fact2}
For $i=0, 1$, let $\bA_i$ and $\bB_i$ be algebras in the variety $\var{V}$ and suppose
$\bB_0 \absorbing_f \bA_0$ and $\bB_1 \absorbing_g \bA_1$.
Then $\bB_0\times \bB_1$ is absorbing in $\bA_0\times \bA_1$ with respect to $f\star g$.
\end{corollary}
\begin{proof}
Since $\bB_0 \absorbing_f \bA_0$, it follows that
$\bB_0\times \bA_1 \absorbing \bA_0\times \bA_1$  with respect to the same term $f$.
Similarly $\bA_0 \times \bB_1 \absorbing_g \bA_0\times \bA_1$.
Hence, $\bB_0\times \bB_1=(\bB_0\times \bA_1) \cap (\bA_0\times \bB_1)$ is 
absorbing in $\bA_0\times \bA_1$ with respect to 
$f\star g$, by Lemma~\ref{lem:bk-prop-2-4}.
\end{proof}
Corollary \ref{cor:fact2} generalizes to finite products and \emph{minimal} 
absorbing subalgebras. We record these observations next. 
(A proof appears in~\ref{sec:proof-cor-min-abs-prod}.)
\begin{lemma}
\label{lem:min-abs-prod}
Let $\bB_i \leq \bA_i$ $(0\leq i < n)$ be algebras in the variety $\var{V}$, 
let $\bB := \bB_0\times \bB_1\times \cdots \times \bB_{n-1}$, and let
$\bA := \bA_0\times \bA_1\times \cdots \times \bA_{n-1}$.
%% \begin{enumerate}
%% \item  If $\bB_i\absorbing_{t_i}\bA_i$ for each $0\leq i < n$,
%% then $\bB \absorbing \bA$ with respect to $t_0\star t_1 \star \cdots \star t_{n-1}$.
%% \item  If $\bB_i\minabsorbing_{t_i}\bA_i$ for each $0\leq i < n$,
%% then $\bB \minabsorbing \bA$ with respect to $t_0\star t_1 \star \cdots \star t_{n-1}$.
%% \end{enumerate}
If $\bB_i\absorbing_{t_i}\bA_i$ (resp., $\bB_i\minabsorbing_{t_i}\bA_i$) for each $0\leq i < n$,
then $\bB \absorbing_s \bA$ (resp., $\bB \minabsorbing_s \bA$) 
where $s:= t_0\star t_1 \star \cdots \star t_{n-1}$.
\end{lemma}
An obvious but important consequence of Lemma~\ref{lem:min-abs-prod} is that a
(finite) product of finite idempotent algebras is absorption-free if
each of its factors is absorption-free.

We conclude this section with a few more properties of absorption that will 
be useful below.
The first is a simple observation with a very easy proof.
\begin{lemma}
\label{lem:restriction}
Let $\bB$ and $\bC$ be subalgebras of a finite idempotent algebra $\bA$.
Suppose $\bB \absorbing_t \bA$, and suppose $D = B\cap C \neq \emptyset$.
Then the restriction of $t$ to $C$ is an absorbing term for $D$ in $C$, whence 
$D\absorbing C$.
\end{lemma}
Despite its simplicity, Lemma~\ref{lem:restriction} is quite useful for
proving that certain algebras are absorption-free.
%% (See Section~\ref{sec:example-absorpt-free} for examples.)
%% >>>> wjd: can't find thie reference sec:example-absorpt-free.  It must have been
%% >>>>      deleted during chb's revisions.  If we find it, we could uncomment
%% >>>>      the reference in parentheses in the line above.
Along with Lemma~\ref{lem:min-abs-prod}, Lemma~\ref{lem:restriction} 
yields the following result (to be applied in Section~\ref{sec:applications}):
\begin{lemma}
\label{lem:gen-abs1}
Let $\bA_0, \bA_1, \dots, \bA_{n-1}$ be finite algebras with $\bB_i\absorbing \bA_i$ for each $i$.
Suppose $\bR \leq \myprod_{i}\bA_i$ and $R \cap \myprod_i B_i \neq \emptyset$.
Then $R \cap \myprod_i B_i$ is an absorbing subuniverse of $\bR$.
\end{lemma}
\begin{proof}
  Let $\bA = \myprod_i \bA_i$ and $\bB = \myprod_i \bB_i$.
  By Lemma~\ref{lem:min-abs-prod}, $\bB \absorbing_t \bA$, so the result
  follows Lemma~\ref{lem:restriction} if we put $C = R$.
\end{proof}
%%% OLD PROOF (omitted once I realized this is an easy consequence of lem:restriction)
%% \begin{proof}
%%   Let $\bA = \myprod_i \bA_i$ and $\bB = \myprod_i \bB_i$, and let $\bR'$ denote
%%   the algebra with universe $R' = R \cap \myprod_i B_i$. By Corollary~\ref{cor:min-abs-prod},
%%   $\bB \minabsorbing_t \bA$, for some $t\in \sansClo(\bA)$ (say, $k$-ary).
%%   Fix $j< k$,  $\br\in R$, and $\br'_i\in R'$ ($0\leq i<k$).  Consider
%%     $\bw=t^{\bA}(\br'_0, \dots, \br'_{j-1}, \br, \br'_{j+1}, \dots, \br'_{k-1})$.
%%   Each of the arguments to $t^{\bA}$ belongs to $R$ and $\bR \leq \bA$,
%%   thus $\bw \in R$. Each $\br'_i$ belongs to $B$ and $\bB \absorbing \bA$,
%%   thus $\bw \in \myprod_i B_i$. Therefore, $\bw\in R \cap \myprod_i B_i = R'$.
%%   This proves $\bR'\absorbing \bR$, as desired.
%%   %% It's possible that $\bR'$ is a \emph{minimal} absorbing subalgebra of $\bR$
%%   %% in this case, but a proof is not obvious to me.
%% \end{proof}

For $0<j\leq k$, we say that a $k$-ary term operation $t$ of an algebra $\bA$
\emph{depends on its $j$-th argument} provided there exist $a_0, a_1, \dots, a_{k-1}$
such that $p(x) := t(a_0, \dots, a_{j-2}, x, a_{j}, \dots, a_{k-1})$ is a
nonconstant polynomial of $\bA$.
Let \bA\ be an algebra, and $s \in C \subseteq A$. We call $s$ a
\emph{sink for} $C$ provided that for any term 
$t \in \sansClo_k(\bA)$ and for any $0< j \leq k$, if 
$t$ depends on its $j$-th argument, then 
$t(c_0, c_1, \dots, c_{j-2}, s, c_{j}, \dots, c_{k-1})=s$
for all $c_i \in C$.  If $C$ is a subuniverse, then $\{s\}$ is absorbing, in a very strong way. In fact,
it is easy to see that if an absorbing subuniverse $B$ intersects
nontrivially with a set containing a sink, then $B$ must also contain the sink.
We record this as
\begin{lemma}
\label{lem:sink}
  If $B \absorbing A$, if $s$ is a sink for $C\subseteq A$, and 
  if $B\cap C \neq \emptyset$, then $s\in B$.
\end{lemma}

Proof of the next lemma %Lemma~\ref{lem:sdp-general}
appears in~\ref{sec:proof-lemma-sdp-general} below.
\begin{lemma}
\label{lem:sdp-general}
  Let $\bA_0, \bA_1, \dots, \bA_{n-1}$ be finite idempotent algebras of the same type, and
  suppose $\bB_i \minabsorbing \bA_i$ for $0\leq i< n$.
  Let $\bR \sdp \bA_0 \times \bA_1 \times \cdots \times \bA_{n-1}$, 
  and let $R' = R \cap (B_0 \times B_1 \times \cdots \times B_{n-1})$.  
  If $R'\neq \emptyset$, then $\bR' \sdp \bB_0 \times \bB_1 \times \cdots \times \bB_{n-1}$.
\end{lemma}

For a proof of the next result, %% Lemma~\ref{lem:linked-absorber}, 
see~\cite[Prop~2.15]{MR2893395}.

\begin{lemma}[\protect{\cite[cf.~Prop~2.15]{MR2893395}}]
\label{lem:linked-absorber}
Let $\bA_0$ and $\bA_1$ be finite idempotent algebras of the same type, let
$\bR \sdp \bA_0 \times \bA_1$ and assume $\bR$ is linked.
If $\bS \absorbing \bR$, then $\bS$ is also linked.
\end{lemma}

Finally, we come to a very useful fact about absorption that we exploit heavily
in the sections that follow. 
(This is proved  %% Lemma~\ref{lem:abelian-AF} is proved 
in~\cite[Lem~4.1]{MR3374664}, as well as~\ref{sec:proof-that-abelian}
  below.)
\begin{lemma}[\protect{\cite[Lem~4.1]{MR3374664}}]
\label{lem:abelian-AF}
  Finite idempotent abelian algebras are absorption-free. 
\end{lemma}


 \subsection{Linking is easy, sometimes}
  \label{sec:linking-lemmas}
  We will apply the Absorption Theorem frequently below, so we pause here to
  consider one of the hypotheses of the theorem that might seem less familiar to
  some of our readers.  Specifically, one might wonder when
   we can expect a subdirect product to be \emph{linked}, as is required 
  of $\bR \sdp \bA_0\times \bA_1$ in the statement of the Absorption Theorem.
  Here we consider a few special cases in which this hypothesis comes
  essentially for free.  Most of the proofs in this section, as well as some
  in later sections, depend on the following elementary observation about subdirect
  products:
  \begin{lemma}
    \label{lem:basic}
    Let $\bA_0$ and $\bA_1$ be algebras.  Suppose
    $\bR \sdp \bA_0 \times \bA_1$ and let $\etaR_i = \ker(\bR \onto \bA_i)$
    denote the kernel of the $i$-th projection of $\bR$ onto $\bA_i$. 
    \begin{enumerate}
    \item 
      If $\bA_0$ is simple, then either $\etaR_0 \join \etaR_1 = 1_R$ or $\etaR_0 \geq \etaR_1$.
    \item If $\bA_0$ and $\bA_1$ are both simple, then either $\etaR_0 \join \etaR_1 = 1_R$
      or $\etaR_0 = 0_R = \etaR_1$.
    \end{enumerate}
  \end{lemma}
  \begin{proof}
    The congruence relation $\etaR_0 \in \Con \bR$ is maximal, since 
    $\bR/\etaR_0 \cong \bA_0$ and $\bA_0$ is simple. 
    Therefore, if $\etaR_0 \join \etaR_1 < 1_R$, then $\etaR_0 =\etaR_0 \join \etaR_1$, so 
    $\etaR_1 \leq \etaR_0$, proving (1). If $\bA_1$ is also simple, then the
    same argument, with the roles of $\etaR_0$ and $\etaR_1$ reversed, shows that
    $\etaR_0 \leq \etaR_1$, so   
    $\etaR_0 = \etaR_1$.  Since $\etaR_0 \meet  \etaR_1 = 0_R$, we see that both 
    projection kernels must be $0_R$ in this case.
  \end{proof}
  An immediate corollary is that, in case both factors in the product
  $\bA_0 \times \bA_1$ are simple,
  the linking required in the Absorption Theorem holds under weaker hypotheses. 
  \begin{corollary}
    \label{cor:rect-two_factors-pre}
    Let $\bA_0$ and $\bA_1$ be simple algebras and suppose $\bR \sdp \bA_0 \times \bA_1$.
    If $\etaR_0 \neq \etaR_1$ or $\bA_0 \ncong \bA_1$, then $\bR$ is linked.
  \end{corollary}

  As another corollary of~\ref{lem:basic}
  holds in a locally finite Taylor variety.
  When one factor of $\bA_0 \times \bA_1$ is
  abelian and the other is simple and nonabelian, then we get linking for
  free. That is, in a locally finite Taylor variety,
  \emph{every subdirect product of $\bA_0 \times \bA_1$ is linked}, as we now prove. 
  \begin{corollary}
    \label{cor:S-NA-AF_A-pre}
    Suppose $\bA_0$ and $\bA_1$ are algebras in a locally finite Taylor variety.
    If $\bA_0$ is abelian and $\bA_1$ is simple and nonabelian, then
    every subdirect product of $\bA_0 \times \bA_1$ is linked.     Moreover, if 
    $\bR \sdp \bA_0 \times \bA_1$ and if $\bB_1 \minabsorbing \bA_1$,
    then  $\bR$ intersects $\bA_0\times \bB_1$ nontrivially and this intersection
    forms a linked subdirect product of $\bA_0 \times \bB_1$.
  \end{corollary}
  \begin{proof}
    Suppose  $\bR \sdp \bA_0 \times \bA_1$ is not linked. Since $\bA_1$ is simple
    we have $\etaR_0 \leq \etaR_1$ by Lemma~\ref{lem:basic}.
    Therefore, $\etaR_0 = \etaR_0 \meet \etaR_1 = 0_R$.  But
    then $\bR \cong \bR/\etaR_0 \cong \bA_0$ is abelian, while
    $\bR/\etaR_1 \cong \bA_1$ is nonabelian. This is a contradiction
    since, by Remark~\ref{rem:abelian-quotients},
    $\bR/\theta$ is abelian for all
    $\theta \in \Con \bR$.  Therefore, $\bR$ is linked.
    For the second part, since $\bR$ is a subdirect product, it follows that
    $R \cap (A_0\times B_1)$ is nonempty and, by
    Lemma~\ref{lem:gen-abs1},  $\bR \cap (\bB_0\times \bA_1)$ is
    absorbing in $\bR$. Therefore, by Lemma~\ref{lem:linked-absorber},
    the intersection must also be linked. 
  \end{proof}
  We can extend the previous result to multiple abelian factors by collecting
  them into a single factor.
  We use the notation $\nn := \{0,1,\dots, n-1\}$ and $k':=\nn-\{k\}$.
  For example,
  \[0' :=  \{1,2 ,\dots, n-1\} \quad \text{ and }\quad \bR_{0'} := \Proj_{0'} \bR.\]  
  \begin{corollary}
    \label{cor:Link-1}
    Let $\bA_0$, $\bA_1$, $\dots$, $\bA_{n-1}$ be algebras in a locally finite Taylor variety.
    Suppose $\bA_0$ is simple nonabelian and $\bA_1, \bA_2, \dots, \bA_{n-1}$ are abelian.
    If $\bR \sdp \myprod_{\nn} \bA_i$, then $\bR \sdp \bA_0 \times \bR_{0'}$ is linked.
  \end{corollary}
  \begin{proof}
    Suppose $\bR \sdp \bA_0 \times \bR_{0'}$ is not linked: $\etaR_0 \join \etaR_{0'} < 1_R$.
    Since $\bA_0$ is simple, $\etaR_0$ is a coatom, so
    $\etaR_0 \geq \etaR_{0'}$. Therefore, $ \etaR_{0'} = \etaR_0 \meet \etaR_{0'} = 0_R$,
    so $\bR \cong \bR/\etaR_{0'} \cong \bR_{0'} \leq \myprod_{0'}\bA_i$.
    This proves that $\bR$ is abelian, and yet %(that is, $\sansC(1_R, 1_R)$),
    the quotient $\bR/\etaR_0 \cong \bA_0$ is nonabelian,
    which contradicts Remark~\ref{rem:abelian-quotients}.
  \end{proof}

  Suppose we add to the respective contexts of the last three results the hypothesis
  %Corollary~\ref{lem:S-NA-AF_A} (respectively, Lem.~\ref{lem:Link-1})
  that the algebras live in an idempotent variety with a Taylor
  \ifthenelse{\boolean{footnotes}}{%
    term.\footnote{Equivalently, assume the variety is idempotent and has a
      nontrivial \malcev condition, equivalently, is idempotent and omits
      tame-congruence type 1.}
  }{term.} 
  As mentioned above, we refer to such varieties as ``Taylor varieties'' and we call the
  algebras they contain ``Taylor algebras.'' In this context, we can apply the
  Absorption Theorem (in combination with other results from above) to
  deduce the following:

  \begin{lemma}
    \label{lem:rect-two_factors}
    Let $\bA_0$ and $\bA_1$ be finite  algebras in a Taylor variety
    with minimal absorbing subalgebras $\bB_i\minabsorbing \bA_i$ ($i =0,1$),
    and suppose $\bR \sdp \bA_0 \times \bA_1$, where $\etaR_0 \neq \etaR_1$.
    \begin{enumerate}[(i)]
    \item If $\bA_0$ and $\bA_1$ are simple and $R\cap (B_0 \times B_1) \neq \emptyset$, then
      $\bB_0 \times \bB_1\leq \bR$. 
    \item  If $\bA_0$ is simple nonabelian and $\bA_1$ is abelian,
      then $\bB_0 \times \bA_1 \leq \bR$.
    \end{enumerate}
  \end{lemma}
    \begin{proof}
      \begin{enumerate}[(i)]
      \item First note that, by Corollary~\ref{cor:rect-two_factors-pre}, $\bR$ is linked.
        Let $\bR':= \bR \cap (\bB_0 \times \bB_1)$. Then by
        Lemma~\ref{lem:sdp-general} $\bR' \sdp \bB_0 \times \bB_1$,
        and by Lemma~\ref{lem:gen-abs1} $\bR'\absorbing \bR$, so $\bR'$ is also
        linked, by Lemma~\ref{lem:linked-absorber}.  The hypotheses of the
        Absorption Theorem---with $\bR'$ in place of $\bR$ and $\bB_i$
        in place of $\bA_i$---are now satisfied.  Therefore, 
        $\bR' = \bB_0 \times \bB_1$.
      \item This follows directly from Corollary~\ref{cor:S-NA-AF_A-pre} and the
        Absorption Theorem.
      \end{enumerate}
    \end{proof}

  %%   \begin{corollary}
  %%   Let $\bA_0$ and $\bA_1$ be finite Taylor algebras
  %%   with minimal absorbing subalgebras $\bB_i\minabsorbing \bA_i$ ($i =0,1$),
  %%   and suppose $\bR \sdp \bA_0 \times \bA_1$.
  %%   If $\bA_0$ and $\bA_1$ are both simple,
  %%   if $\etaR_0 \neq \etaR_1$, and if 
  %%   $\bR$ intersects nontrivially with $\bB_0 \times \bB_1$, then
  %%   $\bR$ contains the full product $\bB_0 \times \bB_1$. 
  %%   That is, $\bB_0 \times \bB_1\leq \bR$. 
  %%   \end{corollary}

  %% \begin{corollary}
  %%   \label{cor:S-NA-AF_A}
  %%   Let $\bA_0$ and $\bA_1$ be finite Taylor algebras with minimal absorbing
  %%   subalgebras $\bB_i\minabsorbing \bA_i$ ($i =0,1$), and  
  %%   suppose $\bR \sdp \bA_0 \times \bA_1$.
  %%   If $\bA_0$ is nonabelian and simple, if $\bA_1$ is abelian,
  %%   and if $\bR \sdp \bA_0 \times \bA_1$,
  %%   then $\bB_0 \times \bA_1 \leq \bR$.
  %% \end{corollary}

  Once again, by collecting multiple abelian factors into a single factor, we obtain
  \begin{corollary}
  \label{cor:S-NA-AF_A-multi}
  Let  %% $\bR \sdp \bA_0 \times \bA_1 \times \cdots \bA_{n-1}$, where
  $\bA_0, \bA_1, \dots, \bA_n$ be finite Taylor algebras, where
  $\bA_0$ is simple nonabelian, $\bB_0 \minabsorbing \bA_0$, 
  and the remaining $\bA_i$ are abelian, % (hence absorption-free)
  and suppose $\bR \sdp \myprod_{\nn}\bA_i$.
  Then
  %% $\bR \sdp \bA_0 \times \bR_{0'}$ and $R \cap (B_0 \times R_{0'}) \neq \emptyset$; in fact,
  $\bB_0 \times \bR_{0'} \leq \bR$.
  \end{corollary}
  \begin{proof}
    Obviously, $\bR \sdp \bA_0 \times \bR_{0'}$ and 
    $R \cap (B_0 \times R_{0'}) \neq \emptyset$.
    Also, $\bR_{0'}:=\Proj_{0'} \bR$ is abelian, so we can
    apply Lemma~\ref{lem:rect-two_factors} (ii), with $\bR_{0'}$ in place of $\bA_1$.
    %% Corollary~\ref{cor:S-NA-AF_A} to
  \end{proof}
  
\section{The Rectangularity Theorem}
\label{sec:tayl-vari-rect}
\subsection{Some recent history}
In late May of 2015 we attended a workshop on ``Open Problems 
in Universal Algebra,'' at which Libor Barto announced a new theorem that he 
and Marcin Kozik proved called the ``Rectangularity Theorem.'' 
At this meeting Barto gave a detailed overview of the proof~\cite{Barto-shanks}.
The authors of the present paper then made a concerted effort over many months
to fill in the details and produce a complete proof.  
Unfortunately, each attempt uncovered a gap that we were unable to fill.
Nonetheless, we found a slightly different route to the theorem.
Our argument is similar to the one presented by Barto in most of 
its key aspects. In particular, we make heavy use of the
absorption idea and the Absorption Theorem plays a key role.  
However, we were unable to complete the proof without a new 
``Linking Lemma'' (Lem.~\ref{lem:Link-2}) that we proved using a powerful 
general result of Kearnes and Kiss.

Thus, our argument is similar in spirit to the original, but
provides alternative evidence that Barto and Kozik's Rectangularity Theorem
is correct, and may shed additional light on the problem.
In the next subsection, we present our proof of the Rectangularity Theorem 
and some new corollaries that we use 
in Section~\ref{sec:applications} to demonstrate 
how to apply the Rectangularity Theorem to \csp problems.
%% On the surface, our proof might seem shorter than the original argument. 
%% However, this is not really a fair comparison because we 
%% rely on a general theorem of Kearnes and Kiss about algebras in a Taylor variety,
%% the proof of which runs well over a page in~\cite{MR3076179}.

\subsection{Preliminaries}
From now on we assume:
\begin{quote}
\emph{all algebras are finite and belong to a single Taylor variety $\var{V}$.}
\end{quote}
%(That is, $\var{V}$ is idempotent and has a Taylor term.) 
As mentioned, our main goal in this section is to prove the  
Rectangularity Theorem of Barto and Kozik, which can be described as follows:
%% \noindent {\bf Rectangularity Theorem.}
Start with finite algebras $\bA_0, \bA_1, \dots, \bA_{n-1}$ in a 
Taylor variety with minimal absorbing subalgebras $\bB_i \minabsorbing \bA_i$.
Suppose at most one of these algebras is abelian and 
suppose the nonabelian algebras are simple.
Let $\bR$ be a subdirect product of $\myprod_{\nn} \bA_i$.
Then (after a few more side assumptions),
if $R$ intersects nontrivially with the product $\myprod_{\nn}B_i$
of minimal absorbing subuniverses,
then $R$ contains the entire product $\myprod_{\nn}B_i$.

%% Let $\bA_0, \bA_1, \dots, \bA_{n-1}$ be finite algebras in a
%% Taylor variety with minimal absorbing subalgebras 
%% $\bB_i \minabsorbing \bA_i$
%% and suppose
%% \begin{itemize}
%% \item at most one $\bA_i$ is abelian,
%% \item all nonabelian factors are simple, 
%% \item $\bR \sdp \bA_0 \times \bA_1 \times \cdots \times \bA_{n-1}$,
%% \item $\etaR_i \neq \etaR_j$ for all $i\neq j$, %where $\etaR_i = \ker(\bR \onto \bA_i)$,
%% \item $R':= R \cap (B_0 \times B_1 \times \cdots \times B_{n-1}) \neq \emptyset$.
%% \end{itemize}
%% Then, $\bR' = \bB_0 \times \bB_1 \times \cdots \times \bB_{n-1}$. % \leq \bR$.


%% OLD VERSION
%% \noindent {\bf Rectangularity Theorem.}
%% Let $\bA_0, \bA_1, \dots, \bA_{n-1}$ be finite simple algebras in a
%% Taylor variety with minimal absorbing subalgebras 
%% $\bB_i \minabsorbing \bA_i$ and suppose
%% \begin{itemize}
%% \item at most one $\bA_i$ is abelian, 
%% \item $\bR \sdp \bA_0 \times \bA_1 \times \cdots \times \bA_{n-1}$,
%% \item $\etaR_i \neq \etaR_j$ for all $i\neq j$, %, where $\etaR_i = \ker(\bR \onto \bA_i)$,
%% \item $\bR' = \bR \cap (\bB_0 \times \bB_1 \times \cdots \times \bB_{n-1}) \neq \emptyset$.
%% \end{itemize}
%% Then, $\bR' = \bB_0 \times \bB_1 \times \cdots \times \bB_{n-1}$.

Before attempting to prove this, we first prove a number of useful lemmas. 
In doing so, we will make frequent use of the following

\noindent {\bf Notation:}
\begin{itemize}
\item $\nn :=\{0, 1, 2, \dots, n-1\}$.
\item For $\sigma \subseteq \nn $, 
  %$\bA = \bA_0 \times \bA_1 \times \cdots \times \bA_{n-1}$,
  $A = A_0 \times A_1 \times \cdots \times A_{n-1}$,
  and $\ba = (a_0, a_1\dots, a_{n-1})$, we define
  \[
  \eta_\sigma := \ker(A \onto \Pi_\sigma A_i) = \{(\ba, \ba') \in A^2 \mid
  a_i = a_i' \text{ for all } i \in \sigma\},
  \]
  the kernel of the projection of $A$ onto coordinates $\sigma$.
\item For
  $\bR \sdp \bA_0 \times \bA_1 \times \cdots \times \bA_{n-1}$, let 
  \[
  \etaR_\sigma := \ker(R \onto \Pi_\sigma A_i) = \{(\br, \br') \in R^2 \mid
  r_i = r_i' \text{ for all } i \in \sigma\},
  \]
  the kernel of the projection of $R$ onto the coordinates $\sigma$. Thus,
  $\etaR_\sigma = \eta_\sigma \cap R^2$.  

\item For $\sigma\subseteq \nn$, we let $\sigma' := \nn -\sigma$, and by
  $\bR \sdp \myprod_\sigma \bA_i \times \myprod_{\sigma'}\bA_i$ we 
  mean that the following three conditions 
  \ifthenelse{\boolean{footnotes}}{%
    hold:\footnote{Note that the expression
    $\bR \sdp \myprod_\sigma \bA_i \times \myprod_{\sigma'}\bA_i$ does not mean
    $\bR$ is a subalgebra of $\myprod_\sigma \bA_i \times \myprod_{\sigma'}\bA_i$. 
    Generally speaking, such an interpretation would require permuting the coordinates 
    of elements of $\bR$, which is feasible but unnecessary if we use the definition 
    given in the text above.}
  }{hold:} 
  \begin{enumerate}
  \item $\bR$ is a subalgebra of $\myprod_{\nn} \bA_i$;
  \item $\Proj_\sigma\bR = \myprod_\sigma \bA_i$;
  \item $\Proj_{\sigma'}\bR = \myprod_{\sigma'} \bA_i$;
  \end{enumerate}
  we say that $\bR$ is a \defn{subdirect product of} 
  $\myprod_\sigma \bA_i$ \emph{and} $\myprod_{\sigma'}\bA_i$ in this case.
\item 
  The subdirect product $\bR \sdp \myprod_\sigma \bA_i \times
  \myprod_{\sigma'}\bA_i$ 
  is said to be \defn{linked} if $\etaR_\sigma \join \etaR_{\sigma'} = 1_R$.
  %where $\etaR_\sigma = \ker(\bR\onto \Pi_\sigma\bA_i)$.
\item We sometimes use $\bR_\sigma$ as shorthand for $\Proj_\sigma\bR$, the projection 
  of $\bR$ onto coordinates $\sigma$.

\end{itemize}


%\input{AF-rectangularity-thm.tex}

\subsection{Rectangularity theorem}
\label{sec:rect-theor}
We are almost ready to prove the general Rectangularity Theorem.
We need just three more results that play a crucial role in the proof.
The first comes from combining Lemma~\ref{lem:min-abs-prod}, transitivity of absorption, and 
Lemmas~\ref{lem:gen-abs1}--\ref{lem:linked-absorber}.
\begin{lemma}
\label{lem:general-linked}
Let $\bA_0, \bA_1, \dots, \bA_{n-1}$ 
be finite algebras in a Taylor variety, 
let $\bB_i\minabsorbing \bA_i$ for each $i\in \nn $, 
and let $\nn  = \sigma \cup {\sigma'}$ be a partition of 
$\{0, 1, \dots, n-1\}$ into two nonempty disjoint subsets.
Assume
$\bR$ is a \emph{linked} subdirect product of 
$\myprod_{\sigma} \bA_i$ and $\myprod_{\sigma'}\bA_i$, and
suppose $R' := R \cap \myprod_i B_i \neq \emptyset$.
Then $\myprod_i B_i\subseteq R$, so $\bR' = \myprod_i \bB_i$.
\end{lemma}
\begin{proof}
  By Lemma~\ref{lem:sdp-general}, $\bR' \sdp \myprod_\sigma \bB_i \times \myprod_{\sigma'} \bB_i$, and
  by Lemma~\ref{lem:gen-abs1}, $\bR' \absorbing \bR$.
  Therefore, Lemma~\ref{lem:linked-absorber} implies $\bR'$ is linked. 
  By Lemma~\ref{lem:min-abs-prod} and transitivity of absorption, it follows that
  $\myprod_{\sigma} \bB_i$ and $\myprod_{\sigma'} \bB_i$ are both absorption-free, so the Absorption
  Theorem %(Thm.~\ref{thm:absorption}) 
implies that $\bR' = \myprod_{\sigma}\bB_i \times \myprod_{\sigma'}\bB_i$.
\end{proof}


Next, we recall a powerful theorem Keith Kearnes and Emil Kiss, proved in~\cite{MR3076179}, 
about algebras in a variety that satisfies a nontrivial idempotent \malcev condition (equivalently,
has a Taylor term).
As above, we use the phrase ``Taylor variety'' to refer to an idempotent variety
with a Taylor term, and we call algebras in such varieties ``Taylor algebras.''

  \begin{theorem}[\protect{\cite[Thm~3.27]{MR3076179}}] 
    \label{thm:kearnes-kiss-3.27}
    Suppose $\alpha$ and $\beta$ are congruences of a Taylor algebra. Then
    $\CC{\alpha}{ \alpha}{ \alpha \meet \beta}$ if and only if
    $\CC{\alpha \join \beta}{ \alpha \join \beta}{ \beta}$.
  \end{theorem}
  \noindent We use this theorem to prove a 
  ``Linking Lemma'' that will be central to our proof of the Rectangularity
  Theorem, and the following corollary of Theorem~\ref{thm:kearnes-kiss-3.27} and 
  Lemma~\ref{lem:common-meets} gives the precise context in which we will apply 
  these results.
  \begin{corollary}
    \label{cor:common-meets}
    Let $\alpha_0$, $\alpha_1$, $\beta$, $\delta$ be congruences of a Taylor 
    algebra $\bA$, and suppose that 
    $\alpha_0 \meet \beta = \delta = \alpha_1 \meet \beta$ and
    % $\beta \leq \alpha_0 \join \alpha_1$, and 
    $\alpha_0 \join \alpha_1 = \alpha_0 \join \beta =\alpha_1 \join\beta =1_A$.
    Then, $\CC{1_A}{ 1_A}{ \alpha_0}$ and $\CC{1_A}{ 1_A}{ \alpha_1}$,
      so $\bA/\alpha_0$ and $\bA/\alpha_1$ are abelian algebras.
  \end{corollary}
  % \begin{remark}
  %   It follows from the hypotheses that $\alpha_0 \join \alpha_1 = 1_A$, since
  %   $1_A = \alpha_0 \join \beta \leq \alpha_0 \join \alpha_1$.
  % \end{remark}
  \begin{proof}
    The hypotheses of Lemma~\ref{lem:common-meets} hold, so
    $\CC{\beta}{\beta}{\delta}$ and 
    $\beta/\delta$ is an abelian congruence of $\bA/\delta$. 
    Now, since $\delta = \alpha_i \meet \beta$, 
    we have $\CC{\beta}{ \beta}{ \alpha_i \meet \beta}$, so 
    Theorem~\ref{thm:kearnes-kiss-3.27} implies
    $\CC{\alpha_i\join \beta}{ \alpha_i\join \beta}{ \alpha_i}$.
    This yields
    $\CC{1_A}{ 1_A}{ \alpha_i}$, 
    since $\alpha_i \join \beta =1_A$.  By 
    Lemma~\ref{lem:centralizers}~(\ref{centralizing_factors}) then,
    $\CC{1_A/\alpha_i}{ 1_A/\alpha_i}{ \alpha_i/\alpha_i}$.
    Equivalently,
    $\CC{1_{A/\alpha_i}}{ 1_{A/\alpha_i}}{ 0_{A/\alpha_i}}$. That is, $\bA/\alpha_i$ is abelian.
  \end{proof}

  Before stating the next result, we remind the reader that 
  $k' := \nn-\{k\}$.
  \begin{lemma}[Linking Lemma]
    \label{lem:Link-2}
    Let $n\geq 2$, let $\bA_0, \bA_1, \dots, \bA_{n-1}$ be finite algebras in a
    Taylor variety, and let $\bB_i \minabsorbing \bA_i$. Suppose
    \begin{itemize}
    \item at most one $\bA_i$ is abelian
    \item all nonabelian factors are simple
    \item $\bR \sdp \bA_0 \times \bA_1 \times \cdots \times \bA_{n-1}$,
    \item $\etaR_i \neq \etaR_j$ for all $i\neq j$. %where $\etaR_i = \ker(\bR \onto \bA_i)$.
    \end{itemize}
    Then there exists %every $\sigma \subseteq \nn$, the subdirect product
    $k$ such that $\bR\sdp \bA_k \times \bR_{k'}$ is linked.
  \end{lemma}
\ifthenelse{\boolean{arxiv}}{%
\begin{remark} 
  To be pedantic, the expression
  $\bR\sdp \bA_k \times \bR_{k'}$ that appears in the conclusion of the lemma 
  only makes sense if we swap the $0$-th and $k$-th coordinates of all elements of 
  $\bR$ so that the ``$0$-th projection'' is projection onto $\bA_k$. This is
  merely a defect of the notation.
\end{remark}
}{}
\begin{proof}
    By way of contradiction, suppose 
    $\etaR_k \join \etaR_{k'} < 1_R$ for all $0\leq k< n$. \\[5pt]
    \noindent \underline{Case 1:} There is no abelian factor.\\
    If $n=2$, the result holds by Corollary~\ref{cor:rect-two_factors-pre}.
    Assume $n>2$.
    Since every factor is simple, each $\etaR_k$ is a coatom, so 
    $\etaR_k \join \etaR_{k'} < 1_R$ implies $\etaR_k \geq \etaR_{k'}$.  
    Then,
    \begin{equation}
      \label{eq:1111}
      \etaR_{k'}:= \Meet_{i \neq k}\etaR_i 
      = \bigl(\Meet_{i \neq k}\etaR_i\bigr) \meet \etaR_k = \etaR_{\nn} = 0_R.
    \end{equation}
    Note that (\ref{eq:1111}) holds for all $k\in \nn$.
    Now, let $\tau \subseteq \nn$ be a subset that is maximal 
    among those satisfying $\etaR_\tau > 0_R$.
    Then by what we just showed, $\tau$ omits at least two indices, say, $j$ and $\ell$.
    Therefore, by maximality of $\tau$,
    \begin{equation*}%\label{eq:1221}
      \etaR_\tau \meet \etaR_j = 0_R = \etaR_\tau \meet \etaR_\ell,
    \end{equation*}  
    and $\etaR_\tau \nleq \etaR_j$, and $\etaR_\tau \nleq \etaR_\ell$.
    Since all factors are simple, $\etaR_j$ and $\etaR_\ell$ are coatoms and distinct by assumption, so
    \begin{equation*}%\label{eq:1222}
      \etaR_\tau \join \etaR_j  = \etaR_\tau \join \etaR_\ell = \etaR_j\join \etaR_\ell = 1_R.
    \end{equation*}  
    % By Lemma~\ref{lem:M3-abelian}, 
    But then, by Corollary~\ref{cor:common-meets}, both $\bA_j \cong \bR/\etaR_j$ and 
    $\bA_\ell\cong \bR/\etaR_\ell$ are abelian, which contradicts the assumption that 
    there are no abelian factors.
    %% Now, $\sansC(\etaR_j, \etaR_\tau;  \etaR_j  \meet \etaR_\tau)$, and by
    %% (\ref{eq:1221}) this means $\sansC(\etaR_j, \etaR_\tau)$. Similarly, $\sansC(\etaR_\ell, \etaR_\tau)$. 
    %% Therefore, by Lemma~\ref{lem:centralizers}~(\ref{fact:centralizing_over_join1}),
    %% $\sansC(\etaR_j \join \etaR_\ell, \etaR_\tau)$.
    %% This and (\ref{eq:1222}) give $\sansC(1_R, \etaR_\tau)$.  It follows by 
    %% Lemma~\ref{lem:centralizers}~(\ref{fact:monotone_centralizers1}) that 
    %% $\sansC(\etaR_\tau, \etaR_\tau)$. %; that is, $\etaR_\tau$ is abelian.
    %% Finally, Theorem~\ref{thm:kearnes-kiss-3.27} yields
    %% \[
    %% \sansC(\etaR_\tau, \etaR_\tau) \equiv 
    %% \sansC(\etaR_\tau, \etaR_\tau; \etaR_\tau \meet \etaR_j) \iff
    %% \sansC(\etaR_\tau \join \etaR_j, \etaR_\tau \join \etaR_j; \etaR_j) \equiv
    %% \sansC(1_R, 1_R; \etaR_j),
    %% \]
    %% which means $\bA_j$ is abelian---a contradiction. 
    \\[5pt]
    \noindent \underline{Case 2:} 
    One factor, say $\bA_0$, is abelian.\\
    For $k > 0$, $\bA_k$ is simple and nonabelian, so $\etaR_k$ is a coatom of $\Con\bR$.
    Therefore, 
    $\etaR_k \join \etaR_{k'} < 1_R$ implies
    $\etaR_{k'}\leq \etaR_k$, %(otherwise $\etaR_k \join \etaR_{k'} =1_{\bR}$), 
    so
    \begin{equation}
      \label{eq:110}
      \etaR_{k'}= \etaR_{k'} \meet \etaR_k = \Meet_{i\in \nn}\etaR_i = 0_R.
    \end{equation}
    (Note that this holds for every $0<k<n$.)

    Now, let $\theta = \etaR_{0'} := \Meet_{i \neq 0} \etaR_i$.
    Let $\tau$ be a maximal subset of $0':=\{1,2,\dots, n-1\}$ such that 
    $\etaR_\tau > \theta$.  
    %(Recall, $\etaR_\tau$ denotes $\Meet_{i \in \tau}\etaR_i > \theta$.)  
    Obviously $\tau$ is a proper subset of $0'$. Moreover, by~(\ref{eq:110})
    there exists $j\in 0'$ such that $\etaR_\tau \nleq \etaR_j$. 
    Therefore,
    \begin{equation}
      \label{eq:12}
      \etaR_\tau \join \etaR_j = 1_R\quad \text{ and } \quad
      \etaR_\tau \meet \etaR_j = \theta.
    \end{equation}
    The first equality in (\ref{eq:12}) holds since $\etaR_j$ is a coatom,
    while the second holds since $\tau$ is a maximal set such that
    $\etaR_{\tau}> \theta$.
    % $j\neq 0$ and since $\bA_0$ is the only possibly nonsimple factor, so 
    By Theorem~\ref{thm:kearnes-kiss-3.27}, we have
    $\CC{\etaR_\tau \join \etaR_j}{ \etaR_\tau \join \etaR_j}{ \etaR_j}$ iff
    $\CC{\etaR_\tau}{ \etaR_\tau}{  \etaR_\tau \meet \etaR_j}$. Using (\ref{eq:12}), this means
    \[
    \CC{1_R}{ 1_R}{ \etaR_j} \quad
    \iff \quad
    \CC{\etaR_\tau}{ \etaR_\tau}{  \theta}.
    \]
    However, $\CC{1_R}{ 1_R}{ \etaR_j}$ implies $\bA_j$ is abelian, but
    $\bA_0$ is the only abelian factor, so
    \begin{equation}
      \label{imp:12}
      %% \CC{\etaR_\tau}{ \etaR_\tau}{  \theta} \quad \Longrightarrow \quad \text{False}.  
      \CC{\etaR_\tau}{ \etaR_\tau}{  \theta} \; \text{{\it does not hold}}.  
    \end{equation}
    Since $\bA_0$ is abelian,
    $\CC{ 1_R}{ 1_R}{ \etaR_0}$; {\it a fortiori},
    $\CC{\etaR_\tau \join \etaR_0}{ \etaR_\tau \join \etaR_0}{ \etaR_0}$.
    The latter holds iff 
    $\CC{\etaR_\tau}{ \etaR_\tau}{  \etaR_\tau \meet \etaR_0}$, 
    by Theorem~\ref{thm:kearnes-kiss-3.27}.
    Now, if $\etaR_0 \meet \etaR_\tau \leq \theta$, then by
    Lemma~\ref{lem:abelian-quotients} we have
    $\CC{\etaR_\tau}{ \etaR_\tau}{ \theta}$, which is false by (\ref{imp:12}).
    Therefore, $\etaR_0 \meet \etaR_\tau \nleq \theta$.
    It follows that $\etaR_0 \meet \etaR_\tau \neq 0_R$. By~(\ref{eq:110}),
    then, there are at least two distinct indices, say, $j$ and $\ell$, in $\nn$ 
    that do not belong to $\tau$.   By maximality of $\tau$, we have
    $\etaR_\tau \meet \etaR_j = \theta = \etaR_\tau \meet \etaR_\ell$.
     By Corollary~\ref{cor:common-meets} (with $\alpha_0 = \etaR_j$, 
    $\alpha_1 = \etaR_\ell$, $\beta = \etaR_\tau$, $\delta = \theta$),
    it follows that $\bR/\etaR_j \cong \bA_j$ and $\bR/\etaR_\ell \cong
    \bA_\ell$ are both abelian---a contradiction,
    since, by assumption, the only abelian factor is~$\bA_0$.
    % \[\bA_j \cong \bR/\etaR_j \cong (\bR/\theta)/(\etaR_j/\theta).\]
    %% Therefore, $\CC{ \etaR_j}{ \etaR_\tau}{  \theta}$ and $\CC{ \etaR_\ell}{ \etaR_\tau}{ \theta}$,  
    %% so $\CC{ \etaR_j\join \etaR_\ell}{ \etaR_\tau}{  \theta}$.
    %% That is, $\CC{ 1_R}{ \etaR_\tau}{  \theta}$. {\it A fortiori}, 
    %% $\CC{ \etaR_\tau}{\etaR_\tau}{  \theta}$ holds, which contradicts (\ref{imp:12}).
\end{proof}
Finally, we have assembled all the tools necessary to accomplish the main goal of this section,
which is to prove the following:
\begin{theorem}[Rectangularity Theorem]
\label{thm:rectangularity}
Let $\bA_0, \bA_1, \dots, \bA_{n-1}$ be finite algebras in a
Taylor variety with minimal absorbing subalgebras 
$\bB_i \minabsorbing \bA_i$
and suppose
\begin{itemize}
\item at most one $\bA_i$ is abelian,
\item all nonabelian factors are simple, 
\item $\bR \sdp \bA_0 \times \bA_1 \times \cdots \times \bA_{n-1}$,
\item $\etaR_i \neq \etaR_j$ for all $i\neq j$, %where $\etaR_i = \ker(\bR \onto \bA_i)$,
\item $R':= R \cap (B_0 \times B_1 \times \cdots \times B_{n-1}) \neq \emptyset$.
\end{itemize}
Then, $\bR' = \bB_0 \times \bB_1 \times \cdots \times \bB_{n-1}$. % \leq \bR$.
\end{theorem}


\begin{proof}
  %% Assume without loss of generality that $\bA_0$ is the only possibly abelian factor.
  Of course it suffices to prove
  $B_0 \times B_1 \times \cdots \times B_{n-1} \subseteq R$.
  We prove this by induction on the number of factors in the product
  $\bA_0 \times \bA_1 \times \cdots \times \bA_{n-1}$.

  For $n=2$ the result holds by Lemma~\ref{lem:rect-two_factors}.
  Fix $n>2$ and assume that for all $2 \leq k < n$ 
  the result holds for subdirect products of 
  $k$ factors. We will prove the result holds for subdirect products of 
  $n$ factors.

  Let $\sigma$ be a nonempty proper subset of $\nn$ %:=\{0, 1, \dots, n-1\}$ 
  (so $1\leq |\sigma| < n$) and let
  $\sigma'$ denote the complement of $\sigma$ in $\nn$.
  Denote by $\bR_\sigma$ the projection of $\bR$ onto the $\sigma$-factors. 
  That is, $\bR_\sigma := \Proj_\sigma \bR$ and  $\bR_{\sigma'}:=\Proj_{\sigma'} \bR$.
  Each of these projections satisfies the assumptions
  of the theorem,
  since $\bR_{\sigma} \sdp \myprod_{\sigma}\bA_i$ and since
  for all $i\neq j$ in~$\sigma$, we have
  %\etaR_i \neq \etaR_j\; \iff \; 
  $\ker(R_{\sigma} \onto A_i) \neq \ker(R_{\sigma}\onto A_j)$.
  Similarly for $\bR_{\sigma'}$.
  %% \begin{equation}
  %%   \label{eq:4}
  %% %\etaR_i \neq \etaR_j\; \iff \; 
  %% \ker(R_{\sigma} \onto A_i) \neq \ker(R_{\sigma}\onto A_j).
  %% \end{equation}
  %% Similarly for $\bR_{\sigma'}$.
  %% (Let's make sure Equation (\ref{eq:4}) is as obvious as it seems:
  %% since $\etaR_i\neq \etaR_j$, either there is a pair 
  %% $(\br, \br')\in R^2\cap \etaR_i$ 
  %% that does not belong to $\etaR_j$,
  %% or there is a pair 
  %% $(\br, \br')\in R^2\cap \etaR_j$ 
  %% that does not belong to $\etaR_i$.  Without loss of generality, 
  %% assume the former, so that $r_i=r_i'$ and $r_j\neq r_j'$.  
  %% Since $i, j \in \sigma$, the projection
  %% of $\br$ and $\br'$ onto $\sigma$ will also satisfy
  %% $r_i=r_i'$ and $r_j\neq r_j'$.)
  %%
  %% Since $\bR_\sigma$ and  $\bR_{\sigma'}$ satisfy the assumptions of the theorem,
  Therefore, the induction hypothesis implies that %% \bB_{\sigma} :=
  $\myprod_{\sigma}\bB_i \leq \bR_{\sigma}$ and
  %% $\bB_{\sigma'} :=
  $\myprod_{\sigma'}\bB_i \leq \bR_{\sigma'}$.
  By Lemma~\ref{lem:min-abs-prod},
  $\myprod_{\sigma}\bB_i  \minabsorbing \myprod_{\sigma} \bA_i$, and
  since $\myprod_{\sigma}\bB_i \leq \bR_{\sigma}\leq \myprod_{\sigma} \bA_i$ 
  it's clear that  $\myprod_{\sigma}\bB_i \absorbing \bR_{\sigma}$.
  In fact, $\myprod_{\sigma}\bB_i \minabsorbing \bR_{\sigma}$ as well, 
  %% To see this,
  %%   suppose $\bS\leq \myprod_{\sigma}\bB_i$ and $\bS\absorbing \bR_{\sigma}$. Then
  %%   $\bS\absorbing \myprod_{\sigma}\bB_i$, so $\bS\absorbing \myprod_{\sigma}\bA_i$, by
  %%   transitivity of absorption. Thus, by minimality of
  %%   $\myprod_{\sigma}\bB_i \minabsorbing \myprod_{\sigma}\bA_{i}$,
  %%   we have $\bS = \myprod_{\sigma}\bB_i$, as desired.
  by minimality of
  $\myprod_{\sigma}\bB_i \minabsorbing \myprod_{\sigma}\bA_{i}$,
  and transitivity of absorption.
  To summarize, for every $\emptyset \subsetneqq \sigma\subsetneqq \nn$,
  \begin{equation}
    \label{eq:20}
  \bR \sdp \bR_{\sigma} \times \bR_{\sigma'}, \quad
  \myprod_\sigma \bB_i \minabsorbing \bR_{\sigma}, \quad
  \myprod_{\sigma'} \bB_i \minabsorbing \bR_{\sigma'}.
  %% \bB_{\nn } \minabsorbing \bR_{\nn } \quad \text{ and } \quad
  %% \bB_n \minabsorbing \bA_n.
  %% \bB_{\nn } \minabsorbing \bR_{\nn } \quad \text{ and } \quad
  %% \bB_n \minabsorbing \bA_n.
  \end{equation}

  Finally, by the Linking Lemma (Lem.~\ref{lem:Link-2}) there
  is a $k$ such that 
  $\bR \sdp \bR_{k} \times \bR_{k'}$ is linked.  Therefore,
  by Lemma~\ref{lem:general-linked} the
  the proof is complete.
\end{proof}


In case there is more than one abelian factor, we have the following slightly more general result.
Recall our notation: $\nn := \{0,1,\dots, n\}$, and if $\alpha\subseteq \nn$, then 
\[
\alpha' := \nn-\alpha \quad \text{and} \quad
\bR_\alpha := \Proj_{\alpha} \bR.\]
\begin{corollary}
\label{cor:tayl-vari-abel-fact}
Let $\bA_0, \bA_1, \dots, \bA_{n-1}$ be algebras in a
Taylor variety with $\bB_i \minabsorbing \bA_i$ ($i\in \nn$),
and let $\alpha \subseteq \nn$.  Suppose
\begin{itemize}
\item $\bA_i$ is abelian for each $i \in \alpha$,
\item $\bA_i$ is nonabelian and simple for each $i \in \alpha'$,
\item $\bR \sdp \bA_0 \times \bA_1 \times \cdots \times \bA_{n-1}$,
\item $\etaR_i \neq \etaR_j$ for all $i\neq j$, %where $\etaR_i = \ker(\bR \onto \bA_i)$,
\item $R':= R \cap (B_0 \times B_1 \times \cdots \times B_{n-1}) \neq \emptyset$.
\end{itemize}
Then $\bR'= \bR_\alpha  \times \myprod_{\alpha'}\bB_i$.
\end{corollary}
\begin{proof}
%% If $|\alpha| = n$, then $\bB_i = \bA_i$ for all $i$, so $R' = R$ and the result holds trivially.
%% Assume $1<|\alpha|<n$.
%% Let $\bR_\alpha := \Proj_{i\in \alpha} \bR$.
Suppose $\alpha' = \{i_0, i_1, \dots, i_{m-1}\}$.
Clearly, 
$\bR \sdp \bR_\alpha \times \bA_{i_0} \times \bA_{i_1} \times \cdots \times \bA_{i_{m-1}}$. 
If $\alpha\neq \emptyset$, then the product 
has a single abelian factor %$\Proj_{i\in \alpha} \bR \leq \myprod_{i\in \alpha} \bA_i$.
$\bR_\alpha \leq \myprod_{\alpha} \bA_i$.
Otherwise, $\alpha= \emptyset$ and the product has no abelian factors.
In either case, the result follows from Theorem~\ref{thm:rectangularity}.
\end{proof}

To conclude this section, we make two more observations that 
facilitate application of the foregoing results to \csp problems. 
\begin{corollary}
  \label{cor:RT-cor}
Let $\bA_0, \bA_1, \dots, \bA_{n-1}$ be algebras in a
Taylor variety with $\bB_i \minabsorbing \bA_i$ for each $i\in \nn$ and suppose
$\bR$ and $\bS$ are subdirect products of $\myprod_{\nn} \bA_i$.
%$\bA_0 \times \bA_1 \times \cdots \bA_{n-1}$.
%% \item $\bS \sdp \bA_0 \times \bA_1 \times \cdots \times \bA_{n-1}$,
Let $\alpha \subseteq \nn$ and assume the following:
\begin{enumerate}
\item $\bA_i$ is abelian for each $i \in \alpha$,
\item $\bA_i$ is nonabelian and simple for each $i \notin \alpha$,
\item $\etaR_i \neq \etaR_j$ for all $i\neq j$, %where $\etaR_i = \ker(\bR \onto \bA_i)$,
\item $R$ and $S$ both intersect $\myprod_{\nn} B_i$ nontrivially, 
\item there exists $\bx \in R_\alpha \cap S_\alpha$.
\end{enumerate}
Then $R \cap S \neq \emptyset$.
\end{corollary}
\begin{proof}
  By Corollary~\ref{cor:tayl-vari-abel-fact},
$\bR'= \bR_\alpha   \times \myprod_{\alpha'}\bB_i$ and
$\bS'= \bS_\alpha   \times \myprod_{\alpha'}\bB_i$. 
Therefore, since $\bx \in R_\alpha \cap S_\alpha$, we have
$\{\bx\} \times \myprod_{\alpha'}\bB_i \subseteq R \cap S$.
\end{proof}
Of course, this can be generalized to more than two subdirect products, as the 
next result states.
\begin{corollary}
  \label{cor:RT-cor-gen}
Let $\bA_0, \bA_1, \dots, \bA_{n-1}$ be algebras in a
Taylor variety with $\bB_i \minabsorbing \bA_i$ ($i\in \nn$).
Suppose $\{\bR_\ell : 0\leq \ell < m\}$ is a set of $m$ subdirect products of 
$\myprod_{\nn} \bA_i$.
%$\bA_0 \times \bA_1 \times \cdots \bA_{n-1}$.
%% \item $\bS \sdp \bA_0 \times \bA_1 \times \cdots \times \bA_{n-1}$,
Let $\alpha \subseteq \nn$ and assume the following:
\begin{enumerate}
\item $\bA_i$ is abelian for each $i \in \alpha$,
\item $\bA_i$ is nonabelian and simple for each $i \notin \alpha$,
\item $\forall \ell \in \mm$, $\forall i\neq j$, 
  $\etaR^\ell_i \neq \etaR^\ell_j$ (where $\etaR^\ell_i := \ker(\bR_\ell \onto \bA_i)$),
\item\label{item:RT-cor-gen-4} each $R_\ell$ intersects $\myprod B_i$ nontrivially, 
\item\label{item:RT-cor-gen-5} there exists $\bx \in \bigcap \Proj_\alpha R_\ell$.
\end{enumerate}
Then $\bigcap R_\ell \neq \emptyset$.
\end{corollary}








%%%% wjd: adding pagebreak for ``draft mode'' to reduce printing costs
%%%%      To turn off these unnecessary page breaks, set `draft` to false
%%%%      near the top of this file.
\ifthenelse{\boolean{draft}}{\newpage}{}



\section{CSP Applications}
\label{sec:applications}
In this section we give a precise definition of what we mean by a ``constraint
satisfaction problem,'' and what it means for such a problem to be ``tractable.''
We then give some examples demonstrating how one uses the algebraic tools we
have developed to prove tractablility.



\subsection{Definition of a constraint satisfaction problem}
\label{sec:defin-constr-satisf}
%%% Splitting off this file for now because I want to use it in the cib4-notes.tex
%%% document and I don't want to make more work for myself. (We will have to merge this
%%% document and cib4-notes.tex eventually.)


%% -------------------------------------------------------------
%% --- wjd (20160727): inserting file
%% \input{inputs/csp-definition.tex}
%%%%% INPUT: CSP DEFINITION %%%%%
We now define a ``constraint satisfaction problem'' in a way that is convenient
for our purposes. This is not the most general definition possible, but for now
we postpone consideration of the scope of our setup.

Let $\bA = \<A, \sF\>$ be a finite idempotent algebra,
and let $\Sub(\bA)$ and $\sansS(\bA)$ denote the set of subuniverses and subalgebras
of $\bA$, respectively. 
\begin{definition}
  \label{def:csp}
  Let $\mathfrak{A}$ be a collection of algebras of the same similarity type.
  We define $\CSP(\mathfrak{A})$ to be the following decision problem:
\begin{itemize}
\item  An  \defn{$n$-variable instance} of $\CSP(\mathfrak{A})$
  is a quadruple $\<\sV, \sA, \sS, \sR\>$ consisting of
  \begin{itemize}
  \item a set $\sV$ of $n$ \emph{variables}; often we take $\sV$ to be
    $\nn= \{0, 1, \dots, n-1\}$;
  \item a list $\sA = (\bA_0, \bA_1, \dots, \bA_{n-1})\in \mfA^n$ of
    algebras from $\mathfrak{A}$, one for each variable;
  \item a list $\sS = (\sigma_0, \sigma_1, \dots, \sigma_{J-1})$
    of \emph{constraint scope functions}
    with arities $\ar(\sigma_j) = m_j$;
  \item a list $\sR = (R_0, R_1, \dots, R_{J-1})$ of \emph{constraint relations}, where
    each $R_j$ is the universe of a subdirect product of the algebras
    in $\sA$ with indices in $\im \sigma_j$; that is,
    \[\bR_j \sdp \prod_{0\leq i < m_j}\bA_{\sigma_j(i)}.\]
  \end{itemize}
\item A \defn{solution} to the instance $\<\sV, \sA, \sS, \sR\>$
  is an assignment %  $f \in \prod_{i\in \sV}A_i$ 
  $f \colon \sV \to \bigcup_{\nn}A_i$ %% $f: \sV \to A$
  of values to variables that satisfies all constraint relations.  More precisely,
  $f\in \myprod_{\nn}A_i$ and $f \circ \sigma_j \in R_j$ holds for all $0\leq j < J$,
  
\end{itemize}
\end{definition}
We will denote the set of solutions to the instance $\<\sV, \sA, \sS,\sR\>$ by $\Sol(\sS,\sR, \nn)$.

\begin{remark}\
  \begin{enumerate}[(i)]
  \item The $i$-th scope function $\sigma_i \colon \mm_i \to \sV$ picks out the $m_i$
    variables from $\sV$ involved in the constraint relation $R_i$. Thus, the
    list $\sS$ of scope functions belongs to $\myprod_{i< J}\sV^{m_i}$.
  \item
    Frequently we require that the arities of the scope functions be bounded
    above, i.e., $\ar(\sigma_j) \leq m$, for all $j<J$. This gives rise to the
    \emph{local constraint satisfaction problem} $\CSP(\mathfrak A,m)$ consisting
    of instances of this restricted type.  
  \item
    If $(\sigma, R) \in \sC$ is a constraint of an $n$-variable instance
    %% $\<\sV, \sA, \sC\>$,
    then we denote by $\Sol((\sigma, R), \nn)$ the set of all tuples in $\prod_{\nn}A_i$
    that satisfy $(\sigma, R)$. 
    In fact, we already introduced the notation $R^{\overleftarrow{\sigma}}$ for this set
    in~(\ref{eq:19}) of Section~\ref{sec:proj-scop-kern}.  Recall, 
    $\bx \in R^{\overleftarrow{\sigma}}$ iff $\bx \circ \sigma \in R$. Thus, 
    \[
    \Sol((\sigma, R), \nn) = R^{\overleftarrow{\sigma}}
    := \bigl\{\bx \in \prod_{i\in \nn}A_i %\prod_{i\in \nn}A_i
    \mid \bx \circ \sigma \in R\bigr\}.
    \]
    Therefore, the set of solutions to the instance  $\<\sV, \sA, \sC\>$ is
    \[\Sol(\sC, \nn) = \bigcap_{(\sigma, R) \in \sC} R^{\overleftarrow{\sigma}}.\]

  %%% --- wjd: (2016/07/25) omitting this point (we never use the notation described).
  %% \item
  %%   On the other hand, suppose we are given a tuple $\bx\in \prod_{\nn}A_i$
  %%   and we wish to test whether $\bx$ satisfies a particular constraint $(\sigma, R)$.
  %%   If we specify the constraint relation $R$ by its characteristic function
  %%   $\chi_R:   \prod_{\mm} A_{\sigma(i)} \to \{0,1\}$, then to test whether
  %%   $\bx$ satisfies this constraint---that is, to test
  %%   $\bx \circ \sigma \in R$---we simply apply $\chi_R$.  In this way, the 
  %%   set $\Sol(\sC, \nn)$ of solutions to the $n$-variable instance
  %%   $\<\sV, \sA, \sC\>$ consists of those $\bx\in \prod_{\nn}A_i$ satisfying
  %%   \[\bigwedge_{0\leq j < J} \chi_{R_j} \circ f \circ \sigma_j = 1.\]
  %% ---

  %%% --- wjd: (2016/07/25) omitting this point (we never use the notation described;
  %%           also, we now have more convenient notation for this, namely, $R^{\overleftarrow{\sigma}}$
  %% \item It can be useful to visualize the pair $(\sigma_i, R_i)$ as specifying a subalgebra
  %%   of the full product $\myprod_{\sV}\bA_j$ as follows:
  %%   $\{\ba \in \myprod_{\nn}A_j \mid \ba\circ\sigma_i \in R_i\}$.
  %%   Following Willard, we denote this subalgebra by $\lb \sigma_i, R_i\rb$.
  %%   This notation is convenient because a solution to the instance is now
  %%   simply an element in the intersection
  %%   $\bigcap_{i\in V}\lb \sigma_i, \bR_i\rb$.  However, when considering tractability of the
  %%   \csp, and the running time of a given algorithm, we cannot assume the input
  %%   size of the instance is determined by the sizes of these subalgebras of the full product.
  %%   (We define the input size of an instance below.)
  %% ---

\item If $\mathfrak A$ contains a single algebra, we write $\CSP(\bA)$
    instead of $\CSP(\{\bA\})$.  It is important to note that, in our definition of an
    instance of $\CSP(\mathfrak A)$, a constraint relation is a subdirect product of
    algebras in $\mathfrak A$. This means that the constraint relations of an
    instance of $\CSP(\bA)$ are subdirect powers of $\bA$.
    In the literature it is conventional to allow constraint relations
    of $\CSP(\bA)$ to be subpowers of $\bA$. 
    Such interpretations would correspond to instances of
    $\CSP(\sansS (\bA))$  in our notation.
  \end{enumerate}
\end{remark}


\subsection{Instance size and tractability}
\label{sec:inst-size-tract}
%% A finite binary sequence (i.e., a finite string of 0's and 1's) is
%% called a \emph{word} in the \emph{alphabet} $\{0, 1\}$.  
%% Let $\sW$ denote the set of all \emph{words},
%% \[ \sW = \bigcup_{n\in \N}\{0, 1\}^n. \]  
%% A function $f : \sW \rightarrow \sW$ is said to be \emph{computable in
%% polynomial time}, or \emph{in the class \P}, if there exists an algorithm
%% $\sansA$ and constants $c$ and $d$ such that given $x \in W$ of size
%% $n$, $\sansA$ stops in less than $cn^d$ steps and outputs $f(x)$.
%% (Occasionally we will use the phrase \emph{\P-time computable} in
%% place of \emph{computable in polynomial time}.)

We measure the computational complexity of an algorithm for solving instances
of $\CSP(\mathfrak A)$ as a function of input size. In order to do this, and to say
what it means for $\CSP(\mathfrak A)$ to be ``computationally tractable,'' we first
need a definition of input size. %Conforming to standard convention in theoretical work,
%we define the size of an instance to be the sum of the sizes of the explicit constraints.
%Informally, this means that the size of a problem instance is the length of a string
%containing all constraint scopes and all tuples of all constraint relations from the instance.
%Alternatively, the \defn{size of the instance} $\<\sV, \sA, \sS, \sR\>$ of $\CSP(\mathfrak A)$ is 
%the number of bytes required to encode $\sA$, $\sS$, and $\sR$.
%
%Here is a slightly more precise definition.
%Recall, if $\<\sV, \sA, \sS, \sR\>$ is an instance of $\CSP(\mathfrak A)$, then for each
%$0\leq i< p$, we represent the $i$-th constraint scope $\sigma_i$ as a list of
%variables from $\sV$, that is, $\sigma_i$ is
%a function from $\mm_i$ to $\sV$, and the $i$-th constraint relation as a set $R_i$ of tuples in
%$\myprod_{j\in \pp}A_{\sigma_i(j)}$.
%\begin{definition}[cf.~Def.~2.3 of~\cite{MR2416347}]
%  The \defn{size of the instance}
%  $I = \<\sV, \sA, \sS, \sR\>$ is 
%  \[size(I) := \sum_{i\in \pp} m_i (\log |\sV| + |R_i| \log |A|).\]
%\end{definition}
%TODO: The formula for $size(I)$ is not right.  Fix it!
%\\\\
In our case, this amounts to determing the number of bits required to completely specify an instance of the problem. In practice, an upper bound on the size is usually sufficient. 

Using the notation in Definition~\ref{def:csp} as a guide, we bound the size of an instance $\sI=\<\sV, \sA, \sS, \sR\>$ of $\CSP(\mathfrak A)$. Let $q=\max(\card{A_0},\, \card{A_1},\dots,\card{A_{n-1}})$, let $r$ be the maximum rank of an operation symbol in the similarity type, and $p$ the number of operation symbols. Then each member of the list $\sA$ requires at most $pq^r\log q$ bits to specify. Thus
\begin{equation*}
\size(\sA) \leq npq^r\log q.
\end{equation*}
Similarly, each constraint scope $\sigma_j\colon \mm_j \to \nn$ can be encoded using $m_j\log n$ bits. Taking $m=\max(m_1,\dots,m_{J-1})$ we have
\begin{equation*}
\size(\sS) \leq Jm\log n.
\end{equation*}
Finally, the constraint relation $R_j$ requires at most $q^{m_j}\cdot m_j \cdot \log q$ bits. Thus
\begin{equation*}
\size(\sR) \leq Jq^m\cdot m\log q.
\end{equation*}
Combining these encodings and using the fact that $\log q \leq q$, we deduce that
\begin{equation}\label{eqn:size}
\size(\sI) \leq npq^{r+1} + Jmq^{m+1} + Jmn. 
\end{equation}
In particular, for the problem $\CSP(\mathfrak A,m)$, the parameter $m$ is considered fixed, as is~$r$.
In this case, we can assume $J\leq n^m$. Consequently $\size(\sI) \in O((nq)^{m+1})$ 
which yields a polynomial bound (in $nq$) for the size of the instance.

A problem is called \defn{tractable} if there exists a deterministic
polynomial-time algorithm solving all instances of that problem.
We can use Definition~\ref{def:csp} above to classify the complexity of an algebra
$\bA$, or collection of algebras $\mathfrak A$, according to the complexity of their
corresponding constraint satisfaction problems.

An algorithm $\sansA$ is called a \defn{polynomial-time algorithm}
for $\CSP(\mathfrak A)$ %(or \defn{runs in polynomial-time}
if there exist constants $c$ and $d$ such that, given an instance $\sI$ of
$\CSP(\mathfrak A)$ of size $S= \size(\sI)$,
$\sansA$ halts in at most $c S^d$ steps and outputs
whether or not $\sI$ has a solution.  In this case, we say $\sansA$
``solves the decision problem $\CSP(\mathfrak A)$ in polynomial time''
and we call the algebras in $\mathfrak A$ ``jointly tractable.''
Some authors say that an algebra $\bA$ as tractable when
$\mathfrak A = \sansS(\bA)$ is jointly tractable,
or when $\mathfrak A = \sansS \sansP_{\mathrm{fin}} (\bA)$
is jointly tractable.
We say that $\mathfrak A$ is \emph{jointly locally tractable} if, for every natural number, $m$, there is a polynomial-time algorithm $\sansA_m$ that solves $\CSP(\mathfrak A,m)$. 

We wish to emphasize that, as is typical in computational complexity, the problem $\CSP(\mathfrak A)$ is a \emph{decision problem,} that is, the algorithm is only required to respond ``yes'' or ``no'' to the question of whether a particular instance has a solution, it does not have to actually produce a solution. However, it is a surprising fact that if $\CSP(\mathfrak A)$ is tractable then the corresponding \emph{search problem} is also tractable, in other words, one can design the algorithm to find a solution in polynomial time, if a solution exists, 
see~\cite[Cor~4.9]{MR2137072}.





\subsection{Sufficient conditions for tractability}
\label{ssec:edge-sdm}
A lattice is called \emph{meet semidistributive} if it satisfies the quasiidentity
\begin{equation*}
x\meet y \approx x\meet z \rightarrow x\meet y \approx x\meet (y\join z).
\end{equation*}
A variety is \sd-\meet\ if every member algebra has a meet semidistributive congruence lattice. Idempotent \sd-\meet\ varieties are known to be Taylor~\cite{HM:1988}. In~\cite{MR2893395}, Barto and Kozik proved the following.

\begin{theorem}\label{thm:sdm-tractable}
Let \bA\ be a finite idempotent algebra lying in an \sd-\meet\ variety. Then \bA\ is tractable. 
\end{theorem}

A second signifcant technique for establishing tractability is the ``few subpowers algorithm,'' which, according to its discoverers, is a broad generalization of Gaussian elimination.

\begin{definition}\label{defn:edge-term}
Let $\var{V}$ be a variety and $k$ an integer, $k>1$. A $(k+1)$-ary term $t$ is called a \emph{$k$-edge term for $\var{V}$} if the following $k$ identities hold in $\var{V}$:
\begin{align*}
t(y,y,x,x,x,\dots,x) &\approx x\\
t(y,x,y,x,x,\dots,x) &\approx x\\
t(x,x,x,y,x,\dots,x) &\approx x\\
&\vdots\\
t(x,x,x,x,x,\dots,y) &\approx x.
\end{align*}
\end{definition}

Clearly every edge term is idempotent and Taylor. It is not hard to see that every \malcev term and every near unanimity term is an edge term. Combining the main results of \cite{MR2563736} and~\cite{MR2678065} yields the following theorem.

\begin{theorem}\label{thm:edge-tractable}
Let \bA\ be a finite idempotent algebra with an edge term. Then \bA\ is tractable. 
\end{theorem}

Finally, we comment that tractability is largely preserved by familiar algebraic constructions.

\begin{theorem}[\cite{MR2137072}]\label{thm:HSP-tract}
Let \bA\ be a finite, idempotent, tractable algebra. Every subalgebra and finite power of \bA\ is tractable. Every homomorphic image of \bA\ is locally tractable. 
\end{theorem}



\subsection{Rectangularity Theorem: obstacles and applications}
The goal of this section is to consider aspects of the Rectangularity Theorem 
that seem to limit its utility as a tool for proving tractability of \csps.
%% instance has a solution, and to prove tractability of $\CSP(\mathfrak A)$.
We first give a brief overview of the potential obstacles, and then consider
each one in more detail in the following subsections.
\begin{enumerate}
\item
  \label{item:abelian-potatoes-tractable}
  {\bf Abelian factors must have easy partial solutions.}
  One potential limitation concerns the abelian factors in the product 
  algebra associated with a \csp instance.
  Indeed, %% Corollaries \ref{cor:RT-cor} and
  Corollary~\ref{cor:RT-cor-gen} assumes that
  when the given constraint relations are projected onto abelian factors,
  we can efficiently find a solution to this restricted instance---that is,
  an element satisfying all constraint relations after projecting these
  relations onto the abelian factors of the product.  
  Section~\ref{sec:tract-abel-algebr} shows that this concern is easily
  dispensed with and is not a real limitation of
  Corollary~\ref{cor:RT-cor-gen}.

\item {\bf Intersecting products of minimal absorbing subalgebras.}
  Another potential obstacle concerns the nonabelian simple factors.
  As we saw in Section~\ref{sec:rect-theor}, the Rectangularity Theorem
  (and its corollaries) assumes that the universes of the subdirect
  products in question all intersect nontrivially
  %% with a minimal absorbing subuniverse of $\myprod \bA_i$; that is, $R$ intersects nontrivial
  with a single product $\myprod B_i$ of minimal absorbing subuniverses.
  (We refer to ``minimal absorbing subuniverses'' quite frequently, so from now
  on we call them \defn{masses}; that is, a \defn{mass} is 
  a {\bf m}inimal {\bf a}bsorbing {\bf s}ubuniver{\bf s}e, and the product of
  \masses will be called a \defn{mass product}.)
  Moreover, assumption~(\ref{item:RT-cor-gen-4}) of Corollary~\ref{cor:RT-cor-gen}
  requires that all constraint relations intersect nontrivially with a single
  \mas product. This is a real
  limitation, as we demonstrate in Section~\ref{sec:mass-products} below.

\item {\bf Nonabelian factors must be simple.}
  This is the most obvious limitation of the theorem and at this point
  we don't have a completely general means of overcoming it.  However,
  some methods that work in particular cases are described below.

\end{enumerate}

In the next two subsections we address potential limitations (1) and (2).
In Section~\ref{sec:var-reduc} we develop some alternative methods for
proving tractability of nonsimple algebras, and then
in Section~\ref{sec:csps-comm-idemp} we apply these methods 
in the special setting of ``commutative idempotent binars.''
%% Subsection~\ref{sec:tract-abel-algebr} will show that requiring
%% that Abelian factors are easy to solve is not a significant limitation.
%% Subsection~\ref{sec:mass-products}

\subsubsection{Tractability of abelian algebras}
\label{sec:tract-abel-algebr}
To address concern~(\ref{item:abelian-potatoes-tractable})
of the previous subsection, we observe that finite abelian algebras yield
tractable \csps.
We will show how to use this fact and the Rectangularity Theorem to find 
solutions to a given \csp instance (or determine that none exists).
%\newcommand{\sims}{\ensuremath{\stackrel{s}{\sim}}}
%Consider the following result (where
%the notation  $\alpha \sims \beta$ means... \todo{fill in definition of $\sims$}) 
%
%
%\begin{theorem}[Theorem 7.12 of \cite{HM:1988}]
%  For any locally finite variety $\var{V}$, the following are equivalent.
%  \begin{enumerate}
%  \item $1 \notin \typ\{\var{V}\}$;
%  \item $\var{V}$ has an idempotent term $p(x, y, z)$  such that for every $\bA \in \var{V}$ and $\theta\in  \Con \bA$,
%    \[\theta \sims 0_A \; \text{ and } \;
%    (a, b) \in \theta \quad \Longrightarrow \quad  p(a, b, b) = a \; \text{ and } \; p(a, a, b) = b.
%    \]
%    \item For every $A \in \var{V}$ and $\alpha, \beta \in \Con \bA$, if $\alpha \sims \beta$ then
%      $\alpha \circ \beta = \beta \circ \alpha$.
%  \end{enumerate}
%\end{theorem}
%If $\bA$ is a finite idempotent abelian algebra, then $1_A \sims 0_A$, so the above
%theorem implies $\bA$ is congruence permutable and therefore has a tractable \csp.
%\todo{insert reference}
To begin, recall the fundamental result of tame congruence
theory~\cite[Thm~7.12]{HM:1988}  that we 
reformulated as Lemma~\ref{lem:HM-thm-7-12}.
As noted in Remark~\ref{rem:abelian-quotients} above,
this result has the following important corollary.
(A proof that avoids tame congruence theory appears in~\cite[Thm~5.1]{MR3374664}.)
\begin{theorem}
  \label{thm:type2cp}
Let $\var{V}$ be a locally finite variety with a Taylor term. Every finite abelian member of $\var{V}$ generates a congruence-permutable variety. Consequently, every finite abelian member of $\var{V}$ is tractable.
\end{theorem}

Let $\bA$ be a finite algebra in a Taylor variety 
and fix an instance $\sI = \<\sV, \sA, \sS, \sR\>$ of $\CSP(\sansS(\bA))$
with $n = |\sV|$ variables.
Suppose all nonabelian algebras in the list $\sA$ are simple.
Let $\alpha \subseteq \nn$ denote the indices of the abelian algebras in $\sA$,
and assume without loss of generality that $\alpha = \{0,1,\dots, q-1\}$.
That is, $\bA_0, \bA_1, \dots, \bA_{q-1}$ are finite idempotent abelian algebras. Consider
now the restricted instance $\sI_\alpha$ obtained by dropping all constraint relations
with scopes that don't intersect $\alpha$, and by restricting the remaining constraint
relations to the abelian factors.
Since the only algebras involved in $\sI_\alpha$ are abelian, this is an instance of a
tractable \csp.  Therefore, we can check in polynomial-time whether or not $\sI_\alpha$
has a solution.
If there is no solution, then the original instance $\sI$ has no solution.
On the other hand, suppose
$f_\alpha\in \myprod_{j\in \alpha}A_j$ is a solution to $\sI_\alpha$.
In Corollary~\ref{cor:RT-cor-gen},
to reach the conclusion that the full instance has a solution, we 
required a partial solution $\bx \in \bigcap \Proj_\alpha R_\ell$.
This is precisely what $f_\alpha$ provides.

To summarize, we try to find a partial 
solution by restricting the instance to abelian factors and, if such a partial solution exists,
we use it for $\bx$ in Corollary~\ref{cor:RT-cor-gen}.
Then, assuming the remaining hypotheses of Corollary~\ref{cor:RT-cor-gen} hold,
we conclude that a solution to the original instance exists.
If no solution to the restricted instance exists, then the original instance has no solution.
Thus we see that assumption~(\ref{item:RT-cor-gen-5}) of
Corollary~\ref{cor:RT-cor-gen} does not limit the application scope of this result.

\subsubsection{Mass products}
\label{sec:mass-products}
This section concerns products of minimal absorbing subalgebras, or
``\mas products.'' Hypothesis~(\ref{item:RT-cor-gen-4}) of
Corollary~\ref{cor:RT-cor-gen} assumes that all constraint relations intersect
nontrivially with a single \mas product.
%% \towjd{I'm not sure what you mean by ``the hypothesis does not hold". As you point out in your example,
%% the hypothesis \emph{does} hold.}
%% \tochb{I mean, we can find examples where the subdirect products do not all
  %% intersect nontrivially with a single mass.  As explained in the next
  %% paragraph, there are some mass products that intersect with $R$, and
  %% others that intersect with $S$, but none that intersects with both $R$
  %% and $S$. Does that make sense?}
However, it's easy to contrive instances where this hypothesis does not hold.
For example, take the algebra $\bA = \<\{0,1\}, m\>$,
where $m \colon A^3 \to A$ is the ternary idempotent majority operation---that is,
$m(x,x,x)\approx x$ and $m(x,x,y)\approx m(x,y,x)\approx m(y,x,x) \approx x$.
Consider subdirect products $\bR = \<R, m\>$ and $\bS=\<S, m\>$ of
$\bA^3$ with universes
  \begin{align*}
  R &= \{(0,0,0), (0,0,1), (0,1,0), (1,0,0)\},\\
  S &= \{(0,1,1), (1,0,1), (1,1,0), (1,1,1)\}.
  \end{align*}
  %% Below we will see an example similar to this one (Example~\ref{ex:mass-products-3}),
  %% so we leave it to the reader to verify that in the present case 
  %% t
  Then there are \mas products that intersect nontrivially with either $R$ or $S$, 
  but no \mas product intersects nontrivially with both $R$ and $S$. 
  As the Rectangularity Theorem demands,
  each of $R$ and $S$ fully contains every \mas product that it intersects, but there is no
  single \mas product intersecting nontrivially with both $R$ and $S$. Hence,
  hypothesis~(\ref{item:RT-cor-gen-4}) of Corollary~\ref{cor:RT-cor-gen} is not
  satisfied so it would seem that we cannot use rectangularity 
  to determine whether a solution exists in such instances.
  
%% \begin{example}
%%   \label{ex:majority}
%%   Let $\bA = \<\{0,1\}, \{m\}\>$, where $m: A^3 \to A$ is a majority operation, that is,
%%   $m(x,x,y)\approx m(x,y,x)\approx m(y,x,x) \approx x$.  Notice that
%%   $\{0\} \minabsorbing \bA$ and
%%   $\{1\} \minabsorbing \bA$.
%%   Let $B_0 = \{0\}$ and $B_1 = \{0\}$.  Then, for each triple $(i,j,k) \in \{0,1\}^3$,
%%   Corollary~\ref{cor:min-abs-prod} implies $\bB_i \times \bB_j \times \bB_k \minabsorbing \bA^3$.
%%   Let
%%   \begin{align*}
%%   R &= \{(0,0,0), (0,0,1), (0,1,0), (1,0,0)\},\\
%%   S &= \{(0,1,1), (1,0,1), (1,1,0), (1,1,1)\}.
%%   \end{align*}
%%   Then $\bR = \<R, \{m\}\>$ and $\bS=\<S, \{m\}\>$ are 
%%   subdirect products of $\bA^3$. %Obviously they are disjoint, $R \cap S = \emptyset$.
%%   Each product $B_i \times B_j \times B_k$ of minimal absorbing subuniverses of $\bA$
%%   intersects either $R$ or $S$, but since $R \cap S = \emptyset$, none of these
%%   products intersects nontrivially with both $R$ and $S$.  In such examples,
%%   although the Rectangularity Theorem correctly predicts that 
%%   $R \cap (B_i \times B_j \times B_k)\neq \emptyset$ implies
%%   $B_i \times B_j \times B_k \subseteq R$, we cannot use this fact to produce a
%%   solution to an instance of $\CSP(\sansS(\bA))$ with constraint relations $\{R, S\}$.
%%   To draw such a conclusion, there must be a single \mas product $\myprod \bB_i$
%%   intersecting nontrivially with all constraint relations.
%% \end{example}

  The example described in the last paragraph is very special.
  For one thing, there is no solution to the instance with
  constraint relations $R$ and $S$.
  We might hope that when an instance \emph{does} have a solution,
  then there should be a solution that passes through a \mas product.
  As we now demonstrate, this is not always the case.
  In fact, Example~\ref{ex:mass-products-3} describes a case in which
  two subdirect powers intersect nontrivially,
  yet each intersects trivially with every \mas product.
\begin{proposition}
  \label{claim:mass-products-2}
There exists an algebra $\bA$ with subdirect powers $\bR$ and $\bS$ 
such that 
$R \cap S \neq \emptyset$ and, for every collection $\{B_i\}$ of \masses,
$R \cap \myprod B_i = \emptyset = S \cap \myprod B_i$.
\end{proposition}

\noindent We prove Proposition~\ref{claim:mass-products-2} by simply producing 
an example that meets the stated conditions.

\begin{example}
  \label{ex:mass-products-3}
Let $\bA = \<\{0,1,2\}, \circ\>$ be an algebra with binary operation $\circ$
given by
\vskip3mm
 \begin{center}
 \begin{tabular}{c|ccc}
      $\circ$ & 0 & 1 & 2 \\
      \hline
      0 & 0 & 1 & 2\\
      1 & 1 & 1 & 0\\
      2 & 2 & 0 & 2
    \end{tabular}
 \end{center}
\vskip3mm
The proper nonempty subuniverses of
$\bA$ are $\{0\}$, $\{1\}$, $\{2\}$, $\{0,1\}$, and $\{0,2\}$.
Note that %$B_1 = \{1\}$ and $B_2=\{2\}$
$\{1\}$ and $\{2\}$ are both minimal absorbing subuniverses with respect to the
term $t(x,y,z,w) = (x \circ y) \circ (z \circ w)$. 
Note also that $\{0\}$ is not an absorbing subuniverse of~$\bA$. For if
$\{0\}\absorbing \bA$, then
$\{0\}\absorbing \{0,1\}$ which is  false, since $\{0,1\}$ is a semilattice with
absorbing element $1$.  

Let $\bA_0 \cong \bA \cong \bA_1$, $R = \{(0,0), (1,1), (2,2)\}$, and
$S = \{(0,0), (1,2), (2,1)\}$.  Then $\bR$ and $\bS$ are subdirect products of
$\bA_0\times \bA_1$ and $R\cap S= \{(0,0)\}$.  There are four minimal absorbing subuniverses of
$\bA_0 \times \bA_1$.  They are
\[
B_{11} = \{1\}\times \{1\} = \{(1,1)\} , \; B_{12} = \{(1, 2)\}, \; B_{21} = \{(2, 1)\},
\; B_{22} = \{(2, 2)\}.
\]
Finally, observe
\[
R\cap B_{ij} = \begin{cases}
  \{(i,j)\}, & i=j,\\
  \emptyset, & i\neq j.
\end{cases}
\qquad 
S\cap B_{ij} = \begin{cases}
  \emptyset, & i=j\\
  \{(i,j)\}, & i\neq j.
\end{cases}
\]
This proves Proposition~\ref{claim:mass-products-2}.
%% \noindent {\bf Computations for the example above.}
%% The following shows
%% $\{1\} \minabsorbing_t \bA$ and $\{2\}\minabsorbing_t \bA$.
%% (Surely there is a smarter way to show this.)
%% \begin{align*}
%%   t(0,1,1,1) &= (0 * 1) * (1*1) = 1 * 1 = 1\\
%%   t(1,0,1,1) &= (1*0) * (1*1) = 1 * 1 = 1\\
%%   t(1,1,0,1) &= (1*1) * (0*1) = 1 * 1 = 1\\
%%   t(1,1,1,0) &= (1*1) * (1*0) = 1 * 1 = 1\\
%%   %
%%   t(2,1,1,1) &= (2 * 1) * (1*1) = 0 * 1 = 1\\
%%   t(1,2,1,1) &= (1*2) * (1*1) = 0 * 1 = 1\\
%%   t(1,1,2,1) &= (1*1) * (2*1) = 1 * 0 = 1\\
%%   t(1,1,1,2) &= (1*1) * (1*2) = 1 * 0 = 1\\
%%   %
%%   t(0,2,2,2) &= (0 * 2) * (2*2) = 2 * 2 = 2\\
%%   t(2,0,2,2) &= (2*0) * (2*2) = 2 * 2 = 2\\
%%   t(2,2,0,2) &= (2*2) * (0*2) = 2 * 2 = 2\\
%%   t(2,2,2,0) &= (2*2) * (2*0) = 2 * 2 = 2\\
%%   %
%%   t(1,2,2,2) &= (1 * 2) * (2*2) = 0 * 2 = 2\\
%%   t(2,1,2,2) &= (2*1) * (2*2) = 0 * 2 = 2\\
%%   t(2,2,1,2) &= (2*2) * (1*2) = 2 * 0 = 2\\
%%   t(2,2,2,1) &= (2*2) * (2*1) = 2 * 0 = 2
%% \end{align*}
%% \vskip5mm

\end{example}

%% ---wjd: 2016.08.11
%% we agreed that the next subsection ``A special case of mass intersection''
%% should be deleted. Although I believe it is correct and possibly useful,
%% it is unnecessary and interrupts the flow of the paper a bit.
%% For now, I'm putting this inside a comment block in case later
%% I want to come back and develop this further.
%% ---
\begin{comment}
\subsection{A special case of mass intersection.}
Example~\ref{ex:mass-products-3} demonstrates that solutions to \csps may or may not instersect
nontrivially with a \mas product.
Despite this, the Rectangularity Theorem is still useful in certain special situations.
In this section is one demonstration of this. (In Section XX we will see another use-case
which might seem more realistic than the toy example we give here.)


Suppose 
$\sA = (\bA_0, \bA_1, \bA_2, \bA_3)$
is a list of finite nonabelian simple algebras, and 
suppose %we have the following subdirect products: %$\bQ$, $\bR$, $\bS$, and $\bT$ satisfy
\begin{align*}
  \bQ &\sdp \bA_0 \times \bA_1 \times \bA_2, \\
  \bR &\sdp \bA_0 \times \bA_1 \times \bA_3,\\
  \bS &\sdp \bA_0 \times \bA_2 \times \bA_3, \\
  \bT &\sdp \bA_1 \times \bA_2 \times \bA_3.
\end{align*}
Assume %% that $Q$, $R$, $S$, and $T$---
the respective universes of these subalgebras have binary projections that satisfy
\begin{align*}
Q_{01} = R_{01}, \quad Q_{02} = S_{02}, \quad Q_{12} &= T_{12},\\
R_{03} = S_{03}, \quad R_{13} &= T_{13},\\
S_{03} &= T_{03}.
\end{align*}
Then we can use the Rectangularity Theorem to prove the next proposition.
\begin{proposition}
  \label{prop:special-case-mass}
  A \csp instance with constraint set
  \[
  \sC = \{((0,1,2), Q), ((0,1,3), R), ((0,2,3), S), ((1,2,3),T)\},
  \]
  where $Q$, $R$, $S$, and $T$ are as above, has a solution.
\end{proposition}

This has a fairly straightforward but rather tedious proof. To simplify
it we introduce some notation and prove a preliminary lemma.
Let $\sigma$ and $\tau$ be disjoint subsets of $\nn$ and let 
$\iota  = \sigma \cup \tau$. % be their disjoint union. 
%% := \{\<0, s\> \mid s\in \sigma\} \cup
%% coproduct $\iota  = \sigma + \tau := \{\<0, s\> \mid s\in \sigma\} \cup
%% \{\<1, t\> \mid t\in \tau\}$.
For $\bx \in \myprod_{\sigma} A_i$ and $\by \in \myprod_{\tau} A_i$,
%% , so $\bx$ is a function from $\sigma$ to $\bigcup_\sigma A_i$ that takes
%% $j \in \sigma$ to $\bx(j)\in A_j$.  
%% Similarly, $\by$ is a function from $\tau$ to $\bigcup_{\tau} A_i$.  
define
$\bx \oplus \by \in \myprod_{\iota} A_i$ as follows:
%% $\<\bx, \by\> \in \myprod_{\iota} A_i$ as follows:
\[
%% \<\bx, \by\>(i) = 
(\bx \oplus \by)(i) = 
\begin{cases}
  \bx(i), & \text{ if $i \in \sigma$},\\
  \by(i), & \text{ if $i \in \tau$}.
\end{cases}
\]
\newcommand{\st}{\ensuremath{\,.\;}}
For $X \subseteq \myprod_\sigma A_i$ and $R\subseteq \myprod_{\nn}A_i$, define
%% R_{\tau}[X]:=\{\by \in \myprod_\tau A_i \mid \exists \bx \in X \st \<\bx, \by\> \in R_\iota\}.
$R_{\tau}[X]:=\{\by \in \myprod_\tau A_i \mid \exists \bx \in X \st \bx \oplus \by \in R_\iota\}$.
For example, if $R\subseteq \myprod_{\nn} A_i$ and if $B_2\subseteq A_2$, then 
\[
%% R_{2'}[B_2] := \{\by \in \myprod_{i\neq 2} A_i \mid \exists b \in B_2 \st \<b, \by\> \in R\};
R_{2'}[B_2] := \{\by \in \myprod_{i\neq 2} A_i \mid \exists b \in B_2 \st b \oplus \by \in R\};
\]
in this case $b\oplus \by = (\by(0), \by(1),\, b,\, \by(3), \dots, \by(n-1))$.
Another example is
\[
R_2[R_{2'}[B_2]] := \{b \in A_2 \mid \exists \by \in R_{2'}[B_2] \st b \oplus \by \in R\}.
\]
Observe that $R_2[R_{2'}[B_2]]$ contains the set 
$B_2$.
%% \footnote{But $R_2[R_{2'}[\cdot]]$ is not a closure operator since it's not idempotent. 
%%   Is this called a \emph{pre-closure operator}?} 

\begin{lemma}
\label{lem:absorbing-adj}
Let $\bA_0, \bA_1$ be finite algebras %% with $\bB_i \minabsorbing \bA_i$ ($i=0,1$)
and suppose $\bR \sdp \bA_0 \times \bA_1$.
\begin{enumerate}
\item If $\bC \absorbing \bA_0$, then $R_1[C] \absorbing A_1$.
\item If $\bD \absorbing \bA_1$, then $R_0[D] \absorbing A_0$.
\end{enumerate}
\end{lemma}
The proof is straightforward and
appears in~\ref{sec:proof-lemma-absorbing-adj}.
We now use Lemma~\ref{lem:absorbing-adj} to prove Proposition~\ref{prop:special-case-mass}.

\begin{proof}[Proof of Prop.~\ref{prop:special-case-mass}]
Let $n = 4$ and $\nn = \{0,1,2,3\}$. As above, we'll use
primes to denote complementation relative to $\nn$;
that is, $0' = \{1,2,3\}$, and $1' = \{0,2,3\}$, etc.

%% Since $\bQ_{01} = \bR_{01}$, each
%% $(q_0, q_1, q_2) \in Q$ can be extended with some $r_3\in A_3$ to obtain 
%% an element $\bx = (q_0, q_1, q_2, r_3)$ satisfying
%% \begin{equation}
%%   \label{eq:18}
%% %% \Proj_{\{0,1,2\}}\bx \in Q \quad \text{ and } \quad \Proj_{\{0,1,3\}}\bx \in R.
%% \Proj_{3'}\bx \in Q \quad \text{ and } \quad \Proj_{2'}\bx \in R.
%% \end{equation}
%% In other words, for each pair
%% $(x_0,x_1)\in Q_{01} = R_{01}$, there exist $x_2 \in A_2$ and $x_3\in A_3$
%% such that $\bx = (x_0, x_1, x_2, x_3)$ satisfies~(\ref{eq:18}).

Fix a \mas $B_0\minabsorbing A_0$.  As
observed in Section~\ref{sec:mass-products},
$Q_1[B_0]\absorbing A_1$, so there exists 
$B_1\minabsorbing A_1$ contained in 
$Q_1[B_0]$. Therefore, by the Rectangularity Theorem, we have
$B_0\times B_1 \subseteq Q_{01} = R_{01}$.
Continuing in this way, since
$Q_2[B_0\times B_1]\absorbing A_2$, there exists $B_2 \minabsorbing A_2$ 
contained in
$Q_2[B_0\times B_1]$, so
the Rectangularity Theorem again asserts that 
$B_0\times B_1 \times B_2 \subseteq Q$.
Similarly, 
$R_3[B_0\times B_1]\absorbing A_3$, so there exists $B_3 \minabsorbing A_3$ 
contained in $R_3[B_0\times B_1]$, so 
$B_0\times B_1 \times B_3 \subseteq R$. Therefore, the set
$B := B_0\times B_1 \times B_2 \times B_3$ satisfies
$B_{3'} \subseteq Q$ and
$B_{2'} \subseteq R$.

Next, observe that $Q_{02} = S_{02}$ and $R_{03} = S_{23}$ imply
$B_0\times B_2 \subseteq S_{02}$ and 
$B_2\times B_3 \subseteq S_{23}$. Thus,
$B_{1'} \subseteq S$. Similarly, 
$Q_{12} = T_{12}$ and $R_{13} = T_{13}$ imply
$B_1\times B_2 \subseteq T_{12}$ and 
$B_2\times B_3 \subseteq T_{23}$. Therefore,
$B_{0'} \subseteq T$.
\end{proof}
\end{comment}
%% \todo{Insert proof for $n=3$ potatoes with one abelian (AF).}
%% \\\\
%% \todo{Insert proof for $n=3$ potatoes with unique minimal absorbing subuniverses.}
%% \\\\
%% \todo{Insert proof for $n=3$ potatoes under hypotheses of $2k$-ary cyclic term.}







%%% Insert heterosynthesis section here
%%%% \input{inputs/pseudo-rectangularity.tex}
\subsection{Algorithm synthesis for heterogeneous problems}
\label{sec:heter-potat}
We now present a new result (Theorem~\ref{thm:fry-pan2})
that, like the Rectangularity Theorem, aims to describe some of the elements that
must belong to certain subdirect products.
The conclusions we draw are weaker than those of
Theorem~\ref{thm:rectangularity}.
However, the hypotheses required here are simpler, and the motivation
is different.

We have in mind subdirect products of ``heterogeneous'' 
families of algebras. What we mean by this is 
described and motivated as follows:
let $\sC_1$, $\dots$, $\sC_m$
be classes of algebras, all of the same signature. Perhaps we are lucky enough
to possess a single algorithm that
proves the algebras in $\bigcup\sC_i$ are jointly
tractable (defined in Section~\ref{sec:inst-size-tract}).
Suppose instead that we are a little less fortunate
and, although we have no such single
algorithm at hand, at least we do happen to know that
the classes $\sC_i$ are \emph{separately tractable}.
By this we mean that for each $i$ we have a 
proof (or algorithm) $\P_i$ witnessing the tractability of
$\CSP(\sC_i)$.
It is natural to ask whether and under what conditions it might be possible to
derive from $\{\P_i : 1\leq i \leq m\}$ a single proof (algorithm)
establishing that the algebras in 
$\bigcup \sC_i$ are \emph{jointly tractable},
so that $\CSP(\bigcup \sC_i)$ is tractable.

The results in this section take a small
step in this direction by considering two special
classes of algebras that are known to be separately tractable,
and demonstrating that they are, in fact, jointly tractable.
We apply this tool in Section~\ref{sec:block-inst} where
we prove tractability of a \csp involving algebras that were
already known to be tractable, but not previously known to be jointly tractable.


The results here involve special term operations called
cube terms and trasitive terms.
A $k$-ary idempotent term $t$ is a \emph{cube term}
if for every
coordinate $i \leq k$, $t$ satisfies an identity of the form
$t(x_1, \dots, x_k) \approx y$, where
$x_1,\dots, x_k \in \{x, y\}$ and $x_i = x$.
%% (See \cite{MR2563736} and \cite{MR2900858}.)
A $k$-ary operation $f$ on a set $A$ is called
\emph{transitive in the $i$-th coordinate} if for every $u, v \in A$,
there exist $a_1,\dots, a_k \in A$ such that $a_i = u$
and $f(a_1 ,\dots, a_n) = v$. Operations
that are transitive in every coordinate are called \emph{transitive}. 

%% The class
%% $\sT = \{\bA : \bA \text{ finite and every subalgebra of $\bA$ has a transitive term op}\}$
%% is closed under homomorphic images ($\sansH$), subalgebras ($\sansS$), and
%% finite products ($\sansP_{\mathrm{fin}}$). %% (See \cite[Prop.~2.1]{MR3374664}.)
%% That is, $\sT$ is a \emph{pseudovariety}.


\begin{Fact}
  \label{fact:cubes-are-trans}
  Let $\bA$ be a finite idempotent algebra and suppose $t$ is a cube term operation on $\bA$.
  Then $t$ is a transitive term operation on $\bA$.
\end{Fact}
\begin{proof}
  Assume $t$ is a $k$-ary cube term operation on $\bA$ and fix $0\leq i<k$.  We will prove that $t$
  is transitive in its $i$-th argument.  Fix $u, v \in A$.
  We want to show there exist $a_0,\dots, a_{k-1} \in A$ such that $a_{i-1} = u$
  and $t(a_0 ,\dots, a_{k-1}) = v$. Since $t$ is a cube term, it satisfies an identity of the form
  % \begin{equation}
  %   \label{eq:2}
  %   t(x_1, \dots, x_k) \approx y
  % \end{equation}
    $t(x_1, \dots, x_k) \approx y$,
  where $(\forall j)(x_j \in \{x, y\})$ and $x_i = x$.
  So we simply substitute $u$ for $x$ and $v$ for $y$ in 
  the argument list in of this identity. %Equation~(\ref{eq:2}).
  Denoting this substition by $t[u/x, v/y]$, we have
  $t[u/x, v/y]= v$, proving that $t$ is transitive.
\end{proof}

\begin{Fact}
  \label{fact:pseudovar}
  The class
  \begin{equation}
    \label{eq:0001}
    \sT = \{\bA : \bA \text{ finite and every subalgebra of $\bA$ has a transitive term op}\}
  \end{equation}
  is closed under the taking of homomorphic images, % ($\sansH$)
  subalgebras,  % ($\sansS$), 
  and finite products. % ($\sansP_{\mathrm{fin}}$).
  %% (See \cite[Prop~2.1]{MR3374664}.)
  That is, $\sT$ is a \emph{pseudovariety}.
\end{Fact}


We also require the following obvious fact about
nontrivial algebras in a Taylor variety. (We call an algebra \emph{nontrivial}
if it has more than one element.)
\begin{Fact}
  \label{fact:nonconstant-terms-exist}
  If $\bA$ and $\bB$ are nontrivial algebras in a Taylor variety $\var{V}$, then for
  some $k>1$ there is a $k$-ary term $t$ in $\var{V}$ such that $t^{\bA}$ and
  $t^{\bB}$ each depends on at least two of its arguments.
\end{Fact}

Finally, we are ready for the main result of this section.

\begin{theorem}
\label{thm:fry-pan2}
  Let $\bA_0, \bA_1, \dots, \bA_{n-1}$ be finite idempotent algebras in a Taylor
  variety and assume there exists a proper nonempty subset $\alpha \subset \nn$ such that
  \begin{itemize}
  \item $\bA_i$ has a cube term for all $i \in \alpha$,
  \item $\bA_j$ has a sink $s_j \in A_j$ for all $j \in \alpha'$;
    let $\bs \in \prod_{\alpha'}A_j$ be a tuple of sinks.
    %% , so $\bs(j) = s_j$ for all $j \in \alpha'$. 
  \end{itemize}
   If $\bR \sdp \prod_{\nn} \bA_i$, then the set $X:=R_\alpha \times \{\bs\}$
  %% if $S := \{\bx \in \prod_{\nn}\bA_i \mid \bx(i) = s_i \text{ for all $i\in \alpha'$}\}$.
  is a subuniverse of $\bR$.
\end{theorem}
\begin{remark}
  %% \begin{enumerate}
  %% \item 
  %%   Technically speaking, the set in the conclusion of the theorem is
  %%   \[
  %%   X = \{\bx \in \prod_{i\in \nn}\bA_i \mid \bx \circ \alpha \in R_\alpha 
  %%   \text{ and } \bx \circ \alpha' = \bs\}.
  %%   \]
  %%   However, modulo a possible rearrangement of factors, this is
  %%   $R_\alpha \times \{\bs\}$.
  %% \item 
  %% \end{enumerate}
To foreshadow applications of
Theorem~\ref{thm:fry-pan2}, imagine we have 
algebras of the type described, and a collection
$\sR$ of subdirect products of these algebras. Suppose also that we have somehow
determined that the intersection of the $\alpha$-projections of these subdirect
products is nonempty, say,
\[\bx_\alpha \in \bigcap_{R \in \sR} R_\alpha.\]
Then the full intersection $\bigcap \sR$ will also be nonempty, since
according to the theorem it must
contain the tuple that is $\bx_\alpha$ on $\alpha$ and $\bs$ off $\alpha$.
\end{remark}

\begin{proof}
  Fix $\bx \in X$, so $\bx\circ \alpha' = \bs$ and $\bx \circ \alpha \in R_\alpha$. 
  We will prove that $\bx \in R$.
  Define $\bA_{\alpha} =\prod_{\alpha}\bA_i$ and $\bA_{\alpha'} =\prod_{\alpha'}\bA_i$, so 
  $R_{\alpha}$ is a subuniverse of $\bA_\alpha$ and $R$ is a subuniverse of
  $\bA_\alpha \times \bA_{\alpha'}$.

  By Fact~\ref{fact:cubes-are-trans}, 
  for each $i \in \alpha$ every subalgebra of $\bA_i$
  has a transitive term operation.  Moreover, the class $\sT$ defined in~(\ref{eq:0001}) 
  is a pseudovariety, so every subalgebra of 
  $\bA_{\alpha}$ has a transitive term operation.  Suppose there are $J$ subalgebras 
  of $\bA_{\alpha}$ and let $\{t_j\mid 0\leq j < J\}$ denote the collection
  of transitive term operations, one for each subalgebra. % of $\bA_\alpha$.
  Then it is not hard to prove that 
  $t := t_0 \star  t_1 \star \cdots \star  t_{J-1}$
  is a transitive term for every subalgebra of $\bA_\alpha$.
  (Recall, $\star$ was defined at the end of Section~\ref{sec:absorption-thm}.)  
  In particular $t$ is transitive for the subalgebra with universe $R_{\alpha}$. 
  %% Fix $N$ and a
  Assume $t$ is $N$-ary.
  
  Fix $j\in \alpha'$.  As usual, $t^{\bA_j}$ denotes the interpretation of 
  $t$ in $\bA_j$. We may assume without loss of generality that $t^{\bA_j}$ 
  depends on its first argument, at position 0. (It must depend on at least one of
  its arguments by idempotence.) 
  Now, since $\bR$ is a subdirect product of $\prod_{\nn}\bA_i$, 
  there exists $\br^{(j)} \in R$ such that $\br^{(j)}(j) = s_j$, the sink in $\bA_j$.  
  Since   $\bx\circ \alpha\in R_\alpha$ and since $t^{\bA_\alpha}$
  is transitive over $\bR_\alpha$, there exist
  $\br_1, \dots, \br_{N-1}$ in $R$ such that 
  \begin{align*}
   \by^{(j)}&:= t^{\bA_{\alpha} \times \bA_{\alpha'}}(\br^{(j)}, \br_1, \dots, \br_{N-1})\\
      &= 
    (t^{\bA_{\alpha}}(\br^{(j)}\circ \alpha, \br_1\circ \alpha, \dots, \br_{N-1}\circ \alpha),
    t^{\bA_{\alpha'}}(\br^{(j)}\circ \alpha', \br_1\circ \alpha', \dots, \br_{N-1}\circ \alpha'))
  \end{align*}
  belongs to $R$  and satisfies
  $\by^{(j)}\circ \alpha = \bx \circ \alpha$ and
  $\by^{(j)}(j) = s_j$.

  (To make the role played by transitivity here
  more transparent, note that we asserted the existence of elements in $R$
  whose ``$\alpha$-segments,'' $\br_1\circ \alpha, \dots, \br_{N-1}\circ \alpha$
  could be plugged in for all but one of the arguments of $t^{\bA_\alpha}$, 
  resulting in a map (unary polynomial) taking $\br^{(j)}\circ \alpha$ to
  $\bx\circ \alpha$. It is the transitivity of $t^{\bA_\alpha}$ over $\bR_\alpha$
  that justifies this assertion.)

  If $\alpha' = \{j\}$, we are done, since $\by^{(j)} = \bx$ in that case.
  If $|\alpha'|>1$, then we repeat the foregoing procedure for each 
  $j \in \alpha'$ and obtain a subset $\{\by^{(j)} : j \in \alpha'\}$ of $R$,
  each member of which agrees with $\bx$ on $\alpha$ and has a sink in some position
  $j\in \alpha'$. 

  Next, choose distinct $j, k \in \alpha'$. Suppose $w$ is a Taylor term for $\var{V}$.
  Then by Fact~\ref{fact:nonconstant-terms-exist} we may assume without loss of
  generality that $w^{\bA_j}$ depends on its $p$-th argument and $w^{\bA_k}$ depends
  on its $q$-th argument, for some $p\neq q$. Consider
  \begin{align*}
    \bz:= w^{\Pi \bA_i}(\by^{(j)}, %% \by^{(j)}, 
    \dots, \by^{(j)}, & \, \by^{(k)}, \by^{(j)}, \dots, \by^{(j)})\\
    %&\uparrow\\
    &\; ^{\widehat{\lfloor}} \, \text{$q$-th argument}
  \end{align*}
  Evidently, $\bz(j) = s_j$, $\bz(k) = s_k$, and 
  $\bz \circ \alpha = \bx \circ \alpha$ by idempotence, since, when restricted to
  indices in $\alpha$, all the input arguments agree and are equal to $\bx \circ \alpha$. 
  If $\alpha' = \{j, k\}$, we are done.  Otherwise, choose
  $\ell \in \alpha'-\{j, k\}$, and again $w^{\bA_\ell}$ depends on at
  least one of its 
  \ifthenelse{\boolean{footnotes}}{%
    arguments,\footnote{It
      clearly doesn't matter on which argument $w^{\bA_\ell}$ depends.}
  }{arguments,}
  say, the $u$-th.  Let
  \begin{align*}
    \bz':= w^{\Pi \bA_i}(\bz, %% \by^{(j)}, 
    \dots, \bz, & \, \by^{(\ell)}, \bz \dots, \bz).\\
    %&\uparrow\\
    &\; ^{\widehat{\lfloor}} \, \text{$u$-th argument}
  \end{align*}
  Then $\bz'$ belongs to $R$, agrees with $\bx$ on $\alpha$, and satisfies
  $\bz'(j) = s_j$, $\bz'(k) = s_k$, and   $\bz'(\ell) = s_\ell$. 
  Continuing in this way until the set $\alpha'$ is exhausted produces
  an element in $R$ that agrees with $\bx$ everywhere. % $\alpha \cup \alpha'$.
  %% , restricted to $\alpha$, agrees with 
  %% $\bx\circ \alpha$, and restricted to $\alpha'$, is equal to $\bs$.
  In other words,  $\bx$ itself belongs to $R$.
\end{proof}

In Section~\ref{sec:block-inst} we apply
Theorem~\ref{thm:fry-pan2} 
in the special case where
``has a cube term''  in the first hypothesis 
is replaced with ``is abelian.'' Let us be explicit.

\begin{corollary}
\label{cor:fry-pan}
  Let $\bA_1, \dots, \bA_{n-1}$ be finite idempotent algebras in a
  locally finite Taylor variety. 
  Suppose there exists $0< k < n-1$ such that 
  \begin{itemize}
  \item $\bA_i$ is abelian for all $i < k$;
  \item $\bA_i$ has a sink $s_i \in A_i$ for all $i \geq k$.
  \end{itemize}
  If $\bR \sdp \prod \bA_i$, then
  $R_{\kk} \times \{s_k\} \times \{s_{k+1}\} \times \cdots \times \{s_{n-1}\}  \subseteq R$.
  %% where $R_{\kk} = \Proj_{\kk}R$.
\end{corollary}
\begin{proof}
  Since $\bA_{\kk} := \prod_{i<k} \bA_i$ is abelian and lives in a locally finite
  Taylor variety, there exists a term $m$ such that $m^{\bA_{\kk}}$ is a \malcev
  operation on $\bA_{\kk}$ (Theorem~\ref{thm:type2cp}).  %% Since we are
  %% working with idempotent terms, we can be sure that for each $i\in \nn$ the
  %% term operation $m^{\bA_i}$ is not constant (so depends on at least one of
  %% its arguments).
  Since a \malcev term is a cube term, the result follows from Theorem~\ref{thm:fry-pan2}.
\end{proof}

\ifthenelse{\boolean{arxiv}}{%
  We include the original direct proof of Corollary~\ref{cor:fry-pan} in the appendix
  Section~\ref{sec:proof-fry-pan-cor}.}{} 







%%%% wjd: adding pagebreak for ``draft mode'' to reduce printing costs
%%%%      To turn off these unnecessary page breaks, set `draft` to false
%%%%      near the top of this file.
















\ifthenelse{\boolean{draft}}{\newpage}{}



%% -------------------------------------------------------------
%% --- wjd (20160727): inserting file
%% \input{inputs/cib4-intro-reductions.tex} 
%%%%% INPUT: PROBLEM INSTANCE REDUCTIONS %%%%%%


\section{Problem Instance Reductions}\label{sec:var-reduc}
In this section we develop some useful notation for taking an instance of 
a \csp and restricting or reducing it in various ways,
%so that we can describe partial solutions to $\sI$, or solutions to some
%alternative or reduced version of $\sI$. 
%That is, we will develop notation for various types of ``partial instances'' that are
%derived from the ``full instance'' $\sI$. Then, in Section XX, we use these tools
%to resolve certain tractability questions. (TODO: we will be more specific about
%what tractability results are proved in Section XX once we write it!)
either by removing variables or by reducing modulo a sequence of congruence relations. 
The utility of these tools will be demonstrated in Section~\ref{sec:csps-comm-idemp}.

Throughout this section, $\bA$ will denote a finite idempotent algebra.
The problem we will focus on is $\CSP(\mathfrak A)$, defined in
Section~\ref{sec:defin-constr-satisf}, and we will be particularly interested
in the special case in which $\mathfrak A = \sansS(\bA)$.

Recall, we denote an $n$-variable instance of $\CSP(\sansS(\bA))$ by
$\sI = \< \nn, \sA, \mathcal C\>$, where $\nn = \{0, 1, \dots, n-1\}$
represents the set of variables,
$\sA = (\bA_0, \bA_1, \dots, \bA_{n-1})$ % \in \sansS(\bA)^n$
is a list of $n$ subalgebras of $\bA$, and
% %$\mathcal C = \{(\sigma_j, R_j) : j\in J\}$ is a finite set of
%% The finite set of constraints of $\sI$ is $\mathcal C = \{(\sigma_j, R_j) : j\in J\}$.
$\sC = ((\sigma_0, R_0), (\sigma_1, R_1), \dots, (\sigma_{J-1}, R_{J-1}))$
is a list of constraints with respective arities $\ar(\sigma_j) = m_j$.
Thus, $R_j\subseteq \prod_{i < m_j} A_{\sigma(i)}$.
%% A \emph{solution} to $\sI$ is a function $f: \nn \to A$ that assigns
%% values in $A$ to variables in $\nn$,  and for all $j \in J$
%% satisfies $f\circ \sigma_j = (f(\sigma_j(0)), \dots, f(\sigma_j(m_j - 1))) \in R_j$.
%% This simply says that for each $j\in J$ the restriction of $f$ to the variables
%% in the $j$-th scope gives a tuple that belongs to the $j$-th constraint relation.
Much of the discussion below refers to an arbitrary constraint
in $\mathcal C$. In such cases it will simplify notation 
to drop subscripts and denote the constraint by $(\sigma, R)$.
%instead of $(\sigma_j, R_j)$.

\subsection{Variable reductions}
\subsubsection{Partial scopes and partial constraints}
Consider the restriction of an
$n$-variable \csp instance $\sI$ to the first $k$ of its variables, for some
$k\leq n$.
To start, we restrict an arbitrary scope $\sigma$ to the first $k$
variables. This results in a new \emph{partial scope} given by the function
\newcommand\kcapsigma{\ensuremath{\kk \, \cap\, \im \sigma}}
$\restr{\sigma}{\sigma^{-1}(\kcapsigma)}$.
Call this the \emph{$k$-partial scope of} $\sigma$ and, 
to simplify the notation, let
\[
\restr{\sigma}{\overleftarrow{\kk}}=\restr{\sigma}{\sigma^{-1}(\kcapsigma)}.
\]
If $\kcapsigma = \emptyset$ then the $k$-partial scope of $\sigma$
is the empty function.
%% The expression (\ref{eq:6}) might seem unnecessarily complicated at first, but
%% this level of precision will be advantageous later.
To make the notation easier to digest, we give a small example below, but first
let's consider how to restrict a constraint to the first $k$ variables.
To obtain the \emph{$k$-partial constraint of} $(\sigma, R)$, we take the $k$-partial
scope of $\sigma$ as the new scope; for the constraint relation we take the
restriction of each tuple in $R$ to its first $p = |\kcapsigma|$ coordinates.  If we let
%% \begin{equation} \label{eq:6}
\[\restr{R}{\overleftarrow{\kk}}=\restr{R}{\sigma^{-1}(\kcapsigma)},\]
%% \end{equation}
then the \emph{$k$-partial constraint} of $(\sigma, R)$ is given by
%% \begin{equation}
%%   \label{eq:3}
%% (\restr{\sigma}{\overleftarrow{\kk}},
%% \restr{R}{\overleftarrow{\kk}}) := 
%% %% \restr{(\sigma, R)}{\sigma^{-1}(\kcapsigma)} := 
%%   (\restr{\sigma}{\sigma^{-1}(\kcapsigma)}, \restr{R}{\sigma^{-1}(\kcapsigma)}).
%% \end{equation}
$(\restr{\sigma}{\overleftarrow{\kk}},
\restr{R}{\overleftarrow{\kk}})$.
The constraint relation $\restr{R}{\overleftarrow{\kk}}$
consists of all $p$-element initial segments of the tuples in $R$.
For example, suppose $\sigma$ is a scope consisting of the variables $2$, $4$, and $7$; that is,
$\sigma$ corresponds to the list $(\sigma(0), \sigma(1),\sigma(2)) = (2,4,7)$.
To find, say, the 5-partial constraint of $(\sigma, R)$, restrict $(\sigma, R)$
to the first $k= 5$ variables of the instance.
We have $\kk = \{0,1,2,3,4\}$ 
  \ifthenelse{\boolean{footnotes}}{%
    and\footnote{Without loss of generality,
      we assume all scope functions are injective, 
      so $\sigma^{-1}$ is well defined on $\im \sigma$.}
  }{and}
\[
\sigma^{-1}(\kcapsigma) = \sigma^{-1}(\{0,1,2,3,4\} \cap \{2,4,7\})
= \sigma^{-1}\{2,4\} = \{0,1\}.
\]
Therefore,
%% $\restr{\sigma}{\sigma^{-1}(\kcapsigma)} = (\sigma(0), \sigma(1)) = (2,4)$,
$\restr{\sigma}{\overleftarrow{\kk}} = (\sigma(0), \sigma(1)) = (2,4)$,
and
$\restr{R}{\overleftarrow{\kk}}$
%% $\restr{R}{\sigma^{-1}(\kcapsigma)}$
consists of the initial pairs of the triples in $R$, that is,
%% $\restr{R}{\sigma^{-1}(\kcapsigma)} = \{(x,y) : (x,y,z) \in R\}$.
$\{(x,y) : (x,y,z) \in R\}$.

\subsubsection{Partial instances}
The \emph{$k$-partial instance} of $\sI$ is the restriction of
$\sI$ to its first $k$ variables.  We will denote this partial instance
by $\sI_{\kk}$. Thus, $\sI_{\kk}$ is the instance with constraint set
$\mathcal{C}_{\kk}$ equal to the set of all $k$-partial constraints of $\sI$.
If we let $\Sol(\sI, \kk)$ denote the set of solutions to $\sI_{\kk}$,
then $f \in  \Sol(\sI, \kk)$ means that for each $j \in J$,
%% $f$ is a solution to this restricted instance, which  means 
\[
\restr{\sigma_j}{\overleftarrow{\kk}}\; \in\;
\restr{R_j}{\overleftarrow{\kk}}. % {\sigma_j^{-1}(\kk\cap \im \sigma_j)}.
\]
We might be tempted to call $\Sol(\sI, \kk)$ a set of ``partial solutions,''
but that's a bit misleading since an $f \in \Sol(\sI, \kk)$ may or may not
extend to a solution to the full instance $\sI$.

\subsection{Quotient reductions}
As usual, let $\bA$ be a finite idempotent algebra, let
$\nn = \{0, 1, \dots, n-1\}$, and let
$\sI = \< \nn, \sA, \mathcal C\>$
denote an $n$-variable instance of $\CSP(\sansS(\bA))$.
As above, we assume
$\sA = (\bA_0, \bA_1, \dots, \bA_{n-1}) \in \sansS(\bA)^n$,
$\sC = ((\sigma_0, R_0), (\sigma_1, R_1), \dots, (\sigma_{J-1}, R_{J-1}))$,
and $\bR_j\sdp \prod_{\mm_j} A_{\sigma(i)}$ for $0\leq j < J$.

\subsubsection{Quotient instances}
Suppose $\theta_i \in \Con (\bA_i)$ for each $0\leq i < n$ and define
\[
\btheta = (\theta_0, \theta_1, \dots, \theta_{n-1}) \in
\Con (\bA_0) \times \Con (\bA_1) \times \cdots \times \Con (\bA_{n-1}).
\]
%% We will find it useful to restrict an $n$-variable \csp
%% instance $I$ to certain products of blocks $B_{ij}$ of $\theta$.
%% Such ``block instances'' are described more precisely below.
%% On the other hand, the concept of a 
%% ``quotient instance'' will also turn out to be useful. We describe
%% this reduction next.
%%
%% Let $I$ be an $n$-variable instance of $\CSP(\bA)$ and suppose
%% $\mathcal C = \{(\sigma_j, R_j): j \in J\}$ is the (finite) set of constraints of $I$.
If $\bx = (x_0, x_1, \dots, x_{n-1}) \in \prod_{\nn} A_i$, then denote
by $\bx/\btheta$ the tuple whose $i$-th component is the $\theta_i$-class of $\bA_i$
that contains $x_i$, so 
\[
\bx/\btheta = (x_0/\theta_0, x_1/\theta_1, \dots, x_{n-1}/\theta_{n-1}) \in
\prod_{i \in \nn} A_i/\theta_i.
\]
We need slightly more general notation than this since our tuples will often 
come from
%% $\prod_{i \in \im \sigma} A_i$
$\prod_{\mm} A_{\sigma(i)}$ for some scope function
$\sigma\colon \mm \to \nn$.  If we view a general tuple $\bx \in \prod_{\nn} A_i$
as a function from $\nn$ to $\bigcup A_i$, and $\btheta$ as a function from
$\nn$ to $\bigcup \Con (\bA_i)$, then we write
$\bx \circ \sigma \in \prod_{\mm} A_{\sigma(i)}$ and
$\btheta \circ \sigma \in \prod_{\mm} \Con (\bA_{\sigma(i)})$ for the corresponding
scope-restricted
\ifthenelse{\boolean{footnotes}}{%
  tuples,\footnote{Here $\circ$ is function composition, not the relational product.}
}{tuples,}
and define
\[
(\bx \circ \sigma)/(\btheta \circ \sigma) =
(x_{\sigma(0)}/\theta_{\sigma(0)}, x_{\sigma(1)}/\theta_{\sigma(1)}, \dots, x_{\sigma(m-1)}/\theta_{\sigma(m-1)})
%% \in  \prod_{i\in \im \sigma} A_i/\theta_i.
\in  \prod_{i\in \mm} A_{\sigma(i)}/\theta_{\sigma(i)}.
\]
%% More generally, if $\bx$ is a tuple in $\myprod_{\sigma}A_i$, then
%% $\bx/\btheta$ denotes the corresponding tuple whose $i$-th entry is
%% $x_{\sigma(i)}/\theta_{\sigma(i)}$.
Given
$\btheta \in \prod_{\nn}\Con (\bA_i)$ and
a constraint $(\sigma, R) \in \mathcal C$ of $\sI$, define the
\emph{quotient constraint} $(\sigma, R/\btheta)$, where
\begin{equation}
  \label{eq:5}
  R/\btheta := \bigl\{\br/(\btheta\circ \sigma)
  %% \in \prod_{\dom(\sigma)} A_{\sigma(i)}/\theta_{\sigma(i)} \mid \br \in R\}.
  \in \prod_{i \in \mm} A_{\sigma(i)}/\theta_{\sigma(i)} \mid \br \in R\bigr\}.
\end{equation}
Define the \emph{quotient instance} $\sI/\btheta$ to be the $n$-variable instance
of $\CSP(\sansH \sansS (\bA))$ with constraint set
\begin{equation}
  \label{eq:100}
  \mathcal{C}/\btheta:=\{(\sigma, R/\btheta): (\sigma, R) \in \mathcal{C}\}.
\end{equation}
Here are a few easily proved facts about quotient instances.
\begin{Fact}
  \label{fact:quotient-sdp}
  If $\btheta \in \prod_{\nn}\Con (\bA_i)$ and if
  $\bR \sdp \prod_{\mm} \bA_{\sigma(i)}$, then $R/\btheta$ defined in (\ref{eq:5})
  %% is a subuniverse of $\prod_{i=0}^{\ar(\sigma)-1} \bA_{\sigma(i)}/\theta_{\sigma(i)}$
  is a subuniverse of $\prod_{\mm}\bA_{\sigma(i)}/\theta_{\sigma(i)}$
  and the corresponding subalgebra is subdirect.
\end{Fact}
%% \begin{proof}  (TODO: insert proof)\end{proof}

\begin{Fact}
  \label{fact:quotient-instance}
  If $\btheta \in \prod_{\nn}\Con (\bA_i)$ and if
  $\sI$ is an $n$-variable instance of $\CSP(\bA)$, then
  the constraint set $\mathcal C/\btheta$ defined in~(\ref{eq:100})
  defines an $n$-variable instance of $\CSP(\sansH \sansS (\bA))$.
\end{Fact}
%% \begin{proof}  (TODO: insert proof)\end{proof}

\begin{Fact}
  \label{fact:soln-quotient}
  If $\bx$ is a solution to $\sI$, then $\bx/\btheta$ is a solution to $\sI/\btheta$.
\end{Fact}
%% \begin{proof}  (TODO: insert proof) \end{proof}
\noindent By Fact~\ref{fact:soln-quotient}, if there is a
quotient instance with no solution, then $\sI$ has no solution. However, the converse is false;
that is, there may be solutions to every proper quotient instance but no solution to the original instance.

\subsubsection{Block instances}
\label{sec:block-inst}
Let $\mathfrak A$ be a collection of finite idempotent algebras.
Let $\sI = \< \nn, \sA, \mathcal C\>$
be an $n$-variable instance of $\CSP(\mathfrak A)$,
where $\sA = (\bA_0, \bA_1, \dots, \bA_{n-1})$ and 
$\mathcal C$ is a finite set of constraints.
If $\btheta \in \prod_{\nn}\Con (\bA_i)$ and  
$\bx = (x_0, x_1, \dots, x_{n-1}) \in \prod_{\nn}A_i$, then by idempotence,
the list of blocks
$\bx/\btheta = (x_0/\theta_0, x_1/\theta_1, \dots, x_{n-1}/\theta_{n-1})$
is actually a list of algebras.
For each constraint $(\sigma, R)\in \mathcal C$, consider
restricting the relation $R$ to the $x_i/\theta_i$-classes in its scope $\sigma$.
In other words, 
replace the constraint $(\sigma, R)$ with the \emph{block constraint}
% $(\sigma, R \cap \prod_{\mm} x_{\sigma(i)}/\theta_{\sigma(i)})$.
$(\sigma, R \cap \Pi_{\sigma} \bx/\btheta)$, where we have defined
\begin{equation}
  \label{eq:theta-block-algebras-notation}
\Pi_{\sigma} \bx/\btheta:= \prod_{i \in \mm} x_{\sigma(i)}/\theta_{\sigma(i)}.
\end{equation}
Finally, let $\sI_{\bx/\btheta} = \< \nn, \bx/\btheta, \sC_{\bx/\btheta}\>$
denote the problem instance of
%% $\CSP(\{x_i/\theta_i : 0\leq i < n\})$
$\CSP(\sansS(\sA))$
specified by the constraint set
\[
\mathcal{C}_{\bx/\btheta} := \bigl\{
% (\sigma, R \cap \prod_{i \in \mm} x_{\sigma(i)}/\theta_{\sigma(i)})
(\sigma, R \cap \Pi_{\sigma}\bx/\btheta)
\mid
(\sigma, R) \in \mathcal{C}\bigr\}.
\]
We call $\sI_{\bx/\btheta}$ the \emph{$\bx/\btheta$-block instance of} $\sI$.
It is obvious that a solution to $\sI_{\bx/\btheta}$ is also a solution to $\sI$.



%% We call $\sI_{\bx}$ the \emph{$\bx$-block instance}.
%% Note that $\sI_\bx$ is an instance of 
The notions ``quotient instance'' and ``block instance'' suggest a
strategy for solving \csps that works in
certain special cases (one of which we will see in Example~\ref{ex:quot-inst}).
First search for a solution $\bx/\btheta$
to the quotient instance $\sI/\btheta$ for some conveniently chosen $\btheta$.
If no quotient solution exists, then the original instance $\sI$ has no solution.
Otherwise, if $\bx/\btheta$ is a solution to $\sI/\btheta$,
then we try to solve the $\bx/\btheta$-block instance of $\sI$.
If we are successful, then the instance $\sI$ has a solution.
%%  since every solution to $\sI_f$ is also a solution to $\sI$.
Otherwise, the strategy is inconclusive and we try again with a different
solution to $\sI/\btheta$.
After we have exhausted all solutions to the quotient instance, if we have still
not found a solution to any of the corresponding block instances,
then $\sI$ has no solution.

This approach is effective as long as we can find a quotient instance
$\sI/\theta$ for which there is a polynomial bound on the number of
quotient solutions.
Although this is not always possible (consider instances involving only simple algebras!),
there are situations in which an appropriate choice of congruences makes
it easy to check that every instance has quotient instance with a very small
number of solutions.  We present an example.

\begin{example}\label{ex:quot-inst}
Let $\bA = \<\{0,1,2,3\}, \cdot \>$, be an algebra with a single binary operation, `$\cdot $', given by the table on the left in Figure~\ref{tab:final-4}.
%% \begin{table}[h!]
\begin{figure}
\centering
 \begin{tabular}{c|cccc}
      $\cdot $ & 0 & 1 & 2 & 3\\
      \hline
      0 & 0 & 0 & 3& 2\\
      1 & 0 & 1 & 3& 2\\
      2 & 3 & 3 & 2 & 1\\
      3 & 2 & 2 & 1 & 3
 \end{tabular}
 \hskip1cm
  \begin{tabular}{c|ccc}
   $\cdot^{\Sq3}$&0&1&2\\
  \hline
  0&0&2&1\\
  1&2&1&0\\
  2&1&0&2
  \end{tabular}
 \hskip1cm
  \begin{tabular}{c|cccc}
      $t$ & 0 & 1 & 2 & 3\\
      \hline
      0 & 0 & 0 & 0& 0\\
      1 & 0 & 1 & 1& 1\\
      2 & 2 & 2 & 2 & 2\\
      3 & 3 & 3 & 3 & 3
    \end{tabular}
  \caption{Operation tables of Example~\ref{ex:quot-inst}:
    for the algebra $\bA$ (left);
    for the quotient $\bA/\Theta \cong \Sq3$ (middle); 
    for $t(x,y) = x\cdot (y\cdot (x\cdot y))$ (right).}
  \label{tab:final-4}
\end{figure}
%% \end{table}
The proper nonempty subuniverses are
$\{0,1\}$, $\{1,2,3\}$, and the singletons.
%% (in case we need it later, it's easy to prove that $\bA$ is absorption-free)
The algebra \bA\ has a single proper nontrivial congruence relation, $\Theta$, with partition $|01|2|3|$. 
The quotient algebra $\bA/\Theta$ is a 3-element Steiner quasigroup, which happens to be abelian.
We denote the latter by \Sq3 and give its binary operation
in the middle table in Figure~\ref{tab:final-4}.
Note that the subalgebra $\{1,2,3\}$ of $\bA$ is also isomorphic to $\Sq3$. 

As Theorem~\ref{thm:type2cp} predicts, the algebra $\bA/\Theta$ has a \malcev term $q(x,y,z)=y\cdot(x\cdot z)$. 
Let $s(x,y) = q(x,y,y) = y\cdot(x\cdot y)$. Then $\bA/\Theta\vDash s(x,y)\approx x$. By iterating the term $s$ on its second variable, we arrive at a term $t(x,y)$ such that, in~\bA, $t(x,t(x,y)) = t(x,y)$. In fact, $t(x,y)=x\cdot(y\cdot(x\cdot y))$. To summarize
\begin{align*}
\bA &\vDash t(x,t(x,y)) \approx t(x,y) \\
\bA/\Theta &\vDash t(x,y) \approx x.
\end{align*}
The table for $t$ appears on the right in Figure~\ref{tab:final-4}.

%%
%% TO CLIFF: I removed these comments.  Feel free to re-insert or address them as you see fit
%% \towjd{We may have to move the defintion of these two binars up here. We could also move this lemma to section 7.2.}
%% \tochb{I added the definitions; this is just a first pass---feel free to edit it or delete it.}
%%
For the next result, we denote
the two-element semilattice
by $\slt=\<\{0,1\},\meet\>$.
%% , and we denote the
%% 3-element Steiner quasigroup by $\Sq3$. The latter is defined by its Cayley table
%% in Figure~\ref{fig:sq3}.

%% \begin{figure}
%% \begin{center}
%%   \begin{tabular}{c|ccc}
%%    $\cdot$&0&1&2\\
%%   \hline
%%   0&0&2&1\\
%%   1&2&1&0\\
%%   2&1&0&2
%%   \end{tabular}
%%   \caption{The Steiner quasigroup $\Sq3$}
%%   \label{fig:sq3}
%% \end{center}
%% \end{figure}

\begin{lemma}\label{lem:sq3s2}
 Let $\mathfrak A = \{\slt,\Sq3\}$. Then $\CSP(\mathfrak A)$ is tractable.
 \end{lemma}


 
 \begin{proof}
 % \towjd{William, I know I've done a horrible job on this argument. We are going
 %   to need to fill in a bunch of details. Maybe by rewriting the corollaries in
 %   section 5. Or maybe we need to introduce $(2,3)$-minimality, or some kind of
 %   preprocessing. Feel free to overhaul the whole thing.}
 % \tochb{Cliff, No problem.  I'll overhaul it and simplify it.
 %   I believe the new lemma (Lemma 0.2 in malcev-by-sink.pdf)
 %   takes care of the whole thing.  We don't even need to mention the
 %   Rectangularity Theorem! I'll commit my latest changes before the overhaul, 
 %   just in case it all goes wrong and we want to resurect the old argument.}
 Let $\sI=\<\nn,\mathcal A, \mathcal C\>$ be an instance of $\CSP(\mathfrak A)$. 
 Recall, the set of solutions to $\sI$ is
 $\Sol(\sC, \nn) = \bigcap_{\sC} R^{\overleftarrow{\sigma}}$.
 We shall apply Corollary~\ref{cor:fry-pan} to establish 
 tractability.  
 We have $\mathcal A = (\bA_0, \bA_1,\dots,\bA_{n-1}) \in \mathfrak A^n$. Let 
 $\alpha=\{\,i : \bA_i \cong \Sq3\,\}$ and $\alpha'=\{\,i : \bA_i \cong \slt\,\}$, so 
 $\nn = \alpha \cup \alpha'$.  
 We may assume without loss of generality that, 
 for each constraint $(\sigma, R)$ appearing in $\mathcal C$, the associated algebra
 $\bR$ is a subdirect product of $\prod_{\mm}\bA_{\sigma(i)}$, where 
 $\bA_{\sigma(i)} \cong \Sq3$ for all $\sigma(i) \in \alpha$, and  
 $\bA_{\sigma(i)} \cong \slt$ for all $\sigma(i) \in \alpha'$.

 % We may as well assume, without loss of generality, that 
 % $\alpha = \{0,1,\dots, k-1\}$ for some $0\leq k<n$, and that 
 % for each constraint $(\sigma, R)$ appearing in $\mathcal C$, the associated algebra
 % $\bR$ is a subdirect product of $\prod_{\mm}\bA_{\sigma(i)}$, where 
 % $\bA_{\sigma(i)} \cong \Sq3$ if $0\leq \sigma(i)<k$ and  
 % $\bA_{\sigma(i)} \cong \slt$ for $k\leq \sigma(i) <n$.
 % and let $R_0, R_1,\dots,R_{m-1}$ be the constraint relations appearing in $\mathcal C$. 

 Let $0$ denote the bottom element of each semilattice.
 For each constraint $(\sigma, R)$,
 let $\restr{R}{\overleftarrow{\alpha}}$ be an abbreviation for 
 $\restr{R}{\sigma^{-1}(\alpha \cap \im \sigma)}$, which is the projection of
 $R$ onto the factors in its scope whose indices lie in $\alpha$.
 Let $\bzero$ denote a tuple of $0$'s of length $|\alpha' \cap \im \sigma|$.
 Then Corollary~\ref{cor:fry-pan} implies that
 for each constraint $(\sigma, R) \in \sC$, we have
 \begin{equation}
   \label{eq:fry-pan}
  \restr{R}{\overleftarrow{\alpha}} \times \{\bzero\} \subseteq R.
 \end{equation}
 Obviously, if $\sI$ has a solution, then the $\alpha$-partial instance
 $\sI_\alpha$ (i.e., the restriction of $\sI$ to the abelian factors) also has a solution.
 Conversely, if $f \in \prod_{\alpha}A_i$ is a solution to $\sI_\alpha$, then 
 (\ref{eq:fry-pan}) implies that $g \in \prod_{\nn} A_i$ defined by
 \[
   g(i) =
   \begin{cases}
     f(i), & \text{if $i \in \alpha$,} \\
     0, & \text{if $i \in \alpha'$}, 
   \end{cases}
 \]
 is a solution to $\sI$.
 We conclude that $\sI$ has a solution if and only if the $\alpha$-partial instance
 %% restricted to the abelian factors
 has a solution.  Since abelian algebras yield tractable \csps, the proof is complete.
\end{proof}

 Now we address the tractability of the example at hand. 
 
\begin{proposition}
  If $\<\bA, \cdot\>$ is the four-element \cib with operation table given in Figure~\ref{tab:final-4},
  then $\CSP(\sansS(\bA))$ is tractable.
\end{proposition}
\begin{proof}
  Let $\sI = \<\nn, \sA, \sC\>$ be an $n$-variable instance of $\CSP(\sansS(\bA))$
  specified by $\sA = (\bA_0, \bA_1, \dots, \bA_{n-1}) \in \sansS(\bA)^n$ and  
  $\mathcal C = ((\sigma_0, R_0), (\sigma_1, R_1), \dots, (\sigma_{J-1}, R_{J-1}))$.
  For each $i \in \nn$, let
  \begin{equation}
    \label{eq:90}
  \theta_i = 
  \begin{cases}
    \Theta, & \text{if $A_i = A$,}\\
    0_{A_i}, & \text{if $A_i \neq A$,}
  \end{cases}
  \end{equation}
  and define $\btheta = (\theta_0, \theta_1, \dots, \theta_{n-1})$.
  The universes $A_i/\theta_i$ of the quotient algebras defined above satisfy
  \begin{equation*}
  A_i/\theta_i = 
  \begin{cases}
    \{\{0,1\}, \{2\}, \{3\}\},  & \text{if $A_i = \{0,1,2,3\}$,}\\
    \{\{1\},\{2\},\{3\}\}, & \text{if $A_i = \{1,2,3\}$,}\\
    \{\{0\},\{1\}\}, & \text{if $A_i = \{0,1\}$.}
  \end{cases}
  \end{equation*}
  In the first two cases $\bA_i/\theta_i$ is isomorphic to $\Sq3$ and in the third case 
  $\bA_i/\theta_i$ is a 2-element meet semilattice.
  (Singleton factors, $\bA_i/\theta_i =\{a\}$, are ignored because all solutions
  must obviously assign the associated variable to the value $a$.)

  Consider the quotient instance $\sI/\btheta$. From the observations in the previous paragraph we see that $\sI/\btheta$ is an instance of $\CSP(\mathfrak A)$, where  $\mathfrak A  = \{\Sq3,\slt\}$. 
  By Lemma~\ref{lem:sq3s2}, $\sI/\btheta$ can be solved in polynomial-time.
  If $\sI/\btheta$ has no solution, then neither does $\sI$.  Otherwise, let
  $f  \in \Sol(\sI/\btheta, \nn)$ be
  a solution to the quotient instance.
    We will use $f$ to construct a solution $g \in \Sol(\sI, \nn)$ to the original
    instance.
  In doing so, we will occasionally map a singleton set to the element it contains; 
  observe that set union is such a map: $\bigcup\{x\} = x$.

  Assume $\nn = \alpha\cup \beta \cup \gamma$ is a disjoint union where
  \[ A_i = 
  \begin{cases}
    \{0,1,2,3\},  & \text{if $i \in \alpha$,}\\
    \{1,2,3\}, & \text{if $i \in \beta$,}\\
    \{0,1\}, & \text{if $i \in \gamma$.}
  \end{cases}
  \]
  
  In other words,
  $\prod_{\nn} A_i = \{0,1,2,3\}^\alpha \times \{1,2,3\}^\beta \times \{0,1\}^\gamma$.
  Then, according to definition (\ref{eq:90}),
  $\theta_i$ is $\Theta$ for $i \in \alpha$ and $0_{A_i}$ for $i\notin \alpha$.
  Thus, the $i$-th entry of $\sA/\btheta$ is
  \[
  A_i/\theta_i = 
  \begin{cases}
    \{\{0,1\}, \{2\}, \{3\}\},  & \text{if $i \in \alpha$,}\\
    \{\{1\},\{2\},\{3\}\}, & \text{if $i \in \beta$,}\\
    \{\{0\},\{1\}\}, & \text{if $i \in \gamma$.}
  \end{cases}
  \]
  The $i$-th value $f(i)$ of the quotient solution must belong to 
  $A_i/\theta_i$.
  %% $\{\{0,1\}, \{2\}, \{3\}\}$ if $0\leq i < k$,
  %% $\{1,2,3\}$ if $k\leq i < \ell$, and $\{0,1\}$ if $\ell \leq i < n$.
  Let $\alpha = \alpha_0 \cup \alpha_1$ be such that 
  $f(i) = \{0,1\}$ when $i \in \alpha_0$ and
  $f(i) \in\{ \{2\}, \{3\}\}$ when $i \in \alpha_1$. Then,
  \begin{equation}
    \label{eq:10}
  f \in \{\{0,1\}\}^{\alpha_0} \times \{\{2\}, \{3\}\}^{\alpha_1} \times
  \{\{1\},\{2\},\{3\}\}^{\beta} \times 
  \{\{0\},\{1\}\}^{\gamma}
  \end{equation}
  Note that $f(i)$ is a singleton for all $i \notin \alpha_0$, and recall that
  $\bigcup \{a\}= a$. Thus, the following defines an element of $\prod_{\nn}A_i$:
  \[
  g(i) = 
  \begin{cases}
    0, &  \text{ if $i \in \alpha_0\cup \gamma$,}\\ 
    \bigcup f(i), &  \text{ if $i\in \alpha_1 \cup \beta$.}
  \end{cases}
  \]
  %% \[g(i) = \begin{cases} 0 &  \text{ if $0\leq i < q$,}\\ Ff(i) &  \text{ if $q\leq i <n$.}\end{cases}\]
  We will prove that $g$ is a solution to the instance $\sI$; that is,
  $g\in \Sol(\sI, \nn)$. Recall, this holds if and only if $g \circ \sigma \in R$
  for each constraint $(\sigma, R) \in \sC$.
  
  Fix an arbitrary constraint $(\sigma, R)\in \sC$, let $S := \im \sigma$ be the set
  of indices in the scope of $R$, and let $S^c := \nn - S$ denote the complement
  of $S$ with respect to $\nn$. Since we assumed $f$ solves $\sI/\btheta$ and satisfies (\ref{eq:10})  
    there must exist $\br\in \prod_{\nn}A_i$ satisfying not only
  $(\sigma, R)$ but also the following: $\br(i) \in \{0,1\}$ for $i \in \alpha_0$ and 
  $\br(i) = \bigcup f(i)$ for $i \notin \alpha_0$. (Otherwise $f$
  wouldn't satisfy the quotient constraint $(\sigma, R/\btheta)$.)

  We now describe other elements satisfying the (arbitrary) constraint $(\sigma, R)$
  that we will use to prove $g$ is a solution.
  By subdirectness of $R$, for each $\ell \in \alpha_0 \cup \gamma$
  there exists $\bx^\ell \in \prod_{\nn}A_i$ satisfying $(\sigma, R)$ and $\bx^\ell(\ell) = 0$.
  Since $\br$ and $\bx^\ell$ satisfy $(\sigma, R)$---that
  is $\br \circ \sigma\in R$ and $\bx^\ell \circ \sigma\in R$---we also have
  $t(\br,  \bx^\ell) \circ \sigma\in R$.
  That is, $t(\br, \bx^\ell)$ satisfies the constraint $(\sigma, R)$.
  Let's compute the entries of
  $t(\br, \bx^\ell)$  using the partition $\nn = \alpha_0 \cup \alpha_1 \cup \beta \cup \gamma$ defined above.
  First, since $\bx^\ell(\ell) = 0$, we have $t(\br, \bx^\ell)(\ell)=0$.  For $i \neq \ell$, we observe the following:
  \begin{itemize}
  \item
    If $i\in \alpha_0 - \{\ell\}$, then $\br(i) \in \{0,1\}$, %% and $\bx^\ell(i) \in \{0,1,2,3\}$, 
    so $t(\br, \bx^\ell)(i)\in\{0,1\}$.
  \item
    If $i\in \alpha_1$, then $\br(i) = \bigcup f(i)\in \{2,3\}$, %% and $\bx^\ell(i) \in \{0,1,2,3\}$, 
    so $t(\br, \bx^\ell)(i)=\bigcup f(i)$.
  \item
    If $i\in \beta$, then $\br(i)= \bigcup f(i)\in \{1,2,3\}$, %% and $\bx^\ell(i) \in A$
    so $t(\br, \bx^\ell)(i) = \bigcup f(i)$.
  \item
    If $i\in \gamma-\{\ell\}$, then $\br(i)= \bigcup f(i)\in \{0,1\}$,  %% and $\bx^\ell(i) \in \{0,1\}$
    so $t(\br, \bx^\ell)(i) \in \{0,1\}$.
  \end{itemize}
  To summarize, for each $\ell \in \alpha_0 \cup \gamma$,
  there exists $t(\br, \bx^\ell) \in \prod_{\nn}A_i$
  satisfying $(\sigma, R)$ and having values
  %% $0$, $1$, or $\bigcup f(i)$ at each index $i$, in particular,
  %%  $t(\br, \bx^\ell)(\ell) = 0$. More precisely,
  $t(\br, \bx^\ell)(i) \in \{0,1\}$ for $i \in \alpha_0\cup \gamma$ and
  \[
  t(\br, \bx^\ell)(i) = 
  \begin{cases}
    0, & i = \ell,\\
    \bigcup f(i), & i\in \alpha_1 \cup \beta.
  \end{cases}
  \]
  %% \[
  %% t(\br, \bx^\ell)(i) \in 
  %% \begin{cases}
  %%   \{0\}, & i = \ell\\
  %%   \{0,1\}, & i\in \alpha_0 - \{\ell\}\\
  %%   \{\bigcup f(i)\}, & i\in \alpha_1 \cup \beta\\
  %%   \{0,1\}, & i\in \gamma - \{\ell\}.
  %% \end{cases}
  %% \]
  %% \{0,1\}^{\alpha_0 - \{\ell\}} \times {0} \times  0$.
  Finally, if we take the product of all members of
  $\{t(\br, \bx^\ell) : \ell \in \alpha_0 \cup \gamma\}$
  with respect to the binary operation of the original algebra $\<A, \cdot\>$,
  then we find
  \[
  t(\br, \bx^{\ell_1}) \cdot t(\br, \bx^{\ell_2}) \cdot \cdots \cdot t(\br, \bx^{\ell_{L}}) = g,
  \]
  where $L = |\alpha_0\cup \gamma|$.
  Therefore, $g \in \Sol((\sigma, R), \nn)$. Since $(\sigma, R)$ was an arbitrary constraint,
  we have proved $g \in \Sol(\sI, \nn)$, as desired.
\end{proof}
\end{example}

%%
%%Least block algorithm. 
%%
\subsection{The least block algorithm}
This section describes an algorithm that was apparently discovered by
``some subset of'' Petar Markovi\'c and Ralph McKenzie.  We first learned about
this by reading Markovi\'c's slides~\cite{Markovic:2011}
from a 2011 talk in Krakow which gives a one-slide description of
the algorithm. Since no proof (or even formal statement of the assumptions) from the 
originators seems to be forthcoming, we include our own version and proof in this section. 

Let \bA\ be a finite idempotent algebra and let $\theta$ be a congruence on \bA. Recall that by 
idempotence, every $\theta$-class of $A$ will be a subalgebra of \bA. Suppose that, as an 
algebra, each $\theta$-class comes from a variety, $E$, possessing a $(k+1)$-ary edge term, 
$e$. Define $s(x,y)= e(y,y,x,x,\dots,x)$. Then according to Definition~\ref{defn:edge-term}, 
\begin{equation*}
E \vDash s(x,y) \approx x.
\end{equation*}
By iterating $s$ on its second variable, we obtain a binary term $t$ such that
\begin{align}
\bA &\vDash t(x,t(x,y)) \approx t(x,y)\label{eqn:ILT} \\
E &\vDash t(x,y) \approx x. \label{eqn:Lz}
\end{align}
Assume, finally, that the induced algebra $\<A/\theta, t^{\bA/\theta}\>$ is a linearly ordered 
meet semilattice. When this occurs, let us call \bA\ an \emph{Edge-by-Chain} (EC) algebra. 
Our objective in this subsection is to prove the following theorem.

\begin{theorem}\label{thm:least-block}
Every finite, idempotent EC algebra is tractable.
\end{theorem}

\begin{proof}
Recall that to call an algebra, \bA, tractable is to assert that the problem $\CSP(\sansS(\bA))$ 
is solvable in polynomial time. Let  \bA\ be finite, idempotent and EC. We continue to 
use the congruence $\theta$ and term $t$ developed above. Let $\sI$ be an instance of the 
problem $\CSP(\sansS(\bA))$. Thus $\sI$ is a triple $\<\sV,\sA,\sC\>$ in which 
$\sA=(\bA_0, \bA_1,\dots,\bA_{n-1})$ is a list of subalgebras of \bA, see Definition~
\ref{def:csp}.  The congruence $\theta$ induces a congruence (which we will continue to call $
\theta$) on each $\bA_i$, and the term $t$ will induce a linearly ordered semilattice structure 
on $A_i/\theta$. In other words, each $\bA_i$ is also an EC algebra.

Therefore, for each $i<n$, the structure of $\<A_i/\theta,t\>$ will be of the form $C_{i,0} < C_{i,
1} < \cdots < C_{i,q_i}$ in which each $C_{i,j}$ is a $\theta$-class of $\bA_i$. We have the 
following useful relationship.
\begin{equation}\label{eqn:almost-meet}
i<n,\, j\leq k,\, u\in C_{i,j},\, v\in C_{i,k} \implies   t(u,v) = u.
\end{equation}
To see this, let $w=t(u,v)$. Since the quotient structure is a semilattice under $t$, $w\in 
C_{i,j}$. Therefore, by \eqref{eqn:ILT}, $t(u,w)=t(u,t(u,v)) = t(u,v) = w$. On the other hand, 
since $u,w \in C_{i,j} \in E$ we have $t(u,w)=u$ by \eqref{eqn:Lz}. Thus $u=w$. 

 Recall that for an index $k\leq n$, the 
$k$th-partial instance of $\sI$ is the instance, $\sI_{\kk}$, obtained by restricting $\sI$ to the first $k$ variables 
(Section~\ref{sec:var-reduc}). 
The set of solutions to $\sI_{\kk}$ is denoted $\Sol(\sI,\kk)$.

For every $k < n$ we recursively define the index $j_k$ by
\begin{equation*}
j_k = \text{least  $j\leq q_k$ such that $\Sol(\sI,\kplus)\cap 
   \prod_{i=1}^k C_{i,j_i} \neq \emptyset$.}
\end{equation*}
If, for some $k$, no such $j$ exists, then $j_k$, $j_{k+1}$,\dots are undefined. The proof 
of the theorem follows from the following two assertions.
\begin{claim}
\hangindent\leftmargini (1)%
 \hskip\labelsep $\sI$ has a solution if and only if $j_{n-1}$ is defined.
\begin{enumerate}\setcounter{enumi}{1}
\item $j_i$ can be computed in polynomial time for all $i<n$.
\end{enumerate}
\end{claim}

One direction of the first claim is immediate: if $j_{n-1}$ is defined then $\Sol(\sI)= \Sol(\sI,\nn)
$ is nonempty. We address the converse. Assume that $\sI$ has solutions. We argue, by 
induction on~$k$, that every $j_k$ is defined. For the base step, choose any $g\in \Sol(\sI)$.  
Since $g(0) \in A_0$, there is some $\ell$ such that $g(0) \in C_{0,\ell}$. Then $j_0$ is 
defined and is at most~$\ell$.

Now assume that $j_0, j_1,\dots,j_{k-1}$ are defined, but $j_k$ is undefined. Let $f\in \Sol(\sI,
\kk)\cap \prod_{i=0}^{k-1} C_{i,j_i}$. Since $j_k$ is undefined, $f$ has no extension to a 
member of $\Sol(\sI,\kplus)$. We shall derive a contradiction. Since $\sI$ has a solution, 
the restricted instance $\sI_{k+1}$ certainly has solutions. Choose any $g\in \Sol(\sI,k+1)$ with $g(k) 
\in C_{k,\ell}$ for the smallest possible~$\ell$. Let $\sC= \bigl((\sigma_0,R_0), 
(\sigma_1,R_1),\dots,(\sigma_{J-1},R_{J-1})\bigr)$ be the list of constraints of $\sI$. 
By assumption, for each $m<J$, $f$ is a solution to $\restr{(\sigma_m,R_m)}{\kk}$ (the 
restriction of $R_m$ to its first $k$ variables). Therefore, there is an extension, $f_m$, of $f$, 
to $k+1$ variables such that $f_m$ is a solution to $\restr{(\sigma_m,R_m)}{\kplus}$. 

Let $m<J$ and set $h_m = t(g,f_m)$. Since both $g$ and $f_m$ are 
solutions to $\restr{(\sigma_m,R_m)}{\kplus}$, and $t$ is a term, $h_m$ is also a solution to $\restr{(\sigma_m,R_m)}{\kplus}$. 
For every $i<k$ we have $f_m(i)= f(i) = f_0(i)$. Thus 
\begin{gather*}
h_m(i) = t\bigl(g(i), f_m(i)\bigr) = t\bigl(g(i),f_0(i)\bigr) = h_0(i), \text{ for $i<k$ and}\\
h_m(k) = t\bigl(g(k),f_m(k)\bigr) = g(k).
\end{gather*}
The latter relation holds by \eqref{eqn:almost-meet} 
and the choice of $g$. It follows that every $h_m$ coincides with $h_0$. Since $h_m$ satisfies 
$R_m$, we have $h_0 \in \Sol(\sI,\kplus)$. This contradicts our assumption that $f$ has no 
extension to a member of $\Sol(\sI,\kplus)$. 

Finally, we must verify the second claim. Assume we have computed $j_i$, for $i<k$. To 
compute $j_k$ we proceed as follows.
\begin{algorithmic}
 \For{$j=0$ to $q_k$}
 \State $\sB \leftarrow (\bC_{0,j_0}, \bC_{1,j_1}, \dots, \bC_{k-1,j_{k-1}},\bC_{k,j})$
 \If{$\sI_{\sB}$ has a solution in $\CSP(\sB)$}
 \Return $j$
 \EndIf
 \EndFor
 \Return \textsc{Failure}
 \end{algorithmic}
 In this algorithm, $\sI_{\sB}$ is the instance of $\CSP(\sB)$ obtained by first restricting $\sI$ to 
 its first $k+1$ variables, and then restricting each constraint relation to $\prod\sB$. Since the 
members of $\sB$ all lie in the edge-term variety $E$, it follows from 
Theorems~\ref{thm:edge-tractable} and~\ref{thm:HSP-tract} that $\CSP(\sB)$ runs in polynomial-time. 
 \end{proof}
 

%%%% wjd: adding pagebreak for ``draft mode'' to reduce printing costs
%%%%      To turn off these unnecessary page breaks, set `draft` to false
%%%%      near the top of this file.
\ifthenelse{\boolean{draft}}{\newpage}{}


%%New Section 8 on CIBs
%%
\section{CSPs of Commutative Idempotent Binars}
\label{sec:csps-comm-idemp}

Taylor terms were defined in Section~\ref{ssec:term-ops}. It is not hard to see that a binary 
term is Taylor if and only if it is idempotent and commutative. This suggests, in light of the 
algebraic \csp-dichotomy conjecture, that we study commutative, idempotent binars (\cib's for 
short), that is, algebras with a single basic binary operation that is commutative and 
idempotent. If the conjecture is true, then every finite \cib should be tractable. 

The associative {\cib}s are precisely the semilattices. Finite semilattices have long been known to be tractable, 
\cite{MR1481313}. The variety of semilattices is \sd-\meet\ and is equationally complete. Every nontrivial semilattice 
must contain a subalgebra isomorphic to the unique two-element semilattice, $\slt=\<\{0,1\},\meet\>$, and conversely, \slt\ generates the variety of all semilattices.

As we shall see, the algebra \slt plays a central role in the structure of {\cib}s. In particular, the omission of \slt implies tractability. 

\begin{theorem}
  \label{thm:omit5-cib}
  If $\bA$ is a finite \cib then the following are equivalent:
  \begin{enumerate}[(1)]
  \item $\bA$ has an edge term.
  \item $\var{V}(\bA)$ is congruence modular.
  \item $\slt \notin \sansH \sansS (\bA)$ 
    \end{enumerate}
\end{theorem}
The proof that (1) implies (2) appears in \cite{MR2563736} and holds for general
finite idempotent algebras. The contrapositive of (2) implies (3) is easy: 
if 
$\slt \in \sansH \sansS (\bA)$, then 
%% $\slt \in \var{V}(\bA)$, then
$\slt^2 \in \var{V}(\bA)$, and the congruence
lattice of $\slt^2$ is not modular.
So it remains to show that (3) implies (1). For this we use the idea of a 
``cube-term blocker'' (\cite{MR2926316}).
  A \emph{cube term blocker} (\ctb) for an algebra $\bA$ is a pair $(D, S)$ of subuniverses
  with the following properties: $\emptyset < D < S\leq A$ and for every term 
  $t(x_0, x_1, \dots, x_{n-1})$ of $\bA$ there is an index $i \in \nn$ such that, 
  for all $\bs = (s_0, s_1, \dots, s_{n-1}) \in S^n$, if $s_i \in D$ 
  then $t(\bs)\in D$.
  %% \[(\forall (s_0, s_1, \dots, s_{n-1}) \in S^n) ((s_i \in D) \longrightarrow t(s_1, \dots, s_n)\in D).\]
%% \end{definition}

\begin{theorem}[\protect{\cite[Thm 2.1]{MR2926316}}]
\label{thm:ctb}
Let $\bA$ be a finite idempotent algebra. Then $\bA$ has an edge term iff
it possesses no cube-term blockers.
\end{theorem}

(The notions of cube term and edge term both originate in~\cite{MR2563736}. In that paper it is proved that a finite algebra has a cube term if and only if it has an edge term.)

\begin{lemma}
\label{lem:cib-ctb}
  A finite \cib $\bA$ has a \ctb if and only if $\slt \in \sansH \sansS (\bA)$.
\end{lemma}
\begin{proof}
  Assume $(D, S)$ is a \ctb for $\bA$. Then there exists $s\in S-D$.  Consider
  $D^+ :=D \cup\{s\}$.  Evidently  $D^+$ is a subuniverse of $\bA$, and 
  if $\bD^+$ denotes the corresponding subalgebra, then 
  $\theta_D := D^2 \cup \{(s,s)\}$ is a congruence of $\bD^+$. It's easy to see that
  $\bD^+/\theta_D \cong \slt$, so $\slt \in \sansH \sansS (\bA)$.
  
  Conversely, if $\slt \in \sansH \sansS (\bA)$, then there exists $\bB \leq \bA$ and
  a surjective homomorphism $h\colon \bB \to \slt$. Let $B_0=h^{-1}(0)$. Then $\emptyset \neq B_0 < B\leq A$, and $(B_0, B)$ is a \ctb for $\bA$.  
  \end{proof}
  
Lemma~\ref{lem:cib-ctb}, along with Theorem~\ref{thm:ctb}, completes the proof of
Theorem~\ref{thm:omit5-cib} by showing (1) is false if and only if (3) is false. 
In fact, something stronger is true. Kearnes has shown that if $\var{V}$ is any variety of {\cib}s that omits \slt, then $\var{V}$ is congruence permutable. 

\begin{corollary}\label{cor:edge-prod}
Let $\bA_0$, $\bA_1$,\dots,$\bA_{n-1}$ be finite {\cib}s satisfying the equivalent conditions in Theorem~\ref{thm:omit5-cib}. Then both $\bA_0\times \cdots\times \bA_{n-1}$ and $\bA_0\times\cdots\times \bA_{n-1} \times \slt$ are tractable.
\end{corollary}

\begin{proof}
By the theorem, $\var{V}(\bA_0)$ and $\var{V}(\bA_1)$ each have an edge term. By Theorem~\ref{thm:robust}, $\sansH\bigl(\var{V}(\bA_0)\circ \var{V}(\bA_1)\bigr)$ has an edge term as well. Since $\bA_0\times \bA_1$ lies in this variety, it possesses that same edge term. Iterating this process, the algebra $\bB=\bA_0 \times \cdots \times \bA_{n-1}$ has an edge term, hence is tractable by Theorem~\ref{thm:edge-tractable}.

Let $t(x_1,\dots,x_{k+1})$ be an edge term for $\bB$. Then according to the first identity in Definition~\ref{defn:edge-term}, $\var{V}(\bB)$ is a strongly irregular variety. Consequently, the algebra $\bB\times \slt$ is a P\l onka sum of two copies of \bB. As a P\l onka sum of tractable algebras, Theorem~4.1 of~\cite{MR3350338} implies that $\bB\times \slt$ is tractable as well.
\end{proof}

It follows from the corollary and Theorem~\ref{thm:HSP-tract} that every finite member of $\var{V}(\bA_0\times\bA_1\times\cdots\times \bA_{n-1}\times \slt)$ is at least locally tractable. 

As a counterpoint to Theorem~\ref{thm:omit5-cib}, we mention the following. The proof requires some basic knowledge of tame congruence theory.

\begin{theorem}\label{thm:no-abelians}
Let \bA\ be a finite \cib\ and suppose that $\sansH\sansS(\bA)$ contains no nontrivial abelian algebras. Then $\var{V}(\bA)$ is \sd-\meet. Consequently, \bA\ is a tractable algebra.
\end{theorem}

\begin{proof}
Since every \cib\ has a Taylor term, and since $\sansH\sansS(\bA)$ has no abelian algebras, $\sansS(\bA)$ omits types $\{1,2\}$. Then by \cite[Cor~2.2]{Freese:2009}, $\var{V}(\bA)$ omits types $\{1,2\}$. But then, by \cite[Thm~9.10]{HM:1988}, $\var{V}(\bA)$ is \sd-\meet.
\end{proof}

In a congruence-permutable variety, an algebra is abelian if and only if it is polynomially equivalent to a faithful, unital module over a ring. See~\cite[Thm~7.35]{MR2839398} for a full discussion. For example, consider the ring $\mathbb Z_3$ as a module over itself. Define the binary polynomial $x\cdot y = 2x+2y$ on this module. The table for this operation is given at the left of Figure~\ref{fig:cib3}. Conversely, we can retrieve the module operations from `$\cdot$' by $x+y = 0\cdot(x\cdot y)$ and $2x=0\cdot x$. Consequently, the commutative idempotent binar $\Sq3$ is abelian. 

In light of Theorem~\ref{thm:type2cp}, every finite, abelian \cib\ will be of this form. We make the following observation. 

\begin{proposition}\label{prop:cib_ab_odd}
A finite, abelian \cib\ has odd order.
\end{proposition}

\begin{proof}
Let \bA\ be a finite, abelian \cib. By Theorem~\ref{thm:type2cp} (since every \cib is Taylor), \bA\ is  polynomially equivalent to a faithful, unital module $M$ over a ring $R$.  The basic operation of $\bA$ must be a polynomial of $M$. Thus there are $r,s\in R$ and $b\in M$ such that $x\cdot y = rx + sy +b$. Let 0 denote the zero element of $M$. Then $0=0\cdot 0 = r0 +s0 +b =b$. The condition $x\cdot y = y \cdot x$ implies  $rx+sy = ry+sx$, so (by taking $y=0$ and since $M$ is faithful), $r=s$. Finally by idempotence, $2r=1$. 

Now, if $A$, hence $M$, has even cardinality, there is an element $a\in A$, $a\neq 0$, of additive order~2. Thus
\begin{equation*}
a= a\cdot a = 2ra = 0
\end{equation*}
which is a contradiction.
\end{proof}

In an effort to demonstrate the utility of the techniques developed in this paper, we shall now show that every \cib of cardinality at most~4 is tractable. Of course, it is known that the algebraic dichotomy conjecture holds for every idempotent algebra of cardinality at most~3~\cite{MR2212000, MR521057}. However, our arguments are relatively short and may indicate why {\cib}s may be more manageable than arbitrary algebras.

For the remainder of this section, let \bA\ be a \cib with universe $\nn$. We shall consider the various possibilities for $n$ and \bA\ (up to isomorphism), and in each case show that it is tractable. In order to make it clear that our inventory is complete, a few of the arguments and computations are postponed until the end.

\casespec{$\bm{n=2}$} It is easy to see that there is a unique 2-element \cib, namely \slt. Idempotence determines the diagonal entries in the table, and by commutativity, the remaining two entries must be equal. The choices $0$ and $1$ for the off-diagonal yield a meet-semilattice and a join-semilattice respectively. As we remarked above, every finite semilattice is tractable, and, in fact, $\Sl$, the variety of semilattices, is \sd-\meet. 

\casespec{$\bm{n=3}$, \bA\ not simple} There is a congruence, $\theta$, on \bA\ with $0_A < \theta < 1_A$. It is important to note that by idempotence, every congruence class is a subalgebra. Based purely on cardinality concerns, $\bA/\theta$ has cardinality~2, one $\theta$ class has 2 elements, the other has~1. From the uniqueness of the 2-element algebra, $\bA/\theta$ and each of the $\theta$-classes are semilattices. Consequently, $\bA \in \Sl \circ \Sl$. By Theorem~\ref{thm:robust}, $\Sl\circ \Sl$ is \sd-\meet, so by Theorem~\ref{thm:sdm-tractable}, \bA\ is tractable. 

\begin{figure}
\centering
\begin{tabular}{ccc}
  \begin{tabular}{c|ccc}
   $\cdot$&0&1&2\\
  \hline
  0&0&2&1\\
  1&2&1&0\\
  2&1&0&2
  \end{tabular} &
  %
  \begin{tabular}{c|ccc}
  $\cdot$&0&1&2\\
  \hline
  0&0&0&1\\
  1&0&1&2\\
  2&1&2&2
  \end{tabular} &
  %
  \begin{tabular}{c|ccc}
  $\cdot$&0&1&2\\
  \hline
  0&0&0&2\\
  1&0&1&1\\
  2&2&1&2
  \end{tabular} \\
  %
  \vrule height 12pt width 0pt $\Sq3$ & $\bT_1$ & $\bT_2$
 \end{tabular}
 \caption{The simple {\cib}s of cardinality 3}\label{fig:cib3}
 \end{figure}

\casespec{$\bm{n=3}$, \bA\ simple} On the other hand, suppose that \bA\ is simple. If \bA\ has no proper nontrivial subalgebras, then in \bA, $x\neq y \implies x\cdot y \notin \{x,y\}$. It follows that \bA\ must be the 3-element Steiner quasigroup, $\Sq3$, of Figure~\ref{fig:cib3}. By Corollary~\ref{cor:edge-prod}, \bA\ is tractable. 

Finally, if \bA\ has a proper nontrivial subalgebra, that subalgebra must be isomorphic to \slt. Thus no nontrivial subalgebra of \bA\ is abelian. By Theorem~\ref{thm:no-abelians}, \bA\ generates an \sdm variety, so is tractable by Theorem~\ref{thm:sdm-tractable}. For future reference, there are 2 algebras meeting the description in this paragraph, $\bT_1$ and $\bT_2$. (This is easy to check by hand.) Their tables are given in Figure~\ref{fig:cib3}.

\casespec{$\bm{n=4}$, \bA\ not simple} Let $\theta$ be a maximal congruence of \bA. Then $\bA/\theta$ is simple, so from the previous few paragraphs, $\bA/\theta$ is isomorphic to one of $\bT_1$, $\bT_2$, $\Sq3$, or $\slt$. If $\bA/\theta \cong \bT_i$, then the $\theta$-classes must have size $1,1,2$. Consequently, each $\theta$-class is a semilattice, so $\bA\in \Sl \circ \var{V}(\bT_i)$ which is \sdm by the computations above and Theorem~\ref{thm:robust}. Thus $\bA$ is tractable.

The next case to consider is $\bA/\theta \cong \Sq3$. Without loss of generality, assume that~$\theta$ partitions the universe as $|01|2|3|$, and further that $0\cdot 1=0$. From this data we deduce that the operation table for \bA\ must be
\begin{equation*}
\begin{tabular}{c|cccc}
      $\cdot $ & 0 & 1 & 2 & 3\\
      \hline
      0 & 0 & 0 & 3& 2\\
      1 & 0 & 1 & 3& 2\\
      2 & 3 & 3 & 2 & $a$\\
      3 & 2 & 2 & $a$ & 3
 \end{tabular}\qquad \text{with $a\in \{0,1\}$.}
\end{equation*}
If $a=0$, then the only way to obtain 1 as a product is $1\cdot1$. In that case there is a homomorphism onto \slt with kernel $\psi=|023|1|$. We have $\theta \cap \psi = 0_A$, so $\bA$ is a subdirect product of $\Sq3 \times \slt$. Therefore, by \ref{cor:edge-prod}, \bA\ is tractable. 

More problematic is the case $a=1$. We established tractability of that algebra in Example~\ref{ex:quot-inst}.

Finally, suppose that $\bA/\theta \cong \slt$. The possible sizes of the $\theta$-classes are 3,1 or 2,2. If neither class is isomorphic to \Sq3 then the classes both lie in an \sd-\meet\ variety, so by 
Theorem~\ref{thm:robust}, \bA\ too lies in an \sd-\meet variety, and is tractable. On the other hand, suppose that one of the classes is isomorphic to \Sq3, (there are 7 such algebras). Then \bA\ is an EC algebra (with $t(x,y)=y\cdot(x\cdot y)$) and we can apply Theorem~\ref{thm:least-block}
to establish that \bA\ is tractable. 

  
\casespec{$\bm{n=4}$, \bA\ simple} By Theorem~\ref{thm:no-abelians}, if $\sansH\sansS(\bA)$ contains no nontrivial abelian algebra, then \bA\ is tractable. Thus, we may as well assume that $\sansH\sansS(\bA)$ contains an abelian algebra. By Proposition~\ref{prop:cib_ab_odd}, \bA\ itself is nonabelian. Consequently (since \bA\ is simple) \bA\ must have a subalgebra isomorphic to \Sq3. Without loss of generality, let us assume that $\{1,2,3\}$ forms this subalgebra.

Similarly, by Corollary~\ref{cor:edge-prod}, we can assume that $\sansH\sansS(\bA)$ contains a copy of \slt. Examining the 2- and 3-element \cibs, it is easy to see that if $\slt\in \sansH\sansS(\bA)$, then already $\slt\in\sansS(\bA)$. By the symmetry of \Sq3, and the fact that it contains no semilattice, we can assume that $\{0,1\}$ forms the semilattice.


 Thus, the table for \bA\ must have one of the following two forms.
\begin{equation*}
\begin{tabular}{c|cccc}
$\cdot$&0&1&2&3\\
\hline
0&0&0&$u_2$&$u_3$\\
1&0&1&3&2\\
2&$u_2$&3&2&1\\
3&$u_3$&2&1&3
\end{tabular} 
\qquad
\begin{tabular}{c|cccc}
$\cdot$&0&1&2&3\\
\hline
0&0&1&$v_2$&$v_3$\\
1&1&1&3&2\\
2&$v_2$&3&2&1\\
3&$v_3$&2&1&3
\end{tabular} 
\end{equation*}
By checking the 32 possible tables, either via the universal algebra calculator~\cite{UAcalc} or directly using Freese's algorithm~\cite{Freese2008}, one determines that there are 7 pairwise nonisomorphic algebras of one of these two forms. Interestingly, every simple algebra of the second form is isomorphic to one of the first form. Thus we will use the first form for all the candidates. These 7 algebras are indicated in Figure~\ref{fig:simple7}.

\begin{figure}
\centering
\begin{tabular}{l|cc|l}
&$u_2$&$u_3$&Proper Nontrivial Subalgebras\\
\hline
$\bA_0$&0&1&$\{0,1\},\, \{0,2\},\, \{1,2,3\}$\\
$\bA_1$&1&1&$\{0,1\},\, \{1,2,3\}$\\
$\bA_2$&1&2&$\{0,1\},\, \{1,2,3\}$\\
$\bA_3$&0&3&$\{0,1\},\, \{0,2\},\, \{0,3\},\, \{1,2,3\}$\\
$\bA_4$&1&3&$\{0,1\},\, \{0,3\},\, \{1,2,3\}$\\
$\bA_5$&2&2&$\{0,1\},\, \{0,2\},\, \{0,3\}\, \{1,2,3\}$\\
$\bA_6$&2&3&$\{0,1\},\, \{0,2\},\, \{0,3\}\, \{1,2,3\}$
\end{tabular}
 \caption{7 simple \cibs of size 4.}\label{fig:simple7}
 \end{figure}
 
Our intent is to apply the rectangularity theorem to establish the tractability of these~7 algebras. For this, we need to determine the minimal absorbing subuniverses of each subalgebra of \bA. First observe that if $\{a,b\}$ forms a copy of \slt\ with $a\cdot b =a$, then $\{a\} \absorbing_t \{a,b\}$, with $t(x,y)=x\cdot y$, while $\{b\}$ is not absorbing since $a$ is a sink (Lemma~\ref{lem:sink}). Second, by Lemma~\ref{lem:abelian-AF}, \Sq3 is absorption-free, which is to say, it is its own minimal absorbing subalgebra. 

 Now let $B$ be a proper minimal absorbing subalgebras of $\bA_i$, for $i<7$. Then $B\cap \{1,2,3\}$ is absorbing in $\{1,2,3\}$ (Lemma~\ref{lem:restriction}). As $\{1,2,3\}\cong \Sq3$ is absorption free and $B$ is a proper subalgebra of \bA, we must have $B=\{1,2,3\}$ or $B=\{0\}$. However the first of these is impossible because $\{1,2,3\} \cap \{0,1\} = \{1\}$, which is not absorbing in $\{0,1\}$ since $0$ is a sink. Thus the only possible absorbing subalgebra of \bA\ is $\{0\}$. 
 
 For $i\leq 2$, $\{0\}$ is indeed absorbing in $\bA_i$ with absorbing term $t(x,y)=\bigl(x(xy)\bigr)\bigl(y(xy)\bigr)$. However, observe that if $i>2$ then according to Figure~\ref{fig:simple7} either $u_2=2$ or $u_3=3$. If $u_2=2$ then $2$ is a sink for the subuniverse $\{0,2\}$. In that case $\{0\}= B\cap \{0,2\}$ is not absorbing in $\{0,2\}$ contradicting the fact that $B$ is absorbing in \bA. A similar argument works for $u_3=3$ and $\{0,3\}$. Thus $\bA_i$ is absorption free for $i>2$. 
 
 We summarize these computations in the following table.
 \begin{center}
 \begin{tabular}{lc}
 Algebra&Minimal Absorbing Subalgebra\\
 $\slt=\{0,1\}$ & $\{0\}$ \\
 $\{1,2,3\}$ & $\{1,2,3\}$ \\
 $\bA_0, \bA_1, \bA_2$ & $\{0\}$\\ 
 $\bA_3,\dots,\bA_6$ & $A$
 \end{tabular}
 \end{center}
 The crucial point here is that every member of $\sansS(\bA_i)$ has a unique minimal absorbing subalgebra. 
 
We now proceed to argue that each algebra from Figure~\ref{fig:simple7} is tractable. Following Definition~\ref{def:csp}, let $\sI=\<\sV, \sA, \sS, \sR\>$ be an instance of $\CSP(\bA_i)$, for some $i<7$. We have $\sA=(\bC_0, \bC_1,\dots,\bC_{n-1})$, in which each $\bC_i$ is a subalgebra of $\bA_i$, and $\sR=(R_0,\dots,R_{J-1})$ is a list of subdirect products of the various $\bC_k$'s. We must show that there is an algorithm that determines, in polynomial time, whether this instance has a solution. For this we will use Corollary~\ref{cor:RT-cor-gen}.

For each $j<J$, let the relation $\bar R_j$ be obtained from $R_j$ as follows. First, for any distinct variables $\ell,k \in \im(\sigma_j)$, if $\etaR^j_\ell = \etaR^j_k$,  remove the $k$-th variable from the scope of $R_j$. Then, define $\bar R_j = R_j \times \prod_{\ell \notin \im\sigma_j} C_i$. Let $\bar \sI$ denote the modified instance with $(\bar R_j : j<J)$ replacing $\sR$ and every scope equal to $\sV$. Then the solution set of $\bar \sI$ is identical to the solution set of the original instance $\sI$, and, in fact, is equal to $\bigcap_j \bar R_j$. 

Dropping variable $k$ does not affect the solution set: if $\etaR^\ell = \etaR^k$ then there is an isomorphism $h\colon \bC_\ell \to \bC_k$ such that $h\circ \Proj_\ell = \Proj_k$. Thus, the $k$-th coordinate of any tuple in $R$ can be recovered by applying $h$ to the $\ell$-th coordinate. The second step in the transformation clearly does not lose any of the original solutions. Furthermore, it is easy to see that it does not create any new pairs $\ell,k$ with $\etaR_\ell = \etaR_k$. We stress that the creation of $\bar R$ from $R$ is not part of the algorithm, so it is not necessary to execute in polynomial-time. It exists only for the purpose of this proof. 

We now consider the 5 conditions in Corollary~\ref{cor:RT-cor-gen}, applied to $\{\,\bar R_j : j<J\,\}$. Set $\alpha=\{\,\ell<n : \bC_\ell \cong \Sq3\,\}$. Since \Sq3 is the only possible abelian subalgebra of $\bA_i$, the first two conditions will be satisfied. Our preproccing ensures that condition~3 holds as well. We argue momentarily that condition~\ref{item:RT-cor-gen-4} holds. Assuming for the moment that it does, then according to the corollary and our computations above, the following statements are equivalent.
\begin{itemize}
\item $\sI$ has a solution;
\item $\bar \sI$ has a solution;
\item $\bigcap_j \bar R_j \neq \emptyset$;
\item $\bigcap_j \Proj_\alpha \bar R_j \neq \emptyset$;
\item $\sI_\alpha$ has a solution.
\end{itemize}
However, $\sI_\alpha$ is an instance of $\CSP(\Sq3)$, which, as an abelian algebra, is known to be tractable. Thus by running the algorithm on this partial instance, we can determine the existence of a solution to the original instance, $\sI$.

Thus, we are left with the task of verifying condition~\ref{item:RT-cor-gen-4}, that is, show that every $\bar R_j$ intersects $\prod B_\ell$. Recall that $B_\ell$ is a minimal absorbing subalgebra of $\bC_\ell$. We have two cases to consider. First, suppose that $i<3$. In that case the minimal absorbing subalgebra of $\bA_i$ is $\{0\}$, which is, in fact, a sink of $\bA_i$. Therefore, for every $\ell \in \alpha$, $B_\ell = \{1,2,3\}= C_\ell$, while for $\ell \notin \alpha$, $B_\ell = \{0\}$. Let $\bar R$ denote any $\bar R_j$. Since $\bar R$ is subdirect, for every $\ell \in \alpha'$ there is a tuple $\br^\ell \in R$ with $r^\ell_\ell = 0$. Let $\br$ be the product of all the $\br^\ell$'s (in any order). Then, since $0$ is a sink, $\br_\ell = 0\in B_\ell$ for every $\ell \in \alpha'$, while for $\ell \in \alpha$, $\br_\ell \in \bC_\ell = B_\ell$. Thus the condition is satisfied.

The argument when $i\geq 3$ is not very different. In this case, let $\beta = \{\,\ell <n : \bC_\ell = \slt\,\}$. Once again, for $\ell \in \beta$ let $\br^\ell$ be a tuple with $\br^\ell_\ell = 0$, and let $\br$ be the product. Then for $\ell \in \beta$, $\br_\ell = 0 \in B_\ell$, while, for $\ell \notin \beta$, $\br_\ell \in B_\ell$ since $B_\ell = \{1,2,3\} = \bC_\ell$ (if $\ell \in \alpha$) and $B_\ell = A_i$ otherwise. 



%%%% wjd: adding pagebreak for ``draft mode'' to reduce printing costs
%%%%      To turn off these unnecessary page breaks, set `draft` to false
%%%%      near the top of this file.
\ifthenelse{\boolean{draft}}{\newpage}{}


%-------------------------------------------------------------------
%~~~~~~~~~~~~~~           APPENDIX          ~~~~~~~~~~~~~~~~~~~~~~~~
%-------------------------------------------------------------------

\appendix

\ifthenelse{\boolean{arxiv}}{
  %%%% omitting this example from the journal version
  %%%% (it will only appear arxiv and extended versions of the paper).
\section{Nonrectangularity of abelian algebras}

The example in this section reveals why the rectangularity theorem
cannot be generalized to products of abelian algebras.
Before examining the example, we state a lemma that will be useful
when discussing the example. The proof is straightforward.
\begin{lemma}
\label{lem:abs2}
Let $\bA_1, \dots, \bA_n$ be finite simple algebras in a Taylor
variety and suppose
\begin{itemize}
\item each $\bA_i$ is absorption-free, 
\item $\bR \sdp \bA_1 \times \cdots \times \bA_n$,
\item $\etaR_i \neq \etaR_j$ for all $i\neq j$, and 
\item $\mu \subseteq \nn $ is minimal among the sets in
$\{\sigma \subseteq \nn  \mid \bigwedge_{\sigma} \etaR_i = 0_R\}$.
\end{itemize}
Then, $|\mu|>1$ and $\Proj_{\tau} \bR = \Pi_\tau \bA_i$ for every set
$\tau \subseteq \nn $ with $1< |\tau|\leq |\mu|$.
\end{lemma}


\begin{example}
  Let $\bA$ be the algebra $\<\{0,1\}, \{f\}\>$, with universe $\{0,1\}$,
  and a single basic operation given by the ternary function $f(x,y,z) = x+y+z$
  where addition is modulo 2.  This algebra is clearly simple and has two proper
  subuniverses, $\{0\}$ and $\{1\}$, neither of which is absorbing, so $\bA$ is
  absorption-free.  Let $\bR \sdp \bA \times \bA \times \bA$ be the subdirect
  power of $\bA$ with universe 
  \[ R = \{(x,y,z)\in A^3: x+y+z=0 \text{ (mod $2$)}\} = \{(0,0,0), (1,1,0), (1,0,1), (0,1,1) \}.\] 
  It's convenient to give short names to the elements of $R$.  
  Let us use the integers that they (as binary tuples) represent.
  That is, $0 = (0,0,0)$, $3 = (1,1,0)$,  $5 = (1,0,1)$,  and 
  $6 = (0,1,1)$,   
  so $R = \{0, 3, 5, 6\}$.
  Let $\etaR_i = \ker(\bR \onto \bA_i)$, and identify each
  congruence with the associated partition of the set $R$. Then,
  %% \[\etaR_1 = |0,6|3,5|, \quad \etaR_2 = |0,5|3,6|, \quad  \etaR_3 = |0,3|5,6|, \]
  $\etaR_1 = |0,6|3,5|; \,  \etaR_2 = |0,5|3,6|; \,  \etaR_3 = |0,3|5,6|$,
  so $\etaR_i \meet \etaR_j = 0_R$ and $\etaR_i \join \etaR_j = 1_R$.
  In fact, the three projection kernels are the only nontrivial congruence
  relations of $\bR$.

  \tikzstyle{lat} = [circle,draw,inner sep=0.8pt]
  \begin{center}
  \begin{tikzpicture}[scale=1.2]
    \node[lat] (bot) at (0,0) {};
    \node[lat] (top) at (0,2) {};
    \node[lat] (a) at (-1,1) {};
    \node[lat] (b) at (0,1) {};
    \node[lat] (c) at (1,1) {};
    \draw[semithick] (bot) -- (a) -- (top) -- (b) -- (bot) -- (c) -- (top);
    \draw (top) node [right]{$1_R$};
    \draw (bot) node [right]{$0_R$};
    \draw (a) node [left]{$\etaR_1$};
    \draw (b) node [right]{$\etaR_2$};
    \draw (c) node [right]{$\etaR_3$};
    \draw (-2.5,1) node {$\Con(\bR) = $};
  \end{tikzpicture}
  \end{center}

  Each projection of $\bR$ onto 2 coordinates of $\bA^3$ is ``linked;'' 
  these binary projections are all readily seen to be
  $\{(0,0), (0,1), (1,0), (1,1)\} =  A \times A$ in this case. 
  Lemma~\ref{lem:abs2} tells us that this must be so.  For the set 
  $\{S \subseteq \nn  \mid \bigwedge_{i\in S} \etaR_i = 0_R\}$ that appears in the
  lemma is, in this example,
  $\sS = \{\{1,2\}, \{1,3\}, \{2,3\}, \{1,2,3\}\}$, and the projection of
  $R$ onto the coordinates in each minimal  set in $\sS$ must
  equal $A \times A$ by the lemma. 
  Now let $\bS \sdp \bA \times \bA \times \bA$ be the subdirect power of $\bA$
  with universe 
  \begin{align*}
    S &= \{(x,y,z)\in A^3: x+y+z=1 \text{ (mod $2$)}\}\\
    &= \{(1,0,0), (0,1,0), (0,0,1), (1,1,1) \}\\
    &= \{1, 2, 4, 7\}.
  \end{align*}
  All of the facts observed above about $\bR$ are also true of $\bS$.  In
  particular, $\Proj_{\{i,j\}} S = A_i \times A_j$ for 
  each pair $i\neq j$ in $\{1,2,3\}$.
  If both $\bR$ and $\bS$ belong to the set of relations 
  (or ``constraints'') of a single $\CSP(\sansS(\bA))$ instance, then the instance
  has no solution since $R \cap S = \emptyset$, despite the fact that  
  $\Proj_{\{i,j\}}\bR = \Proj_{\{i,j\}}\bS$  for each pair $i\neq j$ in
  $\{1,2,3\}$. To put it another way, $\bR$ and $\bS$  
  witness a failure of the \emph{2-intersection property}. The ``potato
  diagram'' of $\bA^3$ below depicts the elements of
  $R$ and $S$ as colored lines (with colors chosen to emphasize equality of
  the projections onto $\{1,2\}$). 

  \begin{center}
  \begin{figure}
    \caption{potatoes}
  \begin{tikzpicture}[scale=1]
    \draw (0,.5) ellipse (4mm and 12mm);
    \draw (2,.5) ellipse (4mm and 12mm);
    \draw (4,.5) ellipse (4mm and 12mm);
    \node[lat] (00) at (0,0) {};
    \node[lat] (01) at (0,1) {};
    \node[lat] (10) at (2,0) {};
    \node[lat] (11) at (2,1) {};
    \node[lat] (20) at (4,0) {};
    \node[lat] (21) at (4,1) {};
    \draw (00) node [below]{$0$};
    \draw (01) node [above]{$1$};
    \draw (10) node [below]{$0$};
    \draw (11) node [above]{$1$};
    \draw (20) node [below]{$0$};
    \draw (21) node [above]{$1$};
    \draw[red,thick] (00) -- (10) -- (20);
    \draw[blue,thick] (00) -- (11) -- (21);
    \draw[green,thick] (01) -- (10) -- (21);
    \draw[yellow,thick] (01) -- (11) -- (20);
    \draw (7,0.5) node {(lines are elements of $R$)};
  \end{tikzpicture}
  %% \nopagebreak[4]
  \vskip2mm
  \begin{tikzpicture}[scale=1]
    \draw (0,.5) ellipse (4mm and 12mm);
    \draw (2,.5) ellipse (4mm and 12mm);
    \draw (4,.5) ellipse (4mm and 12mm);
    \node[lat] (00) at (0,0) {};
    \node[lat] (01) at (0,1) {};
    \node[lat] (10) at (2,0) {};
    \node[lat] (11) at (2,1) {};
    \node[lat] (20) at (4,0) {};
    \node[lat] (21) at (4,1) {};
    \draw (00) node [below]{$0$};
    \draw (01) node [above]{$1$};
    \draw (10) node [below]{$0$};
    \draw (11) node [above]{$1$};
    \draw (20) node [below]{$0$};
    \draw (21) node [above]{$1$};
    \draw[yellow,thick] (01) -- (11) -- (21);
    \draw[red,thick] (00) -- (10) -- (21);
    \draw[green,thick] (01) -- (10) -- (20);
    \draw[blue,thick] (00) -- (11) -- (20);
    \draw (7,0.5) node {(lines are elements of $S$)};
  \end{tikzpicture}
  \end{figure}
  \end{center}

\end{example}
}{}


%\pagebreak

%\input{appendix-centralizers.tex}
%\pagebreak[2]

\section{Miscellaneous Proofs}
\label{sec:proofs-elem-facts}
\subsection{Proof of Lemma~\ref{lem:min-abs-prod}}
\label{sec:proof-cor-min-abs-prod}
{\bf Lemma~\ref{lem:min-abs-prod}.} Let $\bB_i \leq \bA_i$ $(0\leq i < n)$ be algebras in the
variety $\var{V}$, let $\bB := \bB_0\times \bB_1\times \cdots \times \bB_{n-1}$, and let
$\bA := \bA_0\times \bA_1\times \cdots \times \bA_{n-1}$.
If $\bB_i\absorbing_{t_i}\bA_i$ (resp., $\bB_i\minabsorbing_{t_i}\bA_i$) for each $0\leq i < n$,
then $\bB \absorbing_s \bA$ (resp., $\bB \minabsorbing_s \bA$) 
where $s:= t_0\star t_1 \star \cdots \star t_{n-1}$.

\begin{proof}
  In case $n=2$, the fact that $\bB \absorbing_s \bA$ follows
  directly from Corollary~\ref{cor:fact2}.
  We first extend this result to \emph{minimal} absorbing subalgebras, still
  for $n=2$, and then an easy induction argument will
  complete the proof for arbitrary finite $n$.

  Assume $\bB_0 \minabsorbing_{t_0} \bA_0$ and $\bB_1 \minabsorbing_{t_1} \bA_1$.
  Then, as mentioned, Corollary~\ref{cor:fact2} implies $\bB \absorbing_s \bA$, where
  $\bB := \bB_0\times \bB_1$, $\bA := \bA_0\times \bA_1$,
  %% where $t = t_1\star \cdots \star t_n$, and $t_i = f_i$.
  and $s = t_0\star t_1$.
  %% (Below we will assume $s$ has arity $q$.)
  To show that $\bB_0\times \bB_1$ is minimal absorbing, we
  let $\bS$ be a proper subalgebra of $\bB_0\times \bB_1$ and 
  prove that $\bS$ is not absorbing in $\bA_0\times \bA_1$.
  By transitivity of absorption, it suffices to prove that 
  $\bS$ is not absorbing in $\bB_0\times \bB_1$.

  Let $t$ be an arbitrary term of arity $q$.
  For each $b \in B_1$, the set %$\{b_2\}^{-1}$
  \[
  S^{-1}b := \{b_0 \in B_0 \mid (b_0, b) \in S\}
  \]
  is easily seen to be a subuniverse of $\bB_0$.
  \ifthenelse{\boolean{arxiv}}{  
  Indeed, if $v$ is a term of arity $k$ and if $x_0, \dots, x_{k-1}$ belong to 
  $S^{-1}b$, then $(x_i, b)\in S$ and, since $\bS\leq \bA$, the element 
  \begin{align*}
    v^{\bA}((x_0, b), (x_1, b), \dots, (x_{k-1}, b)) &= 
    (v^{\bA_0}(x_0, x_1, \dots, x_{k-1}),v^{\bA_1}(b, \dots, b))\\ 
    &= (v^{\bA_0}(x_0, x_1, \dots, x_{k-1}), b)
  \end{align*}
  belongs to $S$.  Therefore, $v^{\bA_0}(x_0, x_1, \dots, x_{k-1})\in S^{-1}b$.

  Since}{Since}
  $\bS$ is a proper subalgebra, there exists $b^* \in B_1$ such that 
  $S^{-1}b^*$ is not all of $B_0$.
  Therefore, $S^{-1}b^*$
  %% $\{b_2\}^{-1}$
  is a proper subuniverse of $\bB_0$, so $S^{-1}b^*$ is not absorbing in $\bA_0$
  (by minimality of $\bB_0$). 
  Consequently,
  %% \begin{align*}
  %%   \exists& x_0, x_1, \dots, x_{q-1} \in S^{-1}b^*,\quad \exists b\in B_0,\\
  %%   \exists& j< q, \quad \exists  b'\notin S^{-1}b^*, \text{ such that }\\
  %%   & t^{\bA_0}(x_0, x_1, \dots, x_{j-1}, b, x_{j+1}, \dots, x_{q-1}) = b'.
  %% \end{align*}
  \begin{align*}
    \exists x_i \in S^{-1}b^*, &\quad \exists b\in B_0,
    \quad \exists j< q, \quad \exists  b'\notin S^{-1}b^* \\
    \text{ such that }  \;& t^{\bA_0}(x_0, x_1, \dots, x_{j-1}, b, x_{j+1}, \dots, x_{q-1}) = b'.
  \end{align*}
  Therefore,
  \begin{align*}
    %  \label{eq:1}
    &t^{\bA_0\times \bA_1}((x_0,b^*), %% (x_1,b^*),
    \dots, (x_{j-1}, b^*), (b, b^*), (x_{j+1}, b^*), \dots, (x_{q-1}, b^*))\\
    &= (t^{\bA_0}(x_0,%% x_1,
    \dots, x_{j-1}, b, x_{j+1}, \dots, x_{q-1}), t^{\bA_1}(b^*, \dots, b^*))= (b', b^*)
  \end{align*}
  and $(b', b^*)\notin S$ since $b' \notin S^{-1}b^*$.
  Finally, because $(x_i, b^*) \in S$ for all $i$, and since $t$ was an
  arbitrary term, it follows that $\bS$ is not absorbing in $\bB_0\times \bB_1$.

  Now fix $n>2$ and assume the result holds when there are at most $n-1$
  factors.
  Let $\bB' := \bB_0 \times \cdots \times \bB_{n-2}$ and
  $\bA' := \bA_0\times \cdots \times \bA_{n-2}$. By the induction hypothesis, 
  $\bB' \minabsorbing_{s'} \bA'$, and since
  $\bB_{n-1} \minabsorbing_{t_{n-1}} \bA_{n-1}$ we have (by 
  the $n=2$ case) 
  $\bB' \times \bB_{n-1} \minabsorbing_s \bA'\times \bA_{n-1}$,
  where $s = s' \star t_{n-1}$.
\end{proof}

\subsection{Proof of Lemma~\ref{lem:sdp-general}}
\label{sec:proof-lemma-sdp-general}

%%% OMITTING THE 2-fold case from the arxiv and journal version
\ifthenelse{\boolean{extralong}}{
  The argument used to prove the next lemma also works in the more
  general case of $n$-fold products.
  Although the arguments are almost identical, and although we present the
  general proof below, we begin with a proof of the $2$-fold case since it is so
  much easier to read.
  \medskip

  \noindent {\bf Lemma.}
  Let $\bA_1$ and $\bA_2$ be finite algebras in an idempotent variety, and
  suppose $\bB_i \minabsorbing \bA_i$ for $i=1,2$.
  Let $\bR \sdp \bA_1 \times \bA_2$, and let $R' = R \cap (B_1 \times B_2)$.  
  If $R'\neq \emptyset$, then $\bR' \sdp \bB_1 \times \bB_2$.

  \begin{proof}
  If $R'\neq \emptyset$, then $S_1 := \Proj_1 R'$ and 
  $S_2 := \Proj_2 R'$ are also nonempty.  
  We want to show $S_i = B_i$ for $i=1,2$.
  By minimality of $\bB_1 \minabsorbing \bA_1$ and by transitivity of
  absorption, it suffices to prove $\bS_1 \absorbing \bB_1$.
  Assume $\bB_1 \minabsorbing \bA_1$ with respect to $t$, say, $k = \ar(t)$.
  Fix $s_1, \dots, s_k \in S_1$, $b \in B_1$, and $j\leq k$.  Then
  \[
  \tilde{b} := t(s_1, \dots, s_{j-1}, b, s_{j+1}, \dots, s_k) \in B_1,
  \]
  and we must show $\tilde{b} \in S_1$. (For this will prove 
  $\bS_1 \absorbing \bB_1$.)  Since $\bR$ is subdirect, there exists $a\in A_2$
  with $(b,a) \in R$.  Also, there exist $s_1', \dots, s_k'$ in $S_2$ such that
  for all $i=1, \dots, n$ we have $(s_i, s_i')\in R'$. 
  Since all the pairs belong to $R$, the following expression is also in $R$:
  \begin{align}
    %    \label{eq:2}
    &t^{\bA_1\times \bA_2}((s_1,s_1'), \dots, (s_{j-1}, s_{j-1}'), 
    (b, a),  (s_{j+1}, s_{j+1}'), \dots, (s_k, s_k')) \nonumber \\
    &=(t^{\bA_1}(s_1, \dots, s_{j-1}, b, s_{j+1}, \dots, s_k),
    t^{\bA_2}(s_1', \dots, s_{j-1}', a, s_{j+1}', \dots, s_k')) = (\tilde{b},
    \tilde{a}).
  \end{align}
  Since $\bB_2$ is absorbing in $\bA_2$, we see that $\tilde{a}$ belongs to
  $\bB_2$. Therefore, $(\tilde{b}, \tilde{a}) \in R\cap (B_1\times B_2) = R'$, 
  which means $\tilde{b}\in S_1$, as desired. Of course, the same argument works
  to prove $S_2 = B_2$.
  \end{proof}
  As mentioned, the result generalizes to $n$-fold products and the proof is nearly identical
  to the one above.
}{}

\noindent {\bf Lemma~\ref{lem:sdp-general}.} %(Lem.~\ref{lem:sdp-general})
  Let $\bA_1, \dots, \bA_n$ be finite idempotent algebras of the same type, and
  suppose $\bB_i \minabsorbing \bA_i$ for $i=1,\dots, n$.
  Let $\bR \sdp \bA_1 \times \cdots \times \bA_n$, 
  and let $R' = R \cap (B_1 \times \cdots \times B_n)$.  
  If $R'\neq \emptyset$, then $\bR' \sdp \bB_1 \times \cdots \times \bB_n$.

\begin{proof}
  If $R'\neq \emptyset$, then for $1\leq i\leq n$ the projection 
  $S_i := \Proj_i R'$ is also nonempty.  
  We want to show $S_i = B_i$.
  By minimality of $\bB_i \minabsorbing \bA_i$ and by transitivity of
  absorption, it suffices to prove $\bS_i \absorbing \bB_i$.
  Assume $\bB_i \minabsorbing \bA_i$ with respect to $t$, say, $k = \ar(t)$.
  Fix $s_1, \dots, s_k \in S_i$, $b \in B_1$, and $j\leq k$.  Then 
  \[
  \tilde{b_i} := t^{\bA_i}(s_1, \dots, s_{j-1}, b, s_{j+1}, \dots, s_k) \in B_i,
  \]
  and we must show $\tilde{b_i} \in S_i$. (For this will prove 
  $\bS_i \absorbing \bB_i$.)  Since $\bR$ is subdirect, there exist
  $a_i \in A_i$ (for all $i\neq j$) such that 
  $\ba^\ast:= (a_1, \dots, a_{j-1}, b, a_{j+1}, \dots, a_n)\in R$. 
  Also, for each $1\leq j\leq n$ there exist $s_1^{(j)}, \dots, s_k^{(j)}$ 
  in $S_j$ such that
  for all $1\leq \ell \leq k$ we have 
  \[
  \bs_\ell := (s_\ell^{(1)},\dots, s_\ell^{(i-1)}, s_\ell, s_\ell^{(i+1)}, \dots, s_\ell^{(n)})\in R'.
  \] 
  Since all these $n$-tuples belong to $R$, the following expression is
  also in $R$: 
  \begin{align*}
    %    \label{eq:2}
    &t^{\bA_1\times \cdots \times \bA_n}(\bs_1, \dots, \bs_{j-1},\ba^\ast, \bs_{j+1}, \dots, \bs_{k})\\
    %
    &=t^{\bA_1\times \cdots \times \bA_n}((s_1^{(1)},\dots, s_1^{(i-1)}, s_1, s_1^{(i+1)}, \dots, s_1^{(n)}), \dots\\
    &\qquad \qquad \qquad  \dots, (s_{j-1}^{(1)},\dots, s_{j-1}^{(i-1)}, s_{j-1}, s_{j-1}^{(i+1)}, \dots, s_{j-1}^{(n)}),\\ 
    &\qquad \qquad \qquad\qquad \quad (a_1, \dots, a_{j-1}, b, a_{j+1}, \dots, a_n), \\
    & \qquad \qquad \qquad \qquad \quad \quad (s_{j+1}^{(1)},\dots, s_{j+1}^{(i-1)}, s_{j+1}, s_{j+1}^{(i+1)}, \dots, s_{j
+1}^{(n)}), \dots\\
    &\qquad \qquad \qquad \qquad \quad \quad \quad \dots, (s_{k}^{(1)},\dots,
    s_{k}^{(i-1)}, s_{k}, s_{k}^{(i+1)}, \dots, s_{k}^{(n)})).
  \end{align*}
This is equivalent to 
  \begin{align*}
    & \bigl( t^{\bA_1}(s_1^{(1)},\dots,s_{j-1}^{(1)},a_1, s_{j+1}^{(1)},\dots,s_{k}^{(1)}),\dots \\
    & \qquad \qquad \qquad\quad \dots, t^{\bA_{i-1}}(s_1^{(i-1)}, \dots, s_{j-1}^{(i-1)}, a_{j-1}, s_{j+1}^{(i-1)}, \dots, 
s_{k}^{(i-1)}),\\
    & \qquad \qquad \qquad\quad \quad\quad \quad t^{\bA_{i}}(s_1, \dots, s_{j-1}, b, s_{j+1}, \dots, s_{k}),\\
    & \qquad \qquad \qquad\quad \quad \quad \quad \quad t^{\bA_{i+1}}(s_1^{(i+1)}, \dots, s_{j-1}^{(i+1)}, a_{j+1}, s_{j
+1}^{(i+1)}, \dots, s_{k}^{(i+1)}), \dots\\
    & \qquad \qquad \qquad\quad \quad \quad \quad \quad \quad  \dots t^{\bA_n}(s_1^{(n)}, \dots, s_{j-1}^{(n)}, a_n, 
s_{j+1}^{(n)}, \dots, s_{k}^{(n)}) \bigr),
    %% &= (\tilde{b_1},\dots, \tilde{b_{n}}).
  \end{align*}
which reduces to $(\tilde{b_1},\dots, \tilde{b_{n}})$.
  Since $\bB_i$ is absorbing in $\bA_i$, we see that 
  $(\tilde{b_1},\dots, \tilde{b_{n}})\in (B_1 \times \cdots \times B_n)$.
  Therefore, 
  $(\tilde{b_1},\dots, \tilde{b_{n}})\in R\cap B_1 \times \cdots \times B_n$,
  which means $\tilde{b_i}\in S_i$, as desired. Of course, the same argument 
  works for all $1\leq i\leq n$. 
\end{proof}



\subsection{Proof of Lemma~\ref{lem:abelian-AF}}
\label{sec:proof-that-abelian}
An very useful property of abelian algebras is that they are absorption-free.
A proof of this appears in~\cite[Lem~4.1]{MR3374664}, but we include 
a proof in this section for easy reference and to keep the paper somewhat self-contained.
First we require an elementary fact about functions on finite sets.\\[4pt]
\noindent {\bf Fact A.1.}
%% \begin{Fact}
%% \label{fact:idemp-funct-fin-set}
  If $f: X \rightarrow X$ is a (unary) function on a finite set $X$, then there
  is a natural number $k\geq 1$ 
  such that the $k$-fold composition of $f$ with itself
  is the same function as the $2k$-fold composition.  That is, for all 
  $x \in X$, $f^{2k}(x) = f^k(x)$.\\[6pt]
\noindent {\bf Lemma~\ref{lem:abelian-AF}.}
Finite idempotent abelian algebras are absorption-free.

\begin{proof}
  Suppose $\bA$ is a finite idempotent abelian algebra with $\bB \absorbing_t \bA$.
  We show $\bB = \bA$.
  If $t$ is unary, then by idempotence $t$ is the identity function and
  absorption in this case means $t[A] \subseteq B$.  It follows that $A = B$ and
  we're done.  So assume $t$ has arity $k>1$.  We
  will show that there must also be a $(k-1)$-ary term operation 
  $s\in \sansClo(\bA)$ such that $\bB \absorbing_s \bA$.  
  It follows inductively that there must also be a unary absorbing term
  operation. Since a unary idempotent operation is the identity
  function, this will complete the proof.

  Define a sequence of terms $t_0, t_1, \dots$ as follows:
  for each $\bx = (x_1, \dots, x_{k-1}) \in A^{k-1}$ and $y\in A$,
  \begin{align*}
    t_0(\bx, y) &= t(\bx, y),\\
    t_1(\bx, y) &= t(\bx, t_0(\bx, y)) = t(\bx, t(\bx, y)), \\
    t_2(\bx, y) &= t(\bx, t_1(\bx, y)) = t(\bx, t(\bx, t(\bx, y))),\\
     &\vdots\\
    t_m(\bx, y) &= t(\bx, t_{m-1}(\bx, y)) 
                 = t(\bx, \dots, t(\bx, t(\bx, t(\bx, y))) \cdots )).
  \end{align*}
  It is easy to see that $\bB$ is absorbing in $\bA$ with respect to $t_m$, that
  is, $\bB \absorbing_{t_m} \bA$.  

  For each $\bx_i\in A^{k-1}$, 
  define $p_i:A \rightarrow A$ by $p_i(y) = t(\bx_i, y)$.  Then, 
  $p_i^m(y) = t_m(\bx_i, y)$, so by
  Fact~A.1 %% \ref{fact:idemp-funct-fin-set}
  there exists an $m_i\geq 1$ such that 
   $p_i^{2m_i} = p_i^{m_i}$.  That is, 
  $t_{m_i}(\bx_i,t_{m_i}(\bx_i,y)) = t_{m_i}(\bx_i,y)$.
  Let $m$ be the product of all the $m_i$ as $\bx_i$ varies over $A^{k-1}$.
  Then, for all $\bx_i \in A^{k-1}$, we have 
   $p_i^{2m} = p_i^{m}$.  Therefore, 
  for all $\bx \in A^{k-1}$, we have 
  $t_{m}(\bx,t_{m}(\bx,y)) = t_{m}(\bx,y)$.

  We now show that the $(k-1)$-ary term operation $s$, defined for all 
  $x_1, \dots, x_{k-1} \in A$ by 
  \[
  s(x_1, \dots, x_{k-2}, x_{k-1})=
  t(x_1, \dots, x_{k-2}, x_{k-1}, x_{k-1})
  \]
  is absorbing for $\bB$, that is, $\bB \absorbing_{s} \bA$.  It suffices to
  prove that $s[B \times \cdots \times B \times A] \subseteq B$. 
  (For if the factor involving $A$ occurs earlier, we appeal to
  absorption with respect to $t$.)
  So, for $\bb \in B^{k-2}$ and $a\in A$, we will show 
  $s(\bb, a) = t_m(\bb, a,a) \in B$.  For all $b\in B$, we have
  \[
  t_m(\bb, b, a) = t_m(\bb, b, t_m(\bb, b, a)).
  \]
  Therefore, if we apply (at the $(k-1)$-st coordinate) 
  the fact that $\bA$ is abelian, then we have
  \begin{equation}
    \label{eq:14}
    t_m(\bb, a, a) = t_m(\bb, a, t_m(\bb, b, a)).
  \end{equation}
  By absorption, $t_m(\bb, b, a)$ belongs to $B$, thus so does the
  entire expression on the right of~(\ref{eq:14}).  This proves that 
  $s(\bb, a) = t_m(\bb, a,a) \in B$, as desired.
\end{proof}













\ifthenelse{\boolean{arxiv}}{

\subsection{Direct Proof of Corollary~\ref{cor:fry-pan}}
\label{sec:proof-fry-pan-cor}


{\bf Corollary~\ref{cor:fry-pan}.}
  Let $\bA_0, \bA_1, \dots, \bA_{n-1}$ be finite idempotent algebras in a Taylor
  variety and suppose $\bR \sdp \prod \bA_i$.  For some $0< k < n-1$, assume the
  following:
  \begin{itemize}
  \item if $0\leq i < k$ then $\bA_i$ is abelian;
  \item if $k\leq i < n$ then $\bA_i$ has a sink $s_i \in A_i$.
  \end{itemize}
  Then
   $Z := R_{\kk}
  %% \Proj_{\kk}R 
  \times \{s_k\} \times \{s_{k+1}\} \times \cdots \times \{s_{n-1}\}  \subseteq R$,
  where $R_{\kk} = \Proj_{\kk}R$.

\begin{proof}
  Since $\bA_{\kk} := \prod_{i<k} \bA_i$ is abelian and lives in a Taylor variety, 
  there exists a term $m$ such that $m^{\bA_{\kk}}$ is a \malcev term operation on
  $\bA_{\kk}$ (Theorem~\ref{thm:type2cp}).  Since we are working with idempotent terms, we can be sure 
  that for each $i\in \nn$ the term operation $m^{\bA_i}$ is not
  constant (so depends on at least one of its arguments).

  Fix $\bz :=(r_0, r_1, \dots, r_{k-1}, s_k, s_{k+1}, \dots, s_{n-1}) \in Z$.
  We will show that $\bz \in R$.
  Since $\bz_{\kk}\in R_{\kk}$, there exists $\br \in R$ 
  whose first $k$ elements agree with those of $\bz$.  
  That is, $\br_{\kk} =  (r_{0}, r_{1}, \dots, r_{k-1}) = \bz_{\kk}$.

  Now, since $\bR$ is subdirect, there exists $\bx^{(0)} \in R$ such that 
  $\bx^{(0)}(k) = s_k$, the sink in $\bA_k$.
  If the term operation $m^{\bA_k}$ depends on its second or third argument,
  consider $\by^{(0)}= m(\br, \bx^{(0)}, \bx^{(0)}) \in R$.  
  (Otherwise, $m^{\bA_k}$ depends on its
  first argument, so consider $\by^{(0)}= m(\bx^{(0)}, \bx^{(0)}, \br)$.)
  For each $0\leq i < k$ we have 
  $\by^{(0)}(i) = m^{\bA_i}(r_i, \bx^{(0)}(i), \bx^{(0)}(i)) = r_i$,
  since $m^{\bA_i}$ is \malcev.   Thus, $\by^{(0)}_{\kk} = \bz_{\kk}$.
  At index $i = k$,  we have
  $\by^{(0)}(k) = m^{\bA_k}(r_k, s_k, s_k) = s_k$, since $s_k$ is a sink in $\bA_k$.
  By the same argument, but starting with $\bx^{(1)} \in R$ such that
  $\bx^{(1)}(k+1) = s_{k+1}$, there exists $\by^{(1)} \in R$ such that
  $\by^{(1)}_{\kk} = \bz_{\kk}$ and $\by^{(1)}(k+1) = s_{k+1}$.

  Let $t$ be any term of arity $\ell\geq 2$ that depends on at least two of its 
  arguments, say, arguments $p$ and $q$, and 
  consider $t(\by^{(0)}, \dots, \by^{(0)},\by^{(1)}, \by^{(0)}, \dots, \by^{(0)})$, where 
  $\by^{(1)}$ appears as argument $p$ (or $q$) and $\by^{(0)}$ appears elsewhere.
  By idempotence, and by the fact that $s_k$ and $s_{k+1}$ are sinks, we have
  \[
  t\left( \begin{array}{cccccccc}
    (r_0,&\dots,&r_{k-1},&s_k,&\ast,&\ast,& \dots,&\ast)\\
      & & \vdots& &&\\
    (r_0,&\dots,&r_{k-1},&s_k,&\ast,&\ast,& \dots,&\ast)\\
    (r_0,& \dots,& r_{k-1},&\ast,&s_{k+1},&\ast,& \dots,& \ast)\\
    (r_0,&\dots,&r_{k-1},&s_k,&\ast,&\ast,& \dots,&\ast)\\
       & &\vdots & &&\\
    (r_0,&\dots,&r_{k-1},&s_k,&\ast,&\ast,& \dots,&\ast)\\
  \end{array}\right) = (r_0, r_1, \dots, r_{k-1}, s_k, s_{k+1}, \ast, \dots, \ast),
  \]
  where the wildcard $\ast$ represents unknown elements.
Let $\br^{(1)} = (r_0, r_1, \dots, r_{k-1}, s_k, s_{k+1}, \ast, \dots, \ast)$ 
denote this element of $R$.  
Continuing as above, we find 
$\by^{(2)} = (r_0, \dots, r_{k-1}, \ast, \ast, s_{k+2}, \ast, \dots, \ast) \in R$, 
and compute 
\begin{align*}
\br^{(2)}:= t(\br^{(1)}, \dots, \br^{(1)}, & \, \by^{(2)}, \br^{(1)}, \dots, \br^{(1)}) = 
(r_0, \dots, r_{k-1}, s_k, s_{k+1}, s_{k+2}, \ast, \dots, \ast),\\
%&\uparrow\\
&\; ^{\widehat{\lfloor}} \, \text{$p$-th argument}
\end{align*}
which also belongs to $R$.  In general, once we have 
\begin{align*}
\br^{(j)}&:= (r_0,  \dots, r_{k-1}, s_k, \dots, s_{k+j}, \ast, \dots, \ast) \in R, \text{ and }\\
\by^{(j+1)} &:= (r_0,  \dots, r_{k-1}, \ast, \dots, \ast, s_{k+j+1}, \ast, \dots, \ast) \in R,
\end{align*} 
we compute 
\begin{align*}
\br^{(j+1)} &= t(\br^{(j)},  \dots, \br^{(j)}, \by^{(j+1)}, \br^{(j)}, \dots, \br^{(j)}) \\
              & = (r_0,  \dots, r_{k-1}, s_k, \dots, s_{k+j+1}, \ast, \dots, \ast) \in R.  
\end{align*}
Proceeding inductively in this way yields $\bz =
(r_0, \dots, r_{k-1}, s_k, \dots, s_{n-1}) \in R$, as desired.
\end{proof}

}{}








\ifthenelse{\boolean{arxiv}}{

\subsection{Other elementary facts}
The remainder of this section collects some observations that can be useful when
trying to prove that an algebra is abelian.  We have moved the statements and
proofs of these facts to the appendix since we didn't end up using any of them
in the paper.

Denote the diagonal of $A$ by $D(A) := \{(a,a): a \in A\}$. 

\begin{lemma}
\label{lem:diagonal}
An algebra $\bA$ is abelian if and only if there is some $\theta \in \Con (\bA^2)$ that has
the diagonal set $D(A)$ as a congruence class.
\end{lemma}
\begin{proof}
  ($\Leftarrow$) Assume $\Theta$ is such a congruence.  Fix 
  $k<\omega$,
  $t^{\bA}\in \sansClo_{k+1}(\bA)$, 
  $u, v \in A$, and
  $\bx, \by \in A^k$.
  We will prove the implication~(\ref{eq:22}), which in the present context is
  \begin{equation*}
    t^\bA(\bx,u) = t^\bA(\by,u) \quad \Longrightarrow \quad 
    t^{\bA}(\bx,v) = t^{\bA}(\by,v).
  \end{equation*}
  Since $D(A)$ is a class of $\Theta$, we have 
  $(u,u) \mathrel{\Theta} (v,v)$, and since $\Theta$ is a reflexive relation, we have
  $(x_i,y_i)  \mathrel{\Theta} (x_i,y_i)$ for all $i$.  Therefore,
  \begin{equation}
    \label{eq:9}  
    t^{\bA\times \bA}((x_1,y_1), \dots, (x_k,y_k), (u,u))
    \mathrel{\Theta}
    t^{\bA\times \bA}((x_1,y_1), \dots, (x_k,y_k), (v,v)).
  \end{equation}
  since $t^{\bA \times \bA}$ is a term operation of $\bA\times \bA$.
  Note that~(\ref{eq:9}) is equivalent to
  \begin{equation}
    \label{eq:13}
    (t^{\bA}(\bx, u), t^{\bA}(\by,u))
    \mathrel{\Theta}
    (t^{\bA}(\bx, v), t^{\bA}(\by, v)).
  \end{equation}
  If $t^{\bA}(\bx, u)= t^{\bA}(\by, u)$ then 
  the first pair in~(\ref{eq:13}) belongs to the $\Theta$-class
  $D(A)$, so the second pair must also belong this $\Theta$-class.
  That is, $t^{\bA}(\bx, v)= t^{\bA}(\by, v)$, as desired.

  \vskip2mm

  \noindent ($\Rightarrow$) Assume $\bA$ is abelian. We show
  $\Cg^{\bA^2}(D(A)^2)$ has $D(A)$ as a block.  Assume
  \begin{equation}
    \label{eq:16}
  ((x,x), (c,c')) \in \Cg^{\bA^2}(D(A)^2).
  \end{equation}
  It suffices to prove that $c=c'$.  Recall, \malcev's congruence generation
  theorem states that (\ref{eq:16}) holds iff
  \begin{align*}
  \exists \,& (z_0,z_0'), (z_1,z_1'), \dots, (z_n,z_n') \in A^2\\
    \exists \,& ((x_0,x_0'), (y_0,y_0')), ((x_1,x_1'), (y_1,y_1')), \dots, 
    ((x_{n-1},x_{n-1}'), (y_{n-1},y_{n-1}')) \in D(A)^2\\
    \exists \, & f_0, f_1, \dots, f_{n-1}\in F^*_{\bA^2}
  \end{align*}
  such that 
  \begin{align}
    \label{eq:7}
    \{(x, x),(z_1,z_1')\} &= \{f_0(x_0,x_0'), f_0(y_0,y_0')\}\\
    \nonumber
     \{(z_1,z_1'),(z_2,z_2')\} &= \{f_1(x_1,x_1'), f_1(y_1,y_1')\}\\
    \nonumber
     & \vdots\\
    \label{eq:8}
     \{(z_{n-1},z_{n-1}'),(c, c')\} &= \{f_{n-1}(x_{n-1},x_{n-1}'), f_{n-1}(y_{n-1},y_{n-1}')\}
  \end{align}
  The notation $f_i\in F^*_{\bA^2}$ means 
  \begin{align*}
    f_i(x, x') &= g_i^{\bA^2}((a_1, a_1'), (a_2, a_2'), \dots, (a_k, a_k'), (x, x'))\\
               &= (g_i^{\bA}(a_1, a_2, \dots, a_k, x), g_i^{\bA}(a_1', a_2', \dots, a_k', x')),
  \end{align*}
  for some $g_i^{\bA} \in \sansClo_{k+1}(\bA)$ and some constants 
  $\ba = (a_1, \dots, a_k)$ and $\ba' = (a_1', \dots, a_k')$ in $A^k$. 
  Now, $((x_i,x_i'), (y_i,y_i'))\in D(A)^2$ implies 
  $x_i=x_i'$, and $y_i=y_i'$, so in fact we have 
  \[
    \{(z_i,z_i'),(z_{i+1},z_{i+1}')\} = \{f_i(x_i,x_i), f_i(y_i,y_i)\} \quad (0\leq i < n).
  \]
  Therefore, by Equation~(\ref{eq:7}) we have either 
  \[
    (x,x)= (g_i^{\bA}(\ba, x_0), g_i^{\bA}(\ba', x_0)) \quad \text{ or } \quad 
    (x,x)= (g_i^{\bA}(\ba, y_0), g_i^{\bA}(\ba', y_0)).
  \]
  Thus, either $g_i^{\bA}(\ba, x_0) =  g_i^{\bA}(\ba', x_0)$ %\quad \text{ or } \quad 
  or $g_i^{\bA}(\ba, y_0) =  g_i^{\bA}(\ba', y_0)$.
  By the abelian assumption, if one of these equations holds, then so does the
  other. This and and Equation (\ref{eq:7}) imply $z_1 = z_1'$.  Applying the same
  argument inductively, we find that $z_i = z_i'$ for all $1\leq i < n$ and so, by
  (\ref{eq:8}) and the abelian property, we have $c= c'$.
\end{proof}

Lemma~\ref{lem:diagonal} can be used to prove the next result
which states that if there is a congruence of $\bA_1 \times \bA_2$ that has the
graph of a bijection between $A_1$ and $A_2$ as a block, then both $\bA_1$ and
$\bA_2$ are abelian algebras.

\begin{lemma}
  \label{lem:bijection_abelian}
  Suppose $\rho \colon A_0 \to A_1$ is a bijection and suppose the graph
  $\{(x, \rho x) \mid x \in A_0\}$ is a block of some congruence
  $\beta \in \Con (A_0 \times A_1)$.  Then both $\bA_0$ and $\bA_1$ are abelian.
\end{lemma}
\begin{proof}
  Define the relation $\alpha\subseteq (A_1\times A_1)^2$ as follows: for
  $((a,a'), (b,b')) \in (A_1\times A_1)^2$,
  \[
  (a,a')\mathrel{\alpha} (b,b')
  \quad \iff \quad
  (a, \rho a') \mathrel{\beta} (b, \rho b')
  \]
  We prove that the diagonal $D(A_1)$ is a block of $\alpha$ by showing that
  $(a, a) \mathrel{\alpha} (b,b')$ implies $b = b'$.
  Indeed, if $(a, a) \mathrel{\alpha} (b,b')$, then
  $(a, \rho a) \mathrel{\beta} (b, \rho b')$, which means that
  $(b, \rho b')$ belongs to the block and
  $(a, \rho a)/\beta = \{(x, \rho x): x\in A_1\}$.  Therefore,
  $\rho b  = \rho b'$, so $b = b'$ since $\rho$ is injective.
  This proves that $\bA_1$ is abelian.

  To prove $\bA_2$ is abelian, we reverse the roles of $A_1$ and $A_2$ in the
  foregoing argument.  
  If $\{(x, \rho x) \mid x \in A_1\}$ is a block of $\beta$,
  then 
  $\{(\rho^{-1}(\rho x), \rho x) \mid \rho x \in A_2\}$ is a block of $\beta$; that
  is, $\{(\rho^{-1} y, y) \mid y \in A_2\}$ is a block of $\beta$.  Define 
  the relation $\alpha\subseteq (A_2\times A_2)^2$ as follows: for
  $((a,a'), (b,b')) \in (A_2\times A_2)^2$,
  \[
  (a,a')\mathrel{\alpha} (b,b')
  \quad \iff \quad
  (\rho^{-1}a, \rho a') \mathrel{\beta} (\rho^{-1}b, \rho b').
  \]
  As above, we can prove that the diagonal $D(A_2)$ is a block of $\alpha$
  by using the injectivity of $\rho^{-1}$ to show that $(a, a) \mathrel{\alpha}
  (b,b')$
  implies $b = b'$.
\end{proof}


\begin{lemma}
\label{lem:triv-clone-implies-abelian}
If $\sansClo(\bA)$ is trivial (i.e., generated by the projections),
then $\bA$ is abelian.
\end{lemma}
(In fact, it can be shown that $\bA$ is \emph{strongly abelian} in this case. 
We won't need this stronger result, and the proof that $\bA$ is abelian is elementary.)
\begin{proof}
We want to show $\C{1_A}{1_A}$.  Equivalently, we must show
that for all $t\in \sansClo(\bA)$ (say, $(\ell+m)$-ary) 
and all $a, b \in A^\ell$, we have $\ker t(a,\cdot)=\ker t(b,\cdot)$.
We prove this by induction on the height of the term $t$.  Height-one terms are
projections and the result is obvious for these.  Let $n>1$ and assume the result
holds for all terms  of height less than
$n$.  Let $t$ be a term of height $n$, say, $k$-ary.  Then for some terms 
$g_1, \dots, q_k$ of height less than $n$ and for some $j\leq k$, we have
$t = p^k_j [q_1, q_2, \dots, q_k] = q_j$ and since $q_j$ has height less than
$n$, we have
\[
\ker t(a,\cdot)=\ker g_j(a,\cdot) = \ker g_j(b,\cdot)=\ker t(b,\cdot).
\]\end{proof}
}{}


%% wjd: We never use lem:M3-abelian and we prove a stronger result in the paper, so I'm
%% commenting it out. (will probably delete it soon)
% \begin{lemma}[Lemma \ref{lem:M3-abelian}]
% If $\alpha_1$, $\alpha_2$, $\alpha_3 \in \Con(\bA)$ are pairwise complements,
% then $\sansC(1_A, \alpha_i)$ for each $i=1,2,3$.  If, in addition, $\bA$ is
% idempotent and has a Taylor term operation, then $\sansC(1_A, 1_A)$; that is, $\bA$ is abelian.
% \end{lemma}
% \begin{proof}
%   The goal is to prove $\sansC(1_A, 1_A)$.
%   By Lemma~\ref{lem:centralizers}~(\ref{fact:centralizing_over_meet}), we have
%   $\sansC(\alpha_1, \alpha_2; \alpha_1 \meet \alpha_2)$.  
%   Since $\alpha_1 \meet \alpha_2= 0_A$, this means
%   $\sansC(\alpha_1, \alpha_2)$.
%   Similarly, $\sansC(\alpha_3, \alpha_2)$.  Therefore, by 
%   Lemma~\ref{lem:centralizers}~(\ref{fact:centralizing_over_join1}), we have
%   $\sansC(\alpha_1 \join \alpha_3, \alpha_2)$. This is equivalent to 
%   $\sansC(1_A, \alpha_2)$, since $\alpha_1 \join \alpha_3 = 1_A$. 
%   The same argument \emph{mutatis-mutandis} yields
%   $\sansC(1_A,\alpha_1)$ and $\sansC(1_A,\alpha_3)$. 
%   Before proceeding, note that $\sansC(\alpha_1, \alpha_1)$, by 
%   Lemma~\ref{lem:centralizers}~(\ref{fact:monotone_centralizers1}).
%   Now, if $\bA$ is idempotent and has a Taylor term operation, then
%   by \ref{thm:kearnes-kiss-3.27} we have 
%   $\sansC(\alpha_1 \join \alpha_2,\alpha_1 \join \alpha_2; \alpha_2)$.
%   That is, $\sansC(1_A,1_A; \alpha_2)$.
%   Similarly, $\sansC(1_A,1_A; \alpha_3)$.
%   By~\ref{lem:centralizers}~(\ref{fact:centralizing_over_meet2}) then, 
%   $\sansC(1_A,1_A; \alpha_2 \meet\alpha_3)$. 
%   That is, $\sansC(1_A,1_A)$.
% \end{proof}







\ifthenelse{\boolean{extralong}}{





















  
\section{Strongly Tractable Divisors}
Suppose $\bA$ is a finite idempotent algebra with congruence relation $\theta \in \Con(\bA)$.
Let $\sI$ be an $n$-variable instance of $\CSP(\bA)$ and suppose
$\mathcal C = \{(\sigma_j, R_j): 0\leq j < p\}$ is the (finite) set of constraints of $\sI$.
%% For $\bx = (x_0, x_1, \dots, x_{k-1}) \in A^k$, denote by $\bx/\theta$ the corresponding
%% tuple of equivalence classes of components of $\bx$.  That is,
%% \[
%% \bx/\theta = (x_0/\theta, x_1/\theta, \dots, x_{k-1}/\theta) \in (A/\theta)^k.
%% \]
Recall, %% for each constraint $(\sigma_j, R_j) \in \mathcal C$ the
%% quotient constraint $(\sigma_j, R_j/\theta)$ has 
%% $R_j/\theta := \{\br/\theta \in (A/\theta)^{\sigma_j} \mid \br \in R_j\}$,
%% and 
the quotient instance $\sI/\theta$ is the $n$-variable instance of
$\CSP(\bA/\theta)$ with constraint set given in (\ref{eq:100}) above.
We say that $\bA/\theta$ is a \emph{strongly tractable divisor} of $\bA$ if there exist
constants $c$ and $d$ and an algorithm that, given an instance $\sI$ of $\CSP(\bA)$,
\begin{itemize}
\item determines the full set $\mathcal S$ of solutions to the quotient instance $\sI/\theta$, and
\item takes at most $c |\sI|^d$ steps to complete.
\end{itemize}
Here $|\sI|$ denotes the size of the instance $\sI$, which we take to
be the length of a string encoding all the scopes and tuples of constraints
of $\sI$.
We use the adjective ``strongly'' in the definition above because the algorithm must determine
the full set of solutions to $\sI/\theta$, as opposed to just deciding whether or not
there is a solution.
On the other hand, the bound on the running time of the
algorithm is $c |\sI|^d$ and not $c|\sI/\theta|^d$, so this notion of tractability is maybe not
quite as strong as it first appears.


\subsection{Application: strongly tractable atop tractable}
Let $\bA$ be a finite idempotent algebra with a congruence $\theta$ whose
blocks belong to a tractable variety $\var{V}$, and 
suppose that $\bA/\theta$ is a strongly tractable divisor of $\bA$.
Our goal is to prove $\CSP(\bA)$ is tractable.

As a guide to intuition, we imagine applying the algorithm described in this section to situations
in which the number of congruence classes of $\theta$---and therefore the number of
solutions to quotient instances---is relatively small.  To this end, we
call $\theta\in \Con(\bA)$ a \emph{coarse congruence}
if there are constants $c$ and $d$ such that, for every instance $\sI$ of $\CSP(\bA)$,
the number of solutions to the quotient instance $\sI/\theta$ is bounded above by $c|\sI|^d$.


Recall Fact~\ref{fact:soln-quotient} above which says that if the quotient instance
$\sI/\theta$ has no solution, then $\sI$ has no solution.
This and the notion of a ``coarse congruence'' suggest the following algorithm for solving $\CSP(\bA)$.
Given an instance $\sI$ of $\CSP(\bA)$,
\begin{enumerate}
\item compute the solution set $\mathcal S$ of the quotient instance $\sI/\theta$ of $\CSP(\bA/\theta)$;
\item for each solution $\bx/\theta\in \mathcal S$ and corresponding block instance $\sI_{\bx}$,
check %% (in polynomial-time)
whether there is a solution to $\sI_{\bx}$.
%% If there is a solution, then this will also be a solution to the original instance.
%% Since $\bB_i \in \var{V}$, a tractable variety, we can determine in polynomial time whether the restricted instance
(If $\sI_{\bx}$ has a solution, then so does $\sI$---in fact, a solution to $\sI_{\bx}$ \emph{is} a solution to $\sI$.)
\end{enumerate}
If for all $\bx/\theta \in S$ the block restricted instance $\sI_{\bx}$ has no solution, then $\sI$ has no solution.

The number of steps in stage (1) of the algorithm is bounded by a polynomial in $|\sI|$
since we assumed $\bA/\theta$ is a strongly tractable divisor of $\bA$.
In stage (2), for each solution $\bx/\theta \in \mathcal S$,
determining whether there is a solution to $\sI_{\bx}$ has polynomial time complexity since
$\CSP(\bB_{\chi(0)}, \bB_{\chi(1)}, \dots, \bB_{\chi(n-1)})$ is tractable.  Therefore,
if we can find some constants $c$ and $d$ such that the number $|\mathcal S|$ of solutions to the quotient instance
$\sI/\theta$ is bounded above by $c|\sI|^{d}$, then stage (2) can be completed in
polynomial 
  \ifthenelse{\boolean{footnotes}}{%
    time.\footnote{To be a little 
      more precise, the time complexity of stage (2) of the algorithm is bounded
      by $c|\sI|^{d} \cdot c' |\sI|^{d'} \approx C|\sI|^{D}$, where $c'$ and $d'$ are found
      by bounding the complexity of 
      $\CSP(\bB_{\chi(0)}, \bB_{\chi(1)}, \dots, \bB_{\chi(n-1)})$ for a
      ``worst-case product'' of $\theta$ classes.}
  }{time.}
The next proposition summarizes what we have shown.
\begin{proposition}
  \label{prop:chain-atop-tractable}
  Suppose $\bA$ is a finite idempotent algebra with $\theta \in \Con(\bA)$ satisfying the following conditions:
  \begin{itemize}
  \item  $\bA/\theta$ is strongly tractable;
  \item  $\theta$ blocks belong to a tractable variety;
  \item  $\theta$ is coarse.
  \end{itemize}
  Then $\CSP(\bA)$ is tractable.
\end{proposition}
}{}



















\begin{comment}


\subsection{Proof of Lemma~\ref{lem:absorbing-adj}}
\label{sec:proof-lemma-absorbing-adj}
%% If $\sigma$ and $\tau$ are disjoint subsets of $\nn$, let 
%% $\iota$ be the coproduct
%% \[\iota  = \sigma + \tau := \{\<0, s\> \mid s\in \sigma\} \cup\{\<1, t\> \mid t\in \tau\}.\]
%% Suppose $\bx \in \myprod_{\sigma} A_i$ and $\by \in \myprod_{\tau} A_i$, so $\bx$ is a function
%% from $\sigma$ to $\bigcup_\sigma A_i$ that takes $j \in \sigma$ to $\bx(j)\in A_j$.  
%% Similarly, $\by$ is a function from $\tau$ to $\bigcup_{\tau} A_i$.  
%% For such $\bx$ and $\by$, define $\<\bx, \by\> \in \myprod_{\iota} A_i$ as follows:
%% \[\<\bx, \by\>(i) = 
%% \begin{cases}
%%   \bx(s), & \text{ if $i = \<0,s\> \in \sigma$},\\
%%   \by(t), & \text{ if $i = \<1,t\> \in \tau$}.
%% \end{cases}\]
%% For $X \subseteq \myprod_\sigma A_i$ and $R\subseteq \myprod_{\nn}A_i$, we define
%% \[R_{\tau}[X]:=\{\by \in \myprod_\tau A_i \mid \exists \bx \in X \st \<\bx, \by\> \in R_\iota\}.\]
%% For example, if $R\subseteq \myprod_{\nn} A_i$ and if $B_2\subseteq A_2$, then 
%% \[R_{2'}[B_2] := \{\by \in \myprod_{i\neq 2} A_i \mid \exists b \in B_2 \st \<b, \by\> \in R\};\]
%% in this case $\<b,\by\> = (\by(0), \by(1),\, b,\, \by(3), \dots, \by(n-1))$.
%% Another example is
%% \[R_2[R_{2'}[B_2]] := \{b \in A_2 \mid \exists \by \in R_{2'}[B_2] \st \<b,\by\> \in R\}.\]
%% Observe that $R_2[R_{2'}[B_2]]$ contains the set 
%% $B_2$.\footnote{But $R_2[R_{2'}[\cdot]]$ is not a closure operator since it's not idempotent.

\begin{lemma}
Let $\bA_0, \bA_1$ be finite algebras %% with $\bB_i \minabsorbing \bA_i$ ($i=0,1$) 
and suppose $\bR \sdp \bA_0 \times \bA_1$.
\begin{enumerate}
\item If $\bC \absorbing \bA_0$, then $R_1[C] \absorbing A_1$.
\item If $\bD \absorbing \bA_1$, then $R_0[D] \absorbing A_0$.
\end{enumerate}
\end{lemma}
\begin{proof}
  Recall the notation: if 
  $\iota  = \sigma \cup \tau$ is a disjoint union of subsets of $\nn$,
  and if $\bx \in \myprod_{\sigma} A_i$ and $\by \in \myprod_{\tau} A_i$, then
  $\bx \oplus \by \in \myprod_{\iota} A_i$ is defined as follows:
  \[(\bx \oplus \by)(i) = 
  \begin{cases}
    \bx(i), & \text{ if $i \in \sigma$},\\
    \by(i), & \text{ if $i \in \tau$}.
  \end{cases}\]
  For $X \subseteq \myprod_\sigma A_i$
  and $R\subseteq \myprod_{\nn}A_i$, we define
  $R_{\tau}[X]:=\{\by \in \myprod_\tau A_i \mid \exists \bx \in X \st \bx \oplus \by \in R_\iota\}$.

  Let $\bC \absorbing \bA_0$ with respect to $k$-ary term $t$.
  Fix $y_0, y_1, \dots, y_{k-1}$ in $R_1[C]$ and $a' \in A_1$.
  We wish to prove that
\begin{equation}
  \label{eq:11}
\tilde{v}:= t^{\bA_1}(y_0, \dots, y_{i-1}, a', y_{i+1},\dots, y_{k-1})
\end{equation}
belongs to $R_1[C]$ (where the position of $a'$ is arbitrary).
For each $\ell\in \kk$, we have $y_\ell \in R_1[C]$ which means there exists 
$c_\ell \in C$ such that $(c_\ell, y_\ell) \in R$.  Also, since $\bR$ is a 
subdirect product, for each $a' \in A_1$ there exists $a\in A_0$ such that 
$(a,a') \in R$. Therefore, if we use $(\tilde{u}, \tilde{v})\in A_0 \times A_1$ 
to denote
\begin{equation*}
  t^{\bA_0\times \bA_1}((c_0, y_0), \dots, (c_{i-1}, y_{i-1}), (a, a'), (c_{i+1},y_{i+1}),\dots, (c_{k-1}, y_{k-1})),
\end{equation*}
then $\tilde{v}$ is given by (\ref{eq:11}), while
$\tilde{u} = t^{\bA_0}(c_0, \dots, c_{i-1}, a, c_{i+1},\dots, c_{k-1})$.
Notice that $\tilde{u}$ belongs to $C$, since $C$ is absorbing with respect to $t$.
Also, since $\bR$ is a subalgebra, $(\tilde{u}, \tilde{v}) \in R$.
Therefore, $\tilde{v}$ belongs to $R_1[C]$, as desired.
Thus $R_1[C]\absorbing A_1$.
The second item is proved similarly.
\end{proof}


%% Observe that by~\ref{lem:absorbing-adj}
%% $R_0[R_1[C]] \absorbing A_0$ and $R_1[R_0[D]] \absorbing A_1$.
%% In general, let
%% \begin{align*}
%% (R_0 \circ^1 R_1) [C] &:= R_0[R_1[C]],\\
%% (R_0 \circ^2 R_1) [C] &:= R_1[R_0[R_1[C]]], \\
%% &\vdots\\
%% (R_0 \circ^{2k} R_1) [C] &:= R_1[R_0[ \cdots [R_0[R_1[C]]]\cdots ]],\\
%% (R_0 \circ^{2k+1} R_1) [C] &:= R_0[R_1[ \cdots [R_0[R_1[C]]]\cdots ]].
%% \end{align*}
%% Note that $(R_0 \circ^{2k} R_1) [C]\absorbing A_1$ and 
%% $(R_0 \circ^{2k+1} R_1) [C]\absorbing A_0$.
%% Similarly, we have $(R_1 \circ^{2k} R_0) [D]\absorbing A_0$ and 
%% $(R_1 \circ^{2k+1} R_0) [D]\absorbing A_1$.
%% Now, since $\bB_1\minabsorbing \bA_1$ and since $R_1[B_0]\absorbing A_1$, then 
%% either $R_1[B_0]\cap B_1 = \emptyset$, or 
%% $B_1 \subseteq R_1[B_0]$. In the latter case, 
%% $R \cap (B_0 \times B_1) \neq \emptyset$, and we are done. So assume  the former case:
%% $R_1[B_0]\cap B_1 = \emptyset$.
%% ...\\\\
%% \todo{finish the proof of statement~(\ref{eq:3}).}
%% \\\\
Here is a partial generalization of Lemma~\ref{lem:absorbing-adj}.
\begin{lemma}
\label{lem:intersection}
Let $\bA_0, \bA_1, \dots, \bA_{n-1}$ be algebras in a
Taylor variety with $\bB_i \minabsorbing \bA_i$ ($i\in \nn$) and suppose
$\bR$ is a subdirect product of $\myprod_{\nn} \bA_i$. Then,
for each $j\in \nn$, $R_{j'}[B_j]$ is absorbing in $\bR_{j'}$.
\end{lemma}
\begin{proof}
For ease of notation and without loss of generality, we prove the first item for $j=0$. 
Let $\bB_0$ be minimal absorbing 
in $\bA_0$ with respect to $k$-ary term $t$.
Fix $\by_0, \by_1, \dots, \by_{k-1}$ in $R_{0'}[B_0]$, and
%% $\ba$ in $\myprod_{j'}\bA_i$, and denote $\myprod_{j'}\bA_i$ by
$\br'$ in $R_{0'}$, and let $\underline{\bA}':=\myprod_{0'}\bA_i$.
We wish to prove that
\begin{equation}
  \label{eq:110001}
\tilde{\bv}:= t^{\ubA'}(\by_0, \dots, \by_{i-1}, \br', \by_{i+1},\dots, \by_{k-1})
\end{equation}
belongs to $R_{0'}[B_0]$.
For each $\ell\in \kk$, we have 
$\by_\ell \in R_{0'}[B_0]$ which means there exists $b_\ell \in B_0$ such that 
$b_\ell \oplus \by_\ell \in R$.  Also, since $\br'$ belongs to $R_{0'}$, there exists 
$r_0\in R_0$ such that $r_0 \oplus \br' \in R$.  
Therefore, if we use $\tilde{u} \oplus \tilde{\bv}\in A_0 \times \myprod_{0'}A_i$ 
to denote
\begin{equation*}
t^{\ubA}(b_0 \oplus \by_0, \dots, b_{i-1} \oplus \by_{i-1}, r_0 \oplus \br', 
b_{i+1} \oplus \by_{i+1},\dots, b_{k-1} \oplus \by_{k-1}),
\end{equation*}
then $\tilde{\bv}$ is given by (\ref{eq:110001}), while
$\tilde{u} = t^{\bA_0}(b_0, \dots, b_{i-1}, r_0, b_{i+1},\dots, b_{k-1})$
belongs to $B_0$, since $B_0$ is absorbing with respect to $t$.
Also, since $\bR$ is a subalgebra, $\tilde{u} \oplus \tilde{\bv} \in R$.
Therefore, $\tilde{\bv}$ belongs to $R_{0'}[B_0]$, as desired.
\end{proof}
%% \todo{Prove the following under the assumptions of Lemma~\ref{lem:intersection}: 
%%   \begin{equation}
%%     \label{eq:15}
%%     \text{If $\etaR_k \neq \etaR_\ell$ for all $k \neq \ell$, 
%%     then $R \cap \myprod B_i \neq \emptyset$.}
%%   \end{equation}
%% }

\end{comment}
























% We recommend abbrvnat bibliography style.

\bibliographystyle{amsplain} %% or amsalpha
%% \bibliographystyle{plain-url}
%% \bibliographystyle{abbrvnat}
\bibliography{inputs/refs.bib}

\end{document}



















\begin{lemma}
  \label{lem:nonconstant-terms-exist}
  If $\bA$ and $\bB$ are nontrivial algebras in a Taylor variety $\var{V}$, then for
  some $k>1$ there is a $k$-ary term $t$ in $\var{V}$ such that $t^{\bA}$ and
  $t^{\bB}$ each depends on at least two of its arguments.
\end{lemma}
%%% wjd: (20161008) omitting the proof.  It is trivial.
%% \begin{proof}
%%   Since $\var{V}$ is a Taylor variety, there exists a $k$-ary idempotent term $t$ such that
%%   $\var{V}$ satisfies for each $0\leq i < k$ an identity of the form
%%   $t(\ast, \dots, \ast, x, \ast, \dots, \ast)
%%   \approx t(\ast, \dots, \ast, y, \ast, \dots, \ast)$ where $x$ and $y$ are
%%   distinct and appear in position $i$ in the argument list of $t$.
%%   Therefore the term operation $t^{\bA}$ depends on at least two of its
%%   variables.  If not, then without loss of generality assume
%%   $t^{\bA}$ depends only on its first argument. That is, there is a unary
%%   polynomial $p(x)$ in $\sansPol_1(\bA)$ such that 
%%   $t^{\bA}(x, a_1, \dots, a_{k-1}) = p(x)$, for all $a_i\in A$.
%%   Then, for all $x, a_i\in A$ we have
%%   \[
%%   p(x) = t^{\bA}(x, a_1, \dots, a_{k-1}) = 
%%   t^{\bA}(x, x, \dots, x) = x,
%%   \]
%%   by idempotence.
%%   But then the first Taylor identity gives
%%   \[
%%   x = t^{\bA}(x, \ast, \dots, \ast) =
%%   t^{\bA}(y, \ast, \dots, \ast) = y
%%   \]
%%   for all $x, y \in A$, so $A$ is trivial, contradicting our hypothesis.
%%   By the same argument, $t^{\bB}$ depends on at least two arguments.
%% \end{proof}
